% Chapter06 - Alkenes.tex
% Copyright (c) 2014 - 2016, zhiayang@gmail.com

% Preamble.tex
% Copyright (c) 2014 - The Foreseeable Future, zhiayang@gmail.com

\setlength{\parindent}{0pt}
\setlength{\parskip}{\baselineskip}

\setlist{nosep}



\pagebreak
\section{Alkenes}

\subsection{Open Chain}

	Alkenes are simply unsaturated hydrocarbons, with one or more double bonds. They have the general form of
	\ch{C_nH_{2n}}

	\diagram{

		\chemfig{C(-[:135]H)(-[:225]H)=C(-[:45]H)(-[:315]H)}
		\hspace{15mm}
		\chemfig{C(-[:135]H)(-[:225]H)=C(-[:270]H)(-[:0]C(-[:0]H)(-[:90]H)(-[:270]H))}

	}{Ethene (\ch{C2H4}) and propene (\ch{C3H6}) are examples of alkenes.}

% end subsection

\subsection{Cycloalkenes}

	Cycloalkenes are simply cycloalkanes where one or more of the \ch{C-C} bonds have been replaced with a \ch{C=C} double
	bond.

	\diagram{

		\chemfig{*6(=-=---)}
		\hspace{15mm}
		\chemfig{*6(---=--)}

	}{Cyclohex-1,3-ene and cyclohexene are examples of cycloalkenes.}

% end subsection



\subsection{Physical Properties}

	The physical properties of alkenes, including melting and boiling points, density and solubility follow the
	same trends as alkanes. Larger molecules have higher melting and boiling points, and higher densities, and alkenes are
	generally only soluble in non-polar solvents.

% end subsection

\pagebreak
\subsection{Stability of Carbocations}

	Before reactions and mechanisms of alkenes can be discussed, it is important to note the rules governing the formation
	of products, and the behaviour of molecules during the reaction.

	All physical systems have a tendency to move to the lowest energy state –– this state is characterised by the formation
	of the most stable molecules. As such, ions and radicals are inherently unstable.

	\subsubsection{Structure}

		One of the important intermediate products are carbocations, which are alkyl groups with an sp\sps{2} hybridised
		central carbon atom, and carries a positive charge on that atom.

		\diagram{
			\chemfig{C|$\oplus$(<[:315,,1]R1)(<:[:45,,1]R2)(-[:180]R)}
		}

		The 3 substituent groups are arranged in a trigonal planar fashion, with the p-orbitals above and below this
		plane. As such, nucleophiles can attack the carbocation from either the top or bottom.

	% end subsubsection


	\subsubsection{Stability}

		Charged ions are inherently more unstable than their neutral molecule counterparts; any species that stabilises
		the charge on the carbon ion would in turn increase the stability of the entire molecule. The primary reason for
		this is that molecules that are created from a more stable intermediate product have a higher chance to form. Thus,
		the probability of formation of a given product of a reaction can be estimated by looking at the stability of
		the intermediate compound leading to its creation.

		For carbocations, the carbon atom has a positive charge, thus electron-donating substituents such as alkyl
		groups (\ch{-CH3}), would stabilise the ion, as the donated electrons partially disperse the positive charge on the
		central atom.

		Conversely, electron-withdrawing species such as halogens (\ch{F2}, \ch{\chlorine2} etc.) would further destabilise
		the carbocation, and as such products that involve the formation of these intermediates would have constitute much
		lower proportion of the final products.

	% end subsubsection


\subsection{Electrophilic Addition}

	The primary reaction mechanism of alkenes is electrophilic addition. Due to the high electron density of the π-bonds,
	electrophiles are readily attracted –– thus alkenes are far more reactive than alkanes.

	During a reaction, the comparatively weaker π-bond is preferentially broken over the stronger σ-bond; only a
	\ch{C-C} bond remains, and atoms are \textit{added} to the carbons, since they are now able to form an additional
	bond each. Hence, electrophilic \textit{addition}.



	\subsubsection{Electrophilic Addition of Hydrogen Halides}

		In this reaction, the hydrogen and halogen atom are added across the double bond of the alkene. The hydrogen halide
		should be in a gaseous state for this reaction.

		\textbf{Conditions:} Gaseous HX (usually \ch{H\chlorine} or \ch{HBr}).
		\vspace{1.0em}

		\subsubtext{Step 1}

		\diagram[1.0]{
			\schemestart[0, 1.5, thick]
			\chemfig{C(-[:135]H)(-[:225]H)=[@{b1}:0]C(-[:45]H)(-[:315]H)}
			\hspace{5mm} + \hspace{5mm}
			\chemfig{\delp|@{hyd}H-[@{b2}:0]@{hal}{\color{OliveGreen}X}|\delm}
			\arrow{->[slow]}
			\chemfig{C(-[:90]H)(-[:180]H)(-[:270]H)-C|$\oplus$(-[:45,,1]H)(-[:315,,1]H)}
			\hspace{5mm} + \hspace{5mm}
			\chemfig{{{\color{OliveGreen}X}}\mch}
			\schemestop

			\chemmove{\draw[-Stealth,line width=0.4mm,shorten <=2mm,shorten >=1mm](b1) .. controls +(90:20mm) and +(90:20mm) .. (hyd);}
			\chemmove{\draw[-Stealth,line width=0.4mm,shorten <=2mm,shorten >=1mm](b2) .. controls +(270:8mm) and +(270:8mm) .. (hal);}
		}

		This is the rate-determining step, which involves the breaking of the π-bond. The polar \ch{HX} molecule has to
		approach the electron cloud of the π-bond in the correct orientation (hydrogen-facing), where the π-bond electrons
		attack the electron deficient, \ch{"\ox[parse=false]{\delp,H}"} atom.

		The \ch{H-X} bond then undergoes heterolytic fission, producing a carbocation intermediate and a halide ion.


		% look better
		\pagebreak
		\subsubtext{Step 2}

		\diagram[1.0]{
			\schemestart[0, 1.5, thick]
			\chemfig{C(-[:90]H)(-[:180]H)(-[:270]H)(-C|@{pl}$\oplus$(-[:45,,1]H)(-[:315,,1]H))}
			\hspace{5mm} + \hspace{5mm}
			\chemfig{@{hal}\lewis{2:,\color{OliveGreen}X}|\mch}
			\arrow
			\chemfig{C(-[:90]H)(-[:180]H)(-[:270]H)-C(-[:90]{{\color{OliveGreen}X}})(-[:0]H)(-[:270]H)}
			\schemestop

			\chemmove{\draw[-Stealth,line width=0.4mm,shorten <=2mm,shorten >=1mm](hal) .. controls +(90:15mm) and +(35:15mm) .. (pl);}
		}{Note that the electron pair arrow points to the plus charge on the carbon, \textit{not} the carbon atom itself.}

		This is the fast step; the bromide anion acts as a nucleophile, attacking the positively-charged carbocation.
		Since this step involves the reaction of two oppositely-charged species, it is a fast step.


		\vspace{1.0em}
		\subsubtext{Overall Reaction}

		\diagram[1.0]{
			\schemestart[0, 1.5, thick]
			\chemfig{C(-[:135]H)(-[:225]H)=C(-[:45]H)(-[:315]H)}
			\hspace{5mm} + \hspace{5mm}
			\chemfig{H-{{\color{OliveGreen}X}}}
			\arrow
			\chemfig{C(-[:90]H)(-[:180]H)(-[:270]H)-C(-[:90]{{\color{OliveGreen}X}})(-[:0]H)(-[:270]H)}
			\schemestop
		}

	% end subsubsection


	\subsubsection{Markovnikov's Rule}

		This rule governs the major product that is formed when asymmetrical compounds are electrophilically added to alkenes,
		such as hydrogen halides. It does not apply to symmetrical reactants like \ch{\chlorine2} or \ch{Br2}.

		The rule states basically states that when a hydrogen compound with the general form \ch{H-X} is added to an
		alkene that is asymmetrical about the double-bond, the hydrogen atom will be added to the carbon with \textit{more}
		existing hydrogen atoms. Note that \textit{X} can be a halogen, a hydroxide (ie. \ch{H-X} is \ch{H2O}) or some other
		electronegative species.

	% end subsubsection


	% look nice
	\pagebreak


	\subsubsection{Addition of \ch{H-X} to Unsymmetrical Alkenes}

		The behaviour that Markovnikov's rule predicts can be derived from the fact that adding the non-hydrogen species to the least substituted carbon would
		result in a less stable intermediate being formed. This can be illustrated using the electrophilic addition of
		\ch{HBr} to prop-1-ene (\ch{C3H6}).

		If the Markovnikov's rule is \textit{not} followed, the following will take place:


		\diagram[0.825]{
			\schemestart[0, 1.5, thick]
			\chemfig{C(-[:90]H)(-[:180]H)(-[:270]H)-C(-[:270]H)=[@{b1}:0]C(-[:45]H)(-[:315]H)}
			\hspace{5mm} + \hspace{5mm}
			\chemfig{\delp|@{hyd}H-[@{b2}:0]@{hal}{\color{Mahogany}Br}|\delm}
			\arrow{->[slow]}
			\chemfig{$\oplus$|C(-[:90,,2]H)(-[:270,,2]H)-C(-[:90]H)(-[:270]H)-C(-[:90]H)(-[:0]H)(-[:270]H)}
			\hspace{5mm} + \hspace{5mm}
			\chemfig{!{Br}\mch}
			\schemestop

			\chemmove{\draw[-Stealth,line width=0.4mm,shorten <=2mm,shorten >=1mm](b1) .. controls +(90:20mm) and +(90:20mm) .. (hyd);}
			\chemmove{\draw[-Stealth,line width=0.4mm,shorten <=2mm,shorten >=1mm](b2) .. controls +(270:8mm) and +(270:8mm) .. (hal);}
		}{The hydrogen was added to the centre carbon, in violation of Markovnikov's rule.}

		Notice that the positively-charged carbon on the intermediate only has 1 electron-donating alkyl group to
		stabilise its charge. Conversely, if Markovnikov's rule is followed below, the outcome would be different.



		\diagram[0.80]{
			\schemestart[0, 1.5, thick]
			\chemfig{C(-[:90]H)(-[:180]H)(-[:270]H)-C(-[:270]H)=[@{b1}:0]C(-[:45]H)(-[:315]H)}
			\hspace{5mm} + \hspace{5mm}
			\chemfig{\delp|@{hyd}H-[@{b2}:0]@{hal}{\color{Mahogany}Br}|\delm}
			\arrow{->[slow]}
			\chemfig{C(-[:180]H)(-[:90]H)(-[:270]H)-\chemabove{C}{$\oplus$}(-[:270]H)-C(-[:90]H)(-[:0]H)(-[:270]H)}
			\hspace{5mm} + \hspace{5mm}
			\chemfig{!{Br}\mch}
			\schemestop

			\chemmove{\draw[-Stealth,line width=0.4mm,shorten <=2mm,shorten >=1mm](b1) .. controls +(90:20mm) and +(90:20mm) .. (hyd);}
			\chemmove{\draw[-Stealth,line width=0.4mm,shorten <=2mm,shorten >=1mm](b2) .. controls +(270:8mm) and +(270:8mm) .. (hal);}
		}{The hydrogen was added to the terminal carbon, following Markovnikov's rule.}

		In this case, the carbon atom containing the positive charge has 2 electron-donating alkyl groups attached to it,
		making this intermediate molecule more stable than the one above it.

		As such, it is more likely to form, and hence 2-bromopropane will be the \textit{major product}, and
		1-bromopropane will be the \textit{minor product}.


	% end subsubsection

	\subsubsection{Electrophilic Addition of Halogens}

		The electrophilic addition of halogens to alkanes does not involve Markovnikov's rule, since it is symmetrical.
		As the non-polar halogen molecule approaches the high-electron-density π-bond, it is polarised, forming
		\deltap and \deltam partial charges.

		Note that the halogens used are usually either \ch{Br2} or \ch{\chlorine2}, since \ch{F2} is too reactive, and
		\ch{I2} is too \textit{unreactive}.

		The reaction mechanism involves the formation of a \textit{cyclic halonium ion} (bromonium or chloronium) –– the double
		bond breaks, and each carbon forms a single bond with one positive halide ion (both carbons bond to the same atom).

		\textbf{Conditions:} No UV Light, gaseous \ch{X2}.

		\subsubtext{Step 1}

		This initial step is the slow, rate-determining one. It involves breaking of the π-bond, as well as the \ch{X-X} bond.
		Note that the resulting cyclic halonium ion is highly unstable, due to the inherent strain of having a
		three-membered ring.

		\diagram[0.90]{
			\schemestart[0, 1.5, thick]
			\chemfig{C(-[:135]H)(-[:225]H)=[@{b1}:0]C(-[:45]H)(-[:315]H)}
			\hspace{5mm} + \hspace{5mm}
			\chemfig{\delp|@{br1}{\color{OliveGreen}X}-[@{b2}:0]@{br2}{\color{OliveGreen}X}|\delm}
			\arrow{->[slow]}
			\chemfig{{\color{OliveGreen}X}\mch}
			\hspace{5mm} + \hspace{5mm}
			\chemfig{C?(-[:90]H)(-[:180]H)-C?(-[:90]H)(-[:0]H)(-[:240,,,1]{\color{OliveGreen}X}?|\pch)}
			\schemestop

			\chemmove{\draw[-Stealth,line width=0.4mm,shorten <=2mm,shorten >=1mm](b1) .. controls +(90:15mm) and +(90:15mm) .. (br1);}
			\chemmove{\draw[-Stealth,line width=0.4mm,shorten <=2mm,shorten >=1mm](b2) .. controls +(270:8mm) and +(270:8mm) .. (br2);}
		}



		\subsubtext{Step 2}

		In the second, fast step, the negatively-charged bromide ion from the first step attacks the overall
		positively-charged cyclic bromonium ion, resulting in the final product. Note that the bromide ion attacks
		one of the carbons attached to the positive bromide ion, breaking that bond. This is the fast step as it
		involves the reaction of two oppositely-charged species.

		\diagram[1.0]{
			\schemestart[0, 1.5, thick]
			\chemfig{C?(-[:90]H)(-[:180]H)-@{carb}C?(-[:90]H)(-[:0]H)(-[@{b1}:240,,,1]@{br}{\color{OliveGreen}X}?|\pch)}
			\hspace{5mm} + \hspace{5mm}
			\chemfig{@{br1}\lewis{5:,\color{OliveGreen}X}\mch}
			\arrow{->[fast]}
			\chemfig{C(-[:90]H)(-[:180]H)(-[:270]{{\color{OliveGreen}X}})-C(-[:90]H)(-[:0]H)(-[:270]{{\color{OliveGreen}X}})}
			\schemestop

			\chemmove{\draw[-Stealth,line width=0.4mm,shorten <=3mm,shorten >=1mm](br1) .. controls +(215:15mm) and +(315:15mm) .. (carb);}
			\chemmove{\draw[-Stealth,line width=0.4mm,shorten <=2mm,shorten >=1mm](b1) .. controls +(330:8mm) and +(350:8mm) .. (br);}
		}

		The observable change for this reaction would be the decolourisation of the respective halogen; \textit{\color{Mahogany}reddish-brown}
		for bromine, and \textit{\color{YellowGreen}yellowish-green} for chlorine.


		\subsubtext{Overall Reaction}

		\diagram{
			\schemestart
			\chemfig{C(-[:135]H)(-[:225]H)=C(-[:45]H)(-[:315]H)}
			\hspace{5mm} + \hspace{5mm}
			\chemfig{!{X}-!{X}}
			\arrow
			\chemfig{C(-[:270]!{X})(-[:90]H)(-[:180]H)-C(-[:270]!{X})(-[:90]H)(-[:0]H)}
			\schemestop
		}









	% end subsubsection



	\subsubsection{Electrophilic Addition of Aqueous \ch{Br2}}

	% end subsubsection


	\subsubsection{Electrophilic Addition of Steam (Hydration)}

	% end subsubsection


\subsection{Hydrogenation of Alkenes (Non-electrophilic Addition)}

% end subsection



\subsection{Oxidation of Alkenes}

	\subsubsection{Mild Oxidation in an Acidic Medium}
	% end subsubsection

	\subsubsection{Mild Oxidation in an Alkaline Medium}
	% end subsubsection

	\subsubsection{Oxidative Cleavage (Strong Oxidation)}
	% end subsubsection

	\subsubsection{Uses of Oxidative Cleavage}
	% end subsubsection

% end subsection



% end subsection


% end section



























