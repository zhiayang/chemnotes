% Chapter 07 - Acid-Base Equilibrium.tex
% Copyright (c) 2014 - 2016, zhiayang@gmail.com
% Licensed under the Apache License Version 2.0.


\pagebreak
\part{Acid-Base Equilibrium}

	\section{Theory of Acids and Bases}

		\subsection{Brønsted-Lowry}

			Fundamental to understanding this chapter is the Brønsted-Lowry theory of acids and bases... as the section header would lead you
			to conclude.

			\begin{bulletlist}
				& Brønsted acids \textit{donate} a proton (\ch{H+}) to a base
				& Brønsted bases correspondingly \textit{accept} a proton (\ch{H+}) from an acid.
			\end{bulletlist}

			Thus, there are some restrictions placed on each --- acids must contain one or more atoms of \ch{H} to donate, and bases must have
			one or more lone pairs in order to accept the \ch{H+} ion.

			As such, acid-base reactions in the context of this theory involves the transfer of a proton from a Brønsted acid to a Brønsted base.

		% end subsection

		\subsection{Lewis}

			While Brønsted acids and bases are defined in terms of the transfer of protons, Lewis acids and bases are defined in terms of the
			transfer of electron pairs. Thus:

			\begin{bulletlist}
				& Lewis acids \textit{accept} an electron pair from a Lewis base donor.
				& Lewis bases correspondingly \textit{donate} an electron pair to a Lewis acid.
			\end{bulletlist}

			Further references to acids or bases implicitly refer to the \textit{Brønsted} definition.

		% subsection



		\pagebreak
		\subsection{Conjugate Acid-Base Pairs}

			When a acid loses the \ch{H+} ion, an anion is naturally left --- this is the conjugate base of the acid. Conversely, when a
			base accepts a proton, it forms the conjugate acid of the base.

			\txtdiagram{
				\schemestart[0,1.0,thick]
					\chemname{\ch{CH3CO2H}}{\tinytext{acid}}\hspace{7mm} + \hspace{7mm}\chemname{\ch{NH3}}{\tinytext{base}}
					\arrow(.mid east--.mid west){<=>}
					\chemname{\ch{CH3CO2-}}{\tinytext{conjugate base}}\hspace{7mm} + \hspace{7mm}\chemname{\ch{NH4+}}{\tinytext{conjugate acid}}
				\schemestop
			}{\vspace{-1.5em}}

			It should be immediately clear that this is an \textit{equilibrium} reaction.

			In the forward reaction, \ch{CH3CO2H} acts as the acid, donating a proton to the base, \ch{NH3}. In the reverse direction,
			\ch{NH4+} is the acid, donating a proton to the base \ch{CH3CO2-}.

			 Furthermore, \ch{CH3CO2H} and \ch{CH3CO2-} are \textit{conjugate pairs}, as are \ch{NH3} and \ch{NH4+}. Conjugate pairs always
			 differ by a proton, and in any given acid-base reaction, there are two such pairs.

		% subsection


		\subsection{Strength of Acids and Bases}

			The strength of an acid or base is given as the degree of dissociation from the acid or base into ions, in solution. A strong acid
			or base is one that ionises \textit{completely} in solution to give \ch{H+} or \ch{OH-} respectively.

			\txtdiagram{
				\schemestart[0,1.0,thick]
					\chemname{\ch{H\chlorine}}{\tinytext{strong acid}}\hspace{7mm} + \hspace{7mm}\chemname{\ch{H2O}}{}
					\arrow(.mid east--.mid west){->}
					\chemname{\ch{\chlorine-}}{\tinytext{conjugate base}}\hspace{7mm} + \hspace{7mm}\chemname{\ch{H3O+}}{}
					\arrow(@c1.south east--.north east){0}[-90,.5]
					\chemname{\ch{NaOH}}{\tinytext{strong base}}
					\arrow(.mid east--.mid west){->}
					\chemname{\ch{OH-}}{}\hspace{7mm} + \hspace{7mm}\chemname{\ch{Na+}}{}
				\schemestop
			}{\vspace{-1.5em}}

			Since they ionise \textit{completely}, the reverse reaction is negligible, so the reaction is written with a single-headed
			arrow, \ch{->}. This is because the conjugate base \ch{\chlorine-}, in the case of \ch{H\chlorine}, has a low tendency to
			accept a proton.


		% subsection

	% end section

% end part














