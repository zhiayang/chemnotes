% Chapter 09 - Electrolysis.tex
% Copyright (c) 2014 - 2016, zhiayang@gmail.com
% Licensed under the Apache License Version 2.0.


\pagebreak
\part{Electrolysis}

	\section{Overview}

		Electrolysis is, simply put, the reverse of an electrochemical cell --- now current is \itl{supplied}, and the half-reactions
		on either side are simply reversed.

		One difference is that there will often be different ions in solution, and certain ions will be \itl{selectively discharged}, which
		depends on a number of factors --- mostly their \Eo{} value.

		Note that while oxidation still occurs at the anode and reduction still occurs at the cathode, the \itl{polarity} of each
		electrode has been reversed; the external battery removes electrons from the anode causing it to be \itl{positive}, while
		supplying electrons to the cathode causing it to be \itl{negative}.

	% end section

	\section{Electrolysis of Pure Molten Salts}

		The electrolysis of a pure molten salt is very straightforward --- there are only two substances in the electrolyte to consider, so one will
		always be oxidised, and the other reduced.

		If it isn't already obvious, the anion is oxidised and the cation is reduced. For example, with the electrolysis of molten \ch{Na\Cl}:

		\txtreactioneqn[1.2]{
			\ch{2 Na+ \stL} + \ch{2 e-}\arrow{->[\tinytext{reduction}]}\ch{2 Na \stL}
			\arrow(@c1.east--.east){0}[-90,.4]
			\ch{2 \Cl- \stL}\arrow{->[\tinytext{oxidation}]}\ch{\Cl2 \stG} + \ch{2 e-}
		}

		When the current in the circuit flows, the electrode connected to the negative terminal of the battery will be the cathode --- it supplies
		electrons, and \itl{reduces} the \ch{Na+} ions.

		Because the cathode is negatively charged (due to the electrons), \ch{Na+} ions will be attracted
		to it, and migrate there. The reverse is true for the anode --- and \ch{\Cl-} ions migrate to it.

		Note that each half-equation has its own \Eo{} value, and so the supply voltage (emf) of the battery must be greater than the total
		\Ecell{} of the system for the reaction to occur.

	% end section



	\pagebreak
	\section{Electrolysis of Aqueous Ion Solutions}

		This is more complicated, because apart from the ions from whatever salt is dissolved, water can also be oxidised and reduced --- the
		reaction that occurs at each electrode depends on the \Eo{} of the relevant half-equations.

		% align the equals signs instead of the 'E's.
		\txtreactioneqn[1.2]{
			\ch{2 H2O \stL}\arrow{->[\tinytext{oxidation}]}\ch{O2 \stG} + \ch{4 H+ \stAq} + \ch{4 e-}
			\arrow(.east--eq1.west){0}[,.7] \large$\;= \SI{-1.23}{\volt}$\arrow(@eq1.west--.east){0}[-180,0] \large$\MEox$
			\arrow(@c1.east--.east){0}[-90,.4]
			\ch{2 H2O \stL} + \ch{2 e-}\arrow{->[\tinytext{reduction}]}\ch{H2 \stG} + \ch{2 OH- \stAq}
			\arrow(@eq1.east--eq2.east){0}[-90,.4] \large$\;= \SI{-0.83}{\volt}$\arrow(@eq2.west--.east){0}[-180,0] \large$\MEo$
		}

		The half-equations for the reduction and oxidation of water are shown above. These are always the half-equations for oxidation and
		reduction of water, in any solution.

		\subsection{Factors Affecting Selective Discharge}

			\subsubsection{\texorpdfstring{\Eo{}}{Eo} of Half-Reactions}

				The most direct and common way to identify the correct half-equation is by comparing \Eo{} values.
				The electrolysis of \ch{CuBr2 \stAq} will be used as an example here, using inert platinum electrodes.


				\paragraph{Cathode Reactions}

				% align the equals signs instead of the 'E's.
				\txtreactioneqn{
					\ch{2 H2O \stL} + \ch{2 e-}\arrow\ch{H2 \stG} + \ch{2 OH- \stAq}
					\arrow(.east--eq1.west){0}[,.7] \large$\;= \SI{-0.83}{\volt}$\arrow(@eq1.west--.east){0}[-180,0] \large$\MEo$
					\arrow(@c1.east--.east){0}[-90,.4]
					\ch{Cu^2+ \stAq} + \ch{2 e-}\arrow\ch{Cu \stS}
					\arrow(@eq1.east--eq2.east){0}[-90,.4] \large$\;= \SI{+0.34}{\volt}$\arrow(@eq2.west--.east){0}[-180,0] \large$\MEo$
				}[Note that the equations are written with a single-arrow, since it is known that reduction is occurring.]

				Comparing the two half-equations, the \Eo{} value for the reduction of \ch{Cu^2+} is larger than that of the reduction of water.
				Hence, \ch{Cu^2+} will be \itl{preferentially reduced}, depositing solid \ch{Cu} metal at the cathode.



				\paragraph{Anode Reactions}

				% align the equals signs instead of the 'E's.
				\txtreactioneqn{
					\ch{2 H2O \stL}\arrow\ch{O2 \stG} + \ch{4 H+ \stAq} + \ch{4 e-}
					\arrow(.east--eq1.west){0}[,.7] \large$\;= \SI{-1.23}{\volt}$\arrow(@eq1.west--.east){0}[-180,0] \large$\MEox$
					\arrow(@c1.east--.east){0}[-90,.4]
					\ch{2 Br- \stAq}\arrow\ch{Br2 \stL} + \ch{2 e-}
					\arrow(@eq1.east--eq2.east){0}[-90,.4] \large$\;= \SI{-1.07}{\volt}$\arrow(@eq2.west--.east){0}[-180,0] \large$\MEox$
				}

				Comparing the two half-equations, the \Eox{} value for the oxidation of \ch{Br-} is greater than that for
				the oxidation of water, so \ch{Br-} is \itl{preferentially oxidised} at the anode, and \ch{Br2 \stAq} forms around the anode.

				Note that, in this case, the values of \Eox{} instead of \Eo{} were used --- the former is the standard oxidation
				potential, which is simply the reverse of the reduction potential. Hence, the \itl{greater} value of \Eox{} is used.

			% end subsubsection


			\subsubsection{Concentration of Ions}

				The concentration of ions in solution can also affect the products of electrolysis. As covered in a previous chapter,
				ion concentrations can, as determined by Le Châtelier's Principle, increase or decrease the value of \Eo{} by moving
				the position of equilibrium.

				The most common application of this is in the electrolysis of brine, or concentrated \ch{Na\Cl \stAq}. In a dilute solution,
				the anode candidates are as follows:

				% align the equals signs instead of the 'E's.
				\txtreactioneqn{
					\ch{2 H2O \stL}\arrow\ch{O2 \stG} + \ch{4 H+ \stAq} + \ch{4 e-}
					\arrow(.east--eq1.west){0}[,.7] \large$\;= \SI{-1.23}{\volt}$\arrow(@eq1.west--.east){0}[-180,0] \large$\MEox$
					\arrow(@c1.east--.east){0}[-90,.4]
					\ch{2 \Cl- \stAq}\arrow\ch{\Cl2 \stG} + \ch{2 e-}
					\arrow(@eq1.east--eq2.east){0}[-90,.4] \large$\;= \SI{-1.36}{\volt}$\arrow(@eq2.west--.east){0}[-180,0] \large$\MEox$
				}



				The \Eox{} value for the oxidation of water is larger than for the oxidation of \ch{Cl-}, so water is preferentially oxidised.

				However, when the concentration of \ch{\Cl-} is increased, the position of equilibrium in the half-reaction shifts to the
				right, thus increasing the value of \Eox{}.

				Since the difference between \Eox{} for water and for chlorine is quite small, high concentrations of \ch{\Cl-} can allow its
				\Eox{} value to \enquote{surpass} that of water, allowing chlorine gas to be preferentially discharged.

				At the cathode, nothing new happens --- water is still reduced, because the difference between the \Eo{} values of \ch{Na+} and
				\ch{H2O} are too great, at $\SI{-2.71}{\volt}$ and $\SI{-0.83}{\volt}$ respectively --- something a change in concentration
				cannot overcome.

			% end subsubsection



			\subsubsection{Electrode Reactions}

				Electrolysis typically uses inert electrodes, such as platinum or graphite. However, it is sometimes useful to have the
				electrode provide a source of reactants --- this is typically the anode, since the metal atoms can be oxidised to ions.

				For example, in the electrolysis of \ch{CuSO4 \stAq}, a copper anode is often used. Now, at the anode there are three
				competing half-reactions:

				% align the equals signs instead of the 'E's.
				\txtreactioneqn{
					\ch{2 H2O \stL}\arrow\ch{O2 \stG} + \ch{4 H+ \stAq} + \ch{4 e-}
					\arrow(.east--eq1.west){0}[,.7] \large$\;= \SI{-1.23}{\volt}$\arrow(@eq1.west--.east){0}[-180,0] \large$\MEox$
					\arrow(@c1.east--.east){0}[-90,.4]
					\ch{2 SO4^2- \stAq}\arrow\ch{S2O8^2- \stAq} + \ch{2 e-}
					\arrow(@eq1.east--eq2.east){0}[-90,.4] \large$\;= \SI{-2.01}{\volt}$\arrow(@eq2.west--.east){0}[-180,0] \large$\MEox$
					\arrow(@c5.east--.east){0}[-90,.4]
					\ch{Cu \stS}\arrow\ch{Cu^2+ \stAq} + \ch{2 e-}
					\arrow(@eq2.east--eq3.east){0}[-90,.4] \large$\;= \SI{-0.34}{\volt}$\arrow(@eq3.west--.east){0}[-180,0] \large$\MEox$
				}




				The last equation is only relevant because of \ch{Cu} atoms being introduced by the copper anode. Since its \Eox{} value is the
				greatest among the 3, copper is preferentially oxidised to \ch{Cu^2+} ions in solution.

				One final note --- while graphite is typically considered inert in many cases for ionic electrolysis, if \ch{O2 \stG} is formed and
				comes into contact with the graphite, \ch{CO2} and \ch{CO} is often formed.


			% end subsubsection

		% end subsection

	% end section




	\section{Electrolytic Calculations}

		\subsection{Overview}

			\itl{The number of moles of a substance that undergoes oxidation or reduction at each electrode is directly proportional to
			the quantity of charge that passes through the cell.}

			--- \itl{Michael Faraday}

			Faraday's Law of Electrolysis can be applied, along with some other physics formulae, to derive various relationships. Note
			that the number of moles of electrons passed depends on its mole ratio with the number of moles of product formed.

			Faraday also has a constant --- the aptly named Faraday Constant, that is the ratio of charge to number of moles of electrons; the
			quantity of charge carried per mole of electrons is 1 Faraday.

		% end subsection


		\subsection{Formulae}

			\mathdiagram{
				\[ Q = I \times t \]
			}

			\tabto{0mm}$Q$:     \tabto{10mm}the quantity of charge passed, in coulombs (\si{\coulomb})
			\tabto{0mm}$I$:     \tabto{10mm}the current provided, in amperes (\si{\ampere})
			\tabto{0mm}$t$:     \tabto{10mm}the time in seconds

			\vspace{1em}


			\mathdiagram{
				\[ Q = n(e) \times F \]
			}

			\tabto{0mm}$Q$:     \tabto{10mm}the quantity of charge passed, in coulombs (\si{\coulomb})
			\tabto{0mm}$n(e)$:  \tabto{10mm}the number of moles of electrons passed during electrolysis
			\tabto{0mm}$F$:     \tabto{10mm}Faraday Constant (approx. \SI{96500}{\coulomb\per\mole})

			\vspace{1em}


			\mathdiagram{
				\[ F = L \times e \]
			}

			\tabto{0mm}$F$:     \tabto{10mm}Faraday Constant (approx. \SI{96500}{\coulomb\per\mole})
			\tabto{0mm}$L$:     \tabto{10mm}Avogadro Constant (approx. \SI{6.02e-23})
			\tabto{0mm}$e$:     \tabto{10mm}The charge of one electron (approx. \SI{1.60e-19}{\coulomb})


			Note that this last formula is only used when the question requires calculating either $F$, $L$ or $e$ from experimental data.

		% end subsection

	% end section





	\pagebreak
	\section{Applications of Electrolysis}

		\subsection{Electrolysis of Brine}

			As discussed above, electrolysis of \itl{concentrated} \ch{Na\Cl \stAq} yields \ch{\Cl2 \stG}, which is an industrially
			important compound. It also creates \ch{H2 \stG} and \ch{NaOH \stAq}, which are also useful products.

			\imgdiagram{100mm}{../figures/physical/ch09/brine_electrolysis.png}{Electrolysis of Brine}

			Of note is the fact that the level of solution on the left is higher than on the right --- this is important to ensure that
			the \ch{OH-} ions formed at the cathode do not flow back to the anode to be re-oxidised and contaminate the reactants.

		% end subsection



		\subsection{Anodising of Aluminium}

			Aluminium metal naturally forms a thin layer of \ch{Al2O3} at its surface that protects it from corrosion. Anodising the object
			increases the thickness of this layer, which provides more protection.

			The object to be anodised is used as the... anode, and an inert cathode is used, together with a dilute acid electrolyte to conduct
			electricity.

			The reaction at the cathode is unimportant --- the water is reduced to hydrogen gas, and is released. At the anode however,
			both water and aluminium are oxidised (for some \boit{STRANGE}, unexplainable reason), forming the \ch{Al2O3}.

			The structure of this oxide layer is somewhat porous, allowing it to accept dyes more easily than a smooth metal surface.

		% end subsection


		\pagebreak
		\subsection{Electrolytic Purification of Copper}

			Copper minerals often contain small amounts of other metals, like zinc and silver. After a process of oxidation followed by
			reduction, the purely metallic form of these atoms can be obtained --- including the silver and zinc impurities.

			Electrolytic purification takes advantage of the different \Eo{} values between copper and the impurities, to preferentially
			oxidise and reduce only the wanted ions.

			\imgdiagram{100mm}{../figures/physical/ch09/copper_purification.png}{Electrolytic Purification of Copper}

			The relevant half-equations, and more importantly \Eo{} and \Eox{} values, for each metal are shown below:


			% align the equals signs instead of the 'E's.
			\txtreactioneqn{
				\ch{Zn \stS}\arrow\ch{Zn^2+ \stAq} + \ch{2 e-}
				\arrow(.east--eq1.west){0}[,.6] \large$\;= \SI{+0.76}{\volt}\qquad \MEo = \SI{-0.76}{\volt}$
				\arrow(@eq1.west--.east){0}[-180,0] \large$\MEox$
				\arrow(@c1.east--.east){0}[-90,.4]
				\ch{Cu \stS}\arrow\ch{Cu^2+ \stAq} + \ch{2 e-}
				\arrow(@eq1.east--eq2.east){0}[-90,.4] \large$\;= \SI{-0.34}{\volt}\qquad \MEo = \SI{+0.34}{\volt}$
				\arrow(@eq2.west--.east){0}[-180,0] \large$\MEox$
				\arrow(@c5.east--.east){0}[-90,.4]
				\ch{Ag \stS}\arrow\ch{Ag+ \stAq} + \ch{ e-}
				\arrow(@eq2.east--eq3.east){0}[-90,.4] \large$\;= \SI{-0.80}{\volt}\qquad \MEo = \SI{+0.80}{\volt}$
				\arrow(@eq3.west--.east){0}[-180,0] \large$\MEox$
			}


			At the anode, the \Eox{} of zinc is greater than that for copper, so it will be preferentially oxidised into \ch{Zn^2+}. However,
			it only exists in small amounts, and once all the \ch{Zn} has been oxidised, the next highest \Eox{} metal is oxidised --- in
			this case copper. As the copper is oxidised into aqueous ions, the unreacted \ch{Ag} collects as \itl{anode sludge}
			below the anode.

			At the cathode, both \ch{Cu^2+} and \ch{Zn^2+} are present. Looking at the \Eo{} values however, copper is preferentially reduced
			due to its higher \Eo{} value, and \ch{Zn^2+} ions remain in solution.

			Water is also present at both electrodes, but looking at the values for \Eo{}, it is irrelevant.

			A pure copper cathode is used, such that the newly-pure copper collects on already-pure copper --- the impure copper is used as
			the anode. The electrolyte is typically \ch{CuSO4 \stAq}.

		% end subsection


		\subsection{Electroplating of Chromium}

			The process for electroplating some metal object with chromium (to increase shiny-ness) is similar to the process for purifying
			copper. However, to allow for a gradual and \sout{modest appreciation} even coating of the object, \ch{Cr^3+} is only supplied
			from the electrolyte, and is only present at low concentrations.

			The electrolyte thus consists mainly of \ch{CrO4^2-} ions, and a small amount of \ch{Cr^3+} ions. Like with copper, the object
			to be coated is used as the cathode; \ch{Cr^3+} reduces to \ch{Cr \stS} that coats the object, while \ch{CrO4^2-} is reduced to
			\ch{Cr^3+}, which replenishes the ions that were consumed.

		% end subsection

	% end section

% end part






