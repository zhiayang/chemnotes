% Chapter 09 - Electrolysis.tex
% Copyright (c) 2014 - 2016, zhiayang@gmail.com
% Licensed under the Apache License Version 2.0.


\pagebreak
\part{Electrolysis}

	\section{Electrolysis of Pure Molten Salts}

		The electrolysis of pure molten salts is very straightforward --- there are only two substances in the solution to consider, so one will
		always be oxidised, and the other reduced.

		If it isn't already obvious, the anion is oxidised and the cation is reduced. For example, with the electrolysis of molten \ch{Na\Cl}:

		\diagram[1.0]{
			\schemestart[0, 1.5, thick]
				\ch{2 Na+ \stL} + \ch{2 e-}\arrow{->[\tinytext{reduction}]}\ch{2 Na \stL}
				\arrow(@c1.south east--.north east){0}[-90,.25]
				\ch{2 \Cl- \stL}\arrow{->[\tinytext{oxidation}]}\ch{\Cl2 \stG} + \ch{2 e-}
			\schemestop
		}{\vspace{-1.0em}}

		When the current in the circuit flows, the electrode connected to the negative terminal of the battery will be the cathode --- it supplies
		electrons, and \textit{reduces} the \ch{Na+} ions.

		Because it's negatively charged (due to the electrons), \ch{Na+} ions will be attracted
		to it, and migrate to the cathode. The reverse is true for the anode, attracting \ch{\Cl-} to itself.

		Note that each half-equation has its own \Eo{} value, and so the supply voltage (emf) of the battery must be greater than the total
		\Ecell{} of the system for the reaction to occur.

	% end section


	\section{Electrolysis of Aqueous Ion Solutions}

		This is more complicated, because apart from the ions from whatever salt is dissolved, water can also be oxidised and reduced --- the
		reaction that occurs at each electrode depends on the \Eo{} of the relevant half-equations.

		\diagram[1.0]{
			\schemestart[0, 1.5, thick]
				\ch{2 H2O \stL} + \ch{2 e-}\arrow{->[\tinytext{reduction}]}\ch{H2 \stG} + \ch{2 OH- \stAq}\hspace{5.25mm}$E^{\stdst} = +0.40V$
				\arrow(@c1.south east--.north east){0}[-90,.25]
				\ch{2 H2O \stL}\arrow{->[\tinytext{oxidation}]}\ch{O2 \stG} + \ch{4 e-}					\hspace{15mm}$E^{\stdst} = +1.23V$.
			\schemestop
		}{\vspace{-1.0em}}

		The half-equations for the reduction and oxidation of water are shown above.

	% end section

% end part






