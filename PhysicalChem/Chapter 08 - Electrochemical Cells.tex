% Chapter 08 - Electrochemical Cells.tex
% Copyright (c) 2014 - 2016, zhiayang@gmail.com
% Licensed under the Apache License Version 2.0.


\pagebreak
\part{Electrochemical Cells}

	\section{Overview}

		Electrochemical cells are typically composed from two half-cells, each with their own redox reaction occurring, and each having an
		electrode that is typically inert to prevent it from interfering.

		\subsection{Disclaimer}

			Unfortunately, \TeX{} has no good solution for drawing cell diagrams, so this chapter and the next chapter will be relatively...
			diagram-free. Sorry. To be fair only one diagram is needed per chapter, so...

		% end subsection


		\subsection{Electrodes}

			Consistent across both electrolytic cells and electrochemical cells, is the designation of the cathode and anode, in terms of
			redox reactions:

			\begin{bulletlist}
				& Oxidation occurs at the \itl{anode}
				& Reduction occurs at the \itl{cathode}
			\end{bulletlist}

			This comes in useful since the direction of electron flow, and hence the polarity of each electrode, will change between the two
			systems.

		% end subsection

	% end section


	\pagebreak
	\section{Standard Reduction Potential, \MEo{}}

		The standard reduction potential, \Eo{} is defined for a half redox reaction --- either reduction or oxidation. However, per
		convention, the \itl{values} for \Eo{} are always quoted for the reduction reaction. Hence, if the \Eo{} for oxidising
		\ch{Zn} to \ch{Zn^2+} is \SI{+0.76}{\volt}, the quoted \Eo{} value will be $-0.76V$.

		As its name would imply, the \itl{standard} reduction potential is measured using a \itl{standard hydrogen cell}, which is
		an electrochemical cell (covered later), with these properties:

		\begin{bulletlist}
			& \SI{1.0}{\molarConc} of \ch{H+} ions in an aqueous solution
			& Temperature of \SI{298}{\kelvin}
			& Hydrogen gas at \SI{1}{\atm} bubbling over a \itl{platinised platinum} (how obtuse) electrode
		\end{bulletlist}

		The potential on the surface of the platinum electrode is assigned a value of $0$ by convention, and all standard electrode
		potentials are measured relative to this value of $0$.

		This standard allows for the comparison of different half-equations --- the more positive the value of \Eo{}, the more
		feasible the redox reaction. Since it is defined as such, that means the given substance is easier to \itl{reduce}.

		Conversely, if the value of \Eo{} for a certain equation is highly negative, like this:

		\diagram[1.0]{
			\schemestart[0, 1.0, thick]
				\ch{Na+ \stAq}\hspace{2mm} + \hspace{2mm}\ch{e-}\arrow{<=>}\ch{Na \stS}	\hspace{10mm}$E^{\stdst} = -2.71V$
			\schemestop
		}{\vspace{-1.0em}}

		This means that it is far more feasible for the equation to proceed in the \itl{reverse direction} (note that these are
		written with reversible arrows when the direction of progress is still indeterminate), aka. the oxidation of \ch{Na} instead.


		\subsection{Comparing \MEo{} Values}

			Standard reduction potential values of different species can be used as a judge of their reduction or oxidation strength.
			If the \Eo{} value of a species is very high, for instance with \ch{Li}, the \Eo{} value is $-3.04V$ --- meaning that it
			is very very spontaneously \itl{oxidised}, making it a very powerful \itl{reducing agent}.

			Conversely, with \ch{F2}, the \Eo{} value is $+2.87V$, implying that it is very likely to be \itl{reduced}, and hence
			making it a very powerful \itl{oxidising agent}.

		% end subsection

	% end section



	\pagebreak
	\section{Standard Cell Potentials, \MEcell{}}

		Given that electrochemical cells are often made of two half-cells, or at the very least always have two redox half-reactions,
		the \Ecell{} represents the \itl{total} potential of the two half-cells, otherwise known as the \itl{standard cell potential}.

		This standard cell potential is also the output \itl{emf} of the electrochemical cell, and can be measured with a voltmeter
		connected between the two electrodes. \Ecell{} is simply defined as the difference between the \Eo{} of the reduction half cell, and
		the \Eo{} of the oxidation half cell, like so:

		\diagram[1.0]{
			$E^{\stdst}_{cell} = E^{\stdst}_{cathode} - E^{\stdst}_{anode}$
		}{\vspace{-1.0em}}

		However, with some sign manipulation, it can be also expressed as the sum of the standard electrode potential of the reduction
		half equation, and the \itl{oxidation potential} of the oxidation half-cell.


		While \Ecell{} is a measure of the output \itl{emf} of the cell, if it is negative to begin with, the required redox
		reactions are unfeasible, and there will be no \itl{emf} produced at all, and no reaction occurring.

		This is often used to predict the feasibility of a spontaneous redox reaction between two species, even if the produced
		\itl{emf} is not the primary goal.


		\subsection{Calculating Standard Cell Potentials}

			While the definition of \Ecell{} might be simple, figuring out which values to use for each \Eo{} is not quite so simple. To put it
			concisely, each species the reaction mixture will react in the most feasible way it can.

			For example, given the following half-equations:

			\diagram[1.0]{
				\schemestart[0, 1.0, thick]
					\ch{Zn^2+ \stAq}\hspace{2mm} + \hspace{2mm}\ch{2 e-}\arrow{<=>}\ch{Zn \stS}			\hspace{10mm}$E^{\stdst} = -0.76V$
					\arrow(@c1.south east--.north east){0}[-90,.25]
					\ch{Cu^2+ \stAq}\hspace{2mm} + \hspace{2mm}\ch{2 e-}\arrow{<=>}\ch{Cu \stS}			\hspace{10mm}$E^{\stdst} = +0.34V$.
				\schemestop
			}{\vspace{-1.0em}}

			The first objective should be to identify the half-equation at the cathode, or where reduction is taking place. Comparing the two \Eo{}
			values for the reduction of \ch{Zn^2+} and \ch{Cu^2+}, the \Eo{} of the reduction of copper is \itl{more positive} than that for
			the reduction of \ch{Zn^2+} --- hence copper will be reduced at the cathode.

			Thus, the other half-reaction must occur at the anode, and in this case zinc is oxidised.

			Combining the two, we get the overall reaction, and the cell potential, $+0.34 - (-0.76)$.

			\diagram[1.0]{
				\schemestart[0, 1.0, thick]
					\ch{Cu^2+ \stAq}\hspace{2mm} + \hspace{2mm}\ch{Zn \stS}\arrow\ch{Cu \stS}\hspace{2mm} + \hspace{2mm}\ch{Zn^2+ \stAq}
					\hspace{10mm}$E^{\stdst}_{cell} = +1.10V$
				\schemestop
			}{\vspace{-1.0em}}

		% end subsection

		\subsection{Predicting Redox Feasibility with Cell Potentials}

			Since an \Ecell{} value $ < 0$ will not result in any reaction, it can be used to predict the feasibility of arbitrary redox
			reactions, to determine whether they will occur spontaneously when the reactants are mixed.

			\subsubsection{Limitations of Predicting Feasibility}

				\paragraph{Rate of Reaction}

				\Ecell{} only serves to show that the reaction is feasible, and says nothing about the \itl{rate of reaction};
				the most common cause of this is a high activation energy for the reaction.


				\paragraph{Conditions}

				The entire system of values for \Eo{} and hence \Ecell{} are defined only for \itl{standard conditions} --- pressure,
				temperature, etc. However, given that each half-reaction is actually an equilibrium, our old friend Le Châtelier's Principle
				can predict the direction that the position of equilibrium will shift in.

				For example, given the reduction of chlorine:

				\diagram[1.0]{
					\schemestart[0, 1.0, thick]
						\ch{\Cl2 \stG}\hspace{2mm} + \hspace{2mm}\ch{2 e-}\arrow{<=>}\ch{2 \Cl- \stAq}\hspace{10mm}$E^{\stdst} = -+1.36V$
					\schemestop
				}{\vspace{-1.0em}}

				If the pressure of the system were to increase, then by Le Châtelier's Principle, the position of equilibrium would shift to
				the right, favouring the \itl{reduction of chlorine}. Hence, the value of $E$ (note, not $E^{\stdst}$ since it's
				not standard any more) would \itl{increase}.



				\paragraph{Ion Concentration}

				Again, by Le Châtelier's Principle, changing the concentration of ions can also move the position of equilibrium, that changes
				the value of $E$. This can be exploited by connecting two half-cells of the same ion, but at different concentrations. The
				difference in $E$ values will form a net positive \Ecell{} that can be measured.

			% end subsubsection

		% end subsection


		\subsection{Relation to Gibbs Free Energy}

			Given that both the standard cell potential and Gibbs Free Energy predict the feasibility and spontaneity of a reaction, they can
			be related by an equation:


			\diagram[1.0]{
				$∆G^{\stdst} = -nFE^{\stdst}_{cell}$
			}{\vspace{-1.0em}}

			$F$ is the Faraday constant, which is \SI{9.648e5}{\coulomb\per\mole}, and $n$ is the number of moles of electrons transferred in the
			redox reaction.

		% end subsection

	% end section




	\pagebreak
	\section{Electrochemical Cells}

		\subsection{Overview}

			Now that all the crucial terms have been defined, electrochemical cells are simply a combination of two half-cells that can be
			linked together with a \itl{salt bridge} to produce a usable voltage and current.

			\imgdiagram{100mm}{../figures/physical/ch08/galvanic_cell.png}{Electrochemical cell with copper and zinc}

			As the equations show, zinc is being oxidised and copper is being reduced.

		% end subsection


		\subsection{Electrode and Electrolyte Choice}

			Referring back to the diagram, electrodes and electrolytes should be chosen to make life easier. Firstly the choices need to be
			able to supply the relevant reactants for the reaction, and the ions that are not participating in the reaction (ie. \ch{SO4^2-} above)
			should... be inert.

			For the cathode half-reaction, \ch{Cu^2+ \stAq} is reduced to \ch{Cu \stS}. Hence, there needs to be a source of \ch{Cu^2+} ions,
			in this case provided by the \ch{CuSO4 \stAq} electrolyte. Note that while a copper electrode is used, it is not necessary --- any
			inert electrode can be used as well.

			For the anode half-reaction, \ch{Zn \stS} is oxidised to \ch{Zn^2+ \stAq}. Hence, there needs to be a source of \ch{Zn \stS} atoms,
			in this case provided by the anode --- hence the anode will slowly reduce in mass as the reaction proceeds. While the electrolyte
			used is \ch{ZnSO4}, again there is no need for it to actually contain \ch{Zn^2+ \stAq} ions.

			Finally, even though the electrolyte and electrode need not contain the ions being reduced or oxidised, the standardised nature of
			\Ecell{} requires fixed concentrations of these ions, and so they are still used. While the cell will function with different
			reactants, the value of \Ecell{} will likely differ.

		% end subsection


		\subsection{Salt Bridge}

			A preliminary knowledge of electricity will tell you that circuits need to be closed for electricity to flow. This is one of the
			purposes of the salt bridge, to connect the two half-cells so that current will flow.

			The other purpose is to balance the electrical charge in each half-cell. At the anode half-cell, electrons are lost and
			\ch{Zn^2+} is added to the solution, creating a net positive charge; at the cathode, \ch{Cu^2+} ions are removed from solution
			and electrons are added, creating a net negative charge..

			Without a salt bridge, this imbalance would not be corrected. Eventually, the \itl{potential difference} due to the difference in
			charge of the two cells would overpower the \Ecell{} of the redox reactions, and no more electrons would flow.

			The salt bridge typically contains ions that are unreactive with the reactants, in this case \ch{K+} and \ch{\Cl-}. If the
			electrolytes contain for example \ch{Pb^2+}, then obviously \ch{\Cl-} cannot be used in the salt bridge.

			As positive charge at the anode builds up, \ch{\Cl-} ions from the salt bridge move to balance this, and \ch{K+} ions move to the
			cathode cell to balance the corresponding negative charge buildup.


			\subsubsection{Ion Migration Rates}

				An alternative to a salt bridge is simply a porous membrane between the two cells, that reduces the mixing of electrolytes.
				Due to the different migration rates (depending on charge and size) of the ions in the electrolytes, there may be a net potential
				difference, which can impact the performance and \Ecell{} of the overall cell.

				Hence, the ions chosen to fill the salt bridge need to have similar migration rates; examples include \ch{K+} and \ch{NO3-}, or
				as above, \ch{K+} and \ch{\Cl-}.

			% end subsubsection

		% end subsection

	% end section











	\pagebreak
	\section{Applications of Electrochemical Cells}

		\subsection{Batteries}

			Hmm... a reaction that produces an \itl{emf}, what could it be used for...

			Clearly, the most common use for an electrochemical cell is... in a battery. Or rather, a battery is an electrochemical cell.
			There are many different kinds of batteries, rechargeable and non-rechargeable, using different electrolytes and electrodes.


			\subsubsection{Alkaline}

				Quite possibly the most common battery, non-rechargeable alkaline batteries are commonplace in everyday devices. They
				typically employ a \itl{graphite cathode}, and a \itl{zinc anode}. The electrolyte is usually \ch{MnO2} in \ch{KOH}
				paste --- the \ch{OH-} in the paste is used for the oxidation of zinc.

				\diagram[1.0]{
					\schemestart[0, 1.5, thick]
						\ch{Zn \stS}\hspace{2mm} + \hspace{2mm}\ch{2 OH-}\arrow{->[\tinytext{oxidation}]}\ch{ZnO \stAq}\hspace{2mm} + \hspace{2mm}\ch{H2O \stL}\hspace{2mm} + \hspace{2mm}\ch{2 e-}
						\arrow(@c1.south east--.north east){0}[-90,.25]
						\ch{MnO2 \stS}\hspace{2mm} + \hspace{2mm}\ch{2 H2O \stL}\hspace{2mm} + \hspace{2mm}\ch{2 e-}\arrow{->[\tinytext{reduction}]}\ch{Mn(OH)2 \stS}\hspace{2mm} + \hspace{2mm}\ch{2 OH-}
					\schemestop
				}{\vspace{-1.0em}}

				The overall reaction is as such:

				\diagram[1.0]{
					\schemestart[0, 1.0, thick]
						\ch{Zn \stS}\hspace{2mm} + \hspace{2mm}\ch{MnO2 \stS}\hspace{2mm} + \hspace{2mm}\ch{H2O \stL}\arrow{->}\ch{ZnO \stS}\hspace{2mm} + \hspace{2mm}\ch{Mn(OH)2 \stS}
						\hspace{10mm}$E^{\stdst}_{cell} = +1.5V$
					\schemestop
				}{\vspace{-1.0em}}


			% end subsubsection



			\subsubsection{Silver}

				Silver button cells are commonly used for small devices, and employ a zinc anode and a silver oxide (\ch{Ag2O}) cathode. The
				electrolyte is, again, some alkaline paste, for example \ch{KOH} or \ch{NaOH}.



				\diagram[1.0]{
					\schemestart[0, 1.5, thick]
						\ch{Zn \stS}\hspace{2mm} + \hspace{2mm}\ch{2 OH-}\arrow{->[\tinytext{oxidation}]}\ch{ZnO \stAq}\hspace{2mm} + \hspace{2mm}\ch{H2O \stL}\hspace{2mm} + \hspace{2mm}\ch{2 e-}
						\arrow(@c1.south east--.north east){0}[-90,.25]
						\ch{Ag2O \stS}\hspace{2mm} + \hspace{2mm}\ch{H2O \stL}\hspace{2mm} + \hspace{2mm}\ch{2 e-}\arrow{->[\tinytext{reduction}]}\ch{2 Ag \stS}\hspace{2mm} + \hspace{2mm}\ch{2 OH-}
					\schemestop
				}{\vspace{-1.0em}}

				The overall reaction is as such:

				\diagram[1.0]{
					\schemestart[0, 1.0, thick]
						\ch{Zn \stS}\hspace{2mm} + \hspace{2mm}\ch{Ag2O \stS}\arrow{->}\ch{ZnO \stS}\hspace{2mm} + \hspace{2mm}\ch{2 Ag \stS}
						\hspace{10mm}$E^{\stdst}_{cell} = +1.6V$
					\schemestop
				}{\vspace{-1.0em}}

			% end subsubsection



			\pagebreak
			\subsubsection{Lead-Acid}

				Lead-acid batteries are commonly found in cars, and have the nice property of being rechargeable.


				\diagram[1.0]{
					\schemestart[0, 1.5, thick]
						\ch{Pb \stS}\hspace{2mm} + \hspace{2mm}\ch{H2SO4 \stAq}\arrow{->[\tinytext{oxidation}]}\ch{PbSO4 \stS}\hspace{2mm} + \hspace{2mm}\ch{2 H+ \stAq}\hspace{2mm} + \hspace{2mm}\ch{2 e-}
						\arrow(@c1.south east--.north east){0}[-90,.25]
						\ch{PbO2 \stS}\hspace{2mm} + \hspace{2mm}\ch{H2SO4 \stAq}\hspace{2mm} + \hspace{2mm}\ch{2 H+ \stAq}\arrow{->[\tinytext{reduction}]}\ch{2 PbSO4 \stS}\hspace{2mm} + \hspace{2mm}\ch{2 H2O \stL}
					\schemestop
				}{\vspace{-1.0em}}

				The overall reaction is as such:

				\diagram[1.0]{
					\schemestart[0, 1.0, thick]
						\ch{PbO2 \stS}\hspace{2mm} + \hspace{2mm}\ch{Pb \stS}\hspace{2mm} + \hspace{2mm}\ch{2 H2SO4 \stAq}\arrow{->}\ch{2 PbSO4 \stS}\hspace{2mm} + \hspace{2mm}\ch{2 H2O \stL}
						\hspace{10mm}$E^{\stdst}_{cell} = +2.1V$
					\schemestop
				}{\vspace{-1.0em}}

				When the battery is recharged, it functions like an electrolytic cell, and the half-reactions above are simply reversed. Then,
				the overall equation is reversed as well:

				\diagram[1.0]{
					\schemestart[0, 1.0, thick]
						\ch{2 PbSO4 \stS}\hspace{2mm} + \hspace{2mm}\ch{2 H2O \stL}\arrow{->}\ch{PbO2 \stS}\hspace{2mm} + \hspace{2mm}\ch{Pb \stS}\hspace{2mm} + \hspace{2mm}\ch{2 H2SO4 \stAq}
					\schemestop
				}{\vspace{-1.0em}}

			% end subsubsection



			\subsubsection{Lithium-Ion}

				Again, lithium ion batteries are rechargeable. They commonly use a graphite anode --- the hexagonal layers are able to bind
				to the relatively small \ch{Li+} ions, as well as \ch{Li} atoms. The cathode is typically made of a transition metal oxide,
				in this case \ch{CoO2}, which can also bind to \ch{Li+} ions in its structure.

				Note that aqueous solvents (containing water) cannot be used in the battery, because of the high reactivity of lithium metal;
				organic solvents are usually used.

				\diagram[1.0]{
					\schemestart[0, 1.5, thick]
						\ch{Li \stS}\arrow{->[\tinytext{oxidation}]}\ch{Li+}\hspace{2mm} + \hspace{2mm}\ch{e-}
						\arrow(@c1.south east--.north east){0}[-90,.25]
						\ch{Li+}\hspace{2mm} + \hspace{2mm}\ch{CoO2 \stS}\hspace{2mm} + \hspace{2mm}\ch{e-}\arrow{->[\tinytext{reduction}]}
						\ch{LiCoO2 \stS}
					\schemestop
				}{\vspace{-1.0em}}

				The overall reaction is as such:

				\diagram[1.0]{
					\schemestart[0, 1.0, thick]
						\ch{Li \stS}\hspace{2mm} + \hspace{2mm}\ch{CoO2 \stS}\arrow{->}\ch{LiCoO2 \stS}
						\hspace{10mm}$E^{\stdst}_{cell} = +3.4V$
					\schemestop
				}{\vspace{-1.0em}}

				Again, the process for recharging is simply electrolysis, and involves reversing the equations above.

				\diagram[1.0]{
					\schemestart[0, 1.0, thick]
						\ch{LiCoO2 \stS}\arrow{->}\ch{Li \stS}\hspace{2mm} + \hspace{2mm}\ch{CoO2 \stS}
					\schemestop
				}{\vspace{-1.0em}}


			% end subsubsection

		% end subsection


		\pagebreak
		\subsection{Fuel Cells}

			Fuel cells are basically batteries where the reactants are constantly replenished. Of note is the hydrogen-oxygen fuel cell,
			which basically involves the oxidation of hydrogen gas into water. The electrolyte is basically \ch{OH- \stAq}, typically heated
			to increase the rate of reaction.

			The main attraction of a hydrogen fuel cell is to enable the direct conversion of the \ch{H2 \stG} fuel into electricity, instead
			of using the typical burn-the-gas-and-heat-water-to-make-steam-that-turns-a-turbine-that-turns-a-generator-that-finally-makes-electricity design --- this enables much higher levels of efficiency, up to 70\%, over the 40-odd\% of typical generators.

			\diagram[1.0]{
				\schemestart[0, 1.5, thick]
					\ch{H2 \stG}\hspace{2mm} + \hspace{2mm}\ch{2 OH- \stAq}\arrow{->[\tinytext{oxidation}]}\ch{2 H2O \stL}\hspace{2mm} + \hspace{2mm}\ch{2 e-}
					\arrow(@c1.south east--.north east){0}[-90,.25]
					\ch{O2 \stG}\hspace{2mm} + \hspace{2mm}\ch{2 H2O \stL}\hspace{2mm} + \hspace{2mm}\ch{4 e-}\arrow{->[\tinytext{reduction}]}
					\ch{4 OH- \stAq}
				\schemestop
			}{\vspace{-1.0em}}

			At the anode, \ch{H2} gas diffuses through the porous graphite electrode and comes into contact with the \ch{KOH} electrolyte; at
			the cathode, \ch{O2} gas does the same. \ch{H2} is oxidised, and \ch{O2} is reduced.

			The overall equation forms water only, which is a selling point of this design since the product is clean.

			\diagram[1.0]{
				\schemestart[0, 1.0, thick]
					\ch{2 H2 \stG}\hspace{2mm} + \hspace{2mm}\ch{O2 \stG}\arrow{->}\ch{2 H2O \stL}
				\schemestop
			}{\vspace{-1.0em}}


		% end subsection

	% end section






% end part




























