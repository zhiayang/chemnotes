% Chapter 06 - Solubility Equilibrium.tex
% Copyright (c) 2014 - 2016, zhiayang@gmail.com
% Licensed under the Apache License Version 2.0.

\pagebreak
\part{Solubility Equilibrium}

	\section{Overview}

		Ions are typically soluble in a polar solvent, but since systems always tend to approach the lowest energy level, if a solid ionic
		compound can be created from some of those ions, then it will precipitate out of solution.

		The solubility of any given ionic compound is largely determined by \enth{sol}, the enthalpy change of solution; it is the sum
		of the enthalpy changes of hydration of the constituent ions, minus the lattice energy of the ionic compound.

		Hence, for a compound with a highly negative lattice energy, the enthalpy change of solution will be quite endothermic, and it is
		unlikely to be very soluble. Essentially, stronger ionic lattices are less soluble than weaker ones.

		Furthermore, solubility typically increases with temperature, since the increased energy of the solvent can increase the feasibility of
		solution.

		\subsection{Saturation}

			Even for highly soluble salts, eg. \ch{Na\Cl}, solubility is not infinite. Beyond a certain point, additional solid salt added
			to the solvent will no longer dissolve. At this limit, a \itl{saturated solution} is formed.

			When more solid salt is added to a saturated solution, an equilibrium is created, where ions are simultaneously being dissolved from,
			and precipitated into, the solid state. As with any other dynamic equilibrium, the rates of reaction of these two processes are
			\itl{non-zero} and identical.

		% end subsection


		\pagebreak
		\subsection{Solubility of Common Ionic Compounds}

			The solubilities of ionic compounds are often grouped by the anion, and most of them have fairly straightforward trends.

			\begin{center}\begin{table}[htb]\renewcommand{\arraystretch}{1.5}
			\begin{tabu} to \textwidth {X[c,m] | X[c,m] | X[c,m]}

				% headings
				Anion							&	Solubility	&	Exceptions								\\ \hline
				\ch{NO3-}						&	Soluble		&	---										\\ \hline
				\ch{\Cl-}, \ch{Br-}, \ch{I-}	&	Soluble		&	\ch{Ag+} and \ch{Pb^2+}					\\ \hline
				\ch{SO4^2-}						&	Soluble		&	\ch{Ag+}, \ch{Pb^2+}, \ch{Ca^2+}, \ch{Sr^2+}, \ch{Ba^2+}\\ \hline
				\ch{CO3^2-}						&	Insoluble	&	Group \bld{\romannum{I}} cations and \ch{NH4+}	\\ \hline
				\ch{OH-}						&	Insoluble	&	Group \bld{\romannum{I}} cations				\\ \hline
				\ch{O^2-}						&	Insoluble	&	Group \bld{\romannum{I}} cations				\\ \hline

			\end{tabu}
			\end{table}\end{center}\vspace{-10mm}


			Even though a certain ionic salt is said to be \itl{insoluble}, most of the time a small amount of free ions can still be found
			in solution; the concentrations of these ions are just \itl{very, very small}.

			Certain salts, such as \ch{Mg(OH)2} and \ch{Ca(OH)2}, are actually \itl{sparingly soluble} --- not-insignificant amounts of ions
			are dissolved into solution; limewater is a saturated solution of calcium hydroxide, and the \pH{} of a magnesium hydroxide solution
			is actually greater than 7.

		% end subsection

	% end section


	\pagebreak
	\section{Solubility Constants}

		\subsection{Solubility}

			The solubility of a salt is the number of moles, or mass, of the salt that can be dissolved in a given volume of solvent to create
			a saturated solution at a fixed temperature.

			The units of solubility are typically either \si{\molarConc} or \si{\gram\per\cubic\deci\metre}, depending on the
			the scale used.

			If the solubility of a certain salt, for instance \ch{Mg(NO3)2}, is \SI{8.43}{\molarConc}, then the concentration of \ch{Mg^2+}
			ions in solution will be \SI{8.43}{\molarConc}, and the concentration of \ch{OH-} ions will be twice that, \SI{16.9}{\molarConc}.

		% end subsection


		\subsection{Solubility Product, \texorpdfstring{\Ksp{}}{Ksp}}

			The \Ksp{} of a sparingly soluble salt is the molar concentrations of the constituent ions in a saturated solution raised to their stoichiometric coefficients, at a fixed temperature.

			The solubility product of an ionic equilibrium is derived from the equilibrium constant, \Kc{}. Hence, like \Kc{}, the value of the solubility product for any salt is only valid at a certain temperature. For a given equilibrium of solution:

			\txtreactioneqn{
				\ch{Mg(OH)2 \stS} \arrow{<=>} \ch{Mg^2+ \stAq}\hspace{2mm} + \hspace{2mm}\ch{2 OH- \stAq}
			}

			The equilibrium constant is this:

			\mathdiagram{
				\[\MKc = \frac{[\ch{Mg+}][\ch{OH-}]^{2}}{[\ch{Mg(OH)2}]} \hspace{15mm} \MKsp = [\ch{Mg+}][\ch{OH-}]^{2}\]
			}

			Since the \enquote{concentration} of a solid, in this case \ch{Mg(OH)2}, is constant, it can be factored out, giving the equation of \Ksp{}.
			The units of \Ksp{} are naturally dependent on the ions and their powers on the right side, so in this case they are
			\si{\mole\tothe{3}\per\deci\metre\tothe{9}}.

		% end subsection

		\subsection{Ionic Product}

			The ionic product of an ionic salt is defined in the same manner as the solubility product, except the solution need not be
			saturated; it is most often a measure of the \itl{current} concentration of ions in solution.

			The IP can be compared with \Ksp{} to determine if the solution will form a precipitate if more ions are added. If the ionic
			product is greater than \Ksp{}, then ions will precipitate out until IP is equal to \Ksp{}. If it is less, then more solid
			can be added until they are equal, giving a saturated solution.

			The idea here is similar to the relationship between \Qc{} and \Kc{}.

		% end subsection

	% end section


	\pagebreak
	\section{Factors Affecting Solubility}

		\subsection{Position of Equilibrium}

			One of the main factors affecting the solubility of an ionic salt is the position of equilibrium of the solution system. For a
			given equilibrium, for example that of \ch{Mg(OH)2}:

			\txtreactioneqn{
				\ch{Mg(OH)2 \stS} \arrow{<=>} \ch{Mg^2+ \stAq}\hspace{2mm} + \hspace{2mm}\ch{2 OH- \stAq}
			}

			The position of equilibrium depends solely on the concentrations of the ions, since adding or removing solids will not change its
			position. Any action that modifies the concentration of either ion will result in a shift in equilibrium, and hence an increase or
			decrease in the solubility of the salt.

			For example, adding an acid like \ch{H2SO4} to the system will cause the concentration of \ch{OH-} to decrease, and hence by the
			principle of Le Châtelier, the position of equilibrium will shift to the right, causing more of the solid to dissolve.

			Conversely, if the concentration of \ch{Mg^2+} were increased, for instance by adding \ch{Mg(NO3)2}, then the position of equilibrium
			will shift to the left, causing less solid to dissolve, and indeed causing existing ions to precipitate out of solution.

			\subsubsection{Common Ion Effect}

				The common ion effect is really just a fancy name for a special case of shifting the equilibrium by increasing the concentration of an ion;
				if a solution containing \enquote{common ions} is added to an existing equilibrium, then the solubility of the sparingly soluble salt
				will decrease, as a result of Le Châtelier's Principle.

			% end subsubsection

			\subsubsection{Complex Formation}

				This is also just a special case of changing the position of equilibrium by changing concentrations.
				For example, given the following systems of \ch{\Al(OH)3} and \ch{[\Al(OH)4]-}:

				\txtreactioneqn{
					\ch{\Al(OH)3 \stS}\arrow{<=>}\ch{\Al^3+ \stAq}\hspace{2mm} + \hspace{2mm}\ch{3 OH- \stAq}
					\arrow(@c1.east--.east){0}[-90,.4]
					\ch{\Al^3+ \stAq} + \ch{4 OH- \stAq}\arrow{<=>}\ch{[\Al(OH)4]- \stAq}
				}

				If \ch{OH-} ions are introduced to an aqueous solution of \ch{\Al^3+} ions, then the precipitate of \ch{\Al(OH)3} will form since
				the ionic product is greater than the solubility product.

				If excess \ch{OH-} ions are added, however, then the complex ion \ch{[\Al(OH)4]-} forms, which is soluble in water. This
				removes \ch{\Al^3+} ions from solution, and if enough complex ions form, then the position of equilibrium for the solution system
				will shift forward enough to dissolve the precipitate.


			% end subsubsection
		% end subsection
	% end section
% end part















