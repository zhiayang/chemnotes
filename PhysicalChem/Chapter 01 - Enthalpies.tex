% Chapter 01 - Enthalpies.tex
% Copyright (c) 2014 - 2016, zhiayang@gmail.com
% Licensed under the Apache License Version 2.0.

\pagebreak
\part{Enthalpies}

\section{Enthalpy Change}

	Enthalpy refers to the total energy content of a given substance. When reacting two or more compounds together, the bonds within
	must first be broken, before new bonds are formed. This process involves the transfer of energy, which is measured as the
	\textit{enthalpy change}.

	Enthalpy change, or \enth{}, measures the difference in energy content between the reactants and the products of a reaction.

	For an exothermic reaction, the energy content of the products is less than that of the reactants. Since the most stable state
	is that with the least amount of energy, exothermic reactions are said to be \textit{more energetically favoured} –– \enth{} < 0.
	Heat is released into the surroundings, and generally, temperature increases.

	\diagram[1.0]{
		\begin{endiagram}[scale=2.0,offset=1.5,x-label-text=reaction pathway,y-label-text=energy]
			\ENcurve[step=1.5]{2,4.5,0}

			\ShowNiveaus[shift=-1.0,length=2.0,niveau=N1-1]
			\ShowNiveaus[shift=1.0,length=2.0,niveau=N1-3]

			\ShowEa[label,label-side=left,label-pos=0.2]
			\ShowGain[label,offset=-40mm]

			\draw[above] (N1-1) ++ (1,0) node {\small \hspace{-40mm}reactants};
			\draw[above] (N1-3) ++ (1,0) node {\small products};

		\end{endiagram}
	}

	\pagebreak
	Conversely, for an endothermic reaction, the reverse is true –– the energy content of the final products is greater than that
	of the reactants. Thus, these reactions are somewhat \textit{less energetically favoured}, and less likely to happen.
	\enth{} > 0, and heat is absorbed from the surroundings to feed the reaction.


	\diagram[1.0]{
		\begin{endiagram}[scale=2.0,offset=1.5,x-label-text=reaction pathway,y-label-text=energy]
			\ENcurve[step=1.5]{0,4.5,2}

			\ShowNiveaus[shift=-1.0,length=2.0,niveau=N1-1]
			\ShowNiveaus[shift=1.0,length=2.0,niveau=N1-3]

			\ShowEa[label,label-side=left,label-pos=0.2]
			\ShowGain[label,offset=0mm]

			\draw[above] (N1-1) ++ (1,0) node {\small \hspace{-40mm}reactants};
			\draw[above] (N1-3) ++ (1,0) node {\small products};

		\end{endiagram}
	}


	\pagebreak
	\subsection{Activation Energy}

		Regardless of the enthalpy change of the reaction, some energy must always be input, and thus the activation energy, \ea, is
		always positive. This is because bonds must always be broken, which requires energy input, before new bonds can be formed.

		However, the \ea for endothermic reactions is generally much higher than that for exothermic reactions, since its reactants
		are generally more stable.

	% end subsection

	\subsection{Thermochemical Equations}

		A thermochemical equation is simply a normal, balanced chemical equation that has an enthalpy change of reaction associated with
		it. State symbols must also be included, since a change in state necessitates a change in the enthalpy of the substance.

		\txtdiagram{
			\schemestart[0,1.0,thick]
				\ch{H2SO4 \stAq}\hspace{2mm} + \hspace{2mm}\ch{2 NaOH \stAq}
				\arrow
				\ch{Na2SO4 \stAq}\hspace{2mm} + \hspace{2mm}\ch{2 H2O \stL}
			\schemestop
		}{\enth{} = \SI{-114.2}{\kilo\joule\per\mole}}

		Note that the enthalpy change is per mole \textit{of reaction}, not per mole of any one substance. Thus, if the above reaction
		were to be rewritten using 1 mole of NaOH instead:

		\txtdiagram{
			\schemestart[0,1.0,thick]
				\ch{\fracHalf H2SO4 \stAq}\hspace{2mm} + \hspace{2mm}\ch{NaOH \stAq}
				\arrow
				\ch{\fracHalf Na2SO4 \stAq}\hspace{2mm} + \hspace{2mm}\ch{H2O \stL}
			\schemestop
		}{\enth{} = \SI{-57.1}{\kilo\joule\per\mole}}

		The enthalpy change of the reaction is now half the previous value.
	% end subsection


	\pagebreak
	\subsection{Bond Dissociation Energy}

		Bond dissociation energy is the energy required to break \textit{1 mole} worth of \textit{covalent bonds} between two atoms in the
		\textit{gaseous state}. Note that this value is always positive, since energy is required to break bonds. The larger the value,
		the stronger the bond.

		Bonds are complex beasts; breaking seemingly identical bonds successively will require differing amounts of energy. For instance, the
		bond dissociation energy for the first \ch{C-H} bond in \ch{CH4} is \SI{425}{\kilo\joule\per\mole}, while that of the second \ch{C-H}
		bond (in what is now \ch{CH3}) is \SI{470}{\kilo\joule\per\mole}.

		Naturally, the bond dissociation energies can vary by significant amounts between different molecules –– the BDE for a \ch{C-H} bond
		in \ch{CH4} is smaller than that of the \ch{C-H} bond in \ch{CH2=CH2}.

	% end subsection

	\subsection{Bond Energy}

		Due to these problems, the \textit{bond energy} is more frequently used. It is simply the \textit{average} of the bond dissociation
		energies of the particular bond, sampled from a large variety of molecules.

		Thus, it is defined as the \textit{average energy} required to break \textit{1 mole} worth of \textit{covalent bonds} between two
		atoms in the \textit{gaseous state}.

		Furthermore, if the bond in question is between two atoms in a diatomic molecule in the gaseous state, then the bond energy and
		the enthalpy change of atomisation (\enth{atom}, below) are related: The bond energy is twice the enthalpy change of atomisation.

		\txtdiagram{
			\schemestart[0,1.0,thick]
				\ch{\fracHalf H-H \stG}
				\arrow
				\ch{H \stG}
			\schemestop
		}{\enth{atom} = \fracHalf BE(\ch{H-H})}

		Indeed, since reactions at a fundamental level simply involve breaking and forming bonds, the enthalpy change of a given reaction
		can be calculated (somewhat inaccurately, due to bond energy being an average) taking the difference between the energies of the
		bonds formed, and that of the bonds broken.

		\txtdiagram{
			\schemestart[0,1.0,thick]
				\enth{r} = \chemSigma{} BE(broken) - \chemSigma{} BE(formed)
			\schemestop
		}{}

		Accounting for any required changes in state can be done by adding the enthalpy change of either fusion or vaporisation on either
		side.

	% end subsection






	\pagebreak
	\subsection{Standard Enthalpy Change}

		For enthalpy change to have any meaning, it must be defined in terms of a set of standard conditions. In this case, the standard
		enthalpy change, \enthStd{}, is defined as the enthalpy change at \textit{\SI{298}{\kelvin}}, and a pressure of \textit{\SI{1}{\atm}}.

		An element or substance in its standard state has the least amount of energy, and is physically stable. Elements in their standard
		states (\ch{C \stS}, \ch{Br2 \stL}, \ch{N2 \stG}, etc.) are assigned an enthalpy value of \textit{0}.

		The unit of enthalpy change is usually \si{\kilo\joule\per\mole} (kilojoules per mole). This means that it is always determined
		per mole of some substance; the greater the number of moles of substance reacting, the greater the total enthalpy change will be,
		in \si{\kilo\joule}.

	% end subsection

	\pagebreak
	\subsection{The Law of Hess}

		Hess's Law of \textit{Constant Heat Summation} states that the enthalpy change of any given reaction is determined only by the
		enthalpy change between the initial state and final state, regardless of the steps taken in between.

		Therefore, whether the reactants reacted directly or through some convoluted pathway to get to the final state, the enthalpy
		change will be the same. To make use of this, typically an energy cycle diagram is drawn, inserting whatever intermediate reaction
		is required (with known enthalpies) to get to the final product.

		\diagram[1.0]{
			\schemestart[0,1.0,thick]
				\ch{\fracHalf O2 \stG}\hspace{2mm} + \hspace{2mm}\ch{C \stS}\hspace{2mm} + \hspace{2mm}\ch{\fracHalf O2 \stG}
				\arrow(one--two){->[\enth{f} of \ch{CO}]}[0,2]
				\ch{CO2 \stG}\hspace{2mm} + \hspace{2mm}\ch{\fracHalf O2 \stG}
				\arrow(@one.south east--three.north west){->% note here: we need to manually set the parbox width, because latex is stupid.
					[][*{0.east}\parbox{26mm}{\begin{center}\enth{c} of \ch{C}\\ \SI{-394}{\kilo\joule\per\mole}\end{center}}]}[-62.5,1.5]
				\ch{CO2 \stG}
				\arrow(@three.north east--@two.south west){->%
					[][*{0.west}\parbox{26mm}{\begin{center}\enth{c} of \ch{CO}\\ \SI{-283}{\kilo\joule\per\mole}\end{center}}]}[,1.5]
			\schemestop
		}{Half a mole of \ch{O2} was added on both sides of the primary reaction to balance it.}

		Note that the direction of the arrow can be manipulated \textit{at will}, simply by reversing the sign of the associated enthalpy
		change. Indeed, extra reactants can be introduced too, as long as they are accounted for in all reactions.

	% end subsection



% end section


































































