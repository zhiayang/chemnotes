% Chapter 01 - Enthalpies.tex
% Copyright (c) 2014 - 2016, zhiayang@gmail.com
% Licensed under the Apache License Version 2.0.

\pagebreak
\part{Enthalpies}

	\section{Enthalpy Change}

		Enthalpy refers to the total energy content of a given substance. When reacting two or more compounds together, the bonds within
		must first be broken, before new bonds are formed. This process involves the transfer of energy, which is measured as the
		\itl{enthalpy change}.

		Enthalpy change, or \enth{}, measures the difference in energy content between the reactants and the products of a reaction.

		For an exothermic reaction, the energy content of the products is less than that of the reactants. Since the most stable state
		is that with the least amount of energy, exothermic reactions are said to be \itl{more energetically favoured} --- \enth{} < 0.
		Heat is released into the surroundings, and generally, temperature increases.

		\diagram[1.0]{
			\begin{endiagram}[scale=2.0,offset=1.5,x-label-text=reaction pathway,y-label-text=energy]
				\ENcurve[step=1.5]{2,4.5,0}

				\ShowNiveaus[shift=-1.0,length=2.0,niveau=N1-1]
				\ShowNiveaus[shift=1.0,length=2.0,niveau=N1-3]

				\ShowEa[label,label-side=left,label-pos=0.2]
				\ShowGain[label,offset=-40mm]

				\draw[above] (N1-1) ++ (1,0) node {\small \hspace{-40mm}reactants};
				\draw[above] (N1-3) ++ (1,0) node {\small products};

			\end{endiagram}
		}

		\pagebreak
		Conversely, for an endothermic reaction, the reverse is true --- the energy content of the final products is greater than that
		of the reactants. Thus, these reactions are somewhat \itl{less energetically favoured}, and less likely to happen.
		\enth{} > 0, and heat is absorbed from the surroundings to feed the reaction.


		\diagram[1.0]{
			\begin{endiagram}[scale=2.0,offset=1.5,x-label-text=reaction pathway,y-label-text=energy]
				\ENcurve[step=1.5]{0,4.5,2}

				\ShowNiveaus[shift=-1.0,length=2.0,niveau=N1-1]
				\ShowNiveaus[shift=1.0,length=2.0,niveau=N1-3]

				\ShowEa[label,label-side=left,label-pos=0.2]
				\ShowGain[label,offset=0mm]

				\draw[above] (N1-1) ++ (1,0) node {\small \hspace{-40mm}reactants};
				\draw[above] (N1-3) ++ (1,0) node {\small products};

			\end{endiagram}
		}


		\pagebreak
		\subsection{Activation Energy}

			Regardless of the enthalpy change of the reaction, some energy must always be input, and thus the activation energy, \ea, is
			always positive. This is because bonds must always be broken, which requires energy input, before new bonds can be formed.

			However, the \ea for endothermic reactions is generally much higher than that for exothermic reactions, since its reactants
			are generally more stable.

		% end subsection

		\subsection{Thermochemical Equations}

			A thermochemical equation is simply a normal, balanced chemical equation that has an enthalpy change of reaction associated with
			it. State symbols must also be included, since a change in state necessitates a change in the enthalpy of the substance.

			\txtdiagram{
				\schemestart[0,1.0,thick]
					\ch{H2SO4 \stAq}\hspace{2mm} + \hspace{2mm}\ch{2 NaOH \stAq}
					\arrow
					\ch{Na2SO4 \stAq}\hspace{2mm} + \hspace{2mm}\ch{2 H2O \stL}
				\schemestop
			}{\enth{} = \SI{-114.2}{\kilo\joule\per\mole}}

			Note that the enthalpy change is per mole \itl{of reaction}, not per mole of any one substance. Thus, if the above reaction
			were to be rewritten using 1 mole of NaOH instead:

			\txtdiagram{
				\schemestart[0,1.0,thick]
					\ch{\fracHalf H2SO4 \stAq}\hspace{2mm} + \hspace{2mm}\ch{NaOH \stAq}
					\arrow
					\ch{\fracHalf Na2SO4 \stAq}\hspace{2mm} + \hspace{2mm}\ch{H2O \stL}
				\schemestop
			}{\enth{} = \SI{-57.1}{\kilo\joule\per\mole}}

			The enthalpy change of the reaction is now half the previous value.
		% end subsection


		\pagebreak
		\subsection{Bond Dissociation Energy}

			Bond dissociation energy is the energy required to break \itl{1 mole} worth of \itl{covalent bonds} between two atoms in the
			\itl{gaseous state}. Note that this value is always positive, since energy is required to break bonds. The larger the value,
			the stronger the bond.

			Bonds are complex beasts; breaking seemingly identical bonds successively will require differing amounts of energy. For instance, the
			bond dissociation energy for the first \ch{C-H} bond in \ch{CH4} is \SI{425}{\kilo\joule\per\mole}, while that of the second \ch{C-H}
			bond (in what is now \ch{CH3}) is \SI{470}{\kilo\joule\per\mole}.

			Naturally, the bond dissociation energies can vary by significant amounts between different molecules --- the BDE for a \ch{C-H} bond
			in \ch{CH4} is smaller than that of the \ch{C-H} bond in \ch{CH2=CH2}.

		% end subsection

		\subsection{Bond Energy}

			Due to these problems, the \itl{bond energy} is more frequently used. It is simply the \itl{average} of the bond dissociation
			energies of the particular bond, sampled from a large variety of molecules.

			Thus, it is defined as the \itl{average energy} required to break \itl{1 mole} worth of \itl{covalent bonds} between two
			atoms in the \itl{gaseous state}.

			Furthermore, if the bond in question is between two atoms in a diatomic molecule in the gaseous state, then the bond energy and
			the enthalpy change of atomisation (\enth{atom}, below) are related: The bond energy is twice the enthalpy change of atomisation.

			\txtdiagram{
				\schemestart[0,1.0,thick]
					\ch{\fracHalf H-H \stG}
					\arrow
					\ch{H \stG}
				\schemestop
			}{\enth{atom} = \fracHalf BE(\ch{H-H})}

			Indeed, since reactions at a fundamental level simply involve breaking and forming bonds, the enthalpy change of a given reaction
			can be calculated (somewhat inaccurately, due to bond energy being an average) taking the difference between the energies of the
			bonds formed, and that of the bonds broken.

			\txtdiagram{
				\schemestart[0,1.0,thick]
					\enth{r} = \chemSigma{} BE(broken) - \chemSigma{} BE(formed)
				\schemestop
			}{\vspace{-2em}}

			Accounting for any required changes in state can be done by adding the enthalpy change of either fusion or vaporisation on either
			side.

		% end subsection






		\pagebreak
		\subsection{The Law of Hess}

			Hess's Law of \itl{Constant Heat Summation} states that the enthalpy change of any given reaction is determined only by the
			enthalpy change between the initial state and final state, regardless of the steps taken in between.

			Therefore, whether the reactants reacted directly or through some convoluted pathway to get to the final state, the enthalpy
			change will be the same. To make use of this, typically an energy cycle diagram is drawn, inserting whatever intermediate reaction
			is required (with known enthalpies) to get to the final product.

			\diagram[1.0]{
				\schemestart[0,1.0,thick]
					\ch{\fracHalf O2 \stG}\hspace{2mm} + \hspace{2mm}\ch{C \stS}\hspace{2mm} + \hspace{2mm}\ch{\fracHalf O2 \stG}
					\arrow(one--two){->[\enth{f} of \ch{CO}]}[0,2]
					\ch{CO2 \stG}\hspace{2mm} + \hspace{2mm}\ch{\fracHalf O2 \stG}
					\arrow(@one.south east--three.north west){->% note here: we need to manually set the parbox width, because latex is stupid.
						[][*{0.east}\parbox{26mm}{\begin{center}\enth{c} of \ch{C}\\ \SI{-394}{\kilo\joule\per\mole}\end{center}}]}[-62.5,1.5]
					\ch{CO2 \stG}
					\arrow(@three.north east--@two.south west){->%
						[][*{0.west}\parbox{26mm}{\begin{center}\enth{c} of \ch{CO}\\ \SI{-283}{\kilo\joule\per\mole}\end{center}}]}[,1.5]
				\schemestop
			}{Half a mole of \ch{O2} was added on both sides of the primary reaction to balance it.}

			Note that the direction of the arrow can be manipulated \itl{at will}, simply by reversing the sign of the associated enthalpy
			change. Indeed, extra reactants can be introduced too, as long as they are accounted for in all reactions.

		% end subsection

		\subsection{Standard Enthalpy Change}

			For enthalpy change to have any meaning, it must be defined in terms of a set of standard conditions. In this case, the standard
			enthalpy change, \enthStd{}, is defined as the enthalpy change at \itl{\SI{298}{\kelvin}}, and a pressure of \itl{\SI{1}{\atm}}.

			An element or substance in its standard state has the least amount of energy, and is physically stable. Elements in their standard
			states (\ch{C \stS}, \ch{Br2 \stL}, \ch{N2 \stG}, etc.) are assigned an enthalpy value of \itl{0}.

			The unit of enthalpy change is usually \si{\kilo\joule\per\mole} (kilojoules per mole). This means that it is always determined
			per mole of some substance; the greater the number of moles of substance reacting, the greater the total enthalpy change will be,
			in \si{\kilo\joule}.

		% end subsection

		\pagebreak
		\subsection{Determination of Enthalpy Change}

			Since heat is released or absorbed when a reaction takes place, the enthalpy change, which is the change in energy of the compounds
			involved, can be determined through the measurement of temperature. Using the formula to determine heat change, \itl{q}, enthalpy change
			is then heat change per mole.

			\diagram[2.0]{
				\parbox{50mm}{\begin{center}
					\vspace{-5mm}
					$q = mc\Delta T$
					\\
					$\Delta H = \pm\frac{q}{n}$
				\end{center}}
			}

			\vbox{
				\tabto{0mm}\boit{q}\tabto{15mm}	Heat change of the reaction.
				\tabto{0mm}\boit{m}\tabto{15mm}	Mass of the solution — in a polystyrene cup experiment, this excludes solids.
				\tabto{0mm}\boit{c}\tabto{15mm}	Specific heat capacity of the solution (usually assumed to be \SI{4.18}{\joule\per\gram\per\kelvin})
				\tabto{0mm}\boit{∆T}\tabto{15mm}Change in temperature of the solution (before and after reaction)
				\tabto{0mm}\boit{∆H}\tabto{15mm}Enthalpy change of reaction.
			}

			While heat change is a scalar value and is always positive, enthalpy change has a sign, which is decided by whether the reaction is
			exothermic or endothermic. This can be inferred from the direction of the change in temperature.

		% end subsection

		\subsection{Types of Enthalpy Change}

			\subsubsection{Enthalpy Change of Fusion and Vaporisation}

				The enthalpy change of fusion is the change in enthalpy when \itl{1 mole} of a substance in the \itl{solid state} is heated,
				at a constant temperature (the melting point), until it becomes a liquid. Similarly, the enthalpy change of vaporisation is the
				change in enthalpy when \itl{1 mole} of a substance in the \itl{liquid state} is heated at a constant temperature
				(the boiling point), until it becomes a gas.

				These are known as \itl{latent} heats, because the temperature of the substance remains constant at either the melting or
				boiling points — no heat is actually released. However, since enthalpy measures the actual, total energy content of the compound,
				enthalpy still increases since the internal energy increases.

				As one might expect, the enthalpy changes of fusion and vaporisation are almost always positive, since they involve moving a
				compound to a higher energy, and thus less stable, state. There are, however, a certain number of exceptions to this.
				The negative of these enthalpy changes would represent the amount of energy that needs to be removed, per mole, to change the
				substance from liquid to solid, and gas to liquid respectively.


			% end subsubsection

			\pagebreak
			\subsubsection{Enthalpy Change of Formation}

				The standard enthalpy change of formation a compound is the heat evolved when \itl{1 mole} of it is formed from its constituent
				elements, all in their standard states, under standard conditions. This disregards the actual method of forming the compound, and
				instead only accounts for simply mashing elements together. For example:

				\txtdiagram{
					\schemestart[0,1.0,thick]
						\ch{2 C \stS}\hspace{2mm} + \hspace{2mm}\ch{3 H2 \stAq}\hspace{2mm} + \hspace{2mm}\ch{\fracHalf O2 \stG}
						\arrow
						\ch{C2H5OH \stL}
					\schemestop
				}{\enth{} = \SI{-298}{\kilo\joule\per\mole}}

				If the enthalpy change of formation of a substance, \enth{f} is > 0, then the compound is less stable than its constituent elements,
				meaning it is more likely to decompose --- the reverse is true if \enth{f} < 0. Note that \enth{f} of elements in their standard
				states is 0, by definition.

			% end subsubsection

			\subsubsection{Enthalpy Change of Combustion}
				The enthalpy change of combustion of a compound is the heat evolved when \itl{1 mole} of it is completely burned in excess
				oxygen, under standard conditions. This enthalpy change is always negative --- heat is always released when things are combusted.

				\txtdiagram{
					\schemestart[0,1.0,thick]
						\ch{CH4 \stG}\hspace{2mm} + \hspace{2mm}\ch{2 O2 \stG}
						\arrow
						\ch{CO2 \stG}\hspace{2mm} + \hspace{2mm}\ch{2 H2O \stL}
					\schemestop
				}{\enth{c} = \SI{-866}{\kilo\joule\per\mole}}

			% end subsubsection

			\subsubsection{Enthalpy Change of Neutralisation}

				The enthalpy change of neutralisation is the heat evolved when \itl{1 mole} of \ch{H2O} is formed when an acid and a base are
				reacted. Normally, it is the reaction between \ch{H+} ions and \ch{OH-} ions in aqueous solution.

				\txtdiagram{
					\schemestart[0,1.0,thick]
						\ch{H+ \stAq}\hspace{2mm} + \hspace{2mm}\ch{OH- \stAq}
						\arrow
						\ch{H2O \stL}
					\schemestop
				}{\enth{neut} = \SI{-57.1}{\kilo\joule\per\mole}}

				\pagebreak
				This reaction between \ch{H+} ions and \ch{OH-} ions underlies the majority of reactions between acids and bases.

				\txtdiagram{
					\schemestart[0,1.0,thick]
						\ch{H\Cl \stAq}\hspace{2mm} + \hspace{2mm}\ch{NaOH \stAq}
						\arrow
						\ch{H2O \stL}\hspace{2mm} + \hspace{2mm}\ch{Na\Cl \stAq}
					\schemestop
				}{\enth{neut} = \SI{-57.1}{\kilo\joule\per\mole}}

				In general, the enthalpy change of neutralisation of all strong acid-base reactions will be the same,
				at \SI{-57.1}{\kilo\joule\per\mole}. However, for weak acids, the \ch{H+} and \ch{OH-} ions only dissociate
				partially --- energy is thus needed to break these partially dissociated bonds, which decreases the enthalpy change
				of neutralisation of weak acids and bases.


				\txtdiagram{
					\schemestart[0,1.0,thick]
						\ch{CH3CO2H \stAq}\hspace{2mm} + \hspace{2mm}\ch{NaOH \stAq}
						\arrow
						\ch{H2O \stL}\hspace{2mm} + \hspace{2mm}\ch{CH3CO2Na \stAq}
					\schemestop
				}{\enth{neut} = \SI{-55.9}{\kilo\joule\per\mole}}


			% end subsubsection

			\subsubsection{Enthalpy Change of Atomisation}

				The enthalpy change of atomisation for an element is the energy required to create \itl{1 mole} of gaseous atoms from the
				element in question in the standard state.

				\txtdiagram{
					\schemestart[0,1.0,thick]
						\ch{\fracHalf \Cl2 \stG}
						\arrow
						\ch{\Cl \stG}
					\schemestop
				}{\enth{atom} = \SI{+122}{\kilo\joule\per\mole}}

				For elements that are already gaseous at their standard state, such as \ch{F2 \stG}, the enthalpy change of atomisation would
				simply be the bond energy, in this case of the \ch{F-F} bond. For elements that are monatomic in the gaseous state,
				such as noble gases, the \enth{atom} is 0.


				The enthalpy change of atomisation of a non-elemental compound, on the other hand, is the energy required to create gaseous
				atoms of its constituent elements, from the compound in its standard state. For example:

				\txtdiagram{
					\schemestart[0,1.0,thick]
						\ch{CH4 \stG}
						\arrow
						\ch{C \stG}\hspace{2mm} + \hspace{2mm}\ch{4 H \stG}
					\schemestop
				}{\vspace{-2em}}

				Of course, for compounds not initially in the gaseous state, they must first be converted, and the appropriate enthalpy
				change (of fusion or vaporisation) taken into account.


			% end subsubsection


			\pagebreak
			\subsubsection{Enthalpy Change of Hydration}

				The enthalpy change of hydration is the energy released when \itl{1 mole} of gaseous ions is dissolved in an
				\itl{infinite volume} of water, under standard conditions. \enth{hyd} is always negative, as it involves the
				formation of ion-dipole interactions between the ions in question and the water molecules.

				\txtdiagram{
					\schemestart[0,1.0,thick]
						\enth{hyd} $\propto \frac{q}{r}$
					\schemestop
				}{\vspace{-2em}}

				The enthalpy change of hydration is proportional to the charge density of the ions; the higher the charge density,
				the stronger the ion-dipole interactions and thus the larger the enthalpy change.

			% end subsubsection

			\subsubsection{Enthalpy Change of Solution}

				The enthalpy change of solution is the enthalpy change when \itl{1 mole} of a substance is dissolved in an \itl{infinite volume}
				of solvent, under standard conditions. The solute and solvent can be anything, even though the solute is usually an ionic compound,
				and the solvent water.

				If \enth{sol} is highly positive, then the solute is likely to be insoluble in the solvent, since the reaction would be
				endothermic. On the other hand, if \enth{sol} is negative, then the solute is likely to be soluble in the solvent.

			% end subsubsection

			\subsubsection{Ionisation Energy}

				Ionisation energy is the energy required to remove \itl{1 mole} of electrons from \itl{1 mole} of gaseous atoms, forming
				\itl{1 mole} of \itl{gaseous cations}. The first ionisation energy involves the removal of electrons from neutral atoms,
				forming singly-charged gaseous cations; the second ionisation energy involves removing electrons from singly-charged gaseous
				cations, forming doubly-charged gaseous cations, and so on.

				As should be obvious, ionisation energy increases with the number of electrons already removed. It is always positive,
				since energy is required to remove electrons from an atom.

			% end subsubsection

			\subsubsection{Electron Affinity}

				Electron affinity is essentially the reverse of ionisation energy, measuring the enthalpy change when \itl{1 mole} of electrons
				is added to \itl{1 mole} of gaseous atoms, forming \itl{1 mole} of \itl{gaseous anions}. As with ionisation energy, the energy
				required to do this increases with the number of electrons previously added.

				The first electron affinity is usually negative, due to the slight electrostatic attraction between the neutral atom (rather, its
				nucleus) and the incoming electron. Subsequent electron affinities are usually positive, since energy is required to move a negative
				electron towards a negatively charged atom.

			% end subsubsection

		% end subsection

		\subsection{Lattice Energy}

			Lattice energy is the heat released when \itl{1 mole} of an solid ionic compound is formed from its constituent ions in the
			gaseous state. It is \itl{always negative}, since it involves the formation of ionic bonds between atoms.

			The magnitude of lattice energy is a measure of the strength of the ionic bond — the more exothermic it is, the stronger the ionic bond.
			Ionic bond strength depends on two factors; the magnitude of the charge on both ions, and the ionic radius of the atoms.

			\txtdiagram{
				\schemestart[0,1.0,thick]
					$LE \propto \frac{q_{+}\times q_{-}}{r_{+} + r_{-}}$
				\schemestop
			}{\vspace{-2em}}

			\itl{q\sbs{+}} and \itl{q\sbs{-}} represent the charge of the respective ion, and \itl{r\sbs{+}} and \itl{r\sbs{-}} represent
			the ionic radius. Thus, the maximum lattice energy is achieved with the highest charge magnitude and the smallest ionic radius.


			\subsubsection{Theoretical vs. Experimental Lattice Energies}

				There exists no truly ionic compound --- all ionic bonds will have some degree of covalent character, due to the behaviour of
				electrons. Naturally, some ionic bonds exhibit greater covalent character than others. This is related to the charge density
				of the cation and the ionic radius of the anion.

				If the cation has a high charge density and the ionic radius of the anion is relatively large, then the large electron cloud
				will be attracted towards the cation (due to higher polarisability, shielding effect, etc.). This creates a situation where the
				charge is more spread out over the participating ions, resulting in a smaller ionic character and thus greater covalent
				character. Conversely, if the anion as a small electron cloud, it will be less polarisable, and tend to exhibit greater ionic
				character.

				The calculation of lattice energy assumes a model where the bond is completely ionic, ie. zero covalent character. As this is
				clearly \itl{not} the case, there are often discrepancies between theoretical and experimental values (from Born-Haber cycles)
				of lattice energy. If the discrepancy is small, then the ionic bond in question has a large ionic character. Else, it has a
				larger covalent character.

			% end subsubsection


			\pagebreak
			\subsubsection{Born-Haber Cycles}

				Since it is impractical to determine lattice energy directly by reacting gaseous ions together, the \itl{Law of Hess} can be
				applied to calculate lattice energy indirectly, using a Born-Haber cycle, which really is just a glorified energy-level diagram.

				\imgdiagram{120mm}{../figures/physical/ch01/born_haber_cycle.png}{\vspace{-4em}}

				In the example above, the lattice energy of \ch{LiF}, \enth{latt} is being calculated indirectly using various enthalpy changes.
				The initial state is given an enthalpy of 0. To put into words:


				\begin{numberedlist}
					\ListProperties(Space*=-0.2em, Space=-0.2em, Numbers=a)
					&	Atoms are at their initial, standard state; \ch{Li \stS} and \ch{\fracHalf F2 \stG}.		\\
						\enth{} = 0.

					&	\ch{Li \stS} is heated into the gaseous state, forming \ch{Li \stG}.						\\
						\enth{} = \enth{fus} + \enth{vap}.

					&	One mole of electrons is removed from \ch{Li \stG}, forming \ch{Li+ \stG} and \ch{e-}.		\\
						\enth{} = \enth{fus} + \enth{vap} + First IE (\ch{Li}).

					&	The \ch{F-F} bond is broken, forming \ch{F \stG}.											\\
						\enth{} = \enth{fuS} + \enth{vap} + First IE (\ch{Li}) + BE(\ch{F-F}).

					&	One mole of electrons is added to \ch{F \stG}, forming \ch{F- \stG}.						\\
						\enth{} = \enth{fuS} + \enth{vap} + First IE (\ch{Li}) + BE(\ch{F-F}) + First EA (\ch{F}).

					&	\ch{LiF \stS} is formed from its constituent gaseous ions, \ch{Li+ \stG} and \ch{F- \stG}.	\\
						\enth{} = \enth{fuS} + \enth{vap} + First IE (\ch{Li}) + BE(\ch{F-F}) + First EA (\ch{F}) + \enth{latt}.

					&	To determine \enth{latt}, \enth{f} is used --- the enthalpy change of formation of \ch{LiF \stS} from constituent
						elements in their standard state.

					&	Thus, applying the \itl{Law of Hess} --- \enth{f} = \enth{fus} + \enth{vap} + First IE (\ch{Li}) + BE(\ch{F-F}) +
						First EA (\ch{F}) + \enth{latt}. Rearrange to solve for \enth{latt}.

				\end{numberedlist}

			% end subsubsection


			\subsubsection{Lattice Energy, \enth{sol} and \enth{hyd}}

				These three energies are related; dissolving a solid, ionic compound in water can be split into two sequential processes.
				First, the formation of gaseous ions form the solid ionic compound:

				\txtdiagram{
					\schemestart[0,1.0,thick]
						\ch{NaF \stS}
						\arrow
						\ch{Na+ \stG}\hspace{2mm} + \hspace{2mm}\ch{F- \stG}
					\schemestop
				}{\enth{} = $-LE$}

				Second, the gaseous are hydrated.

				\txtdiagram{
					\schemestart[0,1.0,thick]
						\ch{Na+ \stG}\hspace{2mm} + \hspace{2mm}\ch{F- \stG}\hspace{2mm} + \hspace{2mm}\ch{aq}
						\arrow
						\ch{Na+ \stAq}\hspace{2mm} + \hspace{2mm}\ch{F- \stAq}
					\schemestop
				}{\enth{} = \enth{hyd}(\ch{Na+}) + \enth{hyd}(\ch{F-})}

				Therefore, the total enthalpy change for this process is simply the sum of the enthalpy change of hydration of each ion, less
				the lattice energy of the final ionic compound.

				\txtdiagram{
					\schemestart[0,1.0,thick]
						\enth{sol} $= \sum{}$\enth{hyd}$ - LE$
					\schemestop
				}{\vspace{-2em}}

				Therefore, if the magnitude of the lattice energy is greater than the magnitude of the enthalpy changes of hydration, then the
				overall enthalpy change will be endothermic, and vice versa.

				\imgdiagram{120mm}{../figures/physical/ch01/lattice_energy_relationship.png}{\vspace{-4em}}

			% end subsubsection

		% end subsection

	% end section

% end part































































