% Chapter 02 - Gibbs Energy and Entropy.tex
% Copyright (c) 2014 - 2016, zhiayang@gmail.com
% Licensed under the Apache License Version 2.0.

\pagebreak
\part{Gibbs Energy and Entropy}

	\section{Entropy and Spontaneity}

		The Second Law of Thermodynamics states that for a process to be spontaneous, the total entropy of the system must increase. Entropy, represented by \emph{S}, is a measure of the number of ways energy can be distributed in a given system. This is often manifested through
		the motion of particles.

		In a solid, movement is restricted and the kinetic energy of particles can only have so many permutations. Entropy is the smallest in the
		solid state.

		In a gas, movement is almost unrestricted, and the possible range of kinetic energies for particles within the system is huge. Thus,
		entropy is the greatest. The difference in entropy between a gas and a liquid is much greater than that between a solid and a liquid.


		\subsection{Change in Entropy (\entr{})}

			The entropy change of a given reaction or action is determined mainly by the effect on the number of ways to distribute energy in the
			system. The ways in which entropy is changed include changing the \emph{temperature}, \emph{number of moles of gas}, or the
			\emph{state} of one or more compounds present in the system.

			Throughout this terrible, tedious chapter, it is important to note that
			S ≠ \entr{}. The units of \entr{} are typically \si{\joule\per\mole} (joules per mole).

			\pagebreak
			\subsubsection{Change in Temperature}

				Increasing the temperature of a system of molecules increases their average kinetic energy, and hence the range of possible
				energies within the system.

				As such, the amount of ways to distribute energy in the system increases, resulting in greater entropy. (\entr{} > 0)

				% \imgdiagram{120mm}{../figures/physical/ch02/maxwell_boltzmann_multiple_temps.png}{\vspace{-2em}}

				\begin{center}
				\begin{tikzpicture}

					\def\kB{1.3806488e-23}			% boltzmann constant
					\def\tmpone{300}				% room temperature
					\def\tmptwo{600}
					\def\tmpthr{1000}
					\def\Betaone{1/(\kB*\tmpone)}
					\def\Betatwo{1/(\kB*\tmptwo)}
					\def\Betathr{1/(\kB*\tmpthr)}
					\def\amu{1.660538921e-27}		% atomar mass unit in kg
					\def\mass{0.9*\amu}

					\begin{axis}[
						axis lines		= left,
						domain			= 0:12000,
						xlabel			= \textbf{$E_{k}(x)$},
						ylabel			= \textbf{$N(x)$},
						xtick			= \empty,
						ytick			= \empty,
						axis line style	= very thick,
						ymin			= 0,
						ymax			= 0.0004
					]

						\addplot[color = black, name path = fn, very thick]{
							sqrt(2/pi)*(\mass*\Betaone)^(3/2)*x^2*exp(-.5*\mass*\Betaone*x^2)
						};

						\addplot[color = black, name path = fn, very thick]{
							sqrt(2/pi)*(\mass*\Betatwo)^(3/2)*x^2*exp(-.5*\mass*\Betatwo*x^2)
						};

						\addplot[color = black, name path = fn, very thick]{
							sqrt(2/pi)*(\mass*\Betathr)^(3/2)*x^2*exp(-.5*\mass*\Betathr*x^2)
						};

						\node at (axis cs: 4000,0.00032) {300K};
						\node at (axis cs: 5250,0.00023) {600K};
						\node at (axis cs: 8500,0.00012) {1000K};

					\end{axis}

				\end{tikzpicture}
				\end{center}


				The Maxwell-Boltzmann curve above shows that as temperature increases, the possible range of kinetic energies of the system
				also increases, thus increasing entropy.

			% end subsubsection

			\subsubsection{Change in Number of Gas Particles}

				From the diagram above, it should be obvious that, since gaseous molecules have a much greater entropy, an increase in the number
				of gas particles will also result in an increase in the entropy of the system (\entr{} > 0).

				Simply put, if the reaction results in a greater number of gaseous products than reactants, then there will be a net
				increase in entropy. For example:

				\txtdiagram{
					\schemestart[0,1.0,thick]
						\ch{2 NaN3 \stS}
						\arrow
						\ch{2 Na \stS}\hspace{2mm} + \hspace{2mm}\ch{3 N2 \stG}
					\schemestop
				}{\vspace{-2em}}

				Here, the number of moles of gas \emph{increased} from 0 to 3, hence \entr{} > 0. The reverse is also true, naturally --- if the
				number of moles of gas decreases, then the entropy change will be negative.

			% end subsubsection

			\pagebreak
			\subsubsection{Change in Phase}

				When a particular substance melts, boils, condenses or freezes, the entropy of the system changes accordingly.
				The diagram below aptly illustrates this.

				\imgdiagram{100mm}{../figures/physical/ch02/entropy_phase_change.png}{\vspace{-2em}}

				As a substance melts, its molecules are no longer constrained by the structure of the solid state; average kinetic energy increases,
				and \entr{} > 0. As the substance boils, the molecules are completely free to move; naturally, the change in entropy is much greater
				than for melting. \entr{} > 0, obviously.

			% end subsubsection

			\subsubsection{Change in Volume}

				If the available volume of a gas increases (at a constant temperature, which implies a decrease in pressure), then the entropy
				of the system will also increase. The possible distributions of each molecule of gas increases, and hence entropy increases
				(\entr{} > 0). Naturally, the opposite is also true; if the pressure of a system increases at a constant temperature, then the entropy
				will decrease (\entr{} < 0).

				Note that when two containers containing two different gases are mixed, the entropy of the system also increases --- the total
				volume that each gas occupies is increased as well.

			% end subsubsection


			\pagebreak
			\subsubsection{Dissolution of Ionic Solids}

				When solid ionic compounds are dissolved in water (or another polar solvent), the total entropy of the system will also change.
				However, it depends on the interposition of two separate factors; when the ionic compound is dissolved, the constituent ions are
				able to move around, thus increasing the entropy.

				However, the water molecules that are interacting with the dissolved ions are now restricted in movement, and hence entropy decreases
				as well. Thus, the net change in entropy of the system depends on which of the above factors is more significant. For singly-charged
				ions, the net change in entropy is usually an increase (\entr{} > 0).

			% end subsubsection

		% end subsection


		\subsection{Gibbs Free Energy}

			Since the enthalpy change (\enth{}) and the change in entropy (\entr{}) are both crucial in determining whether a given reaction will
			be spontaneous (occurring without external energy input), they are used together to calculate the change in Gibbs free energy, \gibb{}.
			If you paid attention to the cover page, then this equation should be familiar to you:

			\txtdiagram{
				\schemestart[0,1.0,thick]
					$\Delta G = \Delta H - T\times \Delta S$
				\schemestop
			}{\vspace{-2em}}

			In short, the change in Gibbs energy of a given reaction is determined by the temperature, and the changes in both enthalpy and entropy,
			of the reaction. The units of ∆G are typically \si{\kilo\joule\per\mole} (kilojoules per mole), and that for temperature is \si{\kelvin}
			(kelvins).

			Exergonic reactions, where ∆G < 0, are said to be energetically feasible, and take place spontaneously. Conversely, endergonic
			reactions where \gibb{} > 0 are energetically infeasible, and do not take place spontaneously. \gibb{} can be 0, for example during
			melting or boiling.


			\pagebreak
			\subsubsection{Effect of Gibbs Free Energy}

				Given the sign (positive or negative, duh) of both \enth{} and \entr{}, it is possible to determine if \gibb{} will increase
				or decrease with temperature. Thus, it is possible to predict if the reaction will be more or less spontaneous
				(and thus energetically feasible) with a change in temperature.

				A \emph{disgusting} and \emph{abhorrent} table below summarises the four possible scenarios:

				\begin{center}\begin{table}[htb]\renewcommand{\arraystretch}{1.5}
				\begin{tabu} to \textwidth {| X[c,m] | X[c,m] | X[c,m] | X[c,m] |}

					\hline		\enth{}		&	\entr{}		&	\gibb{}		&		Feasibility			\\

					\hline		< 0			&	< 0			&	always < 0	&		All temperatures	\\
					\hline		< 0			&	> 0			&	always > 0	&				Never		\\
					\hline		> 0			&	< 0			&	depends		&		Low temperatures	\\
					\hline		> 0			&	> 0			&	depends		&		High temperatures	\\
					\hline

				\end{tabu}
				\end{table}\end{center}\vspace{-10mm}

			% end subsubsection


		% end subsection


	% end section

% end part















































