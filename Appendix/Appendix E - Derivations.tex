% Appendix D - Distinguishing Tests.tex
% Copyright (c) 2014 - 2016, zhiayang@gmail.com
% Licensed under the Apache License Version 2.0.

\pagebreak
\hypertarget{AppendixDerivations}{}
\part{Derivations}

	This appendix section details the derivations of various formulas that have appeared in the main body.

	\section{Equilibrium Constant}

		For a given equation:

		\txtdiagram{
			\schemestart[0,1.0,thick]
				\ch{p}\hspace{0.1em}\ch{A}\hspace{2mm} + \hspace{2mm}\ch{q}\hspace{0.1em}\ch{B}
				\arrow
				\ch{r}\hspace{0.1em}\ch{C}\hspace{2mm} + \hspace{2mm}\ch{s}\hspace{0.1em}\ch{D}
			\schemestop
		}{}

		The equilibrium constant is defined as such:

		\eqndiagram{
			 \[ K_{c} = \frac{[C]^{c}[D]^{d}}{[A]^{a}[B]^{b}} \]
		}

		Recall that the system reaches a position of equilibrium when the forward and backward rates of reaction are equal. There are two
		scenarios to consider in this case --- when the given equilibrium reaction is an elementary step, and when it is not.

		\subsection{Equilibrium is Elementary}

			The equilibrium of \ch{N2O4} and \ch{NO2} will be studied as an example in this section.

			\txtdiagram{
				\schemestart[0,1.0,thick]
					\ch{N2O4 \stG}
					\arrow{<=>}
					\ch{2 NO2 \stG}
				\schemestop
			}{}

			If the main reaction is an elementary reaction, it implies that both the forward and the backward reactions are elementary. Hence,
			there are two rate constants --- one for the forward reaction, $k_{fwd}$, and one for the reverse $k_{rev}$.

			\eqndiagram{
				\[ R_{fwd} = k_{fwd}[\mathrm{N_{2}O_{4}}]  \qquad\qquad  R_{rev} = k_{rev}[\mathrm{NO_{2}}]^{2} \]
			}[Since both reactions are elementary, the orders of reaction correspond to the stoichiometric coefficients.]

			\pagebreak

			Further recall that the equilibium constant reflects the position of equilibrium of the system (in fixed conditions) --- if its magnitude
			is large, the equilibrium favours the forward reaction, and vice versa.

			Thus, the following relationship can be applied:

			\eqndiagram{
				\[ K_{c} = \frac{k_{fwd}}{k_{rev}} \]
			}

			The following equation can be found by slightly rearranging the two rate equations above and appropriately substituting:

			\eqndiagram{
				\[ K_{c} \quad = \quad \frac{k_{fwd}}{k_{rev}} \quad = \quad \frac{R_{fwd}}{[\mathrm{N_{2}O_{4}}]}
				\; / \; \frac{R_{rev}}{[\mathrm{NO_{2}}]^{2}} \]
			}

			Since the forward and reverse rates of reaction are the same, they can be cancelled, getting the definition of the equilibrium constant:

			\eqndiagram{
				\[ K_{c} = \frac{[\mathrm{NO_{2}}]^{2}}{[\mathrm{N_{2}O_{4}}]} \]
			}

			This also indirectly explains why the equilbrium constant takes the product concentrations over the reactant concentrations, instead of
			the other way around.

		% end subsection

		\subsection{Equilibrium is not Elementary}

			Even if the equilibrium itself is not elementary, the original definition of \Kc{} still applies --- there are simply more substitution
			steps to get to it.

			Taking the following equilibrium:

			\txtdiagram{
				\schemestart[0,1.0,thick]
					\ch{H2 \stG} \hspace{2mm}+\hspace{2mm} \ch{I2 \stG}
					\arrow{<=>}
					\ch{2 HI \stG}
				\schemestop
			}{}

		% end subsection

	% end section


% end part

































