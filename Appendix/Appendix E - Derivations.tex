% Appendix D - Distinguishing Tests.tex
% Copyright (c) 2014 - 2016, zhiayang@gmail.com
% Licensed under the Apache License Version 2.0.

\pagebreak
\hypertarget{AppendixDerivations}{}
\part{Derivations}

	This appendix section details the derivations of various formulas that have appeared in the main body.

	\section{Equilibrium Constant}

		For a given equation:

		\txtdiagram{
			\schemestart[0,1.0,thick]
				\ch{p}\hspace{0.1em}\ch{A}\hspace{2mm} + \hspace{2mm}\ch{q}\hspace{0.1em}\ch{B}
				\arrow
				\ch{r}\hspace{0.1em}\ch{C}\hspace{2mm} + \hspace{2mm}\ch{s}\hspace{0.1em}\ch{D}
			\schemestop
		}{}

		The equilibrium constant is defined as such:

		\eqndiagram{
			 \[ K_{c} = \frac{[\ch{C}]^{c}[\ch{D}]^{d}}{[\ch{A}]^{a}[\ch{B}]^{b}} \]
		}

		Recall that the system reaches a position of equilibrium when the forward and backward rates of reaction are equal. There are two
		scenarios to consider in this case --- when the given equilibrium reaction is an elementary step, and when it is not.

		\subsection{Elementary Equilibrium}

			The equilibrium of \ch{N2O4} and \ch{NO2} will be studied as an example in this section.

			\txtdiagram{
				\schemestart[0,1.0,thick]
					\ch{N2O4 \stG}
					\arrow{<=>}
					\ch{2 NO2 \stG}
				\schemestop
			}{}

			If the main reaction is an elementary reaction, it implies that both the forward and the backward reactions are elementary. Hence,
			there are two rate constants --- one for the forward reaction, $k_{fwd}$, and one for the reverse $k_{rev}$.

			\eqndiagram{
				\[ R_{fwd} = k_{fwd}[\ch{N2O4}]  \qquad\qquad  R_{rev} = k_{rev}[\ch{NO2}]^{2} \]
			}[Since both reactions are elementary, the orders of reaction correspond to the stoichiometric coefficients.]

			\pagebreak

			Further recall that the equilibium constant reflects the position of equilibrium of the system (in fixed conditions) --- if its magnitude
			is large, the equilibrium favours the forward reaction, and vice versa.

			Thus, the following relationship can be applied:

			\eqndiagram{
				\[ K_{c} = \frac{k_{fwd}}{k_{rev}} \]
			}

			The following equation can be found by slightly rearranging the two rate equations above and appropriately substituting:

			\eqndiagram{
				\[ K_{c} \quad = \quad \frac{k_{fwd}}{k_{rev}} \quad = \quad \frac{R_{fwd}}{[\ch{N2O4}]}
				\; / \; \frac{R_{rev}}{[\ch{NO2}]^{2}} \]
			}

			Since the forward and reverse rates of reaction are the same, they can be cancelled, getting the definition of the equilibrium constant:

			\eqndiagram{
				\[ K_{c} = \frac{[\ch{NO2}]^{2}}{[\ch{N2O4}]} \]
			}

			This also indirectly explains why the equilbrium constant takes the product concentrations over the reactant concentrations, instead of
			the other way around.

		% end subsection

		\pagebreak
		\subsection{Non-Elementary Equilibrium}

			Even if the equilibrium itself is not elementary, the original definition of \Kc{} still applies --- there are simply more substitution
			steps to get to it.

			Taking the following equilibrium, along with its 3 constituent steps:

			\txtdiagram{
				%? we want to align the arrows below, but we also want to draw a horizontal line to separate the reaction
				%? from the steps. 1.125em is the correct distance (to roughly ~0.1mm) to offset the reaction by.
				\schemestart[0,1.0,thick]
					\ch{H2 \stG} \hspace{2mm}+\hspace{2mm} \ch{I2 \stG} \arrow{<=>} \ch{2 HI \stG}\hspace{1.125em}
				\schemestop

				\rule{80mm}{0.3mm}\vspace{3mm}

				\schemestart[0,1.0,thick]
					\ch{I2} \arrow{<=>} \ch{2 I}
					\arrow(@c1.south east--.north east){0}[-90,.2]
					\ch{I} \hspace{2mm}+\hspace{2mm} \ch{H2} \arrow{<=>} \ch{H2I}
					\arrow(@c3.south east--.north east){0}[-90,.2]
					\ch{H2I} \hspace{2mm}+\hspace{2mm} \ch{I} \arrow{<=>} \ch{2 HI}
				\schemestop
			}{}

			The steps have the following equilibrium constants, along with the overall equilibrium constant:
			\eqndiagram{
				\[
					K_{1} = \frac{[\ch{I}]^{2}}{[\ch{I_{2}}]} \qquad
					K_{2} = \frac{[\ch{H2I}]}{[\ch{I}][\ch{H2}]} \qquad
					K_{3} = \frac{[\ch{HI}]^{2}}{[\ch{I}][\ch{H2I}]} \qquad
					K_{c} = \frac{[\ch{HI}]^{2}}{[\ch{H_{2}}][\ch{I2}]}
				\]
			}

			Working backwards from the final equation, a number of substitutions can be performed:

			\eqndiagram{
				\begin{alignat*}{3}
					K_{c}           & = \frac{[\ch{HI}^2]}{[\ch{I2}][\ch{H2}]}                              \hspace{20mm}%
										&&\Longleftarrow\quad [\ch{HI}]^{2} = K_{3}[\ch{I}][\ch{H2I}]       \\[0.6em]
									& = \frac{K_{3}[\ch{I}][\ch{H2I}]}{[\ch{I_{2}}][\ch{H2}]}               \hspace{20mm}%
										&&\Longleftarrow\quad [\ch{H2I}] = K_{2}[\ch{I}][\ch{H2}]           \\[0.6em]
									& = \frac{K_{3}K_{2}[\ch{I}]^{2}}{[\ch{I2}]}                            \hspace{20mm}%
										&&\Longleftarrow\quad [\ch{I}]^{2} = K_{1}[\ch{I2}]                 \\[0.6em]
									& = \frac{K_{3}K_{2}K_{1}[\ch{I2}]}{[\ch{I2}]} = K_{1}K_{2}K_{3}        \\[0.6em]
					K_{1}K_{2}K_{3} & = \frac{[\ch{I}]^{2}}{[\ch{I_{2}}]} \times \frac{\cancel{[\ch{H2I}]}}{[\ch{I}][\ch{H2}]} \times%
										\frac{[\ch{HI}]^{2}}{[\ch{I}]\cancel{[\ch{H2I}]}}%
										&& = \frac{\cancel{[\ch{I}]^{2}}}{[\ch{I_{2}}]} \times \frac{1}{\cancel{[\ch{I}]}[\ch{H2}]} \times%
										\frac{[\ch{HI}]^{2}}{\cancel{[\ch{I}]}}                             \\[0.6em]
									& = \frac{[\ch{HI}]^{2}}{[\ch{H_{2}}][\ch{I2}]} = K_{c}
				\end{alignat*}
			}

			Et voila. While this \enquote{proof} might seem suspiciously circular, the starting point has to either assume that
			$K_{c} = K_{1}K_{2}K_{3}$, or assume that the intial definition of $K_{c}$ is true. The former theorem can be shown using
			the latter, ie. the overall equilibrium constant of a series of sequential equilibria is the product of the individual equilibrium
			constants.


		% end subsection

	% end section


% end part

































