% Chapter07 - Alkenes.tex
% Copyright (c) 2014 - 2016, zhiayang@gmail.com


\pagebreak
\section{Arenes}

\subsection{Benzene}

	Benzene is simplest possible aromatic compound, or arene. The first known and isolated arene compounds had pleasant smells,
	such as benzaldehyde. Unfortunately, even though most known arenes now smell terrible and are carcinogenic, the name stuck.

	\diagram[1.0]{
		\chemfig{**6(------)}
	}{The structural representation of benzene.}

	The most common form of arene is the benzene ring, or phenyl functional group. Note that aromatic rings with other configurations
	and structures can also form, such as with nitrogen.

	Its basic structure consists of 6 carbon atoms, arranged in a hexagonal fashion. However, unlike cyclohexane, benzene is a flat,
	planar molecule. All 6 carbon atoms are sp\sps{2} hybridised, forming the following structure:

	\imgdiagram{120mm}{figures/ch07/benzene_delocalisation.png}

	The trigonal structure of the sp\sps{2} hybrid orbitals dictates this structure, since the angle between each orbital is \ang{120},
	which is the internal angle of a regular hexagon. Note that there is one \ch{H} atom bonded to each carbon, making the molecular
	formula of benzene \ch{C6H6}.

	The carbon atoms are bonded to each other through π-bonds with their unhybridised p-orbitals, while the bonds with the hydrogen
	atoms (or other substituents if substituted) are done with the sp\sps{2} orbitals.

	Furthermore, the 6 π-bond electrons are delocalised, and move freely within the ring. This increases the stability of the benzene
	ring, which increases the amount of energy needed to modify it by fiddling with bonds.


	\subsubsection{Physical Properties}

		Since benzene is a regular hexagon, each \ch{C=C} bond is the same length. However, it shorter than a \ch{C-C} bond, but longer
		than a \ch{C+C} bond.

		Benzene is a volatile, flammable and carcinogenic. Don't drink it. Don't eat it. Don't touch it. It also happens to be colourless,
		with a distinct "aromatic" odour.

		As a non-polar molecule, it has relatively low melting and boiling points (\SI{5.5}{\celsius} and \SI{80.1}{\celsius}), as well as low
		solubility in water. Of course, it is soluble in non-polar solvents, and in fact can be used as a solvent in its own right.

	% end subsubsection


	\subsubsection{Chemical Reactions}

		Benzene undergoes substitution reactions rather than addition reactions, since adding atoms to the ring would destroy the
		delocalised π-system of the ring, which is energetically unfavourable. Instead, the \ch{H} atoms on the ring are substituted.

		The main mechanism for this is electrophilic substitution, as seen below.

	% end subsubsection

% end subsection


% UN FUCK THE FORMATTING
\pagebreak
\subsection{Electrophilic Substitution}

	The delocalised π-system of benzene has a very high electron density, and thus is a prime target for electrophiles, which
	will substitute the \ch{H} atoms on the ring. Thus, the most common form of reaction involving benzenes is electrophilic
	substitution, barring special conditions and requirements.


	\vspace{1.0em}
	\subsubtext{Step 1}

	In the first, rate-determining step, the aromaticity of the benzene ring is partially and temporarily broken, disrupted by
	the attacking electrophile.

	\diagram[1.0]{
		\schemestart[0,1.5,thick]
			\chemfig{**6(---@{ring}---)}
			\arrow{0}[,0]		% used for alignment
			\hspace{5mm} + \hspace{5mm}
			\chemfig{@{el}E}
			\arrow(--.base west){->[slow]}
			\chemfig[yshift=\the\dimexpr-1.5em\relax]{**[60,-240]6(----(-[:135]H)(-[:45]E)--)(-[:30,,,,draw=none]+)}
		\schemestop

		\chemmove{\draw[-Stealth,line width=0.4mm,shorten <=-4mm,shorten >=1mm](ring) .. controls +(40:8mm) and +(120:8mm) .. (el);}

	}{Note that the '+' is drawn in the centre of the ring, not on any one carbon.}


	Two electrons out of six from the delocalised π-system are used to form the bond between the electrophile, E, and the carbon.
	Thus, there is a positive charge on the carbon; due to the delocalised nature of the π-system however, this positive charge is
	delocalised across \textit{all 6 carbons}, making it much more stable than a simple carbocation.

	However, the activation energy for this step is still large, and only strong electrophiles are able to attack the benzene ring without catalysts.


	\vspace{1.0em}
	\subsubtext{Step 2}

	Next, a nucleophile (\chlewis{180}{Nu-} in this example) attacks the hydrogen attached to the hydrogen on the carbon atom,
	restoring the aromaticity of the benzene ring. The new substituent is now in place, and the two electrons in the \ch{C-H}
	bond are returned to the π-system.


	\diagram[1.0]{
		\schemestart[0, 1.5, thick]
			\chemfig[yshift=-1.5em]{**[60,-240]6(----(-[@{bond}:135]@{hyd}H)(-[:45]E)--)(-[:30,,,,draw=none]@{pl}+)}
			\arrow(.base east--){0}[,0]		% alignment purposes
			\hspace{5mm} + \hspace{5mm}
			\chemfig{@{nu}\lewis{2:,N}|u\mch}
			\arrow(--.mid west){->[fast]}
			\chemfig{\ch{HNu}}
			\hspace{5mm} + \hspace{5mm}
			\chemfig[yshift=-1.5em]{**6(----(-[:90]E)--)}
		\schemestop

		\chemmove{\draw[-Stealth,line width=0.4mm,shorten <=2mm,shorten >=1mm](nu) .. controls +(90:15mm) and +(45:15mm) .. (hyd);}
		\chemmove{\draw[-Stealth,line width=0.4mm,shorten <=1mm,shorten >=1mm](bond) .. controls +(225:8mm) and +(90:8mm) .. (pl);}
	}


	Note that the arenium ion (which is the partially delocalised benzene) has 5 sp\sps{2} carbons, and one sp\sps{3} carbon. This
	results in a disruption of the planar structure of benzene –– it is restored once the substitution is completed.

% end subsection

\subsection{Nitration of Benzene}

	The nitration of benzene involves the substitution of one of the \ch{H} atoms on the benzene with a nitro (\ch{-NO2}) group.
	It has a number of specific requirements:

	\vspace{1.5em}
	\vbox{\textbf{Conditions:}	\tabto{35mm}Concentrated \ch{HNO3}, concentrated \ch{H2SO4} catalyst.
								\tabto{35mm}\textit{Constant} temperature of \SI{50}{\celsius}.}

	\vbox{\textbf{Observations:}\tabto{35mm}\textit{\color{Goldenrod}Pale yellow} oily liquid, nitrobenzene.}


	\vspace{1.5em}
	\subsubtext{\vbox{IV.\tabto{10mm}A New Electrophile}}

	Since \ch{H2SO4} is a stronger acid than \ch{HNO3}, it donates a proton to \ch{HNO3}, forming \ch{H2O}, \ch{HSO4-}, and \ch{NO2+}, the
	electrophile. Next, another molecule of \ch{H2SO4} then forms \ch{H+ \aq}, or \ch{H3O+}. The overall equation is as such:

	\diagram{
		\schemestart[0, 1.0, thick]
			\ch{2 H2SO4} + \ch{HNO3}
			\arrow
			\ch{NO2+} + \ch{H3O+} + \ch{2 HSO4-}
		\schemestop
	}{The catalyst \ch{H2SO4} is restored in a later step.}


	\vspace{2.0em}
	\subsubtext{\vbox{V.\tabto{10mm}The π-Electrons Strike Back}}

	Now that the electrophile \ch{NO2+} has been formed, it is attacked by the π-system. As with all electrophilic substitutions, this
	involves the breaking of the aromatic system, and is the slow step. The mechanism follows the general mechanism outlined above.

	\diagram[1.0]{
		\schemestart[0, 1.5, thick]
		\chemfig{**6(---@{ring}---)}
		\arrow{0}[,0]		% used for alignment
		\hspace{5mm} + \hspace{5mm}
		\chemfig{@{el}\ch{NO2+}}
		\arrow(--.base west){->[slow]}
		\chemfig[yshift=\the\dimexpr-1.5em\relax]{**[60,-240]6(----(-[:135]H)(-[:45]\ch{NO2})--)(-[:30,,,,draw=none]+)}
		\schemestop

		\chemmove{\draw[-Stealth,line width=0.4mm,shorten <=-4mm,shorten >=1mm](ring) .. controls +(40:8mm) and +(120:8mm) .. (el);}
	}




	% \vspace{2.0em}
	\pagebreak
	\subsubtext{\vbox{VI.\tabto{10mm}Return of the Aromaticity}}

	The \ch{HSO4-} intermediate acts as a nucleophile and attacks the \ch{H} atom bonded to the benzene intermediate. This
	restores both the π-system of the benzene ring, as well as the catalyst, \ch{H2SO4}.


	% TODO: fix alignment (more?)
	\diagram[0.85]{
		\schemestart[0, 1.5, thick]
		\chemfig{!{S}(=[:180]!{O})(=[:270]!{O})(-[:90]@{oxy}\lewis{2:,\color{Red}O}|\mch)(-[:0]!{OH})}
		\hspace{5mm} + \hspace{5mm}
		\chemfig[yshift=-1.5em]{**[60,-240]6(----(-[@{bond}:135]@{hyd}H)(-[:45]\ch{NO2})--)(-[:30,,,,draw=none]@{plus}+)}
		\arrow(.mid east--.mid west){->[fast]}
		\chemfig[yshift=-1.5em]{**6(----(-[:90]\ch{NO2})--)}
		\hspace{5mm} + \hspace{5mm}
		\chemfig{!{S}(=[:180]!{O})(=[:270]!{O})(-[:90]!{OH})(-[:0]!{OH})}
		\schemestop

		\chemmove{\draw[-Stealth,line width=0.4mm,shorten <=2mm,shorten >=1mm](oxy) .. controls +(90:15mm) and +(120:15mm) .. (hyd);}
		\chemmove{\draw[-Stealth,line width=0.4mm,shorten <=1mm,shorten >=1mm](bond) .. controls +(225:8mm) and +(90:8mm) .. (plus);}
	}



	\vspace{1.0em}
	\subsubtext{\vbox{VII.\tabto{10mm}Overall Reaction}}

	\diagram[1.0]{
		\schemestart[0, 1.5, thick]
		\chemfig{\ch{HNO3}}
		\hspace{5mm} + \hspace{5mm}
		\chemfig[yshift=-1.5em]{**6(------)}
		\arrow(.base east--.base west){->[\ch{H2SO4}][\SI{50}{\celsius}]}
		\chemfig[yshift=-1.5em]{**6(----(-[:90]\ch{NO2})--)}
		\hspace{5mm} + \hspace{5mm}
		\chemfig{\ch{H2O}}
		\schemestop
	}


	\pagebreak
	\subsection{Halogenation of Benzene}

	\lipsum[2-4]



% end section




