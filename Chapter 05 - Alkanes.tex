% Chapter05 - Alkanes.tex
% Copyright (c) 2014 - 2016, zhiayang@gmail.com

% Preamble.tex
% Copyright (c) 2014 - 2016, zhiayang@gmail.com

\setlength{\parindent}{0pt}
\setlength{\parskip}{\baselineskip}

\setlist{nosep}


\pagebreak
\section{Alkanes}

\subsection{Open Chain}

	Open chain alkanes have the general formula of \ce{C\sbs{n}H\sbs{2n+2}}. They are called 'open-chain' because
	the two endsof the chain are separate, in contrast with cycloalkanes.

	Open chain alkanes can either be straight-chained or branch-chained. The terminology should be pretty much
	self explanatory.

	\diagram{

		\chemfig{!{H3C}-[:330]-[:30]-[:330]-[:30]-[:330]-[:30]-[:330]!{CH3}}

	}{Octane is an example of a straight-chain alkane.}


	\diagram{

		\chemfig{C(-[:0]!{CH3})(-[:90]!{CH3})(-[:180]!{H3C})(-[:270]!{CH3})}

	}{2,2-dimethylpropane is an example of a branched alkane.}

% end subsection

\subsection{Cycloalkanes}

	Cycloalkanes are alkanes where the carbon atoms at either end of the chain are bonded together, forming a closed loop.
	They are essentially 'closed-chain' alkanes. They take the shape of regular polygons –– cyclopropane is just a triangle,
	and cyclobutane is a square.

	\diagram{

		\chemfig{*5(-----)}		\hspace{15mm}
		\chemfig{*6(------)}

	}{Cyclopentane (left) and cyclohexane (right)}

% end subsection



\subsection{Physical Properties of Alkanes}
	\subsubsection{Melting and Boiling Points}

		The melting and boiling points of alkanes follow a simple pattern. Due to the fact that they rely solely on induced
		dipole interactions for intermolecular bonding, both melting and boiling points increase with the length of the
		carbon chain, and by extension M\sbs{r}. Note that small chains (eg. \ce{CH4}) have very low boiling points.

		However, because branched alkanes have a smaller surface area for a given number of carbon atoms than straight-chained
		alkanes, their melting and boiling points will be lower, due to a smaller area for polarisation.

	% end subsubsection

	\subsubsection{Density}

		Most liquid alkanes, up to a certain point, are less dense than water. Additionally, since they are insoluble,
		they form an immiscible layer above water.

		As the number of carbon atoms increases, the strength of the intermolecular interactions will increase as well –– this
		forces each molecule ever so slightly closer together, marginally increasing density.

	% end subsubsection

	\subsubsection{Solubility}

		As described before, alkanes rely only on induced-dipole interactions for intermolecular bonding; they are insoluble in
		polar solvents such as water. However, they are highly soluble in non-polar solvents like \ce{C\chlorine4}. In fact,
		since solubility is dependent on the strength of solvent-solute bonds, larger alkanes will be more soluble in non-polar solvents.

	% end subsubsection

% end subsection

\pagebreak
\subsection{Free Radical Substitution of Alkanes}

	\subsubsection{Mechanism of Reaction}

		In the steps below, the free radical substitution of methane [\ce{CH4}\ by chlorine [\ce{\chlorine}] will be used. This applies
		for any alkane, and a sufficiently reactive halogen (Usually either \ce{\chlorine} or \ce{Br}).

		The observable change would be a decolourising of the halogen gas, as it is consumed. This is \textit{reddish-brown} for bromine,
		and \textit{yellowish-green} for chlorine. Conditions for the reaction simply include the alkane and gaseous halogen gas,
		as well as UV light.

		\subsubtext{\vspace{10mm}\vbox{\textit{Stage I}\tabto{25mm}Initiation}}

			The \ce{\chlorine}–\ce{\chlorine} bond is \textit{homolytically} broken to form 2 \ce{\chlorine} radicals.
			The energy required to break this bond is provided by the UV light.

			\diagram{
				\schemestart[0, 1.5, thick]
				\chemfig{@{cl1}{\color{OliveGreen}\chlorine}-[@{b}:0]@{cl2}{\color{OliveGreen}\chlorine}}
				\arrow{->}
				\chemfig{2 \lewis{0.,\color{OliveGreen}\chlorine}}
				\schemestop

				\chemmove{\draw[-{Stealth[left]},line width=0.4mm,shorten <=1mm,shorten >=1mm](b).. controls +(90:7mm) and +(90:7mm).. (cl2);}
				\chemmove{\draw[-{Stealth[left]},line width=0.4mm,shorten <=1mm,shorten >=1mm](b).. controls +(270:7mm) and +(270:7mm).. (cl1);}
			}

		% end

		\subsubtext{\vspace{10mm}\vbox{\textit{Stage II}\tabto{25mm}Propagation}}

			The highly reactive \ce{\chlorine} radicals then react with the \ce{CH4} molecules, bonding with one of the
			hydrogen atoms to form \ce{H\chlorine} and a carbocation radical, \ce{CH3}.

			\diagram{
				\schemestart[0, 1.0, thick]
					\chemfig{\lewis{0.,\chlorine}} \hspace{3mm} + \hspace{3mm} \chemfig{CH\sbs{4}}
					\arrow{->}
					\chemfig{\lewis{4.,CH\sbs{3}}} \hspace{3mm} + \hspace{3mm} \chemfig{H\chlorine}
				\schemestop
			}

			\vspace{-15mm}
			\diagram{
				\schemestart[0, 1.0, thick]
					\chemfig{\ce{\chlorine2}} \hspace{3mm} + \hspace{3mm} \chemfig{\lewis{4.,CH\sbs{3}}}
					\arrow{->}
					\chemfig{\ce{CH3\chlorine}} \hspace{3mm} + \hspace{3mm} \chemfig{\lewis{0.,\chlorine}}
				\schemestop
			}

			This newly-minted \ce{CH3} radical can react with a \ce{\chlorine2} molecule to form \ce{CH3Cl} and another
			\ce{\chlorine} radical, thus recreating the consumed radical. As long as the supply of \ce{\chlorine2} gas has
			not been exhausted, this reaction can continue, and no additional UV light is required to sustain it.

		%end

		\pagebreak
		\subsubtext{\vspace{10mm}\vbox{\textit{Stage II}\tabto{25mm}Termination}}

			In the termination stage, two radicals combine to form stable products, effectively terminating the chain reaction.

			\diagram{
				\schemestart[0, 1.0, thick]
					\chemfig{\lewis{0.,\chlorine}} \hspace{3mm} + \hspace{3mm} \chemfig{\lewis{0.,\chlorine}}
					\arrow{->}
					\chemfig{\ce{\chlorine2}}
				\schemestop
			}

			\vspace{-15mm}
			\diagram{
				\schemestart[0, 1.0, thick]
					\chemfig{\lewis{4.,CH\sbs{3}}} \hspace{3mm} + \hspace{3mm} \chemfig{\lewis{4.,CH\sbs{3}}}
					\arrow{->}
					\chemfig{\ce{CH3CH3}}
				\schemestop
			}

			\vspace{-15mm}
			\diagram{
				\schemestart[0, 1.0, thick]
					\chemfig{\lewis{0.,\chlorine}} \hspace{3mm} + \hspace{3mm} \chemfig{\lewis{4.,CH\sbs{3}}}
					\arrow{->}
					\chemfig{\ce{CH3\chlorine}}
				\schemestop
			}

			Once all the radicals have been consumed in this manner, the reaction will stop, unless more UV light is provided (along
			with more \ce{\chlorine2} gas).

		%end

	% end subsubsection


	\subsubsection{Multi-substitution of Alkanes}

		Free radical substitution of alkanes is usually not the preferred way to produce halogenoalkanes, due to the random
		nature of the process and the possibility of multi-substitution, where more than one halogen atom has been substituted
		onto the alkane.

		If there are still \ce{\chlorine} radicals remaining, in the reaction chamber, \ce{CH3\chlorine} formed during the
		termination stage can still react with it, forming a \ce{CH2\chlorine} radical, which can continue to react.

		\diagram{
			\schemestart[0, 1.0, thick]
				\chemfig{\ce{CH3Cl}} \hspace{3mm} + \hspace{3mm} \chemfig{\lewis{0.,\chlorine}}
				\arrow{->}
				\chemfig{\lewis{4.,\ce{CH2\chlorine}}}
			\schemestop
		}

		\vspace{-15mm}
		\diagram{
			\schemestart[0, 1.0, thick]
				\chemfig{\lewis{0.,\ce{CH2\chlorine}}} \hspace{3mm} + \hspace{3mm} \chemfig{\ce{\chlorine2}}
				\arrow{->}
				\chemfig{\lewis{4.,\ce{CH2\chlorine2}}} \hspace{3mm} + \hspace{3mm} \chemfig{\lewis{0.,\chlorine}}
			\schemestop
		}

		Indeed, this can continue \textit{ad-infinitum}, so until all the hydrogen atoms on the alkane have been substituted.
		In the case of methane, this results in the formation of \ce{CH3\chlorine}, \ce{CH2\chlorine2}, \ce{CH\chlorine3} and
		\ce{C\chlorine4}.

		In fact, the will also be a small amount of alkanes with more than 1 carbon in the chain; this is due to the
		possibility of two molecules of •\ce{CH2\chlorine} reacting, which then has its hydrogens further substituted.

	% end subsubsection










% end subsection



















% end section
