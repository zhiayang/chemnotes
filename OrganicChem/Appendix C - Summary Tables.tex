% Appendix C - Summary Tables.tex
% Copyright (c) 2014 - 2016, zhiayang@gmail.com
% Licensed under the Apache License Version 2.0.

\pagebreak
\hypertarget{AppendixSummaryTables}{}
\part{Summary Tables}

	\section{Nature of Substituent Groups}

		The activating and deactivating nature of substituent groups in this table only applies when attached to aromatic rings.

		\begin{center}\begin{table}[htb]\renewcommand{\arraystretch}{1.5}
		\begin{tabu} to \textwidth {| X[-4,c,m] | X[c,m] | X[c,m] |}

			\hline
						Substituent						&	Electron Effect			&	Directing Effect	\\	\hline
			Alkyl/aryl groups (eg. \ch{-CH3})			&	Weakly Activating		&	2,4-directing		\\	\hline
			\ch{-OH}, \ch{-NH2}, \ch{-OCH3}				&	Strongly Activating		&	2,4-directing		\\	\hline
			\ch{-\Cl}, \ch{-Br}					&	Weakly Deactivating		&	2,4-directing		\\	\hline
			\ch{-CHO}, \ch{-NO2}, \ch{-CN}, \ch{-CO2H}	&	Strongly Deactivating	&	3-directing			\\	\hline

		\end{tabu}
		\end{table}\end{center}\vspace{-10mm}



		For the electron-manipulating nature when attached to carbon chains, only the electronegativity of the atom should
		be considered --- \ch{F}, \ch{O}, \ch{N}, and \ch{\Cl}, since there are no overlapping p-orbitals.

	% end section




	\hypertarget{AppendixReducingAgents}{}
	\section{List and Uses of Reducing Agents}

		The strength of the 3 common reducing agents used, \ch{Li\Al4}, \ch{NaBH4} and good old \ch{H2 \stG} with \ch{Ni}
		catalyst, can be defined in terms of their ability to reduce the various functional groups.

		\begin{center}\begin{table}[htb]\renewcommand{\arraystretch}{1.5}
		\begin{tabu} to \textwidth {| X[c,m] | X[c,m] | X[c,m] | X[c,m] |}

			\hline
						Group						&	\ch{Li\Al4}	&	\ch{NaBH4}	&	\ch{H2}, \ch{Pt} catalyst	\\	\hline
\vspace{2mm}		Alkenes (\ch{C=C})				\vspace{2mm}&			No			&		No		&				Yes				\\	\hline
\vspace{2mm}	Aldehydes and Ketones (\ch{C=O})	\vspace{2mm}&			Yes			&		Yes		&				Yes				\\	\hline
\vspace{2mm}	Carboxylic Acids (\ch{CO2H})		\vspace{2mm}&			Yes			&		No		&				No				\\	\hline
\vspace{2mm}		Nitriles (\ch{C+N})				\vspace{2mm}&			Yes			&		No		&				Yes				\\	\hline

		\end{tabu}
		\end{table}\end{center}\vspace{-10mm}

	% end section



	\pagebreak
	\section{Acidic Reactions of Organic Acids}

		Alcohols, Phenols and Carboxylic Acids are weak organic acids (weak acids in the sense that they dissociate only partially), with
		reactivity in increasing order.

		\begin{center}\begin{table}[htb]\renewcommand{\arraystretch}{1.5}
		\begin{tabu} to \textwidth {| X[-4,c,m] | X[c,m] | X[c,m] | X[c,m] |}

			\hline
						Acid Group				&	\vspace{2mm}Reactive Metals (\ch{Na})\vspace{2mm}	&	Bases (\ch{NaOH})	&	Carbonates (\ch{Na2CO3})\\	\hline
					Alcohols (\ch{OH})			&			Yes					&			No			&			No				\\	\hline
					Phenols (\ch{OH})			&			Yes					&			Yes			&			No				\\	\hline
				Carboxylic Acids (\ch{CO2H})	&			Yes					&			Yes			&			Yes				\\	\hline

		\end{tabu}
		\end{table}\end{center}\vspace{-10mm}


	% end section

% end part


























