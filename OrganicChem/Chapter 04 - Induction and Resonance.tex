% Chapter 04 - Induction and Resonance.tex
% Copyright (c) 2014 - 2016, zhiayang@gmail.com


\pagebreak
\hypertarget{ChapterInductionAndResonance}{}
\part{Induction and Resonance}

	\section{Inductive Effect}

		The inductive effect occurs through covalent bonds, where there is a significant difference in the electronegativity of
		participating atoms. Electrons are either withdrawn or donated through a \chemsigma-bond, due to the polarity of the molecule.

		\subsection{Withdrawal}

			Below, \ch{\chlorine} is more electronegative than the carbon it is bonded to. As such, it
			\textit{inductively withdraws} electrons through the \chemsigma-bond.

			\diagram{
				\chemfig{C(-[:0,,,,-Stealth]!\molCl)(-[:90]!\molStar)(-[:180]!\molStar)(-[:270]!\molStar)}

			}{Note that the arrow represents a withdrawal of electrons, \textit{not} a dative bond.}

		% end subsubsection


		\subsection{Donation}

			Alkyl groups, or groups with the general formula \ch{C_nH_{2n+1}}, \textit{inductively donate} electrons. This behaviour
			is due to hyperconjugation, which lowers the total energy of the system, through an interaction between electrons in a
			\chemsigma-bond of the alkyl group, with a partially-filled or empty p-orbital in the adjacent atom.


			\diagram[1.0]{
				\chemfig{C(-[:0,,,,Stealth-]!\molMe)(-[:90]!\molStar)(-[:180]!\molStar)(-[:270]!\molStar)}

			}{Again, the arrow \textit{does not} represent a dative bond.}

		% end subsection
	% end section

	\pagebreak
	\section{Resonance Effect}

		The resonance effect is the withdrawal or donation of electrons through the side-on overlap of \textit{unhybridised} p-orbitals.
		Thus, the resonance effect can only occur when the central atom is \spone or \sptwo hybridised, since only those
		configurations have unhybridised p-orbitals.


		\subsection{Withdrawal}

			In the case below, electrons flow from the double-bonds to the substituent, via the resonance effect. In general, substituents
			that exhibit an electron-withdrawing resonance effect usually take the form of \ch{–Y=Z}, where Z is more electronegative than Y.
			Examples include carbonyls and nitriles.

			\diagram{
				\schemestart[0, 1.5, thick]
				\chemfig{C(-[:180]H)(=[@{b1}:270]@{oxy}{{\color{Red}O}})(-[@{b2}:0]C(-[:90]H)(=[@{b3}:0]C(-[:45]H)(-[:315]H)))}
				\arrow{->}
				\chemfig{C(-[:180]H)(-[:270]!\molO|\sps{-})(-[:0]C(-[:90]H)(=[:0]C|\sps{+}(-[:45,,1]H)(-[:315,,1]H)))}
				\schemestop

				% draw arrows
				\chemmove{\draw[-Stealth,line width=0.4mm,shorten <=2mm,shorten >=1mm](b3).. controls +(90:8mm) and +(90:8mm).. (b2);}
				\chemmove{\draw[-Stealth,line width=0.4mm,shorten <=2mm,shorten >=1mm](b1).. controls +(180:8mm) and +(180:8mm).. (oxy);}

			}{Electrons move from the electron-rich double-bonds, which have unhybridised p-orbitals, to an adjacent atom.}

		% end subection


		\subsection{Donation}

			On the other hand, groups can donate electrons through resonance, flowing from the substituent to a single-bond, forming
			a double-bond. Substituents such as halogens (\ch{F}, \ch{\chlorine}, etc.), hydroxyls (\ch{-OH}), and amines (\ch{-NH2})
			are examples. They usually contain a lone pair of electrons that are free for donation.


			\diagram{
				\schemestart[0, 1.5, thick]
				\chemfig{C(-[:225]H)(-[@{b1}:90]@{oh}\color{Red}\lewis{4:,O}|{\color{Red}H})(=[@{b2}:315]@{oxy}{\color{Red}O})}
				\arrow{<->}
				\chemfig{C(-[:225]H)(=[:90,,,2]!\molHO|\sps{+})(-[:315]!\molO\sps{-})}
				\schemestop

				% draw arrows
				\chemmove{\draw[-Stealth,line width=0.4mm,shorten <=2mm,shorten >=1mm](oh).. controls +(180:8mm) and +(180:8mm).. (b1);}
				\chemmove{\draw[-Stealth,line width=0.4mm,shorten <=2mm,shorten >=1mm](b2).. controls +(45:8mm) and +(45:8mm).. (oxy);}

			}{This example shows both resonant withdrawal and donation.}


		% end subsection

		\pagebreak
		\subsection{Overall Effect}

			Since it is possible for a substituent group to simultaneously withdraw and donate electrons through different mechanisms,
			it can be difficult to determine whether the overall effect serves to withdraw or donate electrons.

			Behold, the second ugly table.

			\begin{center}\begin{table}[htb]\renewcommand{\arraystretch}{1.5}
			\begin{tabu} to \textwidth {| X[-4,c,m] | X[c,m] | X[c,m] |}

				\hline				Substituent Group					&	Strength	&	Overall Effect	\\

				\hline		Alkyl/aryl groups (eg. \ch{-CH3})			&	Weak		&		Donating	\\
				\hline		\ch{-OH}, \ch{-NH2}, \ch{-OCH3}				&	Strong		&		Donating 	\\
				\hline		\ch{-\chlorine}, \ch{-Br}					&	Weak		&	Withdrawing		\\
				\hline		\ch{-CHO}, \ch{-NO2}, \ch{-CN}, \ch{-CO2H}	&	Strong		&	Withdrawing		\\
				\hline

			\end{tabu}
			\end{table}\end{center}\vspace{-10mm}

			\subsubsection{Caveats}
			\hypertarget{CaveatResonanceTable}{}
				Note that this table only applies to benzene rings, which will be covered \hyperlink{ChapterArenes}{\boit{later}}. This is because
				the resonance effect (overlapping of p-orbitals with π-electron clouds) is only commonly applicable to arenes, or where
				the atom is \sptwo hybridised.

				In all other cases, where the substituent is bonded to an \spthree hybridised atom, the electron-donating or withdrawing
				behaviour is determined mostly by the electronegativity, since there is no unhybridised p-orbital to overlap with. Hence,
				alkyl groups are basically the only electron-donating species in this case.

			% end subsubsection
		% end subsection
	% end section
% end part














