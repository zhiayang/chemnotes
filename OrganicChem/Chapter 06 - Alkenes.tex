% Chapter 06 - Alkenes.tex
% Copyright (c) 2014 - 2016, zhiayang@gmail.com


\pagebreak
\hypertarget{ChapterAlkenes}{}
\part{Alkenes}

	\section{Open Chain}

		Alkenes are simply unsaturated hydrocarbons, with one or more double bonds. They have the general form of
		\ch{C_nH_{2n}}

		\diagram{

			\chemfig{C(-[:135]H)(-[:225]H)=C(-[:45]H)(-[:315]H)}
			\hspace{15mm}
			\chemfig{C(-[:135]H)(-[:225]H)=C(-[:270]H)(-[:0]C(-[:0]H)(-[:90]H)(-[:270]H))}

		}{Ethene (\ch{C2H4}) and propene (\ch{C3H6}) are examples of alkenes.}

	% end section

	\section{Cycloalkenes}

		Cycloalkenes are simply cycloalkanes where one or more of the \ch{C-C} bonds have been replaced with a \ch{C=C} double
		bond.

		\diagram{

			\chemfig{*6(=-=---)}
			\hspace{15mm}
			\chemfig{*6(---=--)}

		}{Cyclohex-1,3-ene and cyclohexene are examples of cycloalkenes.}

	% end section

	\section{Physical Properties}

		The physical properties of alkenes, including melting and boiling points, density and solubility follow the
		same trends as alkanes. Larger molecules have higher melting and boiling points, and higher densities, and alkenes are
		generally only soluble in non-polar solvents.

	% end section

	\pagebreak
	\section{Stability of Carbocations}

		Before reactions and mechanisms of alkenes can be discussed, it is important to note the rules governing the formation
		of products, and the behaviour of molecules during the reaction.

		All physical systems have a tendency to move to the lowest energy state –– this state is characterised by the formation
		of the most stable molecules. As such, ions and radicals are inherently unstable.

		\subsection{Structure}

			One of the important intermediate products are carbocations, which are alkyl groups with an \sptwo hybridised
			central carbon atom, and carries a positive charge on that atom.

			\diagram{
				\chemfig{\chemabove{C}{\smfplus}(<[:315,,1]!\molRon)(<:[:45,,1]!\molRtw)(-[:180]!\molR)}
			}

			The 3 substituent groups are arranged in a trigonal planar fashion, with the p-orbitals above and below this
			plane. As such, nucleophiles can attack the carbocation from either the top or bottom.

		% end subsection


		\subsection{Stability}

			Charged ions are inherently more unstable than their neutral molecule counterparts; any species that stabilises
			the charge on the carbon ion would in turn increase the stability of the entire molecule. The primary reason for
			this is that molecules that are created from a more stable intermediate product have a higher chance to form. Thus,
			the probability of formation of a given product of a reaction can be estimated by looking at the stability of
			the intermediate compound leading to its creation.

			For carbocations, the carbon atom has a positive charge, thus electron-donating substituents such as alkyl
			groups (\ch{-CH3}), would stabilise the ion, as the donated electrons partially disperse the positive charge on the
			central atom.

			Conversely, electron-withdrawing species such as halogens (\ch{F}, \ch{\chlorine} etc.) would further destabilise
			the carbocation, and as such products that involve the formation of these intermediates would have constitute much
			lower proportion of the final products.

		% end subsection

	\pagebreak
	\section{Creation of Alkenes}

		There are two methods of creating alkenes that are covered here, both of which involve elimination reactions, where a small
		molecule is \textit{eliminated} along an alkane, and a double bond formed in its place.

		For brevity, only the conditions and overall reaction will be covered.

		\subsection{Elimination of Hydrogen Halide}

			The main section that elaborates on this reaction can be found \hyperlink{HydrogenHalideElimination}{\boit{here}}, in the
			chapter on halogenoalkanes.


			\vspace{1.5em}
			\vbox{\textbf{Conditions:}	\tabto{35mm}Ethanolic \ch{KOH} or \ch{NaOH}, heat.}

			\diagram[1.0]{
				\schemestart[0, 2.0, thick]
					\chemfig{C(-[:180]H)(-[:90]H)(-[:270]!\molX)-C(-[:90]H)(-[:270]H)(-[:0]H)}
					\arrow{->[\ch{OH-}][heat]}
					\chemfig{C(-[:135]H)(-[:225]H)=C(-[:45]H)(-[:315]H)}
					\hspace{5mm} + \hspace{5mm}
					\chemfig{H-!\molX}
				\schemestop
			}

		% end subsection


		\subsection{Dehydration (Elimination of Water)}

			Detail on this reaction, including the handling of isomers via Zaitsev's rule, can be found
			\hyperlink{AlcoholDehydration}{\boit{here}}, in the chapter on alcohols.


			\vspace{1.5em}
			\vbox{\textbf{Conditions:}	\tabto{35mm}Excess concentrated \ch{H2SO4}, \SI{170}{\celsius}, \textit{OR}
										\tabto{35mm}\ch{\aluminium2O3}, heat.}

			\diagram[1.0]{
				\schemestart[0,1.5,thick]
					\chemfig{C(-[:90]H)(-[:270]H)(-[:180]H)-[:0]C(-[:90]H)(-[:0]H)(-[:270]!\molOH)}
					\arrow
					\chemfig{C(-[:135]H)(-[:225]H)=[:0]C(-[:45]H)(-[:315]H)}
					\hspace{2mm} + \hspace{2mm}
					\ch{H2O}
				\schemestop
			}


		% end subsection

	% end section

	\pagebreak
	\section{Alkene Reactions}

	\subsection{Electrophilic Addition}

		The primary reaction mechanism of alkenes is electrophilic addition. Due to the high electron density of the π-bonds,
		electrophiles are readily attracted –– thus alkenes are far more reactive than alkanes.

		During a reaction, the comparatively weaker π-bond is preferentially broken over the stronger \chemsigma-bond; only a
		\ch{C-C} bond remains, and atoms are \textit{added} to the carbons, since they are now able to form an additional
		bond each. Hence, electrophilic \textit{addition}.

		The detailed mechanisms for each type of electrophilic addition, which are \textit{required knowledge}, can be found
		in \hyperlink{AppendixElectrophilicAddition}{\boit{the appendix}}.

		\subsubsection{Electrophilic Addition of Hydrogen Halides}

			In this reaction, the hydrogen and halogen atom are added across the double bond of the alkene. The hydrogen halide
			should be in a gaseous state for this reaction.

			\vspace{1.5em}
			\vbox{\textbf{Conditions:}\tabto{35mm}Gaseous HX (usually \ch{H\chlorine} or \ch{HBr}).}

			\diagram[1.0]{
				\schemestart[0, 1.5, thick]
				\chemfig{C(-[:135]H)(-[:225]H)=C(-[:45]H)(-[:315]H)}
				\hspace{5mm} + \hspace{5mm}
				\chemfig{H-{{\color{OliveGreen}X}}}
				\arrow
				\chemfig{C(-[:90]H)(-[:180]H)(-[:270]H)-C(-[:90]{{\color{OliveGreen}X}})(-[:0]H)(-[:270]H)}
				\schemestop
			}

		% end subsubsection


		\subsubsection{Markovnikov's Rule}

			This rule governs the major product that is formed when asymmetrical compounds are electrophilically added to alkenes,
			such as hydrogen halides. It does not apply to symmetrical reactants like \ch{\chlorine2} or \ch{Br2}.

			The rule states basically states that when a hydrogen compound with the general form \ch{H-X} is added to an
			alkene that is asymmetrical about the double-bond, the hydrogen atom will be added to the carbon with \textit{more}
			existing hydrogen atoms. Note that \textit{X} can be a halogen, a hydroxide (ie. \ch{H-X} is \ch{H2O}) or some other
			electronegative species.

			A detailed illustration of the rule can be found in \hyperlink{AppendixMarkovnikovsRuleIllustration}{\boit{the appendix}}.

		% end subsubsection





		\pagebreak
		\subsubsection{Electrophilic Addition of Halogens}

			The electrophilic addition of halogens to alkanes does not involve Markovnikov's rule, since it is symmetrical.
			As the non-polar halogen molecule approaches the electron-rich π-bond, it is polarised, forming \deltap{} and \deltam{} partial
			charges.

			Note that the halogens used are usually either \ch{Br2} or \ch{\chlorine2}, since \ch{F2} is too reactive, and
			\ch{I2} is too \textit{unreactive}. The reaction mechanism involves the formation of a \textit{cyclic halonium ion}
			(bromonium or chloronium) –– the double bond breaks, and each carbon forms a single bond with one positive halide ion
			(both carbons bond to the same atom).

			\vspace{1.5em}

			\vbox{\textbf{Conditions:}	\tabto{35mm}No UV Light, gaseous \ch{X2}.}	% again, single line, no need for \vspace{0.5em}
			\vbox{\textbf{Observations:}\tabto{35mm}\boit{\color{Mahogany}Reddish-brown} \ch{Br2} / \boit{\color{YellowGreen}yellowish-green} \ch{\chlorine2} decolourises.}

			\diagram[1.0]{
				\schemestart[0, 1.5, thick]
				\chemfig{C(-[:135]H)(-[:225]H)=C(-[:45]H)(-[:315]H)}
				\hspace{5mm} + \hspace{5mm}
				\chemfig{!\molX-!\molX}
				\arrow
				\chemfig{C(-[:270]!\molX)(-[:90]H)(-[:180]H)-C(-[:270]!\molX)(-[:90]H)(-[:0]H)}
				\schemestop
			}

		% end subsubsection



		\subsubsection{Electrophilic Addition of Aqueous \ch{Br2}}

			The mechanics of this reaction are basically the same as that of the electrophilic addition of \ch{Br2}.
			The conditions and observations are fairly similar as well, except for the colour change –– aqueous
			\ch{Br2} is \boit{\color{Dandelion}yellow}, not reddish-brown.

			Also, since the \ch{Br} atoms are not both added across the double bond in a symmetrical manner,
			Markovnikov's Rule applies –– the OH group will preferentially attack the carbon that has more stabilising alkyl groups.

			\vspace{1.5em}
			\vbox{\textbf{Conditions:}	\tabto{35mm}No UV Light, aqueous \ch{Br2}.}	% again, single line, no need for \vspace{0.5em}
			\vbox{\textbf{Observations:}\tabto{35mm}\boit{\color{Dandelion}Yellow} \ch{Br2 \stAq} decolourises.}


			\diagram[0.75]{
				\schemestart[0, 1.5, thick]
				\chemfig{C(-[:135]H)(-[:225]H)=[@{b1}:0]C(-[:45]H)(-[:315]H)}
				\hspace{5mm} + \hspace{5mm}
				\chemname{\chemfig{!\molBr-!\molBr}}{(aq)}
				\arrow
				\chemname{\chemfig{C(-[:180]H)(-[:90]H)(-[:270]!\molOH)-C(-[:90]H)(-[:0]H)(-[:270]!\molBr)}}{(major)}
				\hspace{5mm} + \hspace{5mm}
				\chemname{\chemfig{C(-[:180]H)(-[:90]H)(-[:270]!\molBr)-C(-[:90]H)(-[:0]H)(-[:270]!\molBr)}}{(minor)}
				\schemestop

			}{Both products are formed, except 1,2-dibromoethane is in much lower proportions.}

		% end subsubsection












		\pagebreak
		\subsubsection{Electrophilic Addition of Steam (Hydration)}

			Under certain (usually industrial) conditions, alkenes can react with steam to form alcohols. Since this reaction involves
			the formation of a carbocation intermediate, the proportion of products will be governed by Markovnikov's Rule.

			\vspace{1.5em}
			\vbox{\textbf{Conditions:}	\tabto{35mm}\SI{300}{\celsius}, at \SI{70}{atm}, \ch{H3PO4} catalyst, \textit{OR}
										\tabto{35mm}Concentrated \ch{H2SO4}, \ch{H2O}, warming.}

			\diagram[1.0]{
				\schemestart[0, 1.5, thick]
				\chemfig{C(-[:135]H)(-[:225]!\molMeR)=[:0]C(-[:45]H)(-[:315]H)}
				\hspace{5mm} + \hspace{5mm}
				\chemfig{\water}
				\arrow
				\chemfig{C(-[:270]!\molMe)(-[:180]H)(-[:90]!\molOH)-C(-[:90]H)(-[:0]H)(-[:270]H)}
				\schemestop
			}



		% end subsubsection


	\subsection{Hydrogenation of Alkenes}

		Hydrogen gas can be used to saturate alkenes, through the use of an insoluble metal catalyst such as nickel, palladium or platinum,
		as well as sufficiently high temperatures and pressures.

		It was commonly used in the food industry to produce margarine from unsaturated plant oils. However, the resulting dense fats were
		found to be hazardous to human health, and subsequent usage was discontinued.

		Note that although hydrogen atoms are added across the double bond, this is \textit{not} an electrophilic addition reaction; it
		is actually a reduction-addition reaction.

		\diagram[1.0]{
			\schemestart[0, 2.0, thick]
			\chemfig{C(-[:135]H)(-[:225]H)=[:0]C(-[:45]H)(-[:315]H)}
			\hspace{5mm} + \hspace{5mm}
			\chemfig{\ch{H2}}
			\arrow{->[Ni, Pd, Pt][high T \& P]}
			\chemfig{C(-[:270]H)(-[:180]H)(-[:90]H)-C(-[:90]H)(-[:0]H)(-[:270]H)}
			\schemestop
		}



	% end subsection

	\pagebreak
	\subsection{Mild Oxidation of Alkenes}

		Alkenes can be mildly oxidised with potassium manganate (VII), KMnO4, forming \textit{diols}. This requires a cold environment,
		and either an acidic or alkaline medium (for \ch{MnO4-} to function).

		On a side note, balancing redox reactions involving organic molecules simply involves adding \ch{H2O} to balance the
		\ch{H} or \ch{O} (for oxidation and reduction respectively), and adding an appropriate number of [\ch{O}] or [\ch{H}]
		to balance the remaining oxygen or water.


		\subsubsection{Acidic Medium}

		A dilute acid is used to provide the acidic medium, in the form of \ch{H+} ions. Typically, this is \ch{H2SO4}. The oxidising
		agent, \ch{MnO4-}, is reduced, forming \ch{Mn2+}.

		\vspace{1.5em}
		\vbox{\textbf{Conditions:}	\tabto{35mm}Cold \ch{KMnO4}, acid (\ch{H2SO4}).}	% again, single line, no need for \vspace{0.5em}
		\vbox{\textbf{Observations:}\tabto{35mm}{\boit{\color{Plum}Purple}} \ch{KMnO4} decolourises.}


		\diagram[0.9]{
			\schemestart[0, 2.0, thick]
			\chemfig{C(-[:135]H)(-[:225]H)=[:0]C(-[:45]H)(-[:315]H)}
			\hspace{5mm} + \hspace{5mm}
			\chemfig{\ch{H2O}}
			\hspace{5mm} + \hspace{5mm}
			[{\color{Red}O}]
			\arrow{->[\ch{KMnO4}, \ch{H+}][cold]}
			\chemfig{C(-[:270]!\molOH)(-[:180]H)(-[:90]H)-C(-[:90]H)(-[:0]H)(-[:270]!\molOH)}
			\schemestop
		}


		% end subsubsection

		\subsubsection{Alkaline Medium}

		An aqueous base, such as \ch{NaOH}, is used to provide the \ch{OH-} ions. The oxidising agent \ch{MnO4-} is reduced to
		\ch{MnO2}.

		\vspace{1.5em}
		\vbox{\textbf{Conditions:}	\tabto{35mm}Cold \ch{KMnO4}, aqueous base (\ch{NaOH}).}	% \vspace{0.5em}
		\vbox{\textbf{Observations:}\tabto{35mm}{\boit{\color{Plum}Purple}} \ch{KMnO4} decolourises, forming
			a {\boit{\color{Brown}brown}} precipitate of \ch{MnO2}.}


		\diagram[0.9]{
			\schemestart[0, 2.0, thick]
			\chemfig{C(-[:135]H)(-[:225]H)=[:0]C(-[:45]H)(-[:315]H)}
			\hspace{5mm} + \hspace{5mm}
			\chemfig{\ch{H2O}}
			\hspace{5mm} + \hspace{5mm}
			[{\color{Red}O}]
			\arrow{->[\ch{KMnO4}, \ch{OH-}][cold]}
			\chemfig{C(-[:270]!\molOH)(-[:180]H)(-[:90]H)-C(-[:90]H)(-[:0]H)(-[:270]!\molOH)}
			\schemestop
		}

		% end subsubsection


	\pagebreak
	\hypertarget{OxidativeCleavageOfAlkenes}{}
	\subsection{Oxidative Cleavage (Strong Oxidation) of Alkenes}

		When the \ch{KMnO4} solution containing an alkene is heated, \textit{strong oxidation} will take place. The double bond is
		\textit{cleaved} instead of added to, and each side of the double bond forms its own fragment. In the case of cycloalkenes
		however, the end product might still only be one molecule.

		Furthermore, only a strong oxidising agent like \ch{MnO4-} can be used. The difference between mild and strong oxidation also
		depends \textit{only} on the temperature of the reacting solution. Only heated solutions will result in oxidative cleavage.

		\vspace{1.5em}
		\vbox{\textbf{Conditions:}	\tabto{35mm}Cold \ch{KMnO4}, dilute \ch{H2SO4}, heat.}	% \vspace{0.5em}
		\vbox{\textbf{Observations:}\tabto{35mm}{\boit{\color{Plum}Purple}} \ch{KMnO4} decolourises, forming colourless \ch{Mn^2+}.}

		\vspace{1.0em}

		Depending on the number of substituents (or inversely, the number of hydrogen atoms) attached to the carbon with the double bond,
		there are 3 possible products. Again, a table is the best way to present this, unfortunately.

		\begin{center}\begin{table}[htb]\renewcommand{\arraystretch}{1.0}
		\begin{tabu} to \textwidth {| X[c,m] | X[c,m] | X[c,m] |}

			\hline
			% headings
			Substituents	&		Structure												&	Product			\\		\hline
				0			&		\vspace{2mm}\chemfig{C(-[:135]H)(-[:225]H)=[:0]}			\vspace{2mm}
							&		\vspace{2mm}\ch{CO2} + \ch{H2O}								\vspace{2mm}	\\		\hline


				1
							&		\vspace{2mm}\chemfig{C(-[:135]H)(-[:225]!\molR)=[:0]}			\vspace{2mm}
							&		\vspace{2mm}\chemfig{C(-[:135]!\molHO)(-[:225]!\molR)=[:0]!\molO}	\vspace{2mm}	\\		\hline

				2
							&		\vspace{2mm}\chemfig{C(-[:135]!\molR)(-[:225]!\molR)=[:0]}		\vspace{2mm}
							&		\vspace{2mm}\chemfig{C(-[:135]!\molR)(-[:225]!\molR)=[:0]!\molO}	\vspace{2mm}	\\		\hline



		\end{tabu}
		\end{table}\end{center}\vspace{-10mm}


		\pagebreak
		\subsubsection{Further Oxidation of Ethanedioic Acid}

			The sole special case that must be noted is that if the oxidative cleavage results in the formation of ethanedioic acid
			(\ch{C2H2O4}), it is further oxidised to form \ch{2 CO2} and \ch{H2O}.

			\diagram[1.0]{
				\schemestart[0, 1.5, thick]
				\chemfig{C(-[:135]!\molHO)(=[:225]!\molO)-C(=[:45]!\molO)(-[:315]!\molOH)}
				\arrow
				\ch{2 CO2} \hspace{5mm} + \hspace{5mm} \ch{H2O}
				\schemestop
			}{Ethandioic acid, or oxalic acid.}

		% end subsubsection

		% \pagebreak
		\subsubsection{Uses of Oxidative Cleavage}

			Oxidative cleavage can be used to determine the position of the double bond in carbon chains and cycloalkenes,
			given the products (fragments) of the cleavage reaction.

			Most importantly, if \ch{CO2} is one of the products (gas evolved), then there are 2 possibilities –– either the original
			molecule has a terminal \ch{C=C} double bond, or the cleavage involved the formation of ethanedioic acid that further
			decomposed to form \ch{CO2}.

		% end subsubsection

	% end subsection
	% end section
% end part




























