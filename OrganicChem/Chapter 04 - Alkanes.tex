% Chapter 05 - Alkanes.tex
% Copyright (c) 2014 - 2016, zhiayang@gmail.com


\pagebreak
\hypertarget{ChapterAlkanes}{}
\part{Alkanes}

	\section{Open Chain}

		Open chain alkanes have the general formula of \ch{C_nH_{2n+2}}. They are called 'open-chain' because
		the two ends of the chain are separate, in contrast with cycloalkanes. Open chain alkanes can either be straight or
		branch; the terminology should be pretty much self explanatory.

		\diagram{

			\chemfig{!\molMeR-[:330]-[:30]-[:330]-[:30]-[:330]-[:30]-[:330]!\molMe}

		}Octane is an example of a straight-chain alkane.]


		\diagram{

			\chemfig{C(-[:0]!\molMe)(-[:90]!\molMe)(-[:180]!\molMeR)(-[:270]!\molMe)}

		}[2,2-dimethylpropane is an example of a branched alkane.]

	% end section

	\section{Cycloalkanes}

		Cycloalkanes are alkanes where the carbon atoms at either end of the chain are bonded together, forming a closed loop and
		taking the shape of regular polygons. They have the general formula \ch{C_nH_{2n}}

		\diagram{

			\chemfig{*5(-----)}		\hspace{15mm}
			\chemfig{*6(------)}

		}[Cyclopentane (left) and cyclohexane (right)]

	% end section


	\pagebreak
	\section{Physical Properties}
		\subsection{Melting and Boiling Points}

			The melting and boiling points of alkanes follow a simple pattern. Due to the fact that they rely solely on induced
			dipole interactions for intermolecular bonding, both melting and boiling points increase with the length of the
			carbon chain, and by extension M\sbs{r}.

			However, because branched alkanes have a smaller surface area for a given number of carbon atoms than straight-chained
			alkanes, their melting and boiling points will be lower, due to a smaller area for polarisation.

		% end subsection

		\subsection{Density}

			Most liquid alkanes, up to a certain point, are less dense than water. Additionally, since they are insoluble,
			they form an immiscible layer above water.

			As the number of carbon atoms increases, the strength of the intermolecular interactions will increase as well --- this
			forces each molecule ever so slightly closer together, marginally increasing density.

		% end subsection

		\subsection{Solubility}

			As described before, alkanes rely only on induced-dipole interactions for intermolecular bonding; they are insoluble in
			polar solvents such as water. However, they are highly soluble in non-polar solvents like \ch{C\Cl4}. In fact,
			since solubility is dependent on the strength of solvent-solute bonds, larger alkanes will be more soluble in non-polar solvents.

		% end subsection

	% end section


	\section{Classification of Carbons}

		This applies to many things --- carbons, halogenoalkanes, and alcohols. Essentially, the classification of an atom with
		substituent groups refers to the number of such groups attached to the carbon, and the number of hydrogens (ie. unfilled
		slots so to speak) attached to it.

		As example, there are 4 possible classifications of carbons in an alkane: \itl{primary}, \itl{secondary}, \itl{tertiary},
		and \itl{quaternary}.

		\begin{table}[ht]\renewcommand{\arraystretch}{1.4}\begin{center}
		\begin{tabu} to 0.9\textwidth { X[c,m] | X[c,m] | X[c,m] | X[c,m] }

			Primary (\SI{1}{\degree})		&
			Secondary (\SI{2}{\degree})		&
			Tertiary (\SI{3}{\degree})		&
			Quaternary (\SI{4}{\degree})	\\

			\hline

			\vspace{2mm}	\chemfig{C(-[:0]!\molRon)(-[:90]H)(-[:180]H)(-[:270]H)}							\vspace{2mm}	&
			\vspace{2mm}	\chemfig{C(-[:0]!\molRon)(-[:90]!\molRtw)(-[:180]H)(-[:270]H)}					\vspace{2mm}	&
			\vspace{2mm}	\chemfig{C(-[:0]!\molRon)(-[:90]!\molRtw)(-[:180]!\molRth)(-[:270]H)}			\vspace{2mm}	&
			\vspace{2mm}	\chemfig{C(-[:0]!\molRon)(-[:90]!\molRtw)(-[:180]!\molRth)(-[:270]!\molRfr)}	\vspace{2mm}	\\

		\end{tabu}\end{center}
		\end{table}\vspace{-1em}

		% The table above gives a visual representation. \itl{R} represents an alkyl group substituent. This principle
		% also applies when classifying alkyl radicals.

	% end section


	\pagebreak
	\section{Alkane Reactions}
		\subsection{Free Radical Substitution}

			\subsubsection{Mechanism of Reaction}

				In the steps below, the free radical substitution of methane (\ch{CH4}) by chlorine (\ch{\Cl2}) will be used. This applies
				for any alkane, and a sufficiently reactive halogen.

				\vspace{1.5em}

				\vbox{\textbf{Conditions:}	\tabto{35mm}UV Light, \ch{Br2} / \ch{\Cl2} gas}
				\vbox{\textbf{Observations:}\tabto{35mm}\boit{\color{Mahogany}Reddish-brown} \ch{Br2} / \boit{\color{YellowGreen}yellowish-green} \ch{\Cl2} decolourises.}

				\paragraph{Stage I.\protect\hphantom{II}\hspace{5mm} \itl{Initiation}}

					The \ch{\Cl-\Cl} bond is \itl{homolytically} broken to form 2 \ch{\Cl} radicals.
					The energy required to break this bond is provided by the UV light.

					\diagram{
						\schemestart[0, 1.5, thick]
						\chemfig{@{cl1}{\color{OliveGreen}\Cl}-[@{b}:0]@{cl2}{\color{OliveGreen}\Cl}}
						\arrow
						\chemfig{2 \lewis{0.,\color{OliveGreen}\Cl}}
						\schemestop

						\chemmove{\draw[-{Stealth[left]},line width=0.4mm,shorten <=1mm,shorten >=1mm](b).. controls +(90:7mm) and +(90:7mm).. (cl2);}
						\chemmove{\draw[-{Stealth[left]},line width=0.4mm,shorten <=1mm,shorten >=1mm](b).. controls +(270:7mm) and +(270:7mm).. (cl1);}
					}

				% end

				\paragraph{Stage II.\protect\hphantom{I}\hspace{5mm} \itl{Propagation}}

					The highly reactive \ch{\Cl} radicals then react with the \ch{CH4} molecules, bonding with one of the
					hydrogen atoms to form \ch{H\Cl} and a carbocation radical, \ch{CH3}.

					\txtreactioneqn{
						\chlewis{0.}{\Cl} \hspace{2mm} + \hspace{3.625mm} \ch{CH4}
						\arrow
						\chlewis{180.}{CH3} \hspace{4.375mm} + \hspace{2mm} \ch{H\Cl}
						\arrow(@c1.south east--.north east){0}[-90,.2]
						\ch{\Cl2} \hspace{2mm} + \hspace{2mm} \chlewis{180.}{CH3}
						\arrow
						\ch{CH3\Cl} \hspace{2mm} + \hspace{2mm} \chlewis{0.}{\Cl}
					}

					This newly-minted \ch{CH3} radical can react with a \ch{\Cl2} molecule to form \ch{CH3\Cl} and another
					\ch{\Cl} radical, thus recreating the consumed radical. As long as the supply of \ch{\Cl2} gas has
					not been exhausted, this reaction can continue, and no additional UV light is required to sustain it.

				%end

				\pagebreak
				\paragraph{Stage III.\hspace{5mm} \itl{Termination}}

					In the termination stage, two radicals combine to form stable products, effectively terminating the chain reaction.

					\txtreactioneqn{
						\chlewis{0.}{\Cl} \hspace{2mm} + \hspace{5.75mm} \chlewis{0.}{\Cl}
						\arrow
						\ch{\Cl2}
						\arrow(@c1.south east--.north east){0}[-90,.2]
						\chlewis{180.}{CH3} \hspace{2mm} + \hspace{2mm} \chlewis{180.}{CH3}
						\arrow
						\ch{CH3CH3}
						\arrow(@c3.south east--.north east){0}[-90,.2]
						\chlewis{0.}{\Cl} \hspace{2mm} + \hspace{2mm} \chlewis{180.}{CH3}
						\arrow
						\ch{CH3\Cl}
					}

					Once all the radicals have been consumed in this manner, the reaction will stop, unless more UV light is provided (along
					with more \ch{\Cl2} gas).

				%end

			% end subsubsection


			\subsubsection{Multi-substitution}

				Free radical substitution of alkanes is usually not the preferred way to produce
				halogenated alkanes, due to the random nature of the process and the possibility of multi-substitution,
				where more than one halogen atom has been substituted onto the alkane.

				If there are still \ch{\Cl} radicals remaining in the reaction chamber, \ch{CH3\Cl} formed during the
				termination stage can still react, forming a \ch{CH2\Cl} radical, which can continue to react.

				\txtreactioneqn{
					\ch{CH3Cl} \hspace{2mm} + \hspace{2.125mm} \chlewis{0.}{\Cl}
					\arrow
					\chlewis{180.}{CH2\Cl}
					\arrow(@c1.south east--.north east){0}[-90,.2]
					\chlewis{0.}{CH2\Cl} \hspace{2mm} + \hspace{2mm} \ch{\Cl2}
					\arrow
					\chlewis{4.}{CH2\Cl2} \hspace{2mm} + \hspace{2mm} \chlewis{0.}{\Cl}
				}

				Indeed, this can continue \itl{ad-infinitum}, until all the hydrogen atoms on the alkane have been substituted.
				In the case of methane, this results in the formation of \ch{CH3\Cl}, \ch{CH2\Cl2}, \ch{CH\Cl3} and
				\ch{C\Cl4}.

				In fact, there will also be a small amount of alkanes with more than 1 carbon in the chain; this is due to the
				possibility of two molecules of •\ch{CH2\Cl} reacting, which then has its hydrogens further substituted.

			% end subsubsection




			\pagebreak
			\subsubsection{Isomerism of Products}

				Because every hydrogen atom can be substituted by a halogen atom, alkanes with 3 or more carbon atoms in the chain
				can form isomers. The probability of the formation of each isomer depends on both the number of possible hydrogen atoms
				that can be substituted to result in it, as well as the stability of the carbon radical intermediate.

				\diagram[1.0]{
					\schemestart[0, 1.0, thick]
						\chemfig{C(-[:90]H)(-[:180]H)(-[:270]H)-C(-[:90]@{h1}H)(-[:270]@{h2}H)-C(-[:90]H)(-[:0]H)(-[:270]H)}
						\arrow[270]
						\chemfig{C(-[:90]H)(-[:180]H)(-[:270]H)-\lewis{2.,C}(-[:270]H)-C(-[:90]H)(-[:0]H)(-[:270]H)}
						\arrow[270]
						\chemfig{C(-[:90]H)(-[:180]H)(-[:270]H)-C(-[:90]!\molCl)(-[:270]H)-C(-[:90]H)(-[:0]H)(-[:270]H)}
					\schemestop

					\hspace{30mm}

					\schemestart[0, 1.0, thick]
						\chemfig{C(-[:90]@{h3}H)(-[:180]@{h4}H)(-[:270]@{h5}H)-C(-[:90]H)(-[:270]H)-C(-[:90]@{h6}H)(-[:0]@{h7}H)(-[:270]@{h8}H)}
						\arrow[270]
						\chemfig{C(-[:90]H)(-[:180]H)(-[:270]H)-C(-[:90]H)(-[:270]H)-\lewis{0.,C}(-[:0,,,,draw=none]\vphantom{H})(-[:90]H)(-[:270]H)}
						\arrow[270]
						\chemfig{C(-[:90]H)(-[:180]H)(-[:270]H)-C(-[:90]H)(-[:270]H)-C(-[:0]!\molCl)(-[:90]H)(-[:270]H)}
					\schemestop


					% circle the atoms
					\chemmove{
						\draw[-latex, red, thick]
						(h1.center) circle(4mm)
						(h2.center) circle(4mm)
						(h3.center) circle(4mm)
						(h4.center) circle(4mm)
						(h5.center) circle(4mm)
						(h6.center) circle(4mm)
						(h7.center) circle(4mm)
						(h8.center) circle(4mm);
					}
				}



				In the example above, propane can be substituted by \ch{\Cl} to form 2 different isomers, 2- chloropropane and 1-chloropropane.
				Only 2 hydrogens can be substituted to form the former, while 6 hydrogens can be substituted to form the latter. Thus, the
				\itl{expected} ratio of products from the free radical substitution of propane with chlorine is 2 : 6 in
				favour of 1-chloromethane.

				However, the stability of the carbon radical intermediate also plays a part in the ratio of products. For 1-chloropropane,
				it involves forming a carbon radical on carbon 1, which only has 1 electron-donating alkyl group (through the induction effect)
				to stabilise the positive charge.

				On the other hand, for 2-chloropropane, the positive charge is on carbon 2, which has 2 electron-donating alkyl groups to stabilise
				it. Thus, the intermediate is more stable, and hence 2-chloropropane is more likely to form.
				Therefore, the actual ratio of products is around 4 : 6, not 2 : 6.

			% end subsubsection

			\pagebreak
			\subsubsection{Halogen Reactivity}

				The reactivity of the free radical substitution naturally depends on the species of halogen that is reacting.
				As is typical of halogens, the reactivity increases in this order: \ch{I2} < \ch{Br2} < \ch{\Cl2} < \ch{F2}.
				In fact, the substitution by fluorine is too reactive --- even in the dark and at room temperature. On the other hand,
				the substitution of iodine is not feasible (\gibb{} > 0).

				In the propagation steps of the reaction, several bonds are formed and broken. Using X as a halogen atom:

				\vspace{1em}
				\vbox{\enth{1} = BE(\ch{C-H}) - BE(\ch{H-X}) \tabto{60mm}\ch{CH3-H} + X•                \tabto{85mm} \ch{->} \chlewis{180.}{CH3} + \ch{H-X} }
				\vbox{\enth{2} = BE(\ch{X-X}) - BE(\ch{C-X}) \tabto{60mm}\chlewis{180.}{CH3} + \ch{X-X} \tabto{85mm} \ch{->} \ch{CH3-X} + \chlewis{0.}{X}   }
				\vbox{\enth{r} = \enth{1} + \enth{2}         \tabto{60mm}\ch{CH4} + \ch{X2}             \tabto{85mm} \ch{->} \ch{CH3X} + \ch{HX}  }

				The values of the bond energies for H–X, X–X, and C–X are below:

				\begin{table}[htb]\renewcommand{\arraystretch}{1.5}\begin{center}
				\begin{tabu} to 0.9\textwidth {X[c,m] | X[c,m] | X[c,m] | X[c,m] | X[c,m]}

					&\ch{H-X} / \si{\kJmol}	&\ch{H-X} / \si{\kJmol}	&\ch{H-X} / \si{\kJmol}	&\enth{r} / \si{\kJmol}\\ \hline

					\ch{F2}     &   \num{562}   &   \num{484}   &   \num{158}   &   \num{-478}                      \\
					\ch{\Cl2}   &   \num{431}   &   \num{340}   &   \num{244}   &   \num{-117}                      \\
					\ch{Br2}    &   \num{366}   &   \num{280}   &   \num{193}   &   \num{-43}                       \\
					\ch{I2}     &   \num{299}   &   \num{240}   &   \num{151}   &   \num[retain-explicit-plus]{+22} \\

				\end{tabu}\end{center}
				\end{table}\vspace{-1em}

				As can be seen, the reaction with \ch{F2} is highly exothermic; the F–F bond energy is relatively small, hence the
				reaction is highly spontaneous --- even at room temperature and without UV light, since the initiation stage requires
				little energy to progress.

				On the other hand the \enth{} of the reaction with \ch{I2} is positive, hence the reaction is not as spontaneous, and the rate of
				reaction is very slow.

			% end subsubsection
		% end subsection


		\subsection{Combustion}

			Alkanes are a type of combustible fuel, and there exists a general formula describing the required amounts of \ch{O2}
			to combust a given hydrocarbon.

			\txtreactioneqn{
				\ch{C$_{x}$H$_{y}$}
				\hspace{2mm} + \hspace{2mm}
				$(x + \frac{y}{4})$ \ch{O2}
				\arrow
				$x$ \ch{CO2}
				\hspace{2mm} + \hspace{2mm}
				$\frac{y}{2}$ \ch{H2O}
			}

			Note that while this combustion process is highly exothermic, it still has a high activation energy, and requires an energy
			input like a spark to begin the reaction.

		% end subsection


% end part















