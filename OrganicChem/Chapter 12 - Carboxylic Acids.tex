% Chapter 12 - Carboxylic Acids.tex
% Copyright (c) 2014 - 2016, zhiayang@gmail.com
% Licensed under the Apache License Version 2.0.


\pagebreak
\hypertarget{ChapterCarboxylicAcids}{}
\part{Carboxylic Acids}

	\section{Structure}

		The structure of carboxylic acids is kind of an extension of a ketone or aldehyde --- it features an oxygen atom double-bonded
		to a carbon atom, with an R group and an \ch{OH} group attached as well.

		\diagram[1.0]{
			\chemfig{C(-[:180]!\molR)(=[:60]!\molO)(-[:300]!\molOH)}
		}{The general structure of a carboxylic acid.}

		The central carbon is \sptwo hybridised, so the bond angle between all three groups is \SI{120}{\degree}.

	% end section

	\section{Physical Properties}

		\paragraph{Melting and Boiling Points}

		Compared to both alkanes, carboxylic acids naturally have higher melting and boiling points. This is mostly due to the large permanent
		dipole moment, caused by the existence of two electron-withdrawing groups (the \ch{C=O} and \ch{C-OH}), creating a large partial-positive
		charge (\deltap) on the carbon atom. Thus, hydrogen bonds can form; these bonds are stronger than that in alcohols, because the dipole
		moment is greater.

		Their melting and boiling points are also higher than that of aldehydes and ketones, since carbonyls lack the \ch{O-H} partial charge
		disparity to form hydrogen bonds, and rely solely on permanent dipole interactions.

		Note that hydrogen bonds between carboxylic acid molecules typically form between the \ch{H} on the \ch{OH} group, and the \ch{O} atom
		double-bonded to the central carbon.

		% i hate mid-paragraph breaks, but there is no choice.
		\pagebreak
		Furthermore, in both the liquid and gaseous state, two carboxylic acid molecules can \itl{dimerise}, forming double-bonds between
		each other and effectively increasing molecular mass, which also increases id-id interaction strength.

		% fuck it, we have to draw it as one molecule.
		\diagram[1.0]{
			\chemfig{C(-[:180]!\molR)(=[:60]\lewis{0:2:,\color{Red}O}-#(3mm)[:0,2.0,,,blue,dashed]\chemabove{H}{\hspace{1mm}\smdeltap}-\chemabove{\color{Red}O}{\hspace{1mm}\smdeltam}-[:300]C(-[:0]!\molR)(=[:240]\lewis{4:6:,\color{Red}O}-#(3mm)[:180,2.0,,,blue,dashed]\chemabove{H}{\hspace{1mm}\smdeltap}?))(-[:300]\chemabove{\color{Red}O}{\hspace{1mm}\smdeltam}?)}

		}{An illustration of two carboxylic acids dimerising.}



		\paragraph{Solubility}

		Carboxylic acids are fairly soluble in non-polar solvents, as they exist as the dimerised form shown above. This means that carboxylic acids
		are not acidic in a non-polar solvent, as they do not dissociate.

		In water, the molecules do not dimerise; instead, they form hydrogen bonds with the water, dissociating into \ch{R-CO2-} and \ch{H+} ions.
		This tends to increase the solubility due to the formation of favourable solvent-solute interactions.

		However, with carbon chains longer than 5 atoms, solubility decreases due to the bulky alkyl chain.

	% end section


	\pagebreak
	\section{Creation of Carboxylic Acids}

		Carboxylic acids can be created in various ways, but one of the most common ways is through oxidation. Essentially, carboxylic acids
		are at the \enquote{highest} oxidation state, followed by aldehydes and ketones, then alcohols.

		\subsection{Oxidation of Primary Alcohols}

			When the aldehydes can be oxidised to carboxylic acids. Alternatively, the strong oxidising agent \ch{KMnO4} can be used to
			immediately create a carboxylic acid from a primary alcohol.

			\vspace{1.5em}
			\vbox{\textbf{Conditions:}	\tabto{35mm}\ch{K2Cr2O7} with dilute \ch{H2SO4}, \itl{OR} \ch{KMnO4} with dilute \ch{H2SO4},
										\tabto{35mm}heat under reflux.}
			\vspace{0.75em}
			\vbox{\textbf{Observations:}\tabto{35mm}\boit{\color{BurntOrange}Orange} \ch{Cr2O7^2-}, turns \boit{\color{LimeGreen}green}
													(\ch{Cr^3+} formed), \itl{OR}
										\tabto{35mm}\boit{\color{Plum}Purple} \ch{MnO4-} decolourises (\ch{Mn^2+} formed)}

			\diagram[1.0]{
				\schemestart[0, 1.5, thick]
					\chemfig{C(-[:90]H)(-[:270]H)(-[:180]!\molR)(-[:0]!\molOH)}
					\arrow
					\chemfig{C(-[:180]!\molR)(=[:45]!\molO)(-[:315]!\molOH)}
				\schemestop
			}

			\diagram[1.0]{
				\schemestart[0, 1.5, thick]
					\chemfig{C(-[:210]H)(-[:330]!\molR)(=[:90]!\molO)}
					\arrow
					\chemfig{C(-[:180]!\molR)(=[:45]!\molO)(-[:315]!\molOH)}
				\schemestop
			}{It is also possible to start with aldehydes, and oxidise them to carboxylic acids.}

		% end subsection


		\pagebreak
		\subsection{Side-chain Oxidation of Alkylbenzenes}

			As noted in the chapter on arenes, benzene rings with an alkyl chain can undergo side-chain oxidation to form benzoic acid.

			Again, note that the carbon attached to the benzene ring \itl{cannot} be tertiary, ie. it must have at least one hydrogen atom.


			\vspace{1.5em}

			\vbox{\textbf{Conditions:}\tabto{35mm}Heat, \ch{KMnO4}, dilute acid or alkali.}

			\vspace{0.75em}
			\vbox{\textbf{Observations:}\tabto{35mm}\boit{\color{Plum}Purple} \ch{KMnO4} decolourises (\itl{acid}), or
										\tabto{35mm}forms \boit{\color{Brown}brown} precipitate of \ch{MnO2} (\itl{alkali}).}

			\diagram[1.0]{
				\schemestart[0, 2.0, thick]
				\chemfig[yshift=-1.5em]{**6(----(-[:90]!\molMe)--)}
				\arrow(.base east--.base west){->[\ch{KMnO4}, \ch{H+}][heat]}
				\chemfig[yshift=-1.5em]{**6(----(-[:90]C(=[:150]!\molO)(-[:30]!\molOH))--)}
				\schemestop
			}

		% end subsection


		\subsection{Hydrolysis of Nitriles}

			As covered in previous chapters, nitriles, \ch{-C+N}, can be hydrolysed to give carboxylic acids.

			\subsubsection{Acid Hydrolysis}

				The more straightforward method is acid hydrolysis --- in this case, a carboxylic acid is immediately produced.

				\vspace{1.5em}
				\vbox{\textbf{Conditions:}	\tabto{35mm}Dilute \ch{H2SO4} or \ch{H\Cl}, heat under reflux.}

				\diagram[1.0]{
					\schemestart[0, 2.0, thick]
						\chemfig{!\molR-[:0]C~[:0]!\molN}
						\arrow{->[dil. \ch{H2SO4}][heat with reflux]}
						\chemfig{C(-[:180]!\molR)(=[:60]!\molO)(-[:300]!\molOH)}
					\schemestop
				}
			% end subsubsection


			\subsubsection{Alkaline Hydrolysis}

				The alternative is to conduct the hydrolysis in an alkaline medium using \ch{NaOH}, which gives a carboxylate salt.
				This salt can be acidified using a dilute acid (I mean, why not start with an acid?) to yield the carboxylic acid.

				\vspace{1.5em}
				\vbox{\textbf{Conditions:}	\tabto{35mm}Dilute \ch{NaOH}, heat under reflux.}

				\diagram[1.0]{
					\schemestart[0, 2.0, thick]
						\chemfig{!\molR-[:0]C~[:0]!\molN}
						\arrow{->[dil. \ch{NaOH}][heat with reflux]}
						\chemfig{C(-[:180]!\molR)(=[:60]!\molO)(-[:300]!\molO\mch Na\pch)}
						\arrow(@c1.south east--.north east){0}[-90,.75]
						\chemfig{C(-[:180]!\molR)(=[:60]!\molO)(-[:300]!\molO\mch Na\pch)}
						\arrow{->[dil. \ch{H2SO4}]}
						\chemfig{C(-[:180]!\molR)(=[:60]!\molO)(-[:300]!\molOH)}
					\schemestop
				}{The carboxylate salt can be acidified.}


			% end subsubsection

		% end subsection


		\subsection{Oxidative Cleavage of Alkenes}

			Alkenes can undergo strong oxidation, cleaving the double bond. The details are elaborated on
			\hyperlink{OxidativeCleavageOfAlkenes}{\boit{here}}, but only one kind of alkene can form a carboxylic, where the carbon
			atom has one alkyl group and one hydrogen atom.

			\vspace{1.5em}
			\vbox{\textbf{Conditions:}	\tabto{35mm}\ch{KMnO4}, dilute \ch{H2SO4}, heat.}
			\vbox{\textbf{Observations:}\tabto{35mm}{\boit{\color{Plum}Purple}} \ch{KMnO4} decolourises, forming colourless \ch{Mn^2+}.}


			\diagram[1.0]{
				\schemestart[0, 2.0, thick]
					\chemfig{C(-[:120]!\molMe)(-[:240]H)=[:0]C(-[:60]H)(-[:300]!\molMe)}
					\arrow{->[\ch{MnO4-}, \ch{H+}][heat]}
					\chemfig{2}\hspace{2mm}\chemfig{C(-[:180]!\molMe)(=[:60]!\molO)(-[:300]!\molOH)}
				\schemestop
			}

		% end subsection


		\pagebreak
		\subsection{Hydrolysis of Esters}

			Finally, esters can be hydrolysed to form a carboxylic acid as one of its products. Note that this hydrolysis is preferably
			conducted in an alkaline medium, since the reaction is slow and reversible in an acidic medium, but fast and irreversible under
			alkaline conditions. All that is required is a dilute acid to protonate the carboxylate salt after.

			\vspace{1.5em}
			\vbox{\textbf{Conditions:}	\tabto{35mm}Dilute \ch{H2SO4}, heat under reflux, \itl{OR}
										\tabto{35mm}Dilute \ch{NaOH}, heat under reflux.}

			\diagram[1.0]{
				\schemestart[0, 2.0, thick]
					\chemfig{C(-[:180]!\molR)(=[:45]!\molO)(-[:315]!\molO-[:0]!\molRon)}
					\hspace{2mm} + \hspace{2mm}
					\chemfig{\ch{H2O}}
					\arrow{<=>[dil. \ch{H2SO4}][reflux]}
					\chemfig{C(-[:180]!\molR)(=[:60]!\molO)(-[:300]!\molOH)}
					\hspace{2mm} + \hspace{2mm}
					\chemfig{!\molRon-[:0]!\molOH}
				\schemestop
			}{An alcohol is also formed, but for our purposes the carboxylic acid is the main product.}
			\diagram[1.0]{
				\schemestart[0, 2.0, thick]
					\chemfig{C(-[:180]!\molR)(=[:45]!\molO)(-[:315]!\molO-[:0]!\molR)}
					\hspace{2mm} + \hspace{2mm}
					\chemfig{\ch{H2O}}
					\arrow{->[dil. \ch{NaOH}][reflux]}
					\chemfig{C(-[:180]!\molR)(=[:60]!\molO)(-[:300]!\molO\mch Na\pch)}
					\hspace{2mm} + \hspace{2mm}
					\chemfig{!\molRon-[:0]!\molOH}
				\schemestop
			}

		% end subsection


		\subsection{Hydrolysis of Acyl Chlorides}

			Acyl chlorides, which are a derivative of carboxylic acids where the \ch{OH} group is replaced by a \ch{\Cl} atom, can
			be hydrolysed to form a carboxylic acid, and \ch{H\Cl}.

			\vspace{1.5em}
			\vbox{\textbf{Conditions:}\tabto{35mm}\ch{H2O}, room temperature.}
			\vbox{\textbf{Observations:}\tabto{35mm}Formation of white fumes of \ch{H\Cl} gas.}

			\diagram[1.0]{
				\schemestart[0, 1.5, thick]
					\chemfig{C(-[:180]!\molR)(=[:60]!\molO)(-[:300]!\molCl)}
					\hspace{5mm} + \hspace{5mm}
					\ch{H2O}
					\arrow
					\chemfig{C(-[:180]!\molR)(=[:60]!\molO)(-[:300]!\molOH)}
					\hspace{5mm} + \hspace{5mm}
					\ch{H\Cl}
				\schemestop
			}

		% end subsection

		\pagebreak
		\subsection{Hydrolysis of Amides}

			Amides can be hydrolysed with either acids or bases to give a carboxylic acid, and an amine. If the original amide is unsubstituted,
			then the result will be \ch{NH3}, or, in an acidic medium, \ch{NH4+}. The products illustrated below can have \ch{R} be an alkyl
			substituent or a hydrogen atom.

			Note that the formation of \ch{NH4+} is due to the fact that \ch{NH3} is a base, and will react with \ch{H+} ions in the acidic
			medium.

			\subsubsection{Acid Hydrolysis}

				The hydrolysis of an amide in an acidic medium yields the carboxylic acid directly.

				\vspace{1.5em}
				\vbox{\textbf{Conditions:}	\tabto{35mm}Dilute \ch{H2SO4} or \ch{H\Cl}, heat under reflux.}

				\diagram[1.0]{
					\schemestart[0, 2.0, thick]
						\chemfig{C(-[:180]!\molR)(=[:60]!\molO)(-[:300]N|R\sbs{2})}
						\arrow{->[dil. \ch{H2SO4}][heat with reflux]}
						\chemfig{C(-[:180]!\molR)(=[:60]!\molO)(-[:300]!\molOH)}
						\hspace{5mm} + \hspace{5mm}
						\ch{NH2R2+}
					\schemestop
				}
			% end subsubsection


			\subsubsection{Alkaline Hydrolysis}

				The alternative is to conduct the hydrolysis in an alkaline medium using \ch{NaOH}, which gives a carboxylate salt.
				This salt can be acidified using a dilute acid to yield the carboxylic acid.

				\vspace{1.5em}
				\vbox{\textbf{Conditions:}	\tabto{35mm}Dilute \ch{NaOH}, heat under reflux.}

				\diagram[1.0]{
					\schemestart[0, 2.0, thick]
						\chemfig{C(-[:180]!\molR)(=[:60]!\molO)(-[:300]N|R\sbs{2})}
						\arrow{->[dil. \ch{H2SO4}][heat with reflux]}
						\chemfig{C(-[:180]!\molR)(=[:60]!\molO)(-[:300]!\molO\mch Na\pch)}
						\hspace{5mm} + \hspace{5mm}
						\ch{NHR2}
					\schemestop
				}{The carboxylate salt can be acidified later.}


			% end subsubsection
		% end subsection
	% end section


	\section{Carboxylic Acid Reactions}

		\subsection{Acidity of Carboxylic Acids}

			Carboxylic acids are far more acidic than either alcohols or phenols, since the conjugate base (carboxylate anion) is greatly
			stabilised due to the delocalisation of the negative charge across the two electronegative oxygen atoms.

			\diagram[1.0]{
				% here's the deal: we draw the carboxylic acid as normal.
				% then, we draw a benzene ring attached to the carbon.
				% it is attached with a negative bond length to bring the circle closer to the carbon
				% the circle is cut off "**[130,230,dashed]6" to certain angles
				% none of the benzene ring bonds are actually drawn (hence 6(), without ------ inside)
				% a bond is drawn from the first position to the centre (using bond angle = :0), for the plus.
				% this is stupid.

				\chemfig{C(-[:60]!\molO)(-[:300]!\molO)(-[:180]!\molR)(
					-[:0,-0.10,,,draw=none](**[130,230,dashed]6([,,,,draw=none](-[:-5,0.60,,,draw=none]\fscrm))))}

			}{An illustration of this delocalisation.}

			This delocalisation immensely increases the stability of the anion, thus carboxylic acid is a much stronger acid, with a \pKa of
			\num{4.75}, in contrast with that of ethanol and phenol, which are \num{15.9} and \ch{9.95} respectively.

			\subsubsection{Effect of Substituents}

				In a similar vein to alcohols and phenols, substituents along the carbon chain can act to disperse or intensify the negative
				charge on the anion, modifying its stability and hence the acidity of the group. Naturally, groups closer to the central
				carbon have a greater effect than groups further away.

				Electronegative groups and atoms, such as \ch{\Cl} or \ch{-NO2}, pull electron density away from the carboxyl carbon,
				dispersing the negative charge and stabilising the anion, hence increasing acidity.

				Conversely, electron-donating groups, most notably alkyl chains, have the opposite effect, intensifying the negative charge
				and thus destabilising the anion and decreasing acidity. Note that the length of the alkyl chain does not have any significant
				impact on the acidity.

			% end subsection

		% end subsection

		\subsection{Reactions as an Acid}

			For all intents and purposes, carboxylic acid functions like a weak mineral acid, and can do all the things they can do;
			neutralise bases, react with metals, and liberate \ch{CO2} gas from carbonates (\ch{CO3^2-}) and hydrogencarbonates (\ch{HCO3-}).

			The reactions will not be discussed in detail here, since they're basically normal acid-base reactions.

		% end subsection

		\subsection{Nucleophilic Acyl Substitution}

			\subsubsection{Formation of Acyl Chlorides}

				The carboxyl carbon undergoes a nucleophilic substitution, replacing the \ch{OH} group with a \ch{\Cl} group, forming
				an acyl chloride.

				Similar to alcohols, either \ch{P\Cl5}, \ch{P\Cl3}, or \ch{SO\Cl2} can be used. However, \ch{H\Cl} will
				not work to substitute the \ch{OH} group on a carboxylic acid, unlike an alcohol.


				\paragraph{Phosphorous Pentachloride (\ch{P\Cl5})}

				\vspace{1.5em}
				\vbox{\textbf{Conditions:}\tabto{35mm}Solid \ch{P\Cl5}, room temperature.}
				\vbox{\textbf{Observations:}\tabto{35mm}Formation of white fumes of \ch{H\Cl} gas.}

				\diagram[1.0]{
					\schemestart[0,1.5,thick]
						\chemfig{C(-[:180]!\molR)(=[:60]!\molO)(-[:300]!\molOH)}
						\hspace{5mm} + \hspace{5mm}
						\chemfig{P\Cl\sbs{5}}
						\arrow
						\chemfig{C(-[:180]!\molR)(=[:60]!\molO)(-[:300]!\molCl)}
						\hspace{2mm} + \hspace{2mm}
						\chemfig{PO\Cl\sbs{3}}
						\hspace{2mm} + \hspace{2mm}
						\chemfig{H\Cl}
					\schemestop
				}


				\paragraph{Phosphorous Trichloride (\ch{P\Cl3})}

				\vspace{1.5em}
				\vbox{\textbf{Conditions:}\tabto{35mm}Solid \ch{P\Cl3}, room temperature.}

				\diagram[1.0]{
					\schemestart[0,1.5,thick]
						\chemfig{3}
						\chemfig{C(-[:180]!\molR)(=[:60]!\molO)(-[:300]!\molOH)}
						\hspace{5mm} + \hspace{5mm}
						\chemfig{P\Cl\sbs{3}}
						\arrow
						\chemfig{3}
						\chemfig{C(-[:180]!\molR)(=[:60]!\molO)(-[:300]!\molCl)}
						\hspace{2mm} + \hspace{2mm}
						\chemfig{H\sbs{3}PO\sbs{3}}
					\schemestop
				}

				\pagebreak
				\paragraph{Thionyl Chloride (\ch{SO\Cl2})}

				This reaction is slightly preferred over the others, since both by-products (\ch{SO2} and \ch{H\Cl}) are
				gaseous, and would bubble out of the solution, leaving mainly the halogenoalkane in the reaction mixture.

				\vspace{1.5em}
				\vbox{\textbf{Conditions:}\tabto{35mm}Warm, liquid \ch{SO\Cl2}.}

				\vspace{0.75em}
				\vbox{\textbf{Observations:}\tabto{35mm}Formation of colourless, pungent \ch{SO2} gas,
											\tabto{35mm}white fumes of \ch{H\Cl} gas.}

				\diagram[1.0]{
					\schemestart[0,1.5,thick]
						\chemfig{C(-[:180]!\molR)(=[:60]!\molO)(-[:300]!\molOH)}
						\hspace{5mm} + \hspace{5mm}
						\chemfig{SO\Cl\sbs{2}}
						\arrow
						\chemfig{C(-[:180]!\molR)(=[:60]!\molO)(-[:300]!\molCl)}
						\hspace{2mm} + \hspace{2mm}
						\chemfig{SO\sbs{2}}
						\hspace{2mm} + \hspace{2mm}
						\chemfig{H\Cl}
					\schemestop
				}

			% end subsubsection


			\subsubsection{Esterification}

				As discussed \hyperlink{EsterificationCarboxylicAcids}{\boit{previously}}, carboxylic acids can undergo esterification
				with alcohols, using concentrated \ch{H2SO4} as a catalyst and dehydrating agent.

				Note that this is \itl{not} the preferred method of creating esters, due to the need for heating and catalysis, and the fact
				that phenols cannot be used.

				\vspace{1.5em}
				\vbox{\textbf{Conditions:}	\tabto{35mm}Carboxylic acid and alcohol,
											\tabto{35mm}Several drops of concentrated \ch{H2SO4}, heated under reflux.}

				\diagram[1.0]{
					\schemestart[0, 2.0, thick]
						\chemfig{!\molRon-[:0]C(=[:45]!\molO)(-[:315]@{oh}{\color{Red}O}|{\color{Red}H})}
						\hspace{2mm} + \hspace{2mm}
						\chemfig{!\molRtw(-[:180]!\molO-[:180,0.75]@{hyd}H)}
						\arrow{<=>[conc. \ch{H2SO4}][reflux]}
						\chemfig{C(-[:180]!\molRon)(=[:45]!\molO)(-[:315]!\molO-[:0]!\molRtw)}
						\hspace{2mm} + \hspace{2mm}
						\ch{H2O}
					\schemestop

					% circle the groups
					\chemmove{
						\draw[-latex, red, thick]
						(oh.east) circle(5mm)
						(hyd.center) circle(4mm);
					}
				}{Note that it is the circled groups that form the water.}


			% end subsubsection

		% end subsection

		\pagebreak
		\subsection{Reduction}

			Carboxylic acids can also be reduced to create alcohols. This reduction requires a strong reducing agent, so only
			\ch{Li\Al H4} can be used. See the \hyperlink{AppendixReducingAgents}{\boit{appendix}} for a list of reducing agents
			and their applicable uses.

			\vspace{1.5em}
			\vbox{\textbf{Conditions:}\tabto{35mm}\ch{Li\Al H4} in dry ether (diethyl ether).}

			\diagram[1.0]{
				\schemestart[0,1.5,thick]
					\chemfig{C(-[:180]!\molR)(=[:45]!\molO)(-[:315]!\molOH)}
					\arrow
					\chemfig{C(-[:0]!\molOH)(-[:90]H)(-[:270]H)(-[:180]!\molR)}
				\schemestop
			}

		% end subsection


		\subsection{Oxidation}

			With the strong oxidising agent \ch{KMnO4}, two special carboxylic acids can be \itl{even further oxidised} to give \ch{CO2}
			and \ch{H2O}; they are ethanedioic acid and methanoic acid.


			\vspace{1.5em}
			\vbox{\textbf{Conditions:}	\tabto{35mm}\ch{KMnO4} with dilute \ch{H2SO4},
										\tabto{35mm}heat under reflux.}
			\vspace{0.75em}
			\vbox{\textbf{Observations:}\tabto{35mm}\boit{\color{Plum}Purple} \ch{MnO4-} decolourises (\ch{Mn^2+} formed).}

			\diagram[1.0]{
				\schemestart[0,1.5,thick]
					\chemfig{C(-[:180]H)(=[:45]!\molO)(-[:315]!\molOH)}
					\arrow
					\ch{CO2}
					\hspace{2mm} + \hspace{2mm}
					\ch{H2O}
					\arrow(@c1.south east--.north east){0}[-90,.25]
					\chemfig{C(-[:135]!\molHO)(=[:225]!\molO)-C(=[:45]!\molO)(-[:315]!\molOH)}
					\arrow
					\ch{2 CO2}
					\hspace{2mm} + \hspace{2mm}
					\ch{H2O}
				\schemestop
			}

		% end subsection

	% end section







% end part






