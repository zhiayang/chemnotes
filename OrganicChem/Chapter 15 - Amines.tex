% Chapter 16 - Amines.tex
% Copyright (c) 2014 - 2016, zhiayang@gmail.com
% Licensed under the Apache License Version 2.0.



\pagebreak
\hypertarget{ChapterAmines}{}
\part{Amines}

	\section{Structure}
		\subsection{Aliphatic Amines}

			Aliphatic amines share a similar structure with ammonia --- three substituents around a central nitrogen atom.

			\begin{center}\begin{table}[ht]\renewcommand{\arraystretch}{1.4}
			\begin{tabu} to \textwidth {| X[c,m] | X[c,m] | X[c,m] | X[c,m] |}

				\hline
				% row one: molecules
				\vspace{2mm}\chemfig{\lewis{2:,\color{RoyalBlue}N}(-[:210]!\molR)(<[:300]H)(<:[:350]H)}				\vspace{2mm}	&
				\vspace{2mm}\chemfig{\lewis{2:,\color{RoyalBlue}N}(-[:210]!\molR)(<[:300]!\molR)(<:[:350]H)}		\vspace{2mm}	&
				\vspace{2mm}\chemfig{\lewis{2:,\color{RoyalBlue}N}(-[:210]!\molR)(<[:300]!\molR)(<:[:350]!\molR)}	\vspace{2mm}	&
				\vspace{2mm}\schemestart\chemleft[\chemfig{{\color{RoyalBlue}N}(-[:210]!\molR)(<[:300]!\molR)(<:[:350]!\molR)(-[:90]!\molR)}\chemright]\arrow(qt--){->}[0,0,draw=none]\schemestop\chemmove{\node[xshift=2mm] at (qt.north east) {+};}\vspace{2mm}\\

				\hline
				\vspace{2mm}Primary (\SI{1}{\degree})		\vspace{2mm} &
				\vspace{2mm}Secondary (\SI{2}{\degree})		\vspace{2mm} &
				\vspace{2mm}Tertiary (\SI{3}{\degree})		\vspace{2mm} &
				\vspace{2mm}Quaternary Ammonium Ion			\vspace{2mm} \\
				\hline

			\end{tabu}
			\end{table}\end{center}\vspace{-10mm}

		% end subsection


		\subsection{Phenylamines}

			Phenylamines also exist, where the nitrogen is directly bonded to a benzene ring. These too can be classified as primary, secondary,
			tertiary or quaternary.


			\begin{center}\begin{table}[ht]\renewcommand{\arraystretch}{1.4}
			\begin{tabu} to \textwidth {| X[c,m] | X[c,m] | X[c,m] | X[c,m] |}

				\hline
				% row one: molecules
				\vspace{2mm}\chemfig{**6(----(-[:90]!\molN(-[:30]H)(-[:150]H))--)}				\vspace{2mm}	&
				\vspace{2mm}\chemfig{**6(----(-[:90]!\molN(-[:30]!\molR)(-[:150]H))--)}			\vspace{2mm}	&
				\vspace{2mm}\chemfig{**6(----(-[:90]!\molN(-[:30]!\molR)(-[:150]!\molR))--)}	\vspace{2mm}	&
				\vspace{2mm}\chemfig{**6(----(-[:90]\chemabove{\color{RoyalBlue}N}{\hspace{5mm}\smfplus}(-[:0]!\molR)(-[:90]!\molR)(-[:180]!\molR))--)}	\vspace{2mm}\\

				\hline
				\vspace{2mm}Primary (\SI{1}{\degree})		\vspace{2mm} &
				\vspace{2mm}Secondary (\SI{2}{\degree})		\vspace{2mm} &
				\vspace{2mm}Tertiary (\SI{3}{\degree})		\vspace{2mm} &
				\vspace{2mm}Quaternary Ammonium Ion			\vspace{2mm} \\
				\hline

			\end{tabu}
			\end{table}\end{center}\vspace{-10mm}


		% end subsection

	% end section



	\section{Physical Properties}

		\paragraph{Melting and Boiling Points}

		Primary and secondary amines (and phenylamines) can form intermolecular hydrogen bonds, leading to increased melting and boiling points.
		However, they still have lower melting and boiling points compared to alcohols, due to the less polar nature of the \ch{N-H} bond compared
		to the \ch{O-H} bond.

		Tertiary amines do not have a hydrogen atom bonded to the central nitrogen, and as such cannot form hydrogen bonds, relying only on pd-pd
		and id-id interactions, thus having lower melting and boiling points compared to primary or secondary amines of a similar molar mass.


		\paragraph{Solubility}

		Aliphatic amines with relatively short carbon chains are able to form favourable solvent-solute interactions with water through hydrogen
		bonding, and as such are quite soluble in water.

		Naturally, the longer the carbon chain, the less soluble the amine becomes --- phenylamine is thus insoluble in water. Of course the
		solubility of amines in non-polar solvents depends mostly on the size of the alkyl groups attached.

	% end section



	\pagebreak
	\section{Creation of Amines}

		\subsection{Nucleophilic Substitution of Alkyl Halides}

			As covered \hyperlink{NucleophilicSubstitutionFormingAmines}{\boit{previously}}, the halogen atom can be substituted by \ch{NH3}
			in a nucleophilic reaction, giving a primary amine. Primary and secondary amines can also be used to give substituted amines.
			Excess \ch{NH3} is used to lower the possibility of multi-substitution.

			\vspace{1.5em}
			\vbox{\textbf{Conditions:}\tabto{35mm}Ethanolic concentrated \ch{NH3} in excess, heat in sealed tube.}

			\diagram[1.0]{
				\schemestart[0,1.5,thick]
					\chemfig{!\molR-[:0]!\molX}
					\hspace{2mm} + \hspace{2mm}
					\chemfig{\ch{NH3}}
					\arrow
					\chemfig{!\molR-[:0]\ch{NH2}}
					\hspace{2mm} + \hspace{2mm}
					\chemfig{\ch{HX}}
				\schemestop
			}

		% end subsection


		\subsection{Reduction of Nitriles}

			Nitriles can be reduced to give primary amines.

			\vspace{1.5em}
			\vbox{\textbf{Conditions:}	\tabto{35mm}\ch{Li\Al H4} in dry ether (diethyl ether), \itl{OR}
										\tabto{35mm}\ch{H2 \stG}, \ch{Ni} catalyst, high temperature and pressure.}

			\diagram[1.0]{
				\schemestart[0, 2.0, thick]
					\chemfig{!\molR-[:0]C~[:0]!\molN}
					\arrow{->[reduction][{[}H{]}]}
					\chemfig{C(-[:180]!\molR)(-[:90]H)(-[:270]H)(-[:0]!\molN(-[:30]H)(-[:330]H))}
				\schemestop
			}

		% end subsection


		% \pagebreak
		\subsection{Reduction of Nitrobenzene}

			This reaction forms a phenylamine, and can be had through the reduction of nitrobenzene. While the reaction produces an amine
			salt, excess \ch{NaOH \stAq} can be used to deprotonate the salt.

			\vspace{1.5em}
			\vbox{\textbf{Conditions:}	\tabto{35mm}Tin (\ch{Sn}) catalyst, excess concentrated \ch{H\Cl},
										\tabto{35mm}heat with reflux. Excess \ch{NaOH \stAq} later.}

			% note: size-fudging to fit everything on one page.
			\diagram[0.9]{
				\schemestart[0, 3.0, thick]
					\chemfig[yshift=-1.5em]{**6(----(-[:90]!\molNitro)--)}
					\arrow(.base east--.base west){->[Sn, excess conc. \ch{H\Cl}, reflux][\ch{NaOH \stAq}]}
					\chemfig[yshift=-1.5em]{**6(----(-[:90]!\molAmine)--)}
					\hspace{5mm} + \hspace{5mm}
					\ch{H2O}
				\schemestop
			}


		% end subsection

	% end section



	\pagebreak
	\section{Amine Reactions}

		\subsection{Basicity of Amines}

			Due to the lone pair of electrons on the central \ch{N} atom in amines, they can act as a weak Brønsted base to accept a proton
			via a dative bond.

			Note that the electron-donating nature of the alkyl groups also increases its nucleophilic character.

			\subsubsection{Aliphatic Amines}

				Aliphatic amines can act as a base, and the electron-donating alkyl groups can intensify the electron density on the
				central nitrogen, increasing the ability for it to attract a proton. Furthermore, once the conjugate acid is formed, the
				positive charge is stabilised by the donation of electrons.

			% end subsubsection


			\subsubsection{Phenylamines}

				Compared to aliphatic amines, phenylamines are far weaker bases. The lone pair on the nitrogen atom is delocalised into
				the $\pi$-system of the benzene ring, reducing the amine's ability to attract a proton.

			% end subsubsection



			\subsubsection{Effect of Substituents}

				In what is basically the opposite effect of acids, electron-donating groups for both aliphatic amines and phenylamines will
				increase their basicity, both by increasing the electron density to attract \ch{H+} ions and by stabilising the conjugate acid.

				Conversely, electron-withdrawing groups will reduce electron density and decrease the basicity of the amine.

			% end subsubsection

		% end subsection


		\pagebreak
		\subsection{Reactions as a Base}

			Amines can react with both organic acids (phenols and carboxylic acids) as well as mineral acids (eg. \ch{H2SO4}). Due to the
			formation of ion-dipole interactions, the salts formed are generally soluble in water.

			\vspace{1.5em}
			\vbox{\textbf{Conditions:}	\tabto{35mm}Organic or mineral acid, room temperature.}

			\diagram[1.0]{
				\schemestart[0, 1.0, thick]
					\ch{R-NH2}
					\hspace{2mm} + \hspace{2mm}
					\ch{H\Cl}
					\arrow
					\ch{R-NH3+ \Cl-}
				\schemestop
			}

		% end subsection


		\subsection{Nucleophilic Substitution of Alkyl Halides}

			Similar to how ammonia can be used in a nucleophilic substitution on alkyl halides to produce amines, amines themselves are actually
			stronger nucleophiles and can, themselves, substitute the halogen atom to form secondary and tertiary amines.

			Since more substituted amines are stronger nucleophiles than less substituted ones due to more electron-donating alkyl groups, an
			excess of the original amine is used to prevent further substitution. This is the same reason why excess \ch{NH3} is used.


			\vspace{1.5em}
			\vbox{\textbf{Conditions:}	\tabto{35mm}Ethanolic alkyl halide, heated in sealed tube.}

			\diagram[0.9]{
				\schemestart[0, 1.5, thick]
					\chemfig{!\molN(-[:180]!\molR)(-[:60]H)(-[:300]H)}
					\hspace{5mm} + \hspace{5mm}
					\chemfig{!\molRon-!\molX}
					\arrow
					\chemfig{!\molN(-[:180]!\molR)(-[:60]!\molRon)(-[:300]H)}
					\hspace{5mm} + \hspace{5mm}
					\chemfig{H-!\molX}
					\arrow(@c1.south east--.north east){0}[-90,.5]
					\chemfig{!\molN(-[:180]!\molRon)(-[:60]!\molRtw)(-[:300]H)}
					\hspace{5mm} + \hspace{5mm}
					\chemfig{!\molRth-!\molX}
					\arrow
					\chemfig{!\molN(-[:180]!\molRon)(-[:60]!\molRtw)(-[:300]!\molRth)}
					\hspace{5mm} + \hspace{5mm}
					\chemfig{H-!\molX}
					\arrow(@c3.south east--.north east){0}[-90,.5]
					\chemfig{!\molN(-[:180]!\molRon)(-[:60]!\molRtw)(-[:300]!\molRth)}
					\hspace{5mm} + \hspace{5mm}
					\chemfig{!\molRfr-!\molX}
					\arrow
					\chemfig{!\molN(-[:180]!\molRon)(-[:90]!\molRtw)(-[:0]!\molRth)(-[:270]!\molRfr)}
					\hspace{5mm} + \hspace{5mm}
					\chemfig{H-!\molX}
				\schemestop
			}

		% end subsection




		\pagebreak
		\subsection{Nucleophilic Acyl Substitution of Acyl Chlorides}

			As covered in \hyperlink{AcylChloridesReactionWithAmines}{\boit{acyl chlorides}}, amines can undergo nucleophilic substitution
			with the chlorine atom in an acyl chloride to form amides.

			Only primary and secondary amines can react in this manner due to them having unsubstituted hydrogen atoms.

			\vspace{1.5em}
			\vbox{\textbf{Conditions:}\tabto{35mm}Acyl chloride, amine of choice in excess, room temperature.}
			\vbox{\textbf{Observations:}\tabto{35mm}Formation of white fumes of \ch{H\Cl} gas.}

			\diagram[1.0]{
				\schemestart[0, 1.5, thick]
					\chemfig{C(-[:180]!\molR)(=[:60]!\molO)(-[:300]!\molCl)}
					\hspace{5mm} + \hspace{5mm}
					\chemfig{!\molN(-[:180]H)(-[:60]!\molRon)(-[:300]!\molRtw)}
					\arrow(.mid east--.mid west)
					\chemfig{C(-[:180]!\molR)(=[:60]!\molO)(-[:300]!\molN(-[:0]!\molRon)(-[:240]!\molRtw))}
					\hspace{5mm} + \hspace{5mm}
					\chemfig{H\Cl}
				\schemestop
			}

		% end subsection




		\subsection{Electrophilic Addition of Aqueous \ch{Br2}}

			In a similar manner to phenols, the lone pair on the nitrogen atom is delocalised into the benzene ring, and thus increases its
			electron density. This makes it more susceptible to electrophile attacks, such as those by \ch{Br-}, and thus bromine can be
			electrophilically added to phenylamines without a Lewis acid catalyst.

			\vspace{1.5em}
			\vbox{\textbf{Conditions:}\tabto{35mm}\ch{Br2 \stAq}, room temperature.}\vspace{0.75em}
			\vbox{\textbf{Observations:}\tabto{35mm}\boit{\color{Dandelion}Yellow} \ch{Br2 \stAq} decolourises,
										\tabto{35mm}White precipitate of 2,4,6-tribromophenylamine formed.}

			\diagram[1.0]{
				\schemestart[0, 1.5, thick]
					\chemfig[yshift=-1.5em]{**6(----(-[:90]!\molNitro)--)}
					\hspace{5mm} + \hspace{5mm}
					\ch{3 Br2 \stAq}
					\arrow(.mid east--.mid west)
					\chemfig[yshift=-1.5em]{**6(-(-[:270]!\molBr)--(-[:30]!\molBr)-(-[:90]!\molAmine)-(-[:150]!\molBr)-)}
					\hspace{5mm} + \hspace{5mm}
					\ch{3 HBr}
				\schemestop
			}


		% end subsection


	% end section


% end part
