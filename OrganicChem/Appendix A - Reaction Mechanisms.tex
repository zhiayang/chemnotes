% Appendix 1 - List of Reaction Mechanisms.tex
% Copyright (c) 2014 - 2016, zhiayang@gmail.com
% Licensed under the Apache License Version 2.0.

\pagebreak
\hypertarget{AppendixReactionMechanisms}{}
\part{Reaction Mechanisms}

	\hypertarget{AppendixElectrophilicAddition}{}
	\section{Electrophilic Addition}

		Electrophilic Addition the main reaction mechanism for alkenes, and involves an electrophile attacking the electron-rich
		π-bond of the alkene.


		\subsection{Markovnikov's Rule}

			This rule governs the major product that is formed when asymmetrical compounds are electrophilically added to alkenes,
			such as hydrogen halides. It does not apply to symmetrical reactants like \ch{\chlorine2} or \ch{Br2}.

			The rule states basically states that when a hydrogen compound with the general form \ch{H-X} is added to an
			alkene that is asymmetrical about the double-bond, the hydrogen atom will be added to the carbon with \textit{more}
			existing hydrogen atoms. Note that \textit{X} can be a halogen, a hydroxide (ie. \ch{H-X} is \ch{H2O}) or some other
			electronegative species.

			The mirror of the rule, used more with the cyclic halonium ion, is that the nucleophile will be added to the more substituted
			carbon atom.


			\pagebreak
			\hypertarget{AppendixMarkovnikovsRuleIllustration}{}
			\subsubsection{Addition of \ch{H-X} to Unsymmetrical Alkenes}

				The behaviour that Markovnikov's rule predicts can be derived from the fact that adding the non-hydrogen species to the least substituted carbon would
				result in a less stable intermediate being formed. This can be illustrated using the electrophilic addition of
				\ch{HBr} to prop-1-ene (\ch{C3H6}).

				If the Markovnikov's rule is \textit{not} followed, the following will take place:


				\diagram[0.85]{
					\schemestart[0, 1.5, thick]
					\chemfig{C(-[:90]H)(-[:180]H)(-[:270]H)-C(-[:270]H)=[@{b1}:0]C(-[:45]H)(-[:315]H)}
					\hspace{5mm} + \hspace{5mm}
					\chemfig{@{hyd}\chembelow{H}{\smdeltap}-[@{b2}:0]@{hal}\chembelow{{\color{Mahogany}Br}}{\smdeltam}}
					\arrow{->[slow]}
					\chemfig{\smfplus|C(-[:90,,2]H)(-[:270,,2]H)-C(-[:90]H)(-[:270]H)-!\molMe}
					\hspace{5mm} + \hspace{5mm}
					\chemfig{!\molBr\mch}
					\schemestop

					\chemmove{\draw[-Stealth,line width=0.4mm,shorten <=2mm,shorten >=1mm](b1) .. controls +(90:20mm) and +(90:20mm) .. (hyd);}
					\chemmove{\draw[-Stealth,line width=0.4mm,shorten <=2mm,shorten >=1mm](b2) .. controls +(90:8mm) and +(90:8mm) .. (hal);}
				}{The hydrogen was added to the centre carbon, in violation of Markovnikov's rule.}


				The positively-charged carbon on the intermediate only has 1 electron-donating alkyl group to
				stabilise its charge --- compare this to the example below where Markovnikov's Rule is followed:


				\diagram[0.85]{
					\schemestart[0, 1.5, thick]
					\chemfig{C(-[:90]H)(-[:180]H)(-[:270]H)-C(-[:270]H)=[@{b1}:0]C(-[:45]H)(-[:315]H)}
					\hspace{5mm} + \hspace{5mm}
					\chemfig{@{hyd}\chembelow{H}{\smdeltap}-[@{b2}:0]@{hal}\chembelow{{\color{Mahogany}Br}}{\smdeltam}}
					\arrow{->[slow]}
					\chemfig{!\molMeR-\chemabove{C}{\smfplus}(-[:270]H)-!\molMe-[:90,,,,draw=none]\phantom{H}}
					\hspace{5mm} + \hspace{5mm}
					\chemfig{!\molBr\mch}
					\schemestop

					\chemmove{\draw[-Stealth,line width=0.4mm,shorten <=2mm,shorten >=1mm](b1) .. controls +(90:20mm) and +(90:20mm) .. (hyd);}
					\chemmove{\draw[-Stealth,line width=0.4mm,shorten <=2mm,shorten >=1mm](b2) .. controls +(90:8mm) and +(90:8mm) .. (hal);}
				}{The hydrogen was added to the terminal carbon, following Markovnikov's rule.}

				In this case, the carbon atom containing the positive charge has 2 electron-donating alkyl groups attached to it,
				making this intermediate molecule more stable than the one above it.

				As such, it is more likely to form, and hence 2-bromopropane will be the \textit{major product}, and
				1-bromopropane will be the \textit{minor product}.

			% end subsubsection

		% end subsection




		\pagebreak
		\subsection{Electrophilic Addition of HX}

			In this reaction, the hydrogen and halogen atom are added across the double bond of the alkene. The hydrogen halide
			should be in a gaseous state for this reaction.

			\vspace{1.5em}
			\vbox{\textbf{Conditions:}\tabto{35mm}Gaseous HX (usually \ch{H\chlorine} or \ch{HBr}).}

			\paragraph{Step 1}


			\diagram[0.95]{
				\schemestart[0, 1.5, thick]
				\chemfig{C(-[:135]H)(-[:225]H)=[@{b1}:0]C(-[:45]H)(-[:315]H)}
				\hspace{5mm} + \hspace{5mm}
				\chemfig{@{hyd}\chembelow{H}{\smdeltap}-[@{b2}:0]@{hal}\chembelow{{\color{OliveGreen}X}}{\smdeltam}}
				\arrow{->[slow]}
				\chemfig{C(-[:90]H)(-[:180]H)(-[:270]H)-\chemabove{C}{\smfplus}(-[:45,,1]H)(-[:315,,1]H)}
				\hspace{5mm} + \hspace{5mm}
				\chemfig{{{\color{OliveGreen}X}}\mch}
				\schemestop

				\chemmove{\draw[-Stealth,line width=0.4mm,shorten <=2mm,shorten >=1mm](b1) .. controls +(90:20mm) and +(90:20mm) .. (hyd);}
				\chemmove{\draw[-Stealth,line width=0.4mm,shorten <=2mm,shorten >=1mm](b2) .. controls +(90:8mm) and +(90:8mm) .. (hal);}
			}

			This is the rate-determining step, which involves the breaking of the π-bond. The polar \ch{HX} molecule has to
			approach the electron cloud of the π-bond in the correct orientation (hydrogen-facing), where the π-bond electrons
			attack the electron deficient, \ch{"\ox[parse=false]{\delp,H}"} atom.

			The \ch{H-X} bond then undergoes heterolytic fission, producing a carbocation intermediate and a halide ion.


			% look better

			\paragraph{Step 2}


			\diagram[1.0]{
				\schemestart[0, 1.5, thick]
				\chemfig{C(-[:90]H)(-[:180]H)(-[:270]H)(-C|@{pl}\smfplus(-[:45,,1]H)(-[:315,,1]H))}
				\hspace{5mm} + \hspace{5mm}
				\chemfig{@{hal}\lewis{2:,\color{OliveGreen}X}|\mch}
				\arrow
				\chemfig{C(-[:90]H)(-[:180]H)(-[:270]H)-C(-[:90]{{\color{OliveGreen}X}})(-[:0]H)(-[:270]H)}
				\schemestop

				\chemmove{\draw[-Stealth,line width=0.4mm,shorten <=1.5mm,shorten >=1mm](hal) .. controls +(90:15mm) and +(35:15mm) .. (pl);}

			}{The electron pair arrow points to the plus charge on the carbon, \textit{not} the carbon atom itself.}

			This is the fast step; the bromide anion acts as a nucleophile, attacking the positively-charged carbocation.


			\paragraph{Overall Reaction}

			\diagram[1.0]{
				\schemestart[0, 1.5, thick]
				\chemfig{C(-[:135]H)(-[:225]H)=C(-[:45]H)(-[:315]H)}
				\hspace{5mm} + \hspace{5mm}
				\ch{HX}
				\arrow
				\chemfig{C(-[:90]H)(-[:180]H)(-[:270]H)-C(-[:90]{{\color{OliveGreen}X}})(-[:0]H)(-[:270]H)}
				\schemestop
			}

		% end subsection


		\pagebreak

		\subsection{Electrophilic Addition of \ch{X2}}

			The electrophilic addition of halogens to alkanes does not involve Markovnikov's rule, since it is symmetrical.
			As the non-polar halogen molecule approaches the high-electron-density π-bond, it is polarised, forming
			\deltap{} and \deltam{} partial charges. The reaction mechanism involves the formation of a \textit{cyclic halonium ion} --- the
			double bond breaks, and each carbon forms a single bond with one positive halide ion.

			\vspace{1.5em}

			\vbox{\textbf{Conditions:}	\tabto{35mm}No UV Light, gaseous \ch{X2}.}
			\vbox{\textbf{Observations:}\tabto{35mm}\boit{\color{Mahogany}Reddish-brown} \ch{Br2} / \boit{\color{YellowGreen}yellowish-green} \ch{\chlorine2} decolourises.}

			\paragraph{Step 1}

			This initial step is the slow, rate-determining one. The resulting cyclic halonium ion is highly unstable, due to the
			geometric constraints of having a three-membered ring.

			\diagram[0.90]{
				\schemestart[0, 1.5, thick]
				\chemfig{C(-[:135]H)(-[:225]H)=[@{b1}:0]C(-[:45]H)(-[:315]H)}
				\hspace{5mm} + \hspace{5mm}
				\chemfig{@{x1}\chembelow{{\color{OliveGreen}X}}{\smdeltap}-[@{b2}:0]@{x2}\chembelow{{\color{OliveGreen}X}}{\smdeltam}}
				\arrow{->[slow]}
				\chemfig{{\color{OliveGreen}X}\mch}
				\hspace{5mm} + \hspace{5mm}
				\chemfig{C?(-[:90]H)(-[:180]H)-C?(-[:90]H)(-[:0]H)(-[:240,,,1]{\color{OliveGreen}X}?|\pch)}
				\schemestop

				\chemmove{\draw[-Stealth,line width=0.4mm,shorten <=2mm,shorten >=1mm](b1) .. controls +(90:15mm) and +(90:15mm) .. (x1);}
				\chemmove{\draw[-Stealth,line width=0.4mm,shorten <=2mm,shorten >=1mm](b2) .. controls +(90:8mm) and +(90:8mm) .. (x2);}
			}


			\paragraph{Step 2}

			In the second, fast step, the negatively-charged bromide ion from the first step attacks the overall
			positively-charged cyclic bromonium ion. The bromide ion attacks one of the carbons attached to the positive bromide ion,
			breaking that bond.

			\diagram[1.0]{
				\schemestart[0, 1.5, thick]
				\chemfig{C?(-[:90]H)(-[:180]H)-@{carb}C?(-[:90]H)(-[:0]H)(-[@{b1}:240,,,1]@{x}{\color{OliveGreen}X}?|\pch)}
				\hspace{5mm} + \hspace{5mm}
				\chemfig{@{x1}\lewis{5:,\color{OliveGreen}X}\mch}
				\arrow{->[fast]}
				\chemfig{C(-[:90]H)(-[:180]H)(-[:270]{{\color{OliveGreen}X}})-C(-[:90]H)(-[:0]H)(-[:270]{{\color{OliveGreen}X}})}
				\schemestop

				\chemmove{\draw[-Stealth,line width=0.4mm,shorten <=3mm,shorten >=1mm](x1) .. controls +(215:15mm) and +(315:15mm) .. (carb);}
				\chemmove{\draw[-Stealth,line width=0.4mm,shorten <=2mm,shorten >=1mm](b1) .. controls +(330:8mm) and +(350:8mm) .. (x);}
			}

			\paragraph{Overall Reaction}

			\diagram[1.0]{
				\schemestart[0, 1.5, thick]
				\chemfig{C(-[:135]H)(-[:225]H)=C(-[:45]H)(-[:315]H)}
				\hspace{5mm} + \hspace{5mm}
				\ch{X2}
				\arrow
				\chemfig{C(-[:270]!\molX)(-[:90]H)(-[:180]H)-C(-[:270]!\molX)(-[:90]H)(-[:0]H)}
				\schemestop
			}

		% end subsection



		\pagebreak
		\subsection{Electrophilic Addition of Aqueous \ch{X2}}

			The mechanics of this reaction are basically the same as that of the electrophilic addition of \ch{Br2}, and the conditions and
			observations are fairly similar as well.

			Also, since the \ch{Br} atoms are not both added across the double bond in a symmetrical manner,
			Markovnikov's Rule applies --- the OH group will preferentially attack the carbon that has more stabilising alkyl groups.

			\vspace{1.5em}
			\vbox{\textbf{Conditions:}	\tabto{35mm}No UV Light, aqueous \ch{Br2}.}
			\vbox{\textbf{Observations:}\tabto{35mm}\boit{\color{Dandelion}Yellow} \ch{Br2 \stAq} decolourises.}



			\paragraph{Step 1}
			\diagram[0.90]{
				\schemestart[0, 1.5, thick]
				\chemfig{C(-[:135]H)(-[:225]H)=[@{b1}:0]C(-[:45]H)(-[:315]H)}
				\hspace{5mm} + \hspace{5mm}
				\chemfig{@{br1}\chembelow{{\color{Mahogany}Br}}{\smdeltap}-[@{b2}:0]@{br2}\chembelow{{\color{Mahogany}Br}}{\smdeltam}}
				\arrow{->[slow]}
				\chemfig{{\color{Mahogany}Br}\mch}
				\hspace{5mm} + \hspace{5mm}
				\chemfig{C?(-[:90]H)(-[:180]H)-C?(-[:90]H)(-[:0]H)(-[:240,,,1]{\color{Mahogany}Br}?|\pch)}
				\schemestop

				\chemmove{\draw[-Stealth,line width=0.4mm,shorten <=2mm,shorten >=1mm](b1) .. controls +(90:15mm) and +(90:15mm) .. (br1);}
				\chemmove{\draw[-Stealth,line width=0.4mm,shorten <=2mm,shorten >=1mm](b2) .. controls +(90:8mm) and +(90:8mm) .. (br2);}
			}{The cyclic bromonium ion is formed.}



			\paragraph{Step 2}

			Once the cyclic bromonium ion is formed however, it is susceptible to attack from
			any and all nucleophiles --- including water, which has lone pairs and is a stronger nucleophile \ch{Br-}. Since it is in much higher
			concentrations than \ch{Br-}, the primary product will now have an \ch{OH} group.

			\diagram[0.825]{
				\schemestart[0, 1.5, thick]
				\chemfig{C?(-[:90]H)(-[:180]H)-@{carb}C?(-[:90]H)(-[:0]H)(-[@{b1}:240,,,1]@{br}{\color{Mahogany}Br}?|\pch)}
				\hspace{5mm} + \hspace{5mm}
				\chemfig{@{oxy}\lewis{1:3:,\color{Red}O}(-[:322]H)(-[:218]H)}
				\arrow{->[fast]}
				\chemfig{C(-[:180]H)(-[:90]H)(-[:270]!\molOH)-C(-[:90]H)(-[:0]H)(-[:270]!\molBr)}
				\hspace{5mm} + \hspace{5mm}
				\chemfig{H-!\molBr}
				\schemestop

				\chemmove{\draw[-Stealth,line width=0.4mm,shorten <=2mm,shorten >=1mm](b1) .. controls +(330:8mm) and +(350:8mm) .. (br);}
				\chemmove{\draw[-Stealth,line width=0.4mm,shorten <=2mm,shorten >=1mm](oxy) .. controls +(135:15mm) and +(45:15mm) .. (carb);}
			}{Following Markovnikov's Rule, the \ch{OH-} group will attach to the more substituted carbon.}


			\paragraph{Overall Reaction}

			\diagram[0.75]{
				\schemestart[0, 1.5, thick]
				\chemfig{C(-[:135]H)(-[:225]H)=[@{b1}:0]C(-[:45]H)(-[:315]H)}
				\hspace{5mm} + \hspace{5mm}
				\ch{Br2 \stAq}
				\arrow(.base east--.base west)
				\chemname{\chemfig{C(-[:180]H)(-[:90]H)(-[:270]!\molOH)-C(-[:90]H)(-[:0]H)(-[:270]!\molBr)}}{(major)}
				\hspace{5mm} + \hspace{5mm}
				\chemname{\chemfig{C(-[:180]H)(-[:90]H)(-[:270]!\molBr)-C(-[:90]H)(-[:0]H)(-[:270]!\molBr)}}{(minor)}
				\schemestop

			}

		% end subsection


		\subsection{Electrophilic Addition of Steam (Hydration)}
			Under certain (usually industrial) conditions, alkenes can react with steam to form alcohols. Since this reaction involves
			the formation of a carbocation intermediate, the proportion of products will be governed by Markovnikov's Rule.

			\vspace{1.5em}
			\vbox{\textbf{Conditions:}	\tabto{35mm}\SI{300}{\celsius}, at \SI{70}{atm}, \ch{H3PO4} catalyst, \textit{OR}
										\tabto{35mm}Concentrated \ch{H2SO4}, \ch{H2O}, warming.}


			\paragraph{Step 0}

			The first step is to form \ch{H3O+ \stAq}. Simply put, it is aqueous \ch{H+}.

			\diagram{
				\schemestart[0, 1.5, thick]
				\chemfig{@{oxy}\lewis{1:3:,\color{Red}O}(-[:322]H)(-[:218]H)}
				\hspace{5mm} + \hspace{5mm}
				\chemfig{@{hyd}H\pch}
				\arrow
				\chemfig{!\molO|\pch(-[:315,,1]H)(-[:225]H)(-[:90]H)}
				\schemestop

				\chemmove{\draw[-Stealth,line width=0.4mm,shorten <=2mm,shorten >=1mm](oxy) .. controls +(45:8mm) and +(135:8mm) .. (hyd);}
			}


			\paragraph{Step 1}

			The electron-rich double-bond (\ch{C=C}) is then attacked by one of the \ch{H} atoms on the \ch{H3O+}, acting as an
			electrophile.


			\diagram[0.8]{
				\schemestart[0, 1.5, thick]
				\chemfig{@{oxy}{\color{Red}O}|\pch(-[:315,,1]H)(-[:225]H)(-[@{bond}:90]@{hyd}H)}
				\hspace{5mm} + \hspace{5mm}
				\chemfig{C(-[:135]H)(-[:225]!\molMeR)=[@{db}:0]C(-[:45]H)(-[:315]H)}
				\arrow
				\chemfig{\chemabove{C}{\smfplus}(-[:270]!\molMe)(-[:180]H)-C(-[:270]H)(-[:90]H)(-[:0]H)}
				\hspace{5mm} + \hspace{5mm}
				\chemfig{\water}
				\schemestop

				\chemmove{\draw[-Stealth,line width=0.4mm,shorten <=2mm,shorten >=1mm](bond) .. controls +(180:7mm) and +(160:7mm) .. (oxy);}
				\chemmove{\draw[-Stealth,line width=0.4mm,shorten <=2mm,shorten >=1mm](db) .. controls +(90:20mm) and +(0:8mm) .. (hyd);}
			}

			Since the alkene is not symmetrical, there are two product possibilities; hence, the positive charge will be on the carbon atom
			with more electron donating alkyl groups, in this case carbon 2.


			\pagebreak
			\paragraph{Step 2}

			One of the lone pairs on the oxygen atom of a water molecule acts as an nucleophile, attacking the newly-formed carbocation. This
			results in a protonated alcohol, which is basically an alcohol with an extra \ch{H} atom on the \ch{OH} group.

			\diagram{
				\schemestart[0, 1.5, thick]
				\chemfig{@{oxy}\lewis{1:7:,\color{Red}O}(-[:128]H)(-[:232]H)}
				\hspace{5mm} + \hspace{5mm}
				\chemfig{@{carb}\chemabove{C}{\smfplus}(-[:270]!\molMe)(-[:180]H)-C(-[:270]H)(-[:90]H)(-[:0]H)}
				\arrow
				\chemfig{C(-[:90]\lewis{2:,\color{Red}O}|\pch(-[:150]H)(-[:210]H))(-[:270]!\molMe)(-[:180]H)-C(-[:270]H)(-[:90]H)(-[:0]H)}
				\schemestop

				\chemmove{\draw[-Stealth,line width=0.4mm,shorten <=2mm,shorten >=3mm](oxy) .. controls +(45:12mm) and +(110:12mm) .. (carb);}
			}

			\paragraph{Step 3}

			The acid catalyst is regenerated as a lone pair on another water molecule's \ch{O} atom attacks the extra \ch{H} of the \ch{OH} group.
			This forms the alcohol proper, and the \ch{H3O+} ion.

			\diagram[0.8]{
				\schemestart[0, 1.5, thick]
				\chemfig{C(-[:90]@{o1}\lewis{2:,\color{Red}O}|\pch(-[@{bond}:150]@{hyd}H)(-[:210]H))(-[:270]!\molMe)(-[:180]H)-C(-[:270]H)(-[:90]H)(-[:0]H)}
				\hspace{5mm} + \hspace{5mm}
				\chemfig{@{o2}\lewis{3:5:,\color{Red}O}(-[:52]H)(-[:308]H)}
				\arrow
				\chemfig{C(-[:270]!\molMe)(-[:180]H)(-[:90]!\molOH)-C(-[:90]H)(-[:0]H)(-[:270]H)}
				\hspace{5mm} + \hspace{5mm}
				\chemfig{{\color{Red}O}|\pch(-[:315,,1]H)(-[:225]H)(-[:90]H)}
				\schemestop

				\chemmove{\draw[-Stealth,line width=0.4mm,shorten <=2mm,shorten >=2mm](bond) .. controls +(45:12mm) and +(0:10mm) .. (o1);}
				\chemmove{\draw[-Stealth,line width=0.4mm,shorten <=2mm,shorten >=1mm](o2) .. controls +(135:25mm) and +(45:15mm) .. (hyd);}
			}

			\paragraph{Overall Reaction}



			\diagram[1.0]{
				\schemestart[0, 1.5, thick]
				\chemfig{C(-[:135]H)(-[:225]!\molMeR)=[:0]C(-[:45]H)(-[:315]H)}
				\hspace{5mm} + \hspace{5mm}
				\chemfig{\water}
				\arrow
				\chemfig{C(-[:270]!\molMe)(-[:180]H)(-[:90]!\molOH)-C(-[:90]H)(-[:0]H)(-[:270]H)}
				\schemestop
			}

	% end section

































	\pagebreak
	\hypertarget{AppendixElectrophilicSubstitution}{}
	\section{Electrophilic Substitution}

		In general, benzene rings undergo electrophilic substitution, since it is not possible for more bonds to be formed.

		The delocalised π-system of benzene has a very high electron density, and thus is a prime target for electrophiles, which
		will substitute the \ch{H} atoms on the ring. Thus, the most common form of reaction involving benzenes is electrophilic
		substitution, barring special conditions and requirements.

		\paragraph{Step 1}

		In the first, rate-determining step, the aromaticity of the benzene ring is partially and temporarily broken, disrupted by
		the attacking electrophile.

		\diagram[1.0]{
			\schemestart[0, 1.5, thick]
				\chemfig{**6(---@{ring}---)}
				\arrow{0}[,0]		% used for alignment
				\hspace{5mm} + \hspace{5mm}
				\chemfig{@{el}E}
				\arrow(--.base west){->[slow]}
				\chemfig[yshift=\the\dimexpr-1.5em\relax]{**[60,-240]6(----(-[:135]H)(-[:45]E)--)(-[:30,,,,draw=none]+)}
			\schemestop

			\chemmove{\draw[-Stealth,line width=0.4mm,shorten <=-4mm,shorten >=1mm](ring) .. controls +(40:8mm) and +(120:8mm) .. (el);}

		}{Note that the \enquote{+} is drawn in the centre of the ring, not on any one carbon.}



		Two electrons out of six from the delocalised π-system are used to form the bond between the electrophile, E, and the carbon.
		Thus, there is a positive charge on the carbon; due to the delocalised nature of the π-system however, this positive charge is
		delocalised across \textit{all 6 carbons}, making it much more stable than a simple carbocation.

		However, the activation energy for this step is still large, and only strong electrophiles are able to attack the benzene ring without catalysts.

		\pagebreak
		\paragraph{Step 2}

		Next, a nucleophile (\chlewis{180}{Nu-} in this example) attacks the hydrogen attached to the hydrogen on the carbon atom,
		restoring the aromaticity of the benzene ring. The new substituent is now in place, and the two electrons in the \ch{C-H}
		bond are returned to the π-system.


		\diagram[1.0]{
			\schemestart[0, 1.5, thick]
				\chemfig[yshift=-1.5em]{**[60,-240]6(----(-[@{bond}:135]@{hyd}H)(-[:45]E)--)(-[:30,,,,draw=none]@{pl}+)}
				\arrow(.base east--){0}[,0]		% alignment purposes
				\hspace{5mm} + \hspace{5mm}
				\chemfig{@{nu}\lewis{2:,N}|u\mch}
				\arrow(--.mid west){->[fast]}
				\chemfig{\ch{HNu}}
				\hspace{5mm} + \hspace{5mm}
				\chemfig[yshift=-1.5em]{**6(----(-[:90]E)--)}
			\schemestop

			\chemmove{\draw[-Stealth,line width=0.4mm,shorten <=2mm,shorten >=1mm](nu) .. controls +(90:15mm) and +(45:15mm) .. (hyd);}
			\chemmove{\draw[-Stealth,line width=0.4mm,shorten <=1mm,shorten >=1mm](bond) .. controls +(225:8mm) and +(90:8mm) .. (pl);}
		}


		Note that the arenium ion (which is the partially delocalised benzene) has 5 \sptwo carbons, and one \spthree carbon. This
		results in a disruption of the planar structure of benzene --- it is restored once the substitution is completed.




		\pagebreak
		\subsection{Nitration of Benzene}

			The nitration of benzene involves the substitution of one of the \ch{H} atoms on the benzene with a nitro (\ch{-NO2}) group.
			It has a number of specific requirements:

			\vspace{1.5em}
			\vbox{\textbf{Conditions:}	\tabto{35mm}Concentrated \ch{HNO3}, concentrated \ch{H2SO4} catalyst.
										\tabto{35mm}\textit{Constant} temperature of \SI{50}{\celsius}.}
			\vspace{0.75em}
			\vbox{\textbf{Observations:}\tabto{35mm}\boit{\color{Goldenrod}Pale yellow} oily liquid, nitrobenzene.}


			\vspace{1.0em}
			\paragraph{A New Electrophile}

			Since \ch{H2SO4} is a stronger acid than \ch{HNO3}, it donates a proton to \ch{HNO3}, forming \ch{H2O}, \ch{HSO4-}, and \ch{NO2+}, the
			electrophile. Next, another molecule of \ch{H2SO4} then forms \ch{H+ \stAq}, or \ch{H3O+}. The overall equation is as such:

			\diagram{
				\schemestart[0, 1.0, thick]
					\ch{2 H2SO4} \hspace{2mm} + \hspace{2mm} \ch{HNO3}
					\arrow
					\ch{NO2+} \hspace{2mm} + \hspace{2mm} \ch{H3O+} \hspace{2mm} + \hspace{2mm} \ch{2 HSO4-}
				\schemestop
			}{The catalyst \ch{H2SO4} is restored in a later step.}



			\paragraph{The π-Electrons Strike Back}

			Now that the electrophile \ch{NO2+} has been formed, it is attacked by the π-system. As with all electrophilic substitutions, this
			involves the breaking of the aromatic system, and is the slow step. The mechanism follows the general mechanism outlined above.

			\diagram[1.0]{
				\schemestart[0, 1.5, thick]
				\chemfig{**6(---@{ring}---)}
				\arrow{0}[,0]		% used for alignment
				\hspace{5mm} + \hspace{5mm}
				\chemfig{@{el}\ch{NO2+}}
				\arrow(--.base west){->[slow]}
				\chemfig[yshift=\the\dimexpr-1.5em\relax]{**[60,-240]6(----(-[:135]H)(-[:45]!\molNitro)--)(-[:30,,,,draw=none]+)}
				\schemestop

				\chemmove{\draw[-Stealth,line width=0.4mm,shorten <=-4mm,shorten >=1mm](ring) .. controls +(40:8mm) and +(120:8mm) .. (el);}
			}


			\pagebreak
			\paragraph{Return of the Aromaticity}

			The \ch{HSO4-} intermediate acts as a nucleophile and attacks the \ch{H} atom bonded to the benzene intermediate. This
			restores both the π-system of the benzene ring, as well as the catalyst, \ch{H2SO4}.


			% TODO: fix alignment (more?)
			\diagram[0.85]{
				\schemestart[0, 1.5, thick]
				\chemfig{!\molS(=[:180]!\molO)(=[:270]!\molO)(-[:90]@{oxy}\lewis{2:,\color{Red}O}|\mch)(-[:0]!\molOH)}
				\hspace{5mm} + \hspace{5mm}
				\chemfig[yshift=-1.5em]{**[60,-240]6(----(-[@{bond}:135]@{hyd}H)(-[:45]!\molNitro)--)(-[:30,,,,draw=none]@{plus}+)}
				\arrow(.mid east--.mid west){->[fast]}
				\chemfig[yshift=-1.5em]{**6(----(-[:90]!\molNitro)--)}
				\hspace{5mm} + \hspace{5mm}
				\chemfig{!\molS(=[:180]!\molO)(=[:270]!\molO)(-[:90]!\molOH)(-[:0]!\molOH)}
				\schemestop

				\chemmove{\draw[-Stealth,line width=0.4mm,shorten <=2mm,shorten >=1mm](oxy) .. controls +(90:15mm) and +(120:15mm) .. (hyd);}
				\chemmove{\draw[-Stealth,line width=0.4mm,shorten <=1mm,shorten >=1mm](bond) .. controls +(225:8mm) and +(90:8mm) .. (plus);}
			}



			\vspace{1.0em}
			\paragraph{The Product Awakens}

			\diagram[1.0]{
				\schemestart[0, 2.0, thick]
				\chemfig{\ch{HNO3}}
				\hspace{5mm} + \hspace{5mm}
				\chemfig[yshift=-1.5em]{**6(------)}
				\arrow(.base east--.base west){->[\ch{H2SO4}][\SI{50}{\celsius}]}
				\chemfig[yshift=-1.5em]{**6(----(-[:90]!\molNitro)--)}
				\hspace{5mm} + \hspace{5mm}
				\chemfig{\ch{H2O}}
				\schemestop
			}















		\pagebreak
		\subsection{Halogenation of Benzene}

			Halogenation of benzene requires rather specific conditions, such as anhydrous \ch{FeBr3} or \ch{Fe\chlorine2} (for a reaction
			with bromine and chlorine respectively), and a warm environment.

			Aluminium-based analogues of these catalysts (\ch{\aluminium Br3}, \ch{\aluminium\chlorine3}) can also be used, as can pure
			filings of the metal, in which case the catalyst will be generated \textit{in-situ} (\ch{2 Fe \stS} + \ch{3 Br2 \stL} -> \ch{2 FeBr3}).


			\vspace{1.5em}
			\vbox{\textbf{Conditions:}	\tabto{35mm}Warm, anhydrous \ch{FeBr3}, \ch{\aluminium Br3}, or \ch{Fe} / \ch{\aluminium} filings (for bromine),
										\tabto{35mm}Anhydrous \ch{Fe\chlorine3}, \ch{\aluminium \chlorine3}, or
													\ch{Fe} / \ch{\aluminium} filings (for chlorine)}

			\vspace{0.75em}
			\vbox{\textbf{Observations:}\tabto{35mm}\boit{\color{Mahogany}Reddish-brown} \ch{Br2} / \boit{\color{YellowGreen}yellowish-green} \ch{\chlorine2} decolourises.
										\tabto{35mm}Formation of white fumes of \ch{HX} gas.}


			\hypertarget{BenzeneHalogenationCatalyst}{}
			\subsubsection{Catalysts}

				Lewis acid catalysts must be used, since the \ch{Br-Br} and \ch{\chlorine-\chlorine} are only instantaneously polar (instantaneous
				dipole moments). As such, they are nowhere near strong enough to attack the benzene system on their own.

				Indeed, this can be used to distinguish between alkenes and benzenes, since the former does not require a catalyst for addition
				of halogens.

				Furthermore, the entire reaction must be conducted in the absence of water; the reaction mechanism for the lewis-acid catalyst
				involves accepting a lone pair, the lone pair on water can, and will, in sufficient concentrations, destroy the catalyst.

			% end subsubsection



			\subsubsection{Reaction Mechanism}

				\paragraph{Generation of Electrophile}

				The reaction below uses Iron (III) chlorine (\ch{Fe\chlorine3}) as an example, adding \ch{\chlorine} to benzene,
				and this reaction mechanism applies to aluminium-based catalysts as well.

				\diagram{
					\schemestart[0, 1.0, thick]
						\ch{FeBr3} \hspace{2mm} + \hspace{2mm} \ch{Br2}
						\arrow
						\ch{Br+} \hspace{2mm} + \hspace{2mm} \ch{FeBr4-}
					\schemestop
				}{The catalyst is \ch{FeBr4-}, and the \textit{electrophile} is \ch{Br+}}


				\pagebreak

				\paragraph{Formation of Benzene Intermediate}

				Again, this mechanism is similar in nature to electrophilic substitution in general. Now, the electrophile (\ch{Br+}) attacks
				the π-system, forming the arenium ion.

				\diagram[1.0]{
					\schemestart[0, 1.5, thick]
					\chemfig{**6(---@{ring}---)}
					\arrow{0}[,0]		% used for alignment
					\hspace{5mm} + \hspace{5mm}
					\chemfig{@{el}!\molBr}
					\arrow(--.base west){->[slow]}
					\chemfig[yshift=\the\dimexpr-1.5em\relax]{**[60,-240]6(----(-[:135]H)(-[:45]!\molBr)--)(-[:30,,,,draw=none]+)}
					\schemestop

					\chemmove{\draw[-Stealth,line width=0.4mm,shorten <=-4mm,shorten >=3mm](ring) .. controls +(40:8mm) and +(120:8mm) .. (el);}
				}


				\paragraph{Restoration of π-system and Aromaticity}

				In the final step, the \ch{FeBr4-} acts as a nucleophile, attacking the \ch{H} atom attached to the benzene intermediate. This
				regenerates the catalyst \ch{FeBr3}, and also forms \ch{HBr}.


				\diagram[0.85]{
					\schemestart[0, 1.5, thick]
					\chemfig{Fe(-[:180]!\molBr)(-[:0]!\molBr)(-[:90]@{brom}\lewis{2:,\color{Mahogany}Br}|\mch)(-[:270]!\molBr)}
					\hspace{5mm} + \hspace{5mm}
					\chemfig[yshift=-1.5em]{**[60,-240]6(----(-[@{bond}:135]@{hyd}H)(-[:45]!\molBr)--)(-[:30,,,,draw=none]@{plus}+)}
					\arrow(.mid east--.mid west){->[fast]}
					\chemfig[yshift=-1.5em]{**6(----(-[:90]!\molBr)--)}
					\hspace{5mm} + \hspace{5mm}
					\chemfig{\ch{FeBr3}}
					\hspace{5mm} + \hspace{5mm}
					\chemfig{\ch{HBr}}
					\schemestop

					\chemmove{\draw[-Stealth,line width=0.4mm,shorten <=2mm,shorten >=1mm](brom) .. controls +(90:15mm) and +(120:15mm) .. (hyd);}
					\chemmove{\draw[-Stealth,line width=0.4mm,shorten <=1mm,shorten >=1mm](bond) .. controls +(225:8mm) and +(90:8mm) .. (plus);}
				}



				\paragraph{Overall Reaction}

				\diagram[1.0]{
					\schemestart[0, 1.5, thick]
					\chemfig{\ch{Br2}}
					\hspace{5mm} + \hspace{5mm}
					\chemfig[yshift=-1.5em]{**6(------)}
					\arrow(.base east--.base west)
					\chemfig[yshift=-1.5em]{**6(----(-[:90]!\molBr)--)}
					\hspace{5mm} + \hspace{5mm}
					\chemfig{\ch{HBr}}
					\schemestop
				}

			% end subsubsection

		% end subsection

	% end section































	\pagebreak
	\hypertarget{AppendixNucleophilicAddition}{}
	\section{Nucleophilic Addition}

		Due to the electronegativity of the oxygen atom, the electron density is drawn towards it. Thus, the carbon atom is able
		to be attacked by electrophiles due to its partial positive charge.

		Note that aldehydes will be more reactive than ketones in nucleophilic addition, since they only have one electron-donating
		alkyl group on the central carbon, whereas ketones have two alkyl groups that are able to inductively donate electrons to
		stabilise the positive charge, slightly reducing reactivity.

		The mechanism is a two-step mechanism with the slow step as the first step. Here, the nucleophilic addition of \ch{CN} to
		a carbonyl will be illustrated.

		\paragraph{Step 1}

		Here, the central carbon is attacked by the nucleophile. This also serves to repel the electrons in the π-bond towards the
		electronegative \ch{O} atom, forming a stable alkoxide anion.

		\diagram[1.0]{
			\schemestart[0,1.5,thick]
				\chemfig{@{carb}\chemabove{C}{\smdeltap}(-[:120]!\molR)(-[:240]!\molR)(=[@{bond}:0]@{oxy}\chemabove{\color{Red}O}{\smdeltam})}
				\hspace{5mm} + \hspace{5mm}
				\chemfig{@{cn}\lewis{6:,C}|{\color{RoyalBlue}N}\mch}
				\arrow{->[slow]}
				\chemfig{C(-[:90]!\molR)(-[:180]!\molR)(-[:0]!\molO\mch)(-[:270]!\molCN)}
			\schemestop

			\chemmove{\draw[-Stealth,line width=0.4mm,shorten <=2mm,shorten >=1mm](cn) .. controls +(270:20mm) and +(270:20mm) .. (carb);}
			\chemmove{\draw[-Stealth,line width=0.4mm,shorten <=2mm,shorten >=1mm](bond) .. controls +(270:8mm) and +(270:8mm) .. (oxy);}
		}


		\paragraph{Step 2}

		Now, the alkoxide ion attacks an undissociated \ch{HCN} molecule, protonating itself and regenerating the \ch{CN-} nucleophile.

		\diagram[1.0]{
			\schemestart[0,1.5,thick]
				\chemfig{C(-[:90]!\molR)(-[:180]!\molR)(-[:0]@{oxy}\lewis{2:,\color{Red}O}|\mch)(-[:270]!\molCN)}
				\hspace{5mm} + \hspace{5mm}
				\chemfig{@{hyd}H-[@{bond}:0]@{carb}C|{\color{RoyalBlue}N}}
				\arrow{->[fast]}
				\chemfig{C(-[:90]!\molR)(-[:180]!\molR)(-[:0]!\molOH)(-[:270]!\molCN)}
				\hspace{5mm} + \hspace{5mm}
				\chemfig{!\molCN\mch}
			\schemestop


			\chemmove{\draw[-Stealth,line width=0.4mm,shorten <=2mm,shorten >=1mm](oxy) .. controls +(90:15mm) and +(90:15mm) .. (hyd);}
			\chemmove{\draw[-Stealth,line width=0.4mm,shorten <=2mm,shorten >=1mm](bond) .. controls +(270:8mm) and +(270:8mm) .. (carb);}
		}

		\pagebreak
		\paragraph{Overall Reaction}

		\diagram[1.0]{
			\schemestart[0,1.5,thick]
				\chemfig{C(-[:120]!\molR)(-[:240]!\molR)(=[:0]!\molO)}
				\hspace{5mm} + \hspace{5mm}
				\chemfig{H-!\molCN}
				\arrow
				\chemfig{C(-[:90]!\molR)(-[:180]!\molR)(-[:0]!\molOH)(-[:270]!\molCN)}
			\schemestop
		}


		\subsection{Stereoisomerism of Product}

			Since the carbonyl is trigonal planar, the nucleophile can attack from either the top or bottom side, with equal probability.
			Hence, if the product formed is chiral, the resulting solution will be a \textit{racemic mixture}.

			\diagram[1.0]{
				\schemestart[0, 1.5, thick]
				\chemfig{@{carb}\chemabove{C}{\hspace{7mm}\smdeltap}(=[:0]\chemabove{\color{Red}O}{\hspace{5mm}\smdeltam})(<:[:120]!\molPropyl)(<[:240]!\molEthyl)}
				\hspace{5mm}
				\chemfig{@{cn}\lewis{4:,C\color{RoyalBlue}N}\mch}
				\arrow(main.east--top.west){->[50\%]}[30]
				\chemfig{C(-[:90]!\molCN)(-[:210]!\molOH)(<:[:-15]!\molPropyl)(<[:315]!\molEthyl)}
				\arrow(@main.east--bottom.west){->[][50\%]}[-30]
				\chemfig{C(-[:270]!\molCN)(-[:150]!\molOH)(<:[:45]!\molPropyl)(<[:15]!\molEthyl)}
				\schemestop

			    \chemmove{\draw[-Stealth,line width=0.4mm,shorten <=2mm,shorten >=1mm]($(cn) + (-0.5,0)$) .. controls
			    	+(90:20mm) and +(90:20mm) .. (carb) node [pos=0.5,above,font=\footnotesize] {top face attack};}

			    \chemmove{\draw[-Stealth,line width=0.4mm,shorten <=2mm,shorten >=1mm]($(cn) + (-0.5,0)$) .. controls
			    	+(270:20mm) and +(270:20mm) .. (carb) node [pos=0.5,below,font=\footnotesize] {bottom face attack};}
			}


		% end subsection

	% end section









































	\pagebreak
	\hypertarget{AppendixNucleophilicSubstitution}{}
	\section{Nucleophilic Substitution}

		Due to the high relatively electronegativity of the halogen atom attached to the carbon atom, the carbon atom is
		\textit{electron-deficient}, which makes it more susceptible to attacks from nucleophiles (\textit{electron-rich}), and
		as such halogenoalkanes are fairly reactive.

		There are two primary mechanisms of nucleophilic substitution --- single-step and two-step reactions. Generally, primary
		halogenoalkanes (molecules with 1 or 0 alkyl groups attached to the halogen-containing carbon) react via a one-step
		mechanism, while tertiary halogenoalkanes react via a two-step mechanism. Secondary halogenoalkanes can react via either
		mechanism, depending on the specific molecule.

		The primary differentiating factor is the size of the substituents on the central carbon --- larger groups will hinder
		\sntwo{} substitution, forcing an \snone{} reaction.

		Either way, the overall reaction is as such:

		\diagram[1.0]{
			\schemestart[0, 1.5, thick]
			\chemfig{C(-[:0]!\molX)(-[:90]!\molR)(-[:180]!\molR)(-[:270]!\molR)}
			\hspace{5mm} + \hspace{5mm}
			\chemfig{!\molOH\mch}
			\arrow
			\chemfig{C(-[:0]!\molOH)(-[:90]!\molR)(-[:180]!\molR)(-[:270]!\molR)}
			\hspace{5mm} + \hspace{5mm}
			\chemfig{!\molX\mch}
			\schemestop
		}

		\pagebreak
		\subsection{Nucleophilic Substitution (\sntwo{}, single-step)}

			The nucleophilic substitution of \ch{CH3CH2Br} by \ch{OH-} is a one-step reaction. Take note that \sntwo{} is a
			\textit{one}-step reaction --- it is \sntwo{} because \textit{two} molecules are involved in the slow (only) step.

			\diagram[0.95]{
				\chemnameinit{\chemleft[
					\chemfig{C(-[:90]H)(<[:290]!\molMe)(<:[:250]H)(-[:0,,,,dash pattern=on 1.2mm off 1.2mm]!\molBr)
					(-[:180,,,,dash pattern=on 1.2mm off 1.2mm]!\molHO)}
				\chemright]}

				\schemestart[0, 1.0, thick]
				\chemname{\chemfig{\mch @{oh}\lewis{0:,\color{Red}HO}}}{}
				\hspace{5mm}
				\chemname{\chemfig{@{carb}\chemabove{C}{\hspace{2mm}\smdeltap}(<[:270]!\molMe)(<:[:225]H)(-[:120]H)
							(-[@{bond}:0]@{br}\chemabove{\color{Mahogany}Br}{\smdeltam})}}{}
				\arrow(.mid east--.mid west)
				\chemname{\chemleft[
					\chemfig{C(-[:90]H)(<[:290]!\molMe)(<:[:250]H)(-[:0,,,,dash pattern=on 1.2mm off 1.2mm]!\molBr)
					(-[:180,,,,dash pattern=on 1.2mm off 1.2mm]!\molHO)}
				\chemright]}{transition state}
				\arrow(.mid east--.mid west)
				\chemname{\chemfig{C(<[:270]!\molMe)(<:[:315]H)(-[:60]H)(-[:180]!\molHO)}}{}
				\hspace{5mm}
				\chemname{\chemfig{\lewis{4:,\color{Mahogany}Br}\mch}}{}
				\schemestop

			    \chemmove{\node[xshift=3mm] at (c2.north east) {–};}
			    \chemmove{\draw[-Stealth,line width=0.4mm,shorten <=2mm,shorten >=1mm](oh) .. controls +(0:5mm) and +(150:15mm) .. (carb);}
			    \chemmove{\draw[-Stealth,line width=0.4mm,shorten <=2mm,shorten >=1mm](bond) .. controls +(270:8mm) and +(270:8mm) .. (br);}
			}


			Due to the difference in electronegativity between the \ch{C} and \ch{Br} atoms, the \ch{C-Br} bond is polar. Thus,
			nucleophiles will be attracted towards the partially-positive carbon atom. They attack from the side opposite the
			leaving atom (\ch{Br} in this case) due to spatial constraints.

			As the nucleophile attacks, it donates a lone pair to the carbon atom to stabilise it, forming a bond. At the same time, this
			weakens the \ch{C-Br} bond, and it begins to break. At this stage, the \textit{transition state} is formed, which is an
			unstable, activated complex where bond-breaking and bond-forming occur simultaneously --- this is a \textit{one-step} mechanism.

			\subsubsection{Stereoisomerism of Product}

				Note that the chirality of the molecule will be \textit{inverted}, if it is chiral. Imagine the attacking nucleophile
				\enquote{pushing} the other groups away from itself.

				\diagram[1.0]{
					\schemestart[0, 1.5, thick]
					\chemfig{C(<[:270]!\molMe)(<:[:225]H)(-[:120,,,1]!\molEthyl)(-[@{bond}:0]!\molBr)}
					\arrow(.mid east--.mid west)
					\chemfig{C(<[:270]!\molMe)(<:[:315]H)(-[:60]!\molEthyl)(-[:180]!\molHO)}
					\schemestop
				}

			% end subsubsection



			\pagebreak
			\subsubsection{Reaction Pathway}
				Since this is a single-step reaction, there is one activation energy step, and the energy profile diagram looks like
				this:

				\diagram[1.0]{
					\begin{endiagram}[scale=2.0,offset=2,x-label-text=\footnotesize reaction progress,y-label-text=\footnotesize energy]
						\ENcurve[step=2]{2,4.5,0}
						\ShowNiveaus[shift=-1.5,length=3.0,niveau=N1-1]
						\ShowNiveaus[shift=1.5,length=3.0,niveau=N1-3]

						\ShowEa[label]
						\ShowGain[label,offset=-40mm]

						\draw[above] (N1-1) ++ (1,0) node {\small\hspace{-50mm}\ch{OH-} + \ch{CH3CH2Br}};
						\draw[above] (N1-3) ++ (1,0) node {\small\hspace{5mm}\ch{Br-} + \ch{CH3CH2OH}};

						\draw[above] (N1-2) ++ (0,0) node {\small\chemname{\chemleft[
							\chemfig[][scale=0.75]{C(-[:90]H)(<[:290]!\molMe)(<:[:250]H)(-[:0,,,,dash pattern=on 0.3mm off 0.3mm]!\molBr)
							(-[:180,,,,dash pattern=on 0.3mm off 0.3mm]!\molHO)}\chemright]}{transition state}};

					\end{endiagram}
				}
			% end subsubsection
		% end subsection


		\pagebreak
		\subsection{Nucleophilic Substitution (\snone{}, two-step)}

			The \snone{} reaction is a \textit{two-step} reaction, because larger groups attached to the central carbon
			often hinder the direct attack of nucleophiles, and thus require the formation of a carbocation. Only one molecule
			is involved in the slow (first) step, hence it is a unimolecular reaction.

			Again, the reaction hinges on the polar \ch{C-Br} bond, due to the differences in electronegativity.

			\paragraph{Step 1}

			\diagram[1.0]{
				\schemestart[0, 1.5, thick]
				\chemfig{\chembelow{C}{\hspace{5mm}\smdeltap}(-[:90]!\molPropyl)(-[:180]!\molMe)(-[:270]!\molEthyl)
					(-[@{bond}:0]@{br}\chembelow{\color{Mahogany}Br}{\hspace{5mm}\smdeltam})}
				\arrow{<=>[slow]}
				\chemfig{C|\pch(-[:90]!\molPropyl)(-[:180]!\molMe)(-[:270]!\molEthyl)}
				\hspace{5mm} + \hspace{5mm}
				\chemfig{!\molBr\mch}
				\schemestop

			    \chemmove{\draw[-Stealth,line width=0.4mm,shorten <=2mm,shorten >=1mm](bond) .. controls +(90:8mm) and +(90:8mm) .. (br);}
			}

			The \ch{C-Br} bond undergoes heterolytic fission, giving a carbocation intermediate, and a bromide ion. The reaction is more
			feasible due to the 3 electron-donating alkyl groups that stabilise the positive charge on the carbocation intermediate, which
			would not be possible in a primary alcohol.

			\paragraph{Step 2}

			\diagram[1.0]{
				\schemestart[0, 1.5, thick]
				\chemfig{@{c}C|\pch(-[:90]!\molPropyl)(-[:180]!\molMe)(-[:270]!\molEthyl)}
				\chemfig{@{oh}\lewis{4:,\color{Red}OH}\mch}
				\arrow{->[fast]}
				\chemfig{C(-[:90]!\molPropyl)(-[:180]!\molMe)(-[:270]!\molEthyl)(-[:0]!\molOH)}
				\schemestop

			    \chemmove{\draw[-Stealth,line width=0.4mm,shorten <=2mm,shorten >=1mm](oh) .. controls +(180:5mm) and +(60:15mm) .. (c);}
			}

			The carbocation intermediate is now attacked by the \ch{OH-} ion, forming the product.


			\pagebreak
			\subsubsection{Stereoisomerism of Product}
				Since the carbocation is trigonal planar (wrt. the central carbon), the nucleophile can attack from both sides of the
				carbocation, with equal probability. Hence, if the product formed is chiral, the resulting solution will be a
				\textit{racemic mixture}.

				\diagram[1.0]{
					\schemestart[0, 1.5, thick]
					\chemfig{@{carb}\chemabove{C}{\hspace{-5mm}\smfplus}(-[:180]!\molMe)(<:[:30]!\molPropyl)(<[:330]!\molEthyl)}
					\hspace{2mm}
					\chemfig{@{oh}\lewis{4:,\color{Red}OH}\mch}
					\arrow(main.east--top.west){->[50\%]}[30]
					\chemfig{C(-[:90]!\molOH)(-[:210]!\molMe)(<:[:-15]!\molPropyl)(<[:315]!\molEthyl)}
					\arrow(@main.east--bottom.west){->[][50\%]}[-30]
					\chemfig{C(-[:270]!\molOH)(-[:150]!\molMe)(<:[:45]!\molPropyl)(<[:15]!\molEthyl)}
					\schemestop

				    \chemmove{\draw[-Stealth,line width=0.4mm,shorten <=2mm,shorten >=1mm]($(oh) + (-0.5,0)$) .. controls
				    	+(90:20mm) and +(90:20mm) .. (carb) node [pos=0.5,above,font=\footnotesize] {top face attack};}

				    \chemmove{\draw[-Stealth,line width=0.4mm,shorten <=2mm,shorten >=1mm]($(oh) + (-0.5,0)$) .. controls
				    	+(270:20mm) and +(270:20mm) .. (carb) node [pos=0.5,below,font=\footnotesize] {bottom face attack};}
				}
			% end subsubsection


			\subsubsection{Reaction Pathway}
				Since this is a two-step reaction, there are two activation energies that must be overcome, and isolatable intermediates
				(the carbocation and bromide ion).


				\diagram[1.0]{
					\begin{endiagram}[scale=2.0,offset=2,x-label-text=\footnotesize reaction progress,y-label-text=\footnotesize energy]
						\ENcurve[looseness=0.40,step=1.25]{1.5,5.0,2.5,4.0,0}
						\ShowNiveaus[shift=-1.0,length=2.0,niveau=N1-1]
						\ShowNiveaus[shift=1.0,length=2.0,niveau=N1-5]

						\ShowEa[label,from={(N1-3) to (N1-4)},label-side=left,label-pos=0.80,label=$E_{a1}$]
						\ShowEa[label,from={(N1-1) to (N1-2)},label-pos=0.60,label=$E_{a2}$]

						\ShowGain[label,offset=-25mm]

						\draw[above] (N1-1) ++ (1,0) node {\small\hspace{-40mm}\ch{OH-} + \ch{R-Br}};
						\draw[above] (N1-5) ++ (1,0) node {\small\hspace{5mm}\ch{Br-} + \ch{R-OH}};
						\draw[below] (N1-3) ++ (1,0) node {\small\hspace{-15mm}\ch{R+} + \ch{Br-} + \ch{OH-}};

						% \draw[above] (N1-2) ++ (0,0) node {\small\chemname{\chemleft[
						% 	\chemfig[][scale=0.75]{C(-[:90]H)(<[:290]!\molMe)(<:[:250]H)(-[:0,,,,dash pattern=on 0.3mm off 0.3mm]!\molBr)
						% 	(-[:180,,,,dash pattern=on 0.3mm off 0.3mm]!\molHO)}\chemright]}{transition state}};

					\end{endiagram}
				}
			% end subsubsection
		% end subsection

	% end section

% end part





















