% Chapter 08 - Halogen Derivatives.tex
% Copyright (c) 2014 - 2016, zhiayang@gmail.com


\pagebreak
\hypertarget{ChapterHalogenDerivatives}{}
\part{Halogen Derivatives}

	\section{Halogenoalkanes (Alkyl Halides)}

		Halogenoalkanes are simply alkanes, with one or more halogen atoms taking the place of a hydrogen atom. They are classified in a
		similar way to alkanes and alcohols.

		\begin{center}\begin{table}[ht]\renewcommand{\arraystretch}{1.4}
		\begin{tabu} to \textwidth {| X[c,m] | X[c,m] | X[c,m] |}

			\hline
			% row one: molecules
			\vspace{2mm}	\chemfig{C(-[:0]!\molX)(-[:90]H)(-[:180]!\molR)(-[:270]H)}				\vspace{2mm}	&
			\vspace{2mm}	\chemfig{C(-[:0]!\molX)(-[:90]!\molR)(-[:180]!\molR)(-[:270]H)}			\vspace{2mm}	&
			\vspace{2mm}	\chemfig{C(-[:0]!\molX)(-[:90]!\molR)(-[:180]!\molR)(-[:270]!\molR)}	\vspace{2mm}	\\

			\hline
			Primary (\SI{1}{\degree})		&
			Secondary (\SI{2}{\degree})		&
			Tertiary (\SI{3}{\degree})		\\
			\hline

		\end{tabu}
		\end{table}\end{center}\vspace{-10mm}

		Methane-based molecules are classified as primary halogenoalkanes.

		\subsection{Physical Properties}

			Halogenoalkanes are generally colourless like their alkane counterparts, and might have a sweetish smell.

			Compared to their alkane counterparts, halogenoalkanes have a higher boiling point for the same carbon chain length,
			mainly due to the increased size of the halogen atom. The primary method of intermolecular bonding is still via
			instantaneous dipole-induced dipole interactions, which is strengthened by the larger electron cloud, and thus results in
			a higher \textit{polarisability} of the molecule.

			However, permanent-dipole interactions are still present, and contribute slightly to the increase in melting and boiling points.
			This is of more significance for chloroalkanes, since \ch{\chlorine} is more negative than both \ch{Br} and \ch{I}, hence creating
			a larger dipole moment.

			When comparing between different halogens with otherwise identical carbon chains, the melting and boiling points increases as
			the halogen atom's size increases –– the instantaneous dipole-induced dipole interaction plays a larger role in determining
			that.

		% end section

	% end section

	\pagebreak
	\section{Halogenoarenes}

		Halogenoarenes are simply molecules where the halogen atom is \textit{directly attached} to the benzene ring.

		\diagram[1.0]{

			\chemfig{**6(----(-[:90]!\molCl)--)}

		}{Chlorobenzene is a halogenoarene.}

	% end section



	\section{Creation of Halogenated Molecules}

		\subsection{Free Radical Substitution of Alkanes}

			As covered above, alkanes can undergo free radical substitution, replacing one or more of their hydrogen atoms with halogen
			atoms.

			\vspace{1.5em}

			\vbox{\textbf{Conditions:} \tabto{35mm}UV Light, \ch{Br2} / \ch{\chlorine2} gas}	% single line, not needed \vspace{0.5em}
			\vbox{\textbf{Observations:} \tabto{35mm}\boit{\color{Mahogany}Reddish-brown} \ch{Br2} / \textit{\color{YellowGreen}yellowish-green} \ch{\chlorine2} decolourises.}

			\diagram[1.0]{
				\schemestart[0,1.5,thick]
					\chemfig{!\molR(-[:0]!\molMe)}
					\hspace{2mm} + \hspace{2mm}
					\chemfig{\ch{X2}}
					\arrow
					\chemfig{\ch{HX}}
					\hspace{2mm} + \hspace{2mm}
					\chemfig{!\molR(-[:0]CH\sbs{2}X)}
				\schemestop
			}

			However, this is usually a \textit{bad} way to prepare halogenoalkanes, due to the mix of products that will be formed, including
			isomers and multisubstituted molecules. It is difficult (or nigh impossible) to control the extent or position of substitution,
			but to decreases the chances of multisubstitution, the alkane should be used in excess, as a last resort.

		% end section


		\pagebreak
		\subsection{Electrophilic Addition to Alkenes}

			\subsubsection{Addition of \ch{X2}}

				Halogen molecules (\ch{X2}) can be added to alkenes to form halogenoalkanes, as covered above. Naturally, this results in
				having two halogen atoms in the molecule, or a \textit{dihalide}.

				\vspace{1.5em}
				\vbox{\textbf{Conditions:}\tabto{35mm}No UV Light, gaseous \ch{X2}.}	% again, single line, no need for \vspace{0.5em}
				\vbox{\textbf{Observations:}\tabto{35mm}\boit{\color{Mahogany}Reddish-brown} \ch{Br2} / \boit{\color{YellowGreen}yellowish-green} \ch{\chlorine2} decolourises.}

				\diagram[1.0]{
					\schemestart[0, 1.5, thick]
					\chemfig{C(-[:135]H)(-[:225]H)=C(-[:45]H)(-[:315]H)}
					\hspace{5mm} + \hspace{5mm}
					\chemfig{!\molX-!\molX}
					\arrow
					\chemfig{C(-[:270]!\molX)(-[:90]H)(-[:180]H)-C(-[:270]!\molX)(-[:90]H)(-[:0]H)}
					\schemestop
				}
			% end subsubsection

			\subsubsection{Addition of \ch{HX}}

				Gaseous \ch{HX} can also be added, resulting in a monohalogenated molecule.

				\vspace{1.5em}
				\vbox{\textbf{Conditions:}\tabto{35mm}Gaseous HX (usually \ch{H\chlorine} or \ch{HBr}).}

				\diagram[1.0]{
					\schemestart[0, 1.5, thick]
					\chemfig{C(-[:135]H)(-[:225]H)=C(-[:45]H)(-[:315]H)}
					\hspace{5mm} + \hspace{5mm}
					\chemfig{H-{{\color{OliveGreen}X}}}
					\arrow
					\chemfig{C(-[:90]H)(-[:180]H)(-[:270]H)-C(-[:90]{{\color{OliveGreen}X}})(-[:0]H)(-[:270]H)}
					\schemestop
				}
			% end subsubsection
		% end section

		\pagebreak
		\hypertarget{NSubAlcohols}{}
		\subsection{Nucleophilic Substitution of \ch{OH} groups (Chlorine)}

			\subsubsection{Phosphorous Pentachloride (\ch{P\chlorine5})}

				\vspace{1.5em}
				\vbox{\textbf{Conditions:}\tabto{35mm}Solid \ch{P\chlorine5}, room temperature.}
				\vbox{\textbf{Observations:}\tabto{35mm}Formation of white fumes of \ch{H\chlorine} gas.}

				\diagram[1.0]{
					\schemestart[0,1.5,thick]
						\chemfig{!\molR-[:0]!\molOH}
						\hspace{2mm} + \hspace{2mm}
						\chemfig{P\chlorine\sbs{5}}
						\arrow
						\chemfig{!\molR-[:0]!\molCl}
						\hspace{2mm} + \hspace{2mm}
						\chemfig{PO\chlorine\sbs{3}}
						\hspace{2mm} + \hspace{2mm}
						\chemfig{H\chlorine}
					\schemestop
				}
			% end subsubsection


			\subsubsection{Phosphorous Trichloride (\ch{P\chlorine3})}

				\vspace{1.5em}
				\vbox{\textbf{Conditions:}\tabto{35mm}Solid \ch{P\chlorine3}, room temperature.}

				\diagram[1.0]{
					\schemestart[0,1.5,thick]
						\chemfig{3}
						\chemfig{!\molR-[:0]!\molOH}
						\hspace{2mm} + \hspace{2mm}
						\chemfig{P\chlorine\sbs{3}}
						\arrow
						\chemfig{3}
						\chemfig{!\molR-[:0]!\molCl}
						\hspace{2mm} + \hspace{2mm}
						\chemfig{H\sbs{3}PO\sbs{3}}
					\schemestop
				}
			% end subsubsection

			\subsubsection{Thionyl Chloride (\ch{SO\chlorine2})}

				This reaction is slightly preferred over the others, since both by-products (\ch{SO2} and \ch{H\chlorine}) are
				gaseous, and would bubble out of the solution, leaving mainly the halogenoalkane in the reaction mixture.

				\vspace{1.5em}
				\vbox{\textbf{Conditions:}\tabto{35mm}Warm, liquid \ch{SO\chlorine2}.}

				\vspace{0.75em}
				\vbox{\textbf{Observations:}\tabto{35mm}Formation of colourless, pungent \ch{SO2} gas,
											\tabto{35mm}white fumes of \ch{H\chlorine} gas.}

				\diagram[1.0]{
					\schemestart[0,1.5,thick]
						\chemfig{!\molR-[:0]!\molOH}
						\hspace{2mm} + \hspace{2mm}
						\chemfig{SO\chlorine\sbs{2}}
						\arrow
						\chemfig{!\molR-[:0]!\molCl}
						\hspace{2mm} + \hspace{2mm}
						\chemfig{SO\sbs{2}}
						\hspace{2mm} + \hspace{2mm}
						\chemfig{H\chlorine}
					\schemestop
				}
			% end subsubsection

			\pagebreak
			\subsubsection{Hydrogen Chloride (\ch{H\chlorine})}

				Here, a distinction must be made between \textit{primary}, \textit{secondary}, and \textit{tertiary} alcohols;
				tertiary alcohols will react fine with simply concentrated \ch{H\chlorine}.

				\vspace{1.5em}
				\vbox{\textbf{Conditions:}	\tabto{35mm}Concentrated \ch{H\chlorine \stAq} / gaseous \ch{H\chlorine}, \textit{OR}
											\tabto{35mm}Solid \ch{Na\chlorine}, concentrated \ch{H2SO4}, heat.}

				\diagram[1.0]{
					\schemestart[0,1.5,thick]
						\chemfig{!\molR-[:0]!\molOH}
						\hspace{2mm} + \hspace{2mm}
						\chemfig{\ch{H\chlorine}}
						\arrow
						\chemfig{!\molR-[:0]!\molCl}
						\hspace{2mm} + \hspace{2mm}
						\chemfig{\ch{H2O}}
					\schemestop
				}{\ch{H\chlorine} can be prepared \textit{in-situ} with \ch{Na\chlorine} and \ch{H2SO4}.}


				For primary and secondary alcohols, anhydrous \ch{Zn\chlorine2} catalyst must be used, and the reaction mixture
				must be heated. Apart from that, the reaction is similar.

				\vspace{1.5em}
				\vbox{\textbf{Conditions:}	\tabto{35mm}Concentrated \ch{H\chlorine \stAq} / gaseous \ch{H\chlorine}.
											\tabto{35mm}Anhydrous \ch{Zn\chlorine2} catalyst, heat.}

				\diagram[1.0]{
					\schemestart[0,1.5,thick]
						\chemfig{!\molR-[:0]!\molOH}
						\hspace{2mm} + \hspace{2mm}
						\chemfig{\ch{H\chlorine}}
						\arrow
						\chemfig{!\molR-[:0]!\molCl}
						\hspace{2mm} + \hspace{2mm}
						\chemfig{\ch{H2O}}
					\schemestop
				}
			% end subsubsection
		% end section

		\pagebreak
		\subsection{Nucleophilic Substitution of \ch{OH} groups (Bromine)}

			\subsubsection{Hydrogen Bromide (\ch{HBr})}

				Gaseous hydrogen bromide (\ch{HBr}) can be reacted with alcohols to give bromoalkanes.

				\vspace{1.5em}
				\vbox{\textbf{Conditions:}	\tabto{35mm}Gaseous \ch{HBr}.}

				\diagram[1.0]{
					\schemestart[0,1.5,thick]
						\chemfig{!\molR-[:0]!\molOH}
						\hspace{2mm} + \hspace{2mm}
						\chemfig{\ch{HBr}}
						\arrow
						\chemfig{!\molR-[:0]!\molBr}
						\hspace{2mm} + \hspace{2mm}
						\chemfig{\ch{H2O}}
					\schemestop
				}

				\ch{HBr} can be prepared by reacting concentrated \ch{H2SO4} with solid \ch{NaBr}.

				\vspace{1.5em}
				\vbox{\textbf{Conditions:}	\tabto{35mm}Solid \ch{NaBr}, concentrated \ch{H2SO4}, heat.}


				\diagram[1.0]{
					\schemestart[0,1.5,thick]
						\ch{NaBr}
						\hspace{2mm} + \hspace{2mm}
						\ch{H2SO4}
						\arrow
						\ch{HBr}
						\hspace{2mm} + \hspace{2mm}
						\ch{NaHSO4}
					\schemestop
				}
			% end subsubsection

			\subsubsection{Phosphorous Tribromide (\ch{PBr3})}

				\ch{PBr3} is typically prepared \textit{in-situ} by heating red phosphorous with liquid bromine. It then proceeds to
				react with the alcohol, in a manner similar to that of \ch{P\chlorine3}.

				\vspace{1.5em}
				\vbox{\textbf{Conditions:}	\tabto{35mm}Liquid \ch{PBr3}, \textit{OR}
											\tabto{35mm}Liquid \ch{Br2}, {\color{Red}red} phosphorous, heat.}


				\diagram[1.0]{
					\schemestart[0,1.5,thick]
						\ch{2 P}
						\hspace{2mm} + \hspace{2mm}
						\ch{3 Br2}
						\arrow
						\ch{2 PBr3}
					\schemestop
				}{\textit{In-situ} formation of \ch{PBr3}.}

				\diagram[1.0]{
					\schemestart[0,1.5,thick]
						\chemfig{3}
						\chemfig{!\molR-[:0]!\molOH}
						\hspace{2mm} + \hspace{2mm}
						\chemfig{PBr\sbs{3}}
						\arrow
						\chemfig{3}
						\chemfig{!\molR-[:0]!\molBr}
						\hspace{2mm} + \hspace{2mm}
						\chemfig{H\sbs{3}PO\sbs{3}}
					\schemestop
				}
			% end subsubsection
		% end section


		\pagebreak
		\subsection{Nucleophilic Substitution of \ch{OH} groups (Iodine)}

			\subsubsection{Phosphorous Triiodide (\ch{PI3})}

				This reaction is identical to the reaction involving \ch{PBr3}, above.

				\vspace{1.5em}
				\vbox{\textbf{Conditions:}	\tabto{35mm}Liquid \ch{PI3}, \textit{OR}
											\tabto{35mm}Solid \ch{I2}, {\color{Red}red} phosphorous, heat.}


				\diagram[1.0]{
					\schemestart[0,1.5,thick]
						\ch{2 P}
						\hspace{2mm} + \hspace{2mm}
						\ch{3 I2}
						\arrow
						\ch{2 PI3}
					\schemestop
				}{\textit{In-situ} formation of \ch{PI3}.}

				\diagram[1.0]{
					\schemestart[0,1.5,thick]
						\chemfig{3}
						\chemfig{!\molR-[:0]!\molOH}
						\hspace{2mm} + \hspace{2mm}
						\chemfig{PI\sbs{3}}
						\arrow
						\chemfig{3}
						\chemfig{!\molR-[:0]!\molI}
						\hspace{2mm} + \hspace{2mm}
						\chemfig{H\sbs{3}PO\sbs{3}}
					\schemestop
				}
			% end subsubsection
		% end section


		\subsection{Electrophilic Substitution of Arenes}

			Halogens can be substituted onto a benzene ring through the use of a catalyst, in an electrophilic substitution reaction.

			\vspace{1.5em}
			\vbox{\textbf{Conditions:}	\tabto{35mm}Warm, anhydrous \ch{FeBr3}, \ch{\aluminium Br3}, or \ch{Fe} /
										\ch{\aluminium} filings (for bromine),
										\tabto{35mm}Anhydrous \ch{Fe\chlorine3}, \ch{\aluminium \chlorine3}, or
													\ch{Fe} / \ch{\aluminium} filings (for chlorine)}

			\vspace{0.75em}
			\vbox{\textbf{Observations:}\tabto{35mm}\boit{\color{Mahogany}Reddish-brown} \ch{Br2} /
										\boit{\color{YellowGreen}yellowish-green} \ch{\chlorine2} decolourises.
										\tabto{35mm}Formation of white fumes of \ch{HX} gas.}

			\diagram[1.0]{
				\schemestart[0, 1.5, thick]
				\chemfig{\ch{X2}}
				\hspace{5mm} + \hspace{5mm}
				\chemfig[yshift=-1.5em]{**6(------)}
				\arrow(.base east--.base west)
				\chemfig[yshift=-1.5em]{**6(----(-[:90]!\molX)--)}
				\hspace{5mm} + \hspace{5mm}
				\chemfig{\ch{HX}}
				\schemestop
			}
		% end section
	% end section

	\pagebreak
	\section{Halogenoalkane Reactions}

		\subsection{Nucleophilic Substitution}

			Due to the high relatively electronegativity of the halogen atom attached to the carbon atom, the carbon atom is
			\textit{electron-deficient}, which makes it more susceptible to attacks from nucleophiles (\textit{electron-rich}), and
			as such halogenoalkanes are fairly reactive.

			The full mechanism of this reaction can be found in \hyperlink{AppendixNucleophilicSubstitution}{\boit{the appendix}}.

			\subsubsection{Formation of Alcohols}

				The \ch{OH-} ion acts as a nucleophile, substituting the attached halogen atom.

				\vspace{1.5em}
				\vbox{\textbf{Conditions:}\tabto{35mm}Aqueous \ch{NaOH} or \ch{KOH}, heat.}

				\diagram[1.0]{
					\schemestart[0,1.5,thick]
						\chemfig{!\molR-[:0]!\molX}
						\hspace{2mm} + \hspace{2mm}
						\chemfig{!\molOH\mch}
						\arrow
						\chemfig{!\molR-[:0]!\molOH}
						\hspace{2mm} + \hspace{2mm}
						\chemfig{!\molX\mch}
					\schemestop
				}
			% end subsubsection


			\subsubsection{Formation of Amines}

				This reaction forms a primary amine. Ethanolic \ch{NH3} is simply a solution of \ch{NH3} in ethanol. Also, the reaction
				mixture must be heated in a sealed tube, to prevent the \ch{NH3} from escaping.

				\vspace{1.5em}
				\vbox{\textbf{Conditions:}\tabto{35mm}Ethanolic concentrated \ch{NH3}, heat in sealed tube.}

				\diagram[1.0]{
					\schemestart[0,1.5,thick]
						\chemfig{!\molR-[:0]!\molX}
						\hspace{2mm} + \hspace{2mm}
						\chemfig{\ch{NH3}}
						\arrow
						\chemfig{!\molR-[:0]\ch{NH2}}
						\hspace{2mm} + \hspace{2mm}
						\chemfig{\ch{HX}}
					\schemestop
				}


				However, this amine turns out to be a \textit{stronger} nucleophile than \ch{NH3},
				which opens the possibility for multiply-substituted amines, forming secondary amines, tertiary amines, and even
				quaternary ammonium salts.

				\diagram[1.0]{
					\schemestart[0,1.0,thick]
						\ch{RX}\hspace{2mm} + \hspace{2mm}\ch{RNH2}\arrow\ch{R2NH}\hspace{2mm} + \hspace{2mm}\ch{HX}
						\arrow(@c1.south east--.north east){0}[-90,.1]
						\ch{RX}\hspace{2mm} + \hspace{2mm}\ch{R2NH}\arrow\ch{R3N}\hspace{2mm} + \hspace{2mm}\ch{HX}
						\arrow(@c3.south east--.north east){0}[-90,.1]
						\ch{RX}\hspace{2mm} + \hspace{2mm}\ch{R3N}\arrow\ch{R4N+ X-}
					\schemestop
				}

				Thus, to prevent this from happening, \ch{NH3} should be used in excess, in essence 'crowding-out' the amines formed
				from reacting further.



			% end subsubsection

			\pagebreak
			\subsubsection{Formation of Nitriles}

				In this reaction, the lone pair is donated not by the nitrogen atom, but by the \textit{carbon} atom. This reaction is
				important, as it serves to increase the length of the carbon chain, forming a cyanohydrin –– a \textit{step-up}
				reaction –– which is useful in the synthesis of organic molecules.

				\vspace{1.5em}
				\vbox{\textbf{Conditions:}\tabto{35mm}Ethanolic \ch{KCN}, heat.}

				\diagram[1.0]{
					\schemestart[0,1.5,thick]
						\chemfig{!\molR-[:0]!\molX}
						\hspace{2mm} + \hspace{2mm}
						\chemfig{\ch{CN-}}
						\arrow
						\chemfig{!\molR-[:0]\ch{CN}}
						\hspace{2mm} + \hspace{2mm}
						\chemfig{\ch{X-}}
					\schemestop
				}



				\hypertarget{NitrileUses}{}
				Indeed, the nitrile that is formed can be hydrolysed to form carboxylic acids.

				\vspace{1.5em}
				\vbox{\textbf{Conditions:}	\tabto{35mm}Dilute \ch{H2SO4}, heat, \textit{OR}
											\tabto{35mm}Dilute \ch{NaOH}, heat.}

				\diagram[1.0]{
					\schemestart[0, 2.0, thick]
						\chemfig{!\molR-[:0]C~[:0]!\molN}
						\arrow{->[dil. \ch{H2SO4}][heat]}
						\chemfig{C(-[:180]!\molR)(=[:30]!\molO)(-[:330]!\molOH)}
					\schemestop
				}{If \ch{NaOH} is used instead of \ch{H2SO4}, a carboxylate salt is formed instead.}
				\diagram[1.0]{
					\schemestart[0, 2.0, thick]
						\chemfig{!\molR-[:0]C~[:0]!\molN}
						\arrow{->[dil. \ch{NaOH}][heat]}
						\chemfig{C(-[:180]!\molR)(=[:30]!\molO)(-[:330]!\molO\mch Na\pch)}
					\schemestop
				}


				Furthermore, it can also be reduced to form amines, either through the very strong reducing agent \ch{Li\aluminium H4}
				(dissolved in dry, diethyl ether due to its high reactivity with water), the slightly weaker \ch{NaBrH4} dissolved in
				methanol, or the traditional, copious application of \ch{H2} gas with a metal catalyst at high temperatures and pressures.

				\vspace{1.5em}
				\vbox{\textbf{Conditions:}	\tabto{35mm}\ch{Li\aluminium H4} in dry ether (diethyl ether), \textit{OR}
											\tabto{35mm}\ch{NaBH4} in methanol, \textit{OR}
											\tabto{35mm}\ch{H2 \stG}, \ch{Ni} catalyst, high temperature and pressure.}


				\diagram[1.0]{
					\schemestart[0, 2.0, thick]
						\chemfig{!\molR-[:0]C~[:0]!\molN}
						\arrow{->[reduction][{[}H{]}]}
						\chemfig{C(-[:180]!\molR)(-[:90]H)(-[:270]H)(-[:0]!\molN(-[:30]H)(-[:330]H))}
					\schemestop
				}{}



			% end subsubsection

		% end subsection

		\pagebreak
		\hypertarget{HydrogenHalideElimination}{}
		\subsection{Elimination}

			\ch{OH-} behaves as a Brønsted-Lowry base, accepting, or in this case removing, an \ch{H+} ion from the halogenoalkane.
			At the same time that this \ch{C-H} bond breaks, the \ch{C-X} bond in the adjacent carbon atom breaks, thus forming
			\ch{HX}, and an alkene.

			\vspace{1.5em}
			\vbox{\textbf{Conditions:}	\tabto{35mm}Ethanolic \ch{KOH} or \ch{NaOH}, heat.}

			\diagram[1.0]{
				\schemestart[0, 2.0, thick]
					\chemfig{C(-[:180]!\molR)(-[:90]H)(-[:270]@{x}{\color{OliveGreen}X})-C(-[:90]H)(-[:270]@{h}H)(-[:0]H)}
					\arrow{->[\ch{OH-}][heat]}
					\chemfig{C(-[:135]!\molR)(-[:225]H)=C(-[:45]H)(-[:315]H)}
					\hspace{5mm} + \hspace{5mm}
					\chemfig{H-!\molX}
				\schemestop

				\chemmove{
					\draw[-latex, red, thick]
					(x.center) circle(4mm)
					(h.center) circle(4mm);
				}
			}{The alkane can be of any length, the important bit happens here. The circled atoms are the ones forming the \ch{HX} molecule.}

			Naturally, isomerism, either optical or cis/trans (E/Z), may occur, and a mix of products can be formed.

			One might notice that these reaction conditions are similar to that for the nucleophilic substitution of alcohols. Indeed,
			both reactions will occur at the same time, forming both alcohols and alkenes.

			To favour one reaction over the other, there are a number of factors that can be controlled

			\begin{bulletlist}
				& Type of halogenoalkane (primary, secondary or tertiary)
				& Temperature of reaction mixture
				& Concentration of \ch{KOH} or \ch{NaOH}
				& Solvent used (ethanol or water)
			\end{bulletlist}

			Primary halogenoalkanes tend to mainly favour nucleophilic substitution, while tertiary halogenoalkanes tend to favour the
			elimination reaction –– secondary halogenoalkanes favour either reaction.

			Furthermore, the higher the temperature of the mixture, the more likely it is that the elimination reaction will take place.
			A higher concentration of \ch{NaOH} or \ch{KOH} will also increase the likelihood of the elimination reaction occurring.
			Finally, using ethanol as a solvent for the hydroxide, versus water, will also favour elimination.

		% end subsection
	% end section


	\pagebreak
	\section{Halogenoarene Reactions}

		Since the halogen atom is directly bonded to the benzene ring, its p-orbitals overlap in a parallel manner to the π-system of the
		benzene ring, and thus get delocalised.

		\imgdiagram{120mm}{../figures/organic/ch08/halogen_benzene_delocalistion.png}

		This makes the bond stronger than a typical \ch{C-X} bond, and also reduces the effect of the electronegativity of the halogen atom.
		Thus, it is less susceptible to nucleophile attacks, and is fairly unreactive. The high electron density also tends to repel the
		electron-rich nucleophiles.


		\subsection{Electrophilic Substitution}

			Of course, being attached to a benzene ring, halogenoarenes can undergo electrophilic substitution –– this is really a reaction
			of the benzene ring, not specific to the halogen.

			However, it must be noted that since halogens are deactivating, the conditions for some reactions become slightly harsher;
			nitration would require temperatures \textit{above} \SI{50}{\celsius}. Also, it is \textit{ortho/para}, or 2,4-directing.

			The two kinds electrophilic substitution reactions that the benzene ring can undergo can be found in the
			\hyperlink{AreneReactions}{\boit{relevant chapter}}.

		% end subsection

	% end section

	\pagebreak
	\section{Distinguishing Tests}

		\subsection{Comparing Colour of Precipitate}
			The primary means of determining the identity of the attached halogen is by looking at the colour of the precipitate formed, when
			reacted with \ch{AgNO3} –– silver halides are insoluble.

			\begin{numberedlist}
				&	Aqueous \ch{NaOH} or \ch{KOH} is added with heat, to substitute the halogen with an \ch{OH-} group, forming the \ch{X-} ion.
				&	Excess, dilute \ch{HNO3} is added to neutralise unreacted \ch{OH-}.
				&	Aqueous \ch{AgNO3} is added, to form the \ch{AgX} precipitate.
			\end{numberedlist}

			\vspace{1.5em}
			\vbox{\textbf{Conditions:}	\tabto{35mm}Aqueous \ch{NaOH} or \ch{KOH}, dilute \ch{HNO3}, aqueous \ch{AgNO3}.}

			\vspace{0.75em}
			\vbox{\textbf{Observations:}\tabto{35mm}Chlorine\tabto{60mm}White precipitate of \ch{Ag\chlorine}.
										\tabto{35mm}Bromine	\tabto{60mm}Cream precipitate of \ch{AgBr}.
										\tabto{35mm}Iodine	\tabto{60mm}Yellow precipitate of \ch{AgI}.}
		% end subsection

		\subsection{Comparing Rate of Formation of Precipitate}

			The steps are identical to that above, but instead of comparing the colour, the rate of formation is measured instead. Since the
			rate-determining step in nucleophilic substitution involves breaking the \ch{C-X} bond, the weaker this bond, the faster the
			precipitate will form.

			Hence, iodoalkanes will form the precipitate the fastest, followed by bromoalkanes, then chloroalkanes.

		% end subsection
	% end section


	\pagebreak
	\section{Chlorofluorocarbons (CFCs)}

		CFCs were commonly used as refrigerants and aerosol propellants, due to their inert, odourless, and non-toxic properties.
		Indeed, Teflon is a fluoroalkane, and is typically used to coat non-stick surfaces.

		However, under the Montreal Protocol, the use of CFCs were banned due to their harmful effect on the ozone layer. Naturally, there
		is the following equilibrium in the ozone layer:

		\diagram[1.0]{
			\schemestart[0,1.0,thick]
				\ch{O}\hspace{2mm} + \hspace{2mm}\ch{O2}\arrow{<=>}\ch{O3}
			\schemestop
		}


		Also, in the presence of UV light in the stratosphere, the previously inert chlorofluorocarbons undergo photodecomposition.

		\diagram[1.0]{
			\schemestart[0,1.5,thick]
				\ch{CF3\chlorine}\arrow{<=>[uv]}\ch{\chlorine•}\hspace{2mm} + \hspace{2mm}\ch{•CF3}
			\schemestop
		}

		When the ozone equilibrium is exposed to these chlorine radicals, they are disrupted, since the radicals can act as a
		homogeneous catalyst, reforming afterwards.

		\diagram[1.0]{
			\schemestart[0,1.0,thick]
				\ch{\chlorine•}\hspace{2mm} + \hspace{2mm}\ch{O3}\arrow\ch{\chlorine O•}\hspace{2mm} + \hspace{2mm}\ch{O2}
				\arrow(@c1.south east--.north east){0}[-90,.1]
				\ch{\chlorine O•}\hspace{2mm} + \hspace{2mm}\ch{O}\arrow\ch{\chlorine•}\hspace{2mm} + \hspace{2mm}\ch{O2}
			\schemestop
		}

		Thus, the ozone molecules are destroyed, and depletes the free oxygen atoms required in the formation of ozone.

	% end section







% end section




