% OrganicChem.tex
% Copyright (c) 2014 - 2016, zhiayang@gmail.com

\documentclass[11pt]{article}

\usepackage{studynotes}
% \usepackage{lua-visual-debug}

\title{Organic Chemistry}



\begin{document}
\pagenumbering{gobble}


% title page
\hypersetup{pageanchor=false}

\begin{center}

	\vspace*{10mm}
	\hugetext{Organic Chemistry}
	\vspace*{10mm}

	\diagram[2.0]{
		\chemfig{!\molO=[:30]*6(-!\molN(-!\molMe)-(*5(-!\molN=-!\molN(-!\molMe)-))=-(=!\molO)-!\molN(-!\molMeR)-)}
	}

	\vspace*{8mm}

	\linespread{1.0}
	\subtext{\ch{C8H10N4O2}} \break \break
	\subtext{1,3,7-trimethylpurine-2,6-dione}

	\vspace{5mm}
	\subtext{-}
	\vspace{5mm}

	\subtext{Caffeine}
	\vspace{10mm}

	Chapters 8 to 12, 15 to 18

\end{center}

% end title

\fancyhead{}
\afterpage{\blankpage}


\pagebreak

\pagenumbering{roman}
\fancyfoot{}
\fancyfoot[C]{\thepage}

\tableofcontents
\pagebreak

\hypersetup{pageanchor=true}

% reset

\pagenumbering{arabic}
\fancyfoot{}
\fancyfoot[R]{\thepage}

\fancyhead[R]{Chapter \arabic{part} - \parttitle}

\input{"Chapter 01 - Functional Groups"}
\input{"Chapter 02 - Isomers"}
\input{"Chapter 03 - Reaction Mechanisms"}
\input{"Chapter 04 - Induction and Resonance"}
\input{"Chapter 05 - Alkanes"}
\input{"Chapter 06 - Alkenes"}
\input{"Chapter 07 - Arenes"}
\input{"Chapter 08 - Halogen Derivatives"}
\input{"Chapter 09 - Alcohols"}
\input{"Chapter 10 - Phenols"}
\input{"Chapter 11 - Aldehydes and Ketones"}
\input{"Chapter 12 - Carboxylic Acids"}
\input{"Chapter 13 - Acyl Chlorides"}
\input{"Chapter 14 - Esters"}
\input{"Chapter 15 - Amides"}
\input{"Chapter 16 - Amines"}
\input{"Chapter 17 - Amino Acids"}
\input{"Chapter 18 - Proteins"}








\pagebreak

\vspace*{65mm}
\begin{center}\hugetext{Appendices}\end{center}
\normtext

Herein lie things that \itl{must} be known, but for brevity are excluded from the main text.

% disable page numbers for this "cover" page.
\thispagestyle{empty}
\pagebreak





\setcounter{part}{0}
\renewcommand\thepart{\Alph{part}}
\renewcommand\theHpart{appendix.\thepart}

\addtocontents{toc}{\protect\setcounter{tocdepth}{2}}
\addtocontents{toc}{\protect\newpage}


% change the header style
\fancyhead[R]{Appendix \Alph{part} - \parttitle}


% change the part heading.
\titleformat{\part}[display]
	{\LatoLight\fontsize{40pt}{48pt}\selectfont\bfseries}
	{\LatoLight\fontsize{30pt}{36pt}\selectfont\bfseries\hfill A\fontsize{24pt}{36pt}\selectfont\bfseries PPENDIX\hspace{5mm}\normalfont\Alexandria\fontsize{30pt}{36pt}\selectfont\raisebox{0mm}{\thepart}}
	{-0.5em}
	{\vspace{5mm}}
	[\vspace{-7.5mm}{\makebox[0mm][l]{\rule[10mm]{\textwidth}{0.5mm}}}\vspace{15mm}]





\addtocontents{toc}{\protect\newpage}
\input{"Appendix A - Reaction Mechanisms"}

\addtocontents{toc}{\protect\newpage}
\input{"Appendix B - List of Reactions"}

\addtocontents{toc}{\protect\newpage}
\input{"Appendix C - Summary Tables"}

% \addtocontents{toc}{\protect\newpage}
\input{"Appendix D - Distinguishing Tests"}

\input{"Appendix E - List of Alpha Amino Acids"}






% \pagebreak
% \part{Functional Groups}
% 	\section{Overview}
% 	\section{Basic Functional Groups}

% \pagebreak
% \part{Isomers}
% 	\section{Overview}
% 	\section{Structural Isomerism}
% 		\subsection{Chain Isomerism}
% 		\subsection{Positional Isomerism}
% 		\subsection{Functional Group Isomerism}
% 	\section{Stereoisomerism}
% 		\subsection{Cis/Trans Isomerism}
% 		\subsection{E/Z Isomerism}
% 		\subsection{Optical Isomerism}

% \pagebreak
% \part{Reaction Mechanisms}
% 	\section{Bond Breaking}
% 		\subsection{Homolytic Fission}
% 		\subsection{Heterolytic Fission}
% 	\section{Bond Forming}
% 		\subsection{Single Electrons}
% 		\subsection{Electron Pairs}
% 	\section{Electrophiles and Nucleophiles}
% 		\subsection{Electrophiles}
% 		\subsection{Nucleophiles}

% \pagebreak
% \part{Induction and Resonance}
% 	\section{Inductive Effect}
% 		\subsection{Withdrawal}
% 		\subsection{Donation}
% 	\section{Resonance Effect}
% 		\subsection{Withdrawal}
% 		\subsection{Donation}
% 		\subsection{Overall Effect}
% 			\subsubsection{Caveats}

% \pagebreak
% \part{Alkanes}
% 	\section{Open Chain}
% 	\section{Cycloalkanes}
% 	\section{Physical Properties}
% 		\subsection{Melting and Boiling Points}
% 		\subsection{Density}
% 		\subsection{Solubility}
% 	\section{Classification of Carbons}
% 	\section{Alkane Reactions}
% 		\subsection{Free Radical Substitution}
% 			\subsubsection{Mechanism of Reaction}
% 			\subsubsection{Multi-substitution}
% 			\subsubsection{Isomerism of Products}
% 			\subsubsection{Halogen Reactivity}
% 		\subsection{Combustion}

% \pagebreak
% \part{Alkenes}
% 	\section{Open Chain}
% 	\section{Cycloalkenes}
% 	\section{Physical Properties}
% 	\section{Stability of Carbocations}
% 	\section{Creation of Alkenes}
% 		\subsubsection{Elimination of Hydrogen Halide}
% 		\subsubsection{Dehydration (Elimination of Water)}
% 	\section{Alkene Reactions}
% 		\subsubsection{Electrophilic Addition}
% 			\subsubsection{Electrophilic Addition of Hydrogen Halides}
% 			\subsubsection{Markovnikov's Rule}
% 			\subsubsection{Electrophilic Addition of Halogens}
% 			\subsubsection{Electrophilic Addition of Aqueous \ch{Br2}}
% 			\subsubsection{Electrophilic Addition of Steam (Hydration)}
% 		\subsubsection{Reduction (Hydrogenation)}
% 		\subsubsection{Mild Oxidation}
% 			\subsubsection{Acidic Medium}
% 			\subsubsection{Alkaline Medium}
% 		\subsubsection{Strong Oxidation (Oxidative Cleavage)}
% 			\subsubsection{Further Oxidation of Ethanedioc Acid}
% 			\subsubsection{Uses of Oxidative Cleavage}

% \pagebreak
% \part{Arenes}
% 	\section{Benzene}
% 		\subsection{Physical Properties}
% 	\section{Substituted Benzenes}
% 		\subsection{Effect of Reactivity}
% 		\subsection{Effect on Positions of Further Substituents}
% 		\subsection{Directing Mechanism}
% 			\subsubsection{Electron-withdrawing Groups}
% 			\subsubsection{Electron-donating Groups}
% 			\subsubsection{Halogen Substituents}
% 	\section{Arene Reactions}
% 		\subsection{Nitration of Benzene}
% 		\subsection{Halogenation of Benzene}
% 		\subsection{Alkylbenzene Reactions}
% 			\subsubsection{Halogenation}
% 			\subsubsection{Nitration}
% 			\subsubsection{Free Radical Substitution}
% 			\subsubsection{Side-chain Oxidation}

% \pagebreak
% \part{Halogen Derivatives}
% 	\section{Halogenoalkanes (Alkyl Halides)}
% 		\subsection{Physical Properties}
% 	\section{Halogenoarenes}
% 	\section{Creation of Halogenated Molecules}
% 		\subsection{Free Radical Substitution}
% 		\subsection{Electrophilic Addition to Alkenes}
% 			\subsubsection{Addition of \ch{X2}}
% 			\subsubsection{Addition of \ch{HX}}
% 		\subsection{Nucleophilic Substitution of OH Groups (Chlorine)}
% 			\subsubsection{Phosphorous Pentachloride (\ch{P\Cl5})}
% 			\subsubsection{Phosphorous Trichloride (\ch{P\Cl3})}
% 			\subsubsection{Thionyl Chloride (\ch{SO\Cl2})}
% 			\subsubsection{Hydrogen Chloride (\ch{H\Cl})}
% 		\subsection{Nucleophilic Substitution of OH Groups (Bromine)}
% 			\subsubsection{Hydrogen Bromide (\ch{HBr})}
% 			\subsubsection{Phosphorous Tribromide (\ch{PBr3})}
% 		\subsection{Nucleophilic Substitution of OH Groups (Iodine)}
% 			\subsubsection{Phosphorous Triiodide (\ch{PI3})}
% 		\subsection{Electrophilic Substitution of Arenes}
% 	\section{Halogenoalkane Reactions}
% 		\subsection{Nucleophilic Substitution}
% 			\subsubsection{Formation of Alcohols}
% 			\subsubsection{Formation of Amines}
% 			\subsubsection{Formation of Nitriles}
% 		\subsection{Elimination}
% 	\section{Halogenoarene Reactions}
% 		\subsection{Electrophilic Substitution}
% 	\section{Distinguishing Tests}
% 		\subsection{Comparing Colour of Precipitate}
% 		\subsection{Comparing Rate of Formation of Precipitate}
% 	\section{Chlorofluorocarbons (CFCs)}

% \pagebreak
% \part{Alcohols}
% 	\section{Aliphatic Alcohols}
% 		\subsection{Physical Properties}
% 	\section{Phenols}
% 	\section{Creation of Alcohols}
% 		\subsection{Electrophilic Addition (Hydration) of Alkenes}
% 		\subsection{Nucleophilic Substitution of Halogenoalkanes}
% 		\subsection{Reduction of Aldehydes and Ketones}
% 		\subsection{Reduction of Carboxylic Acids}
% 		\subsection{Hydrolysis of Esters}
% 	\section{Alcohol Reactions}
% 		\subsection{Acidity of Alcohols}
% 			\subsubsection{Effect of Substituents}
% 		\subsection{Reactions as an Acid}
% 			\subsubsection{Reaction with Metals}
% 			\subsubsection{Reaction with Bases and Carbonates}
% 		\subsection{Esterification (Nucleophilic Acyl Substitution)}
% 			\subsubsection{With Carboxylic Acids}
% 			\subsubsection{With Acyl Chlorides}
% 		\subsection{Nucleophilic Substitution with Halogens}
% 		\subsection{Dehydration (Elimination of Water)}
% 			\subsubsection{Zaitsev's Rule}
% 			\subsubsection{Elimination of Water in Gem-diols}
% 		\subsection{Combustion}
% 		\subsection{Oxidation of Primary Alcohols}
% 			\subsubsection{Controlled Oxidation to Aldehydes}
% 			\subsubsection{Complete Oxidation to Carboxylic Acids}
% 		\subsection{Oxidation of Secondary Alcohols}
% 		\subsection{Oxidation of Tertiary Alcohols}
% 		\subsection{Tri-iodomethane (Iodoform) Formation}

% \pagebreak
% \part{Phenols}
% 	\section{Structure}
% 		\subsection{Physical Properties}
% 	\section{Phenol Reactions}
% 		\subsection{Acidity of Phenols}
% 			\subsubsection{Effect of Substituents}
% 		\subsection{Reactions as an Acid}
% 			\subsubsection{Reaction with Metals}
% 			\subsubsection{Reaction with Bases}
% 		\subsection{Oxidation of Phenols}
% 		\subsection{Electrophilic Substitution}
% 			\subsubsection{Halogenation of}
% 			\subsubsection{Nitration of}
% 		\subsection{Formation of Complex with \ch{Fe\Cl3}}
% 		\subsection{Esterification (Nucleophilic Acyl Substitution)}

% \pagebreak
% \part{Aldehydes and Ketones}
% 	\section{Structure}
% 		\subsection{Ketones}
% 		\subsection{Aldehydes}
% 		\subsection{Hybridisation}
% 	\section{Physical Properties}
% 	\section{Creation of Aldehydes and Ketones}
% 		\subsection{Oxidation of Alcohols}
% 			\subsubsection{Oxidation of Primary Alcohols}
% 		\subsection{Oxidation of Secondary Alcohols}
% 		\subsection{Oxidative Cleavage of Alkenes}
% 	\section{Aldehyde and Ketone Reactions}
% 		\subsection{Nucleophilic Addition}
% 			\subsubsection{Formation of Nitriles (Cyanohydrins)}
% 		\subsection{Condensation}
% 			\subsubsection{Distinguishing test with 2,4-DNPH}
% 		\subsection{Reduction}
% 			\subsubsection{Reduction of Aldehydes}
% 			\subsubsection{Reduction of Ketones}
% 		\subsection{Oxidation}
% 			\subsubsection{Oxidation of Aldehydes to Carboxylic Acids}
% 			\subsubsection{Tollens' Reagent}
% 			\subsubsection{Fehling's Solution}
% 			\subsubsection{Tri-iodomethane (Iodoform) Formation}

% \pagebreak
% \part{Carboxylic Acids}
% 	\section{Structure}
% 	\section{Physical Properties}
% 	\section{Creation of Carboxylic Acids}
% 		\subsection{Oxidation of Primary Alcohols}
% 		\subsection{Side-chain Oxidation of Alkylbenzenes}
% 		\subsection{Hydrolysis of Nitriles}
% 			\subsubsection{Acid Hydrolysis}
% 			\subsubsection{Alkaline Hydrolysis}
% 		\subsection{Oxidative Cleavage of Alkenes}
% 		\subsection{Hydrolysis of Esters}
% 		\subsection{Hydrolysis of Acyl Chlorides}
% 		\subsection{Hydrolysis of Amides}
% 			\subsubsection{Acid Hydrolysis}
% 			\subsubsection{Alkaline Hydrolysis}
% 	\section{Carboxylic Acid Reactions}
% 		\subsection{Acidity of Carboxylic Acids}
% 			\subsubsection{Effect of Substituents}
% 		\subsection{Reactions as an Acid}
% 		\subsection{Nucleophilic Acyl Substitution}
% 			\subsubsection{Formation of Acyl Chlorides}
% 			\subsubsection{Esterification}
% 		\subsection{Reduction}
% 		\subsection{Oxidation}

% \pagebreak
% \part{Acyl Chlorides}
% 	\section{Structure}
% 	\section{Physical Properties}
% 	\section{Creation of Acyl Chlorides}
% 	\section{Acyl Chloride Reactions}
% 		\subsection{Hydrolysis}
% 		\subsection{Nucleophilic Acyl Substitution}
% 			\subsubsection{Esterification}
% 			\subsubsection{Formation of Amides}

% \pagebreak
% \part{Esters}
% 	\section{Structure}
% 	\section{Physical Properties}
% 	\section{Creation of Esters}
% 		\subsection{From Carboxylic Acids}
% 		\subsection{From Acyl Chlorides}
% 	\section{Ester Reactions}
% 		\subsection{Hydrolysis}
% 			\subsubsection{Acid Hydrolysis}
% 			\subsubsection{Alkaline Hydrolysis (Saponification)}

% \pagebreak
% \part{Amides}
% 	\section{Structure}
% 	\section{Physical Properties}
% 	\section{Creation of Amides}
% 		\subsection{From Acyl Chlorides}
% 	\section{Amide Reactions}
% 		\subsection{General Reactivity}
% 		\subsection{Hydrolysis}
% 			\subsubsection{Acid Hydrolysis}
% 			\subsubsection{Alkaline Hydrolysis}

% \pagebreak
% \part{Amines}
% 	\section{Structure}
% 	\section{Physical Properties}
% 	\section{Creation of Amines}
% 		\subsection{Nucleophilic Substitution of Alkyl Halides}
% 		\subsection{Reduction of Nitriles}
% 		\subsection{Reduction of Nitrobenzene}
% 	\section{Amine Reactions}
% 		\subsection{Basicity of Amines}
% 			\subsubsection{Aliphatic Amines}
% 			\subsubsection{Phenylamines}
% 			\subsubsection{Effect of Substituents}
% 		\subsection{Reactions as a Base}
% 		\subsection{Nucleophilic Substitution of Alkyl Halides}
% 		\subsection{Nucleophilic Acyl Substitution of Acyl Chlorides}
% 		\subsection{Nucleophilic Addition of Aqueous \ch{Br2}}

% \pagebreak
% \part{Amino Acids}
% 	\section{Stuff}
% 	\section{Stuff}

% \pagebreak
% \part{Proteins}
% 	\section{Stuff}
% 	\section{Stuff}




% \pagebreak
% \part{Reaction Mechanisms}
% 	\section{Electrophilic Addition}
% 		\subsection{Markovnikov's Rule}
% 			\subsubsection{Addition of \ch{H-X} to Unsymmetrical Alkenes}
% 		\subsection{Electrophilic Addition of HX}
% 		\subsection{Electrophilic Addition of \ch{X2}}
% 		\subsection{Electrophilic Addition of Aqueous \ch{X2}}
% 		\subsection{Electrophilic Addition of Steam (Hydration)}
% 	\section{Electrophilic Substitution}
% 		\subsection{Nitration of Benzene}
% 		\subsection{Halogenation of Benzene}
% 			\subsubsection{Catalysts}
% 			\subsubsection{Reaction Mechanism}
% 	\section{Nucleophilic Addition}
% 		\subsection{Stereoisomerism of Product}
% 	\section{Nucleophilic Substitution}
% 		\subsection{Nucleophilic Substitution (\sntwo{}, single-step)}
% 			\subsubsection{Stereoisomerism of Product}
% 			\subsubsection{Reaction Pathway}
% 		\subsection{Nucleophilic Substitution (\snone{}, two-step)}
% 			\subsubsection{Stereoisomerism of Product}
% 			\subsubsection{Reaction Pathway}

% \pagebreak
% \part{Grand List of Reactions}
% 	\section{Alkanes}
% 		\subsection{Free Radical Substitution}
% 		\subsection{Combustion}
% 	\section{Alkenes}
% 		\subsection{Electrophilic Addition of \ch{X2}}
% 		\subsection{Electrophilic Addition of \ch{HX}}
% 		\subsection{Electrophilic Addition of Aqueous \ch{Br2}}
% 		\subsection{Electrophilic Addition of Steam (Hydration)}
% 		\subsection{Reduction (Hydrogenation)}
% 		\subsection{Mild Oxidation}
% 			\subsubsection{Acidic Medium}
% 			\subsubsection{Alkaline Medium}
% 		\subsection{Strong Oxidation (Oxidative Cleavage)}
% 	\section{Arenes}
% 		\subsection{Electrophilic Substitution of \ch{NO2} (Nitration)}
% 		\subsection{Electrophilic Substitution of Halogens}
% 		\subsection{Side-chain Oxidation of Alkylbenzenes}
% 	\section{Nitriles}
% 		\subsection{Acid Hydrolysis}
% 		\subsection{Alkaline Hydrolysis}
% 		\subsection{Reduction to Amines}
% 	\section{Alkyl Halides}
% 		\subsection{Nucleophilic Substitution of \ch{OH}}
% 		\subsection{Nucleophilic Substitution of \ch{NH2}}
% 		\subsection{Nucleophilic Substitution of \ch{CN}}
% 		\subsection{Elimination of \ch{HX}}
% 	\section{Alcohols}
% 		\subsection{Nucleophilic Acyl Substitution (Esterification)}
% 			\subsubsection{With Carboxylic Acids}
% 			\subsubsection{With Acyl Chlorides}
% 		\subsection{Nucleophilic Substitution of Halogens}
% 			\subsubsection{Chlorine (\ch{\Cl2} Substitution)}
% 			\subsubsection{Bromine (\ch{Br2} Substitution)}
% 			\subsubsection{Iodine (\ch{I2} Substitution)}
% 		\subsection{Oxidation}
% 			\subsubsection{Controlled Oxidation of Primary Alcohols}
% 			\subsubsection{Complete Oxidation of Primary Alcohols}
% 			\subsubsection{Oxidation of Secondary Alcohols}
% 		\subsection{Dehydration (Elimination of \ch{H2O})}
% 		\subsection{Tri-iodomethane Formation}
% 	\section{Phenols}
% 		\subsection{Electrophilic Substitution of \ch{NO2} (Nitration)}
% 		\subsection{Electrophilic Substitution of Halogens}
% 		\subsection{\ch{Fe\Cl3} Complex Formation}
% 		\subsection{Nucleophilic Acyl Substitution (Esterification)}
% 	\section{Aldehydes}
% 		\subsection{Nucleophilic Addition of \ch{CN}}
% 		\subsection{Condensation with 2,4-DNPH}
% 		\subsection{Reduction}
% 		\subsection{Oxidation to Carboxylic Acids}
% 		\subsection{Oxidation by Tollens' Reagent}
% 		\subsection{Oxidation by Fehling's Solution}
% 		\subsection{Tri-iodomethane Formation}
% 	\section{Ketones}
% 		\subsection{Nucleophilic Addition of \ch{CN}}
% 		\subsection{Condensation with 2,4-DNPH}
% 		\subsection{Reduction}
% 		\subsection{Tri-iodomethane Formation}
% 	\section{Carboxylic Acids}
% 		\subsection{Nucleophilic Acyl Substitution (Acyl Chloride)}
% 		\subsection{Nucleophilic Acyl Substitution (Esterification)}
% 		\subsection{Reduction}
% 		\subsection{Oxidation}
% 	\section{Acyl Chlorides}
% 		\subsection{Hydrolysis}
% 		\subsection{Nucleophilic Acyl Substitution (Amide Formation)}
% 		\subsection{Nucleophilic Acyl Substitution (Esterification)}
% 	\section{Esters}
% 		\subsection{Acid Hydrolysis}
% 		\subsection{Alkaline Hydrolysis (Saponification)}
% 	\section{Amides}
% 		\subsection{Acid Hydrolysis}
% 		\subsection{Alkaline Hydrolysis}
% 	\section{Amines}
% 		\subsection{Nucleophilic Substitution of Alkyl Halides}
% 		\subsection{Nucleophilic Acyl Substitution of Acyl Chlorides}
% 		\subsection{Electrophilic Addition of Bromine to Phenylamines}

% \pagebreak
% \part{Summary Tables}
% 	\section{Nature of Substituent Groups}
% 	\section{List and Uses of Reducing Agents}
% 	\section{Acidic Reactions of Organic Acids}

% \pagebreak
% \part{Distinguishing Tests}
% 	\section{Oxidising Agents}
% 	\section{Liquid or Aqueous \ch{Br2}}
% 	\section{Neutral \ch{Fe\Cl3}}
% 	\section{2,4-DNPH, Tollens' and Fehling's Solution}


































\end{document}
