% Chapter 07 - Alkenes.tex
% Copyright (c) 2014 - 2016, zhiayang@gmail.com


\pagebreak
\hypertarget{ChapterArenes}{}
\part{Arenes}

	\section{Benzene}

		Benzene is simplest possible aromatic compound, or arene. The first known and isolated arene compounds had pleasant smells,
		such as benzaldehyde. Unfortunately, even though most known arenes now smell terrible and are carcinogenic, the name stuck.

		\diagram[1.0]{
			\chemfig{**6(------)}
		}{The structural representation of benzene.}

		The most common form of arene is the benzene ring, or phenyl functional group. Note that aromatic rings with other configurations
		and structures can also form, such as with nitrogen.

		Its basic structure consists of 6 carbon atoms, arranged in a hexagonal fashion. However, unlike cyclohexane, benzene is a flat,
		planar molecule. All 6 carbon atoms are \sptwo hybridised, forming the following structure:

		\imgdiagram{120mm}{../figures/organic/ch07/benzene_delocalisation.png}

		The trigonal structure of the \sptwo hybrid orbitals dictates this structure, since the angle between each orbital is \ang{120},
		which is the internal angle of a regular hexagon. Note that there is one \ch{H} atom bonded to each carbon, making the molecular
		formula of benzene \ch{C6H6}.

		The carbon atoms are bonded to each other through π-bonds with their unhybridised p-orbitals, while the bonds with the hydrogen
		atoms (or other substituents if substituted) are done with the \sptwo orbitals.

		Furthermore, the 6 π-bond electrons are delocalised, and move freely within the ring. This increases the stability of the benzene
		ring, which increases the amount of energy needed to modify it by fiddling with bonds.


		\subsection{Physical Properties}

			Since benzene is a regular hexagon, each \ch{C=C} bond is the same length. However, it shorter than a \ch{C-C} bond, but longer
			than a \ch{C+C} bond.

			Benzene is a volatile, flammable and carcinogenic. Don't drink it. Don't eat it. Don't touch it. It also happens to be colourless,
			with a distinct \enquote{aromatic} odour.

			As a non-polar molecule, it has relatively low melting and boiling points (\SI{5.5}{\celsius} and \SI{80.1}{\celsius}), as well as low
			solubility in water. Of course, it is soluble in non-polar solvents, and in fact can be used as a solvent in its own right.

		% end subsection
	% end section


	\pagebreak
	\hypertarget{AreneReactions}{}
	\section{Arene Reactions}

		Benzene undergoes substitution reactions rather than addition reactions, since adding atoms to the ring would destroy the
		delocalised π-system of the ring, which is energetically unfavourable. Instead, the \ch{H} atoms on the ring are substituted.

		The main mechanism for this is electrophilic substitution, the details of which can be found in
		\hyperlink{AppendixElectrophilicSubstitution}{\boit{the appendix}}.


		\subsection{Nitration of Benzene}

			The nitration of benzene involves the substitution of one of the \ch{H} atoms on the benzene with a nitro (\ch{-NO2}) group.
			It has a number of specific requirements:

			\vspace{1.5em}
			\vbox{\textbf{Conditions:}	\tabto{35mm}Concentrated \ch{HNO3}, concentrated \ch{H2SO4} catalyst.
										\tabto{35mm}\textit{Constant} temperature of \SI{50}{\celsius}.}
			\vspace{0.75em}
			\vbox{\textbf{Observations:}\tabto{35mm}\boit{\color{Goldenrod}Pale yellow} oily liquid, nitrobenzene.}

			\diagram[1.0]{
				\schemestart[0, 2.0, thick]
				\chemfig{\ch{HNO3}}
				\hspace{5mm} + \hspace{5mm}
				\chemfig[yshift=-1.5em]{**6(------)}
				\arrow(.base east--.base west){->[\ch{H2SO4}][\SI{50}{\celsius}]}
				\chemfig[yshift=-1.5em]{**6(----(-[:90]!\molNitro)--)}
				\hspace{5mm} + \hspace{5mm}
				\chemfig{\ch{H2O}}
				\schemestop
			}


			\pagebreak
			\subsection{Halogenation of Benzene}

				Halogenation of benzene requires rather specific conditions, such as anhydrous \ch{FeBr3} or \ch{Fe\chlorine2} (for a reaction
				with bromine and chlorine respectively), and a warm environment.

				Aluminium-based analogues of these catalysts (\ch{\aluminium Br3}, \ch{\aluminium\chlorine3}) can also be used, as can pure
				filings of the metal, in which case the catalyst will be generated \textit{in-situ} (\ch{2 Fe \stS} + \ch{3 Br2 \stL} -> \ch{2 FeBr3}).

				The in-depth explanation about the need for an anhydrous catalyst can be found \hyperlink{BenzeneHalogenationCatalyst}{\boit{here}}.

				\vspace{1.5em}
				\vbox{\textbf{Conditions:}	\tabto{35mm}Warm, anhydrous \ch{FeBr3}, \ch{\aluminium Br3}, or \ch{Fe} / \ch{\aluminium} filings (for bromine),
											\tabto{35mm}Anhydrous \ch{Fe\chlorine3}, \ch{\aluminium \chlorine3}, or
														\ch{Fe} / \ch{\aluminium} filings (for chlorine)}

				\vspace{0.75em}
				\vbox{\textbf{Observations:}\tabto{35mm}\boit{\color{Mahogany}Reddish-brown} \ch{Br2} / \boit{\color{YellowGreen}yellowish-green} \ch{\chlorine2} decolourises.
											\tabto{35mm}Formation of white fumes of \ch{HX} gas.}

				\diagram[1.0]{
					\schemestart[0, 1.5, thick]
					\chemfig{\ch{Br2}}
					\hspace{5mm} + \hspace{5mm}
					\chemfig[yshift=-1.5em]{**6(------)}
					\arrow(.base east--.base west)
					\chemfig[yshift=-1.5em]{**6(----(-[:90]!\molBr)--)}
					\hspace{5mm} + \hspace{5mm}
					\chemfig{\ch{HBr}}
					\schemestop
				}

			% end subsection

		\pagebreak
		\section{Substituted Benzenes}

			The primary reaction mechanism of benzenes is electrophilic substitution, which involves the electrophiles attacking the
			electron-rich π-system of the benzene ring. As mentioned in Chapter 4 on Induction and Resonance, certain groups and atoms
			have the ability to withdraw or donate electrons, which affects the characteristics of the benzene ring.


			\subsection{Effect of Reactivity}

				If a benzene has electron-donating substituents, (such as \ch{CH3}) it will be more reactive, since it would increase
				the electron density of the π-system, making it a more appealing target for electrophiles. Thus, the ring is said to
				be \textit{activated}. Conversely, electron-withdrawing substituents (such as \ch{-NO2} or \ch{-CO2H}) \textit{deactivate}
				the benzene, which decreases the reactivity of the benzene ring by making it less susceptible to electrophilic attacks.


				Importantly, it \textit{must be noted} that for the nitration of benzene, when the benzene ring is \textit{activated}, the
				required temperature for reaction is only \boit{\SI{30}{\celsius}}, whereas for \textit{deactivated} rings, the required
				temperature is \boit{above \SI{50}{\celsius}}.

			% end subsection

			\subsection{Effect on Positions of Further Substituents}

				Since the main way substituents affect the benzene ring is through the distortion of its π-system electrons, naturally
				this distortion can affect the positions of additional substituents on the ring.

				For instance, an electrophilic substitution, of an electrophile \ch{R} on methylbenzene can produce 3 possible products:

				\diagram[0.9]{
					\chemnameinit{\chemfig{**6(-(-[:270]!\molR)---(-[:90]!\molMe)--)}}

					\chemname{\chemfig{**6(---(-[:30]!\molR)-(-[:90]!\molMe)-(-[:150,,,,draw=none]\phantom{R})-)}}{2-directed}	\hspace{15mm}
					\chemname{\chemfig{**6(-(-[:270]!\molR)---(-[:90]!\molMe)--)}}{4-directed}\hspace{15mm}
					\chemname{\chemfig{**6((-[:210,,,,draw=none]\phantom{R})--(-[:-30]!\molR)--(-[:90]!\molMe)--)}}{3-directed}

				}{In this case, the \ch{CH3} is considered to be attached to carbon 1.}

			The exact reasoning for this directing behaviour is complex, and has to deal with the resonance structures of the intermediate
			benzene, and the distribution of electrons within the π-system. Furthermore, there are only two types of substituents:
			\textit{2,4-directing} and \textit{3-directing}. Also note that this is similar in concept to major and minor products; both
			will be produced, except one in much larger quantities.

			The directing effects of various groups are summarised below.


			\begin{center}\begin{table}[htb]\renewcommand{\arraystretch}{1.5}
			\begin{tabu} to \textwidth {| X[-4,c,m] | X[c,m] | X[c,m] |}

				\hline
							Substituent						&	Electron Effect			&	Directing Effect	\\	\hline
				Alkyl/aryl groups (eg. \ch{-CH3})			&	Weakly Activating		&	2,4-directing		\\	\hline
				\ch{-OH}, \ch{-NH2}, \ch{-OCH3}				&	Strongly Activating		&	2,4-directing		\\	\hline
				\ch{-\chlorine}, \ch{-Br}					&	Weakly Deactivating		&	2,4-directing		\\	\hline
				\ch{-CHO}, \ch{-NO2}, \ch{-CN}, \ch{-CO2H}	&	Strongly Deactivating	&	3-directing			\\	\hline

			\end{tabu}
			\end{table}\end{center}\vspace{-10mm}


			For a quick-and-easy way to remember which groups are withdrawing and which are donating, electronegative atoms (\ch{O},
			\ch{N}, halogens, etc.) are electron withdrawing (deactivating) when they are indirectly attached to the benzene ring, but
			are generally electron donating (activating) when directly attached to the benzene ring.

			This can be explained by the fact that, when directly attached, there is the possibility of a p-orbital overlap with the
			π-electron cloud of the benzene, thus allowing for the electron density to be added to the benzene ring.

			Conversely, when indirectly attached, the effect of electronegativity generally \textit{pulls} the electron density
			through the \chemsigma-bonds. For example, even though \ch{N} can have its p-orbital overlap with the π-electron cloud,
			\ch{-NO2} substituents are still strongly deactivating, as the two highly electronegative \ch{O} atoms can still act
			to retract electron density away from the benzene ring, \textit{through} the \ch{N} atom.

			% end subsection


			\subsection{Directing Mechanism}

				The exact mechanism behind the directing effects of substituents can be explored through the resonance structure
				of the substituted ring. Technically, the 2, 3, and 4 positions are called \textit{ortho}, \textit{meta}, and \textit{para}
				respectively. It's just a naming thing.

				\pagebreak
				\subsubsection{Electron-withdrawing Groups}

				Taking nitrobenzene as an example, the attached \ch{NO2} group is electron-withdrawing. As such, based on the resonance
				structure of the π-system below, there will be three points with a partial positive charge (\deltap). Since the
				substitution requires the attack of an \textit{electrophile}, these positions are \textit{less favourable}. Hence, the
				electrophile will tend to target the meta (or 3-directed) position, and the \ch{NO2} group is said to be meta-directing,
				or 3-directing.

				\diagram[0.6125]{
					\schemestart[0, 1.5, thick]
					\chemleft[
						\subscheme{
							\chemfig{*6(-(!\invisbond\phantom{+})=-=[@{db1}](-[@{b1}:90]
								\chemabove{\color{RoyalBlue}N}{+}(=[@{db2}:150]@{oxy}{\color{Red}O})(-[:30]!\molO|\mch))-=)}
							\arrow{<->}
							\chemfig{*6(-(!\invisbond\phantom{+})=[@{db3}]-[@{b2}](!\invisbond+)-(=[:90]
								\chemabove{\color{RoyalBlue}N}{+}(-[:150]\mch|{\color{Red}O})(-[:30]!\molO|\mch))-=)}
							\arrow{<->}
							\chemfig{*6(-[@{b3}](!\invisbond+)-=-(=[:90]
								\chemabove{\color{RoyalBlue}N}{+}(-[:150]\mch|{\color{Red}O})(-[:30]!\molO|\mch))-=[@{db4}])}
							\arrow{<->}
							\chemfig{*6(=(!\invisbond\phantom{+})-=-(=[:90]
								\chemabove{\color{RoyalBlue}N}{+}(-[:150]\mch|{\color{Red}O})(-[:30]!\molO|\mch))-(!\invisbond+)-)}
						}
					\chemright]
					\arrow
					\chemfig[yshift=-9em]{**6(-(!\invisbond\molDeltap)--(!\invisbond\molDeltap)-(-[:90]N|\ch{O2})-(!\invisbond\molDeltap)-)}
					\schemestop

					% first molecule
					\chemmove{\draw[-Stealth,line width=0.4mm,shorten <=2mm,shorten >=1mm](db2) .. controls +(225:6mm) and +(270:6mm) .. (oxy);}
					\chemmove{\draw[-Stealth,line width=0.4mm,shorten <=2mm,shorten >=1mm](db1) .. controls +(45:8mm) and +(0:8mm) .. (b1);}

					% second molecule
					\chemmove{\draw[-Stealth,line width=0.4mm,shorten <=2mm,shorten >=1mm](db3) .. controls +(315:10mm) and +(0:10mm) .. (b2);}

					% third molecule
					\chemmove{\draw[-Stealth,line width=0.4mm,shorten <=2mm,shorten >=1mm](db4) .. controls +(180:10mm) and +(225:10mm) .. (b3);}

				}{The \deltap{} positions represent areas of low electron density.}

			% end subsubsection

			\subsubsection{Electron-donating Groups}

			On the other hand, for an electron-donating group such as \ch{NH2}, the reverse is true; there will be 3 areas of
			\textit{high electron density} (actually the same 3 positions), which \textit{attracts} electrophiles, and as such favours
			substituting further groups on the ortho/para positions, or 2,4 positions. Thus, \ch{NH2} is said to be ortho/para-directing,
			or 2,4-directing.

				\diagram[0.6125]{
					\schemestart[0, 1.5, thick]
					\chemleft[
						\subscheme{
							\chemfig{*6(-(!\invisbond\phantom{\text{–}})=-(!\invisbond @{l1}\phantom{\text{–}})=[@{db1}](-[@{b1}:90]
								@{nitro}\lewis{5:,\color{RoyalBlue}N}(-[:150]H)(-[:30]H))-=)}
							\arrow{<->}
							\chemfig{*6(-(!\invisbond @{l3}\phantom{\text{–}})=[@{db2}]-[@{b2}](!\invisbond @{l2}\lewis{5:,\text{–}})-(=[:90]
								\chemabove{\color{RoyalBlue}N}{+}(-[:150]H)(-[:30]H))-=)}
							\arrow{<->}
							\chemfig{*6(-[@{b3}](!\invisbond @{l5}\lewis{2:,\text{–}})-=-(=[:90]
								\chemabove{\color{RoyalBlue}N}{+}(-[:150]H)(-[:30]H))-(!\invisbond @{l4}\phantom{\text{–}})=[@{db3}])}
							\arrow{<->}
							\chemfig{*6(=(!\invisbond\phantom{\text{–}})-=-(=[:90]
								\chemabove{\color{RoyalBlue}N}{+}(-[:150]H)(-[:30]H))-(!\invisbond\lewis{7:,\text{–}})-)}
						}
					\chemright]
					\arrow
					\chemfig[yshift=-9em]{**6(-(!\invisbond\molDeltam)--(!\invisbond\molDeltam)-(-[:90]N|\ch{H2})-(!\invisbond\molDeltam)-)}
					\schemestop

					% first molecule
					\chemmove{\draw[-Stealth,line width=0.4mm,shorten <=2mm,shorten >=1mm](nitro) .. controls +(225:6mm) and +(180:6mm) .. (b1);}
					\chemmove{\draw[-Stealth,line width=0.4mm,shorten <=2mm,shorten >=-3mm](db1) .. controls +(45:6mm) and +(45:6mm) .. (l1);}

					% second molecule
					\chemmove{\draw[-Stealth,line width=0.4mm,shorten <=-2mm,shorten >=1mm](l2) .. controls +(45:2mm) and +(0:6mm) .. (b2);}
					\chemmove{\draw[-Stealth,line width=0.4mm,shorten <=2mm,shorten >=-3mm](db2) .. controls +(315:6mm) and +(270:6mm) .. (l3);}

					% third molecule
					\chemmove{\draw[-Stealth,line width=0.4mm,shorten <=2mm,shorten >=-3mm](db3) .. controls +(180:6mm) and +(150:6mm) .. (l4);}
					\chemmove{\draw[-Stealth,line width=0.4mm,shorten <=-0mm,shorten >=1mm](l5) .. controls +(270:6mm) and +(225:6mm) .. (b3);}

				}{The \deltam{} positions represent areas of high electron density.}

			% end subsubsection

			\subsubsection{Halogen Substituents}

				Halogens are a special case, since they can donate electrons through resonance, but can also withdraw electrons through
				the induction effect due to their substantial electronegativity difference, compared to carbon.

				The overall effect is that halogens are \textit{ortho/para} (2,4) directors, and are weakly electron-withdrawing.

			% end subsubsection

		% end subsection

		\pagebreak
		\subsection{Alkylbenzene Reactions}

			\subsubsection{Halogenation}
				The reagents and conditions for the halogenation of alkylbenzenes is similar to that of normal, unsubstituted benzenes. This
				time, however, there are two major products, and one minor product, due to the 2,4-directing nature of alkyl groups.

				\diagram[0.9]{
					\chemnameinit{\chemfig{**6(-(-[:270]!\molX)---(-[:90]!\molMe)--)}}

					\chemname{\chemfig{**6(---(-[:30]!\molX)-(-[:90]!\molMe)-(-[:150,,,,draw=none]\phantom{X})-)}}
					{2-directed, \textit{ortho} (major)}	\hspace{15mm}

					\chemname{\chemfig{**6(-(-[:270]!\molX)---(-[:90]!\molMe)--)}}
					{4-directed, \textit{para} (major)}		\hspace{15mm}

					\chemname{\chemfig{**6((-[:210,,,,draw=none]\phantom{X})--(-[:-30]!\molX)--(-[:90]!\molMe)--)}}
					{3-directed, \textit{meta} (minor)}
				}

			% end subsubsection

			\subsubsection{Nitration}

				Similarly, the nitration of alkylbenzenes also gives two major products, and one minor product.

				\diagram[0.9]{
					\chemnameinit{\chemfig{**6(-(-[:270]!\molNitro)---(-[:90]!\molMe)--)}}

					\chemname{\chemfig{**6(---(-[:30]!\molNitro)-(-[:90]!\molMe)-(-[:150,,,,draw=none]\phantom{\ch{NO2}})-)}}
					{2-directed, \textit{ortho} (major)}	\hspace{15mm}

					\chemname{\chemfig{**6(-(-[:270]!\molNitro)---(-[:90]!\molMe)--)}}
					{4-directed, \textit{para} (major)}		\hspace{15mm}

					\chemname{\chemfig{**6((-[:210,,,,draw=none]\phantom{\ch{NO2}})--(-[:-30]!\molNitro)--(-[:90]!\molMe)--)}}
					{3-directed, \textit{meta} (minor)}
				}

			% end subsubsection

			\pagebreak
			\subsubsection{Free Radical Substitution}

				In the absence of Lewis-acid catalysts, halogens will not react with the benzene ring. In this case, the halogens will
				react with the side-chain of the alkylbenzene, with UV light.

				\vspace{1.5em}

				\vbox{\textbf{Conditions:} \tabto{35mm}UV Light, \ch{Br2} / \ch{\chlorine2} gas}
				\vbox{\textbf{Observations:} \tabto{35mm}\boit{\color{Mahogany}Reddish-brown} \ch{Br2} / \boit{\color{YellowGreen}yellowish-green} \ch{\chlorine2} decolourises.}

				\diagram[1.0]{
					\schemestart[0, 2.0, thick]
					\chemfig[yshift=-1.5em]{**6(----(-[:90]!\molMe)--)}
					\hspace{2mm} + \hspace{2mm}
					\chemfig{\ch{\chlorine2}}
					\arrow(.base east--.base west){->[UV Light]}
					\chemfig{\ch{H\chlorine}}
					\hspace{2mm} + \hspace{2mm}
					\chemfig[yshift=-1.5em]{**6(----(-[:90]\ch{C}|\ch{H2}\ch{\chlorine})--)}
					\schemestop
				}

				Naturally, the alkyl side-chain can also undergo multiple substitutions.


			% end subsubsection

			\subsubsection{Side-chain Oxidation}

				When reacted with the strong oxidising agent, \ch{KMnO4}, and heated, the alkyl chain attached to the benzene will be oxidised.
				Regardless of the length of the chain, benzoic acid is always formed.

				Note that the carbon attached to the benzene ring \textit{must} have \textit{at least 1} hydrogen atom (ie. it cannot be a
				teritary carbon) for this oxidation to occur.

				\vspace{1.5em}

				\vbox{\textbf{Conditions:}\tabto{35mm}Heat, \ch{KMnO4}, dilute acid or alkali.}

				\vspace{0.75em}
				\vbox{\textbf{Observations:}\tabto{35mm}\boit{\color{Plum}Purple} \ch{KMnO4} decolourises, forming \ch{Mn^2+} (\textit{acid}), or
											\tabto{35mm}forms \boit{\color{Brown}brown} precipitate of \ch{MnO2} (\textit{alkali}).}


				\diagram[1.0]{
					\schemestart[0, 2.0, thick]
					\chemfig[yshift=-1.5em]{**6(----(-[:90]!\molMe)--)}
					\arrow(.base east--.base west){->[\ch{KMnO4}, \ch{H+}][heat]}
					\chemfig[yshift=-1.5em]{**6(----(-[:90]C(=[:150]!\molO)(-[:30]!\molOH))--)}
					\schemestop
				}{In this case, an acidic medium is used, hence \ch{H+}.}

				\pagebreak
				Alternatively, an alkali medium can be used, for instance with \ch{NaOH \stAq}. Instead of forming benzoic acid however, the
				benzoate ion is formed, which would form an ionic bond with \ch{Na}.


				\diagram[1.0]{
					\schemestart[0, 2.0, thick]
					\chemfig[yshift=-1.5em]{**6(----(-[:90]!\molMe)--)}
					\arrow(.base east--.base west){->[\ch{KMnO4}, \ch{OH-}][heat]}
					\chemfig[yshift=-1.5em]{**6(----(-[:90]C(=[:150]!\molO)(-[:30]!\molOH))--)}
					\schemestop
				}



				If the alkyl chain is 2-long, (ie. ethylbenzene), then \ch{CO2} will be formed from the oxidation of the second carbon,
				in addition to benzoic acid.

				\diagram[1.0]{
					\schemestart[0, 2.0, thick]
					\chemfig[yshift=-1.5em]{**6(----(-[:90]!\molEthyl)--)}
					\arrow(.base east--.base west){->[\ch{KMnO4}, \ch{H+}][heat]}
					\chemfig[yshift=-1.5em]{**6(----(-[:90]C(=[:150]!\molO)(-[:30]!\molOH))--)}
					\hspace{2mm} + \hspace{2mm}
					\chemfig{\ch{CO2}}
					\schemestop
				}

				If the chain is 3 or longer, then the rest of the chain (apart from the first) will be oxidised to form a carboxylic acid.

				\diagram[1.0]{
					\schemestart[0, 2.0, thick]
					\chemfig[yshift=-1.5em]{**6(----(-[:90]\ch{C}|\ch{H2}-[:0]!\molR)--)}
					\arrow(.base east--.base west){->[\ch{KMnO4}, \ch{H+}][heat]}
					\chemfig[yshift=-1.5em]{**6(----(-[:90]C(=[:150]!\molO)(-[:30]!\molOH))--)}
					\hspace{5mm} + \hspace{5mm}
					\chemfig{!\molR-[:0]C(=[:60]!\molO)(-[:300]!\molOH)}
					\schemestop
				}


			% end subsubsection
		% end subsection
	% end section
% end part










































