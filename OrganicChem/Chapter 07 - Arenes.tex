% Chapter 07 - Alkenes.tex
% Copyright (c) 2014 - 2016, zhiayang@gmail.com


\pagebreak
\part{Arenes}
	\section{Benzene}

		Benzene is simplest possible aromatic compound, or arene. The first known and isolated arene compounds had pleasant smells,
		such as benzaldehyde. Unfortunately, even though most known arenes now smell terrible and are carcinogenic, the name stuck.

		\diagram[1.0]{
			\chemfig{**6(------)}
		}{The structural representation of benzene.}

		The most common form of arene is the benzene ring, or phenyl functional group. Note that aromatic rings with other configurations
		and structures can also form, such as with nitrogen.

		Its basic structure consists of 6 carbon atoms, arranged in a hexagonal fashion. However, unlike cyclohexane, benzene is a flat,
		planar molecule. All 6 carbon atoms are sp\sps{2} hybridised, forming the following structure:

		\imgdiagram{120mm}{../figures/organic/ch07/benzene_delocalisation.png}

		The trigonal structure of the sp\sps{2} hybrid orbitals dictates this structure, since the angle between each orbital is \ang{120},
		which is the internal angle of a regular hexagon. Note that there is one \ch{H} atom bonded to each carbon, making the molecular
		formula of benzene \ch{C6H6}.

		The carbon atoms are bonded to each other through π-bonds with their unhybridised p-orbitals, while the bonds with the hydrogen
		atoms (or other substituents if substituted) are done with the sp\sps{2} orbitals.

		Furthermore, the 6 π-bond electrons are delocalised, and move freely within the ring. This increases the stability of the benzene
		ring, which increases the amount of energy needed to modify it by fiddling with bonds.


		\subsection{Physical Properties}

			Since benzene is a regular hexagon, each \ch{C=C} bond is the same length. However, it shorter than a \ch{C-C} bond, but longer
			than a \ch{C+C} bond.

			Benzene is a volatile, flammable and carcinogenic. Don't drink it. Don't eat it. Don't touch it. It also happens to be colourless,
			with a distinct "aromatic" odour.

			As a non-polar molecule, it has relatively low melting and boiling points (\SI{5.5}{\celsius} and \SI{80.1}{\celsius}), as well as low
			solubility in water. Of course, it is soluble in non-polar solvents, and in fact can be used as a solvent in its own right.

		% end subsection
	% end section


	\pagebreak
	\section{Arene Reactions}

		Benzene undergoes substitution reactions rather than addition reactions, since adding atoms to the ring would destroy the
		delocalised π-system of the ring, which is energetically unfavourable. Instead, the \ch{H} atoms on the ring are substituted.

		The main mechanism for this is electrophilic substitution, as seen below.

		\subsection{Electrophilic Substitution}

			The delocalised π-system of benzene has a very high electron density, and thus is a prime target for electrophiles, which
			will substitute the \ch{H} atoms on the ring. Thus, the most common form of reaction involving benzenes is electrophilic
			substitution, barring special conditions and requirements.

			% \vspace{1.0em}
			\paragraph{Step 1}

			In the first, rate-determining step, the aromaticity of the benzene ring is partially and temporarily broken, disrupted by
			the attacking electrophile.

			\diagram[1.0]{
				\schemestart[0,1.5,thick]
					\chemfig{**6(---@{ring}---)}
					\arrow{0}[,0]		% used for alignment
					\hspace{5mm} + \hspace{5mm}
					\chemfig{@{el}E}
					\arrow(--.base west){->[slow]}
					\chemfig[yshift=\the\dimexpr-1.5em\relax]{**[60,-240]6(----(-[:135]H)(-[:45]E)--)(-[:30,,,,draw=none]+)}
				\schemestop

				\chemmove{\draw[-Stealth,line width=0.4mm,shorten <=-4mm,shorten >=1mm](ring) .. controls +(40:8mm) and +(120:8mm) .. (el);}

			}{Note that the '+' is drawn in the centre of the ring, not on any one carbon.}



			Two electrons out of six from the delocalised π-system are used to form the bond between the electrophile, E, and the carbon.
			Thus, there is a positive charge on the carbon; due to the delocalised nature of the π-system however, this positive charge is
			delocalised across \textit{all 6 carbons}, making it much more stable than a simple carbocation.

			However, the activation energy for this step is still large, and only strong electrophiles are able to attack the benzene ring without catalysts.


			\pagebreak
			\paragraph{Step 2}

			Next, a nucleophile (\chlewis{180}{Nu-} in this example) attacks the hydrogen attached to the hydrogen on the carbon atom,
			restoring the aromaticity of the benzene ring. The new substituent is now in place, and the two electrons in the \ch{C-H}
			bond are returned to the π-system.


			\diagram[1.0]{
				\schemestart[0, 1.5, thick]
					\chemfig[yshift=-1.5em]{**[60,-240]6(----(-[@{bond}:135]@{hyd}H)(-[:45]E)--)(-[:30,,,,draw=none]@{pl}+)}
					\arrow(.base east--){0}[,0]		% alignment purposes
					\hspace{5mm} + \hspace{5mm}
					\chemfig{@{nu}\lewis{2:,N}|u\mch}
					\arrow(--.mid west){->[fast]}
					\chemfig{\ch{HNu}}
					\hspace{5mm} + \hspace{5mm}
					\chemfig[yshift=-1.5em]{**6(----(-[:90]E)--)}
				\schemestop

				\chemmove{\draw[-Stealth,line width=0.4mm,shorten <=2mm,shorten >=1mm](nu) .. controls +(90:15mm) and +(45:15mm) .. (hyd);}
				\chemmove{\draw[-Stealth,line width=0.4mm,shorten <=1mm,shorten >=1mm](bond) .. controls +(225:8mm) and +(90:8mm) .. (pl);}
			}


			Note that the arenium ion (which is the partially delocalised benzene) has 5 sp\sps{2} carbons, and one sp\sps{3} carbon. This
			results in a disruption of the planar structure of benzene –– it is restored once the substitution is completed.

		% end subsection

		% \pagebreak
		\subsection{Nitration of Benzene}

			The nitration of benzene involves the substitution of one of the \ch{H} atoms on the benzene with a nitro (\ch{-NO2}) group.
			It has a number of specific requirements:

			\vspace{1.5em}
			\vbox{\textbf{Conditions:}	\tabto{35mm}Concentrated \ch{HNO3}, concentrated \ch{H2SO4} catalyst.
										\tabto{35mm}\textit{Constant} temperature of \SI{50}{\celsius}.}\vspace{0.5em}

			\vbox{\textbf{Observations:}\tabto{35mm}\textit{\color{Goldenrod}Pale yellow} oily liquid, nitrobenzene.}


			\vspace{1.0em}
			\subsubtext{\vbox{IV.\tabto{10mm}A New Electrophile}}

			Since \ch{H2SO4} is a stronger acid than \ch{HNO3}, it donates a proton to \ch{HNO3}, forming \ch{H2O}, \ch{HSO4-}, and \ch{NO2+}, the
			electrophile. Next, another molecule of \ch{H2SO4} then forms \ch{H+ \stAq}, or \ch{H3O+}. The overall equation is as such:

			\diagram{
				\schemestart[0, 1.0, thick]
					\ch{2 H2SO4} \hspace{2mm} + \hspace{2mm} \ch{HNO3}
					\arrow
					\ch{NO2+} \hspace{2mm} + \hspace{2mm} \ch{H3O+} \hspace{2mm} + \hspace{2mm} \ch{2 HSO4-}
				\schemestop
			}{The catalyst \ch{H2SO4} is restored in a later step.}


			% \vspace{2.0em}
			\pagebreak
			\subsubtext{\vbox{V.\tabto{10mm}The π-Electrons Strike Back}}

			Now that the electrophile \ch{NO2+} has been formed, it is attacked by the π-system. As with all electrophilic substitutions, this
			involves the breaking of the aromatic system, and is the slow step. The mechanism follows the general mechanism outlined above.

			\diagram[1.0]{
				\schemestart[0, 1.5, thick]
				\chemfig{**6(---@{ring}---)}
				\arrow{0}[,0]		% used for alignment
				\hspace{5mm} + \hspace{5mm}
				\chemfig{@{el}\ch{NO2+}}
				\arrow(--.base west){->[slow]}
				\chemfig[yshift=\the\dimexpr-1.5em\relax]{**[60,-240]6(----(-[:135]H)(-[:45]!\molNitro)--)(-[:30,,,,draw=none]+)}
				\schemestop

				\chemmove{\draw[-Stealth,line width=0.4mm,shorten <=-4mm,shorten >=1mm](ring) .. controls +(40:8mm) and +(120:8mm) .. (el);}
			}




			\vspace{1.0em}
			\subsubtext{\vbox{VI.\tabto{10mm}Return of the Aromaticity}}

			The \ch{HSO4-} intermediate acts as a nucleophile and attacks the \ch{H} atom bonded to the benzene intermediate. This
			restores both the π-system of the benzene ring, as well as the catalyst, \ch{H2SO4}.


			% TODO: fix alignment (more?)
			\diagram[0.85]{
				\schemestart[0, 1.5, thick]
				\chemfig{!\molS(=[:180]!\molO)(=[:270]!\molO)(-[:90]@{oxy}\lewis{2:,\color{Red}O}|\mch)(-[:0]!\molOH)}
				\hspace{5mm} + \hspace{5mm}
				\chemfig[yshift=-1.5em]{**[60,-240]6(----(-[@{bond}:135]@{hyd}H)(-[:45]!\molNitro)--)(-[:30,,,,draw=none]@{plus}+)}
				\arrow(.mid east--.mid west){->[fast]}
				\chemfig[yshift=-1.5em]{**6(----(-[:90]!\molNitro)--)}
				\hspace{5mm} + \hspace{5mm}
				\chemfig{!\molS(=[:180]!\molO)(=[:270]!\molO)(-[:90]!\molOH)(-[:0]!\molOH)}
				\schemestop

				\chemmove{\draw[-Stealth,line width=0.4mm,shorten <=2mm,shorten >=1mm](oxy) .. controls +(90:15mm) and +(120:15mm) .. (hyd);}
				\chemmove{\draw[-Stealth,line width=0.4mm,shorten <=1mm,shorten >=1mm](bond) .. controls +(225:8mm) and +(90:8mm) .. (plus);}
			}



			\vspace{1.0em}
			\subsubtext{\vbox{VII.\tabto{10mm}Overall Reaction}}

			\diagram[1.0]{
				\schemestart[0, 1.5, thick]
				\chemfig{\ch{HNO3}}
				\hspace{5mm} + \hspace{5mm}
				\chemfig[yshift=-1.5em]{**6(------)}
				\arrow(.base east--.base west){->[\ch{H2SO4}][\SI{50}{\celsius}]}
				\chemfig[yshift=-1.5em]{**6(----(-[:90]!\molNitro)--)}
				\hspace{5mm} + \hspace{5mm}
				\chemfig{\ch{H2O}}
				\schemestop
			}


			\pagebreak
			\subsection{Halogenation of Benzene}

				Halogenation of benzene requires rather specific conditions, such as anhydrous \ch{FeBr3} or \ch{Fe\chlorine2} (for a reaction
				with bromine and chlorine respectively), and a warm environment.

				Aluminium-based analogues of these catalysts (\ch{\aluminium Br3}, \ch{\aluminium\chlorine3}) can also be used, as can pure
				filings of the metal, in which case the catalyst will be generated \textit{in-situ} (\ch{2 Fe \stS + 3 Br2 \stL -> 2 FeBr3}).


				\vspace{1.5em}
				\vbox{\textbf{Conditions:}	\tabto{35mm}Warm, anhydrous \ch{FeBr3}, \ch{\aluminium Br3}, or \ch{Fe} / \ch{\aluminium} filings (for bromine),
											\tabto{35mm}Anhydrous \ch{Fe\chlorine3}, \ch{\aluminium \chlorine3}, or
														\ch{Fe} / \ch{\aluminium} filings (for chlorine)}\vspace{0.5em}

				\vbox{\textbf{Observations:}\tabto{35mm}\textit{\color{Mahogany}Reddish-brown} \ch{Br2} / \textit{\color{YellowGreen}yellowish-green} \ch{\chlorine2} decolourises.
											\tabto{35mm}Formation of white fumes of \ch{HX} gas.}


				\subsubsection{Conditions for Reaction}

					Lewis acid catalysts must be used, since the \ch{Br-Br} and \ch{\chlorine-\chlorine} are only instantaneously polar (instantaneous
					dipole moments). As such, they are nowhere near strong enough to attack the benzene system on their own.

					Indeed, this can be used to distinguish between alkenes and benzenes, since the former does not require a catalyst for addition
					of halogens.

					Furthermore, the entire reaction must be conducted in the absence of water; the reaction mechanism for the lewis-acid catalyst
					involves accepting a lone pair, the lone pair on water can, and will, in sufficient concentrations, destroy the catalyst.

				% end subsubsection



				\subsubsection{Reaction Mechanism}

					% \vspace{1.0em}
					\paragraph{Generation of Electrophile}

					The reaction below uses Iron (III) chlorine (\ch{Fe\chlorine3}) as an example, adding \ch{\chlorine} to benzene, and this reaction
					mechanism applies to aluminium-based catalysts as well.

					\diagram{
						\schemestart[0, 1.0, thick]
							\ch{FeBr3} \hspace{2mm} + \hspace{2mm} \ch{Br2}
							\arrow
							\ch{Br+} \hspace{2mm} + \hspace{2mm} \ch{FeBr4-}
						\schemestop
					}{The catalyst is \ch{FeBr4-} –– the \textit{electrophile} is \ch{Br+}}


					\pagebreak
					\paragraph{Formation of Benzene Intermediate}

					Again, this mechanism is similar in nature to electrophilic substitution in general. Now, the electrophile (\ch{Br+}) attacks
					the π-system, forming the arenium ion.

					\diagram[1.0]{
						\schemestart[0, 1.5, thick]
						\chemfig{**6(---@{ring}---)}
						\arrow{0}[,0]		% used for alignment
						\hspace{5mm} + \hspace{5mm}
						\chemfig{@{el}!\molBr}
						\arrow(--.base west){->[slow]}
						\chemfig[yshift=\the\dimexpr-1.5em\relax]{**[60,-240]6(----(-[:135]H)(-[:45]!\molBr)--)(-[:30,,,,draw=none]+)}
						\schemestop

						\chemmove{\draw[-Stealth,line width=0.4mm,shorten <=-4mm,shorten >=3mm](ring) .. controls +(40:8mm) and +(120:8mm) .. (el);}
					}


					% \vspace{1.0em}
					\paragraph{Restoration of π-system and Aromaticity}

					In the final step, the \ch{FeBr4-} acts as a nucleophile, attacking the \ch{H} atom attached to the benzene intermediate. This
					regenerates the catalyst \ch{FeBr3}, and also forms \ch{HBr}.


					\diagram[0.85]{
						\schemestart[0, 1.5, thick]
						\chemfig{Fe(-[:180]!\molBr)(-[:0]!\molBr)(-[:90]@{brom}\lewis{2:,\color{Mahogany}Br}|\mch)(-[:270]!\molBr)}
						\hspace{5mm} + \hspace{5mm}
						\chemfig[yshift=-1.5em]{**[60,-240]6(----(-[@{bond}:135]@{hyd}H)(-[:45]!\molBr)--)(-[:30,,,,draw=none]@{plus}+)}
						\arrow(.mid east--.mid west){->[fast]}
						\chemfig[yshift=-1.5em]{**6(----(-[:90]!\molBr)--)}
						\hspace{5mm} + \hspace{5mm}
						\chemfig{\ch{FeBr3}}
						\hspace{5mm} + \hspace{5mm}
						\chemfig{\ch{HBr}}
						\schemestop

						\chemmove{\draw[-Stealth,line width=0.4mm,shorten <=2mm,shorten >=1mm](brom) .. controls +(90:15mm) and +(120:15mm) .. (hyd);}
						\chemmove{\draw[-Stealth,line width=0.4mm,shorten <=1mm,shorten >=1mm](bond) .. controls +(225:8mm) and +(90:8mm) .. (plus);}
					}



					% \vspace{1.0em}
					\paragraph{Overall Reaction}

					\diagram[1.0]{
						\schemestart[0, 1.5, thick]
						\chemfig{\ch{Br2}}
						\hspace{5mm} + \hspace{5mm}
						\chemfig[yshift=-1.5em]{**6(------)}
						\arrow(.base east--.base west)
						\chemfig[yshift=-1.5em]{**6(----(-[:90]!\molBr)--)}
						\hspace{5mm} + \hspace{5mm}
						\chemfig{\ch{HBr}}
						\schemestop
					}

				% end subsubsection

			% end subsection

		\pagebreak
		\section{Substituted Benzenes}

			The primary reaction mechanism of benzenes is electrophilic substitution, which involves the electrophiles attacking the
			electron-rich π-system of the benzene ring. As mentioned in Chapter 4 on Induction and Resonance, certain groups and atoms
			have the ability to withdraw or donate electrons, which affects the characteristics of the benzene ring.


			\subsection{Effect of Reactivity}

				If a benzene has electron-donating substituents, (such as \ch{CH3}) it will be more reactive, since it would increase
				the electron density of the π-system, making it a more appealing target for electrophiles. Thus, the ring is said to
				be \textit{activated}. Conversely, electron-withdrawing substituents (such as \ch{-NO2} or \ch{-CO2H}) \textit{deactivate}
				the benzene, which decreases the reactivity of the benzene ring by making it less susceptible to electrophilic attacks.


				Importantly, it \textit{must be noted} that for the nitration of benzene, when the benzene ring is \textit{activated}, the
				required temperature for reaction is only \boit{\SI{30}{\celsius}}, whereas for \textit{deactivated} rings, the required
				temperature is \boit{above \SI{50}{\celsius}}.

			% end subsection

			\subsection{Effect on Positions of Further Substituents}

				Since the main way substituents affect the benzene ring is through the distortion of its π-system electrons, naturally
				this distortion can affect the positions of additional substituents on the ring.

				For instance, an electrophilic substitution, of an electrophile \ch{R} on methylbenzene can produce 3 possible products:

				\diagram[0.9]{
					\chemnameinit{\chemfig{**6(-(-[:270]!\molR)---(-[:90]!\molMe)--)}}

					\chemname{\chemfig{**6(---(-[:30]!\molR)-(-[:90]!\molMe)-(-[:150,,,,draw=none]\phantom{R})-)}}{2-directed}	\hspace{15mm}
					\chemname{\chemfig{**6(-(-[:270]!\molR)---(-[:90]!\molMe)--)}}{4-directed}\hspace{15mm}
					\chemname{\chemfig{**6((-[:210,,,,draw=none]\phantom{R})--(-[:-30]!\molR)--(-[:90]!\molMe)--)}}{3-directed}

				}{In this case, the \ch{CH3} is considered to be attached to carbon 1.}

			The exact reasoning for this directing behaviour is complex, and has to deal with the resonance structures of the intermediate
			benzene, and the distribution of electrons within the π-system. Furthermore, there are only two types of substituents:
			\textit{2,4-directing} and \textit{3-directing}. Also note that this is similar in concept to major and minor products; both
			will be produced, except one in much larger quantities.

			The directing effects of various groups are summarised below.


			\begin{center}\begin{table}[htb]\renewcommand{\arraystretch}{1.5}
			\begin{tabu} to \textwidth {| X[-4,c,m] | X[c,m] | X[c,m] |}

				\hline
							Substituent						&	Electron Effect			&	Directing Effect	\\	\hline
				Alkyl/aryl groups (eg. \ch{-CH3})			&	Weakly Activating		&	2,4-directing		\\	\hline
				\ch{-OH}, \ch{-NH2}, \ch{-OCH3}				&	Strongly Activating		&	2,4-directing		\\	\hline
				\ch{-\chlorine}, \ch{-Br}					&	Weakly Deactivating		&	2,4-directing		\\	\hline
				\ch{-CHO}, \ch{-NO2}, \ch{-CN}, \ch{-CO2H}	&	Strongly Deactivating	&	3-directing			\\	\hline

			\end{tabu}
			\end{table}\end{center}\vspace{-10mm}


			% end subsection


			\subsection{Directing Mechanism}

				The exact mechanism behind the directing effects of substituents can be explored through the resonance structure
				of the substituted ring. Technically, the 2, 3, and 4 positions are called \textit{ortho}, \textit{meta}, and \textit{para}
				respectively. It's just a naming thing.

				\subsubsection{Electron-withdrawing Groups}

				Taking nitrobenzene as an example, the attached \ch{NO2} group is electron-withdrawing. As such, based on the resonance
				structure of the π-system below, there will be three points with a partial positive charge (\deltap). Since the
				substitution requires the attack of an \textit{electrophile}, these positions are \textit{less favourable}. Hence, the
				electrophile will tend to target the meta (or 3-directed) position, and the \ch{NO2} group is said to be meta-directing,
				or 3-directing.

				\diagram[0.6125]{
					\schemestart[0, 1.5, thick]
					\chemleft[
						\subscheme{
							\chemfig{*6(-(!\invisbond\phantom{+})=-=[@{db1}](-[@{b1}:90]
								\chemabove{\color{RoyalBlue}N}{+}(=[@{db2}:150]@{oxy}{\color{Red}O})(-[:30]!\molO|\mch))-=)}
							\arrow{<->}
							\chemfig{*6(-(!\invisbond\phantom{+})=[@{db3}]-[@{b2}](!\invisbond+)-(=[:90]
								\chemabove{\color{RoyalBlue}N}{+}(-[:150]\mch|{\color{Red}O})(-[:30]!\molO|\mch))-=)}
							\arrow{<->}
							\chemfig{*6(-[@{b3}](!\invisbond+)-=-(=[:90]
								\chemabove{\color{RoyalBlue}N}{+}(-[:150]\mch|{\color{Red}O})(-[:30]!\molO|\mch))-=[@{db4}])}
							\arrow{<->}
							\chemfig{*6(=(!\invisbond\phantom{+})-=-(=[:90]
								\chemabove{\color{RoyalBlue}N}{+}(-[:150]\mch|{\color{Red}O})(-[:30]!\molO|\mch))-(!\invisbond+)-)}
						}
					\chemright]
					\arrow
					\chemfig[yshift=-9em]{**6(-(!\invisbond\molDeltap)--(!\invisbond\molDeltap)-(-[:90]N|\ch{O2})-(!\invisbond\molDeltap)-)}
					\schemestop

					% first molecule
					\chemmove{\draw[-Stealth,line width=0.4mm,shorten <=2mm,shorten >=1mm](db2) .. controls +(225:6mm) and +(270:6mm) .. (oxy);}
					\chemmove{\draw[-Stealth,line width=0.4mm,shorten <=2mm,shorten >=1mm](db1) .. controls +(45:8mm) and +(0:8mm) .. (b1);}

					% second molecule
					\chemmove{\draw[-Stealth,line width=0.4mm,shorten <=2mm,shorten >=1mm](db3) .. controls +(315:10mm) and +(0:10mm) .. (b2);}

					% third molecule
					\chemmove{\draw[-Stealth,line width=0.4mm,shorten <=2mm,shorten >=1mm](db4) .. controls +(180:10mm) and +(225:10mm) .. (b3);}

				}{The \deltap positions represent areas of low electron density.}

			% end subsubsection

			\pagebreak
			\subsubsection{Electron-donating Groups}

			On the other hand, for an electron-donating group such as \ch{NH2}, the reverse is true; there will be 3 areas of
			\textit{high electron density} (actually the same 3 positions), which \textit{attracts} electrophiles, and as such favours
			substituting further groups on the ortho/para positions, or 2,4 positions. Thus, \ch{NH2} is said to be ortho/para-directing,
			or 2,4-directing.

				\diagram[0.6125]{
					\schemestart[0, 1.5, thick]
					\chemleft[
						\subscheme{
							\chemfig{*6(-(!\invisbond\phantom{\text{–}})=-(!\invisbond @{l1}\phantom{\text{–}})=[@{db1}](-[@{b1}:90]
								@{nitro}\lewis{5:,\color{RoyalBlue}N}(-[:150]H)(-[:30]H))-=)}
							\arrow{<->}
							\chemfig{*6(-(!\invisbond @{l3}\phantom{\text{–}})=[@{db2}]-[@{b2}](!\invisbond @{l2}\lewis{5:,\text{–}})-(=[:90]
								\chemabove{\color{RoyalBlue}N}{+}(-[:150]H)(-[:30]H))-=)}
							\arrow{<->}
							\chemfig{*6(-[@{b3}](!\invisbond @{l5}\lewis{2:,\text{–}})-=-(=[:90]
								\chemabove{\color{RoyalBlue}N}{+}(-[:150]H)(-[:30]H))-(!\invisbond @{l4}\phantom{\text{–}})=[@{db3}])}
							\arrow{<->}
							\chemfig{*6(=(!\invisbond\phantom{\text{–}})-=-(=[:90]
								\chemabove{\color{RoyalBlue}N}{+}(-[:150]H)(-[:30]H))-(!\invisbond\lewis{7:,\text{–}})-)}
						}
					\chemright]
					\arrow
					\chemfig[yshift=-9em]{**6(-(!\invisbond\molDeltam)--(!\invisbond\molDeltam)-(-[:90]N|\ch{H2})-(!\invisbond\molDeltam)-)}
					\schemestop

					% first molecule
					\chemmove{\draw[-Stealth,line width=0.4mm,shorten <=2mm,shorten >=1mm](nitro) .. controls +(225:6mm) and +(180:6mm) .. (b1);}
					\chemmove{\draw[-Stealth,line width=0.4mm,shorten <=2mm,shorten >=-3mm](db1) .. controls +(45:6mm) and +(45:6mm) .. (l1);}

					% second molecule
					\chemmove{\draw[-Stealth,line width=0.4mm,shorten <=-2mm,shorten >=1mm](l2) .. controls +(45:2mm) and +(0:6mm) .. (b2);}
					\chemmove{\draw[-Stealth,line width=0.4mm,shorten <=2mm,shorten >=-3mm](db2) .. controls +(315:6mm) and +(270:6mm) .. (l3);}

					% third molecule
					\chemmove{\draw[-Stealth,line width=0.4mm,shorten <=2mm,shorten >=-3mm](db3) .. controls +(180:6mm) and +(150:6mm) .. (l4);}
					\chemmove{\draw[-Stealth,line width=0.4mm,shorten <=-0mm,shorten >=1mm](l5) .. controls +(270:6mm) and +(225:6mm) .. (b3);}

				}{The \deltam positions represent areas of high electron density.}

			% end subsubsection

			\subsubsection{Halogen Substituents}

				Halogens are a special case, since they can donate electrons through induction, but can also withdraw electrons through
				the resonance effect due to their substantial electronegativity difference, compared to carbon.

				The overall effect is that halogens are \textit{ortho/para} directors.

			% end subsubsection

		% end subsection

		\pagebreak
		\subsection{Alkylbenzene Reactions}

			\subsubsection{Halogenation}
				The reagents and conditions for the halogenation of alkylbenzenes is similar to that of normal, unsubstituted benzenes. This
				time, however, there are two major products, and one minor product, due to the 2,4-directing nature of alkyl groups.

				\diagram[0.9]{
					\chemnameinit{\chemfig{**6(-(-[:270]!\molX)---(-[:90]!\molMe)--)}}

					\chemname{\chemfig{**6(---(-[:30]!\molX)-(-[:90]!\molMe)-(-[:150,,,,draw=none]\phantom{X})-)}}
					{2-directed, \textit{ortho} (major)}	\hspace{15mm}

					\chemname{\chemfig{**6(-(-[:270]!\molX)---(-[:90]!\molMe)--)}}
					{4-directed, \textit{para} (major)}		\hspace{15mm}

					\chemname{\chemfig{**6((-[:210,,,,draw=none]\phantom{X})--(-[:-30]!\molX)--(-[:90]!\molMe)--)}}
					{3-directed, \textit{meta} (minor)}
				}

			% end subsubsection

			\subsubsection{Nitration}

				Similarly, the nitration of alkylbenzenes also gives two major products, and one minor product.

				\diagram[0.9]{
					\chemnameinit{\chemfig{**6(-(-[:270]!\molNitro)---(-[:90]!\molMe)--)}}

					\chemname{\chemfig{**6(---(-[:30]!\molNitro)-(-[:90]!\molMe)-(-[:150,,,,draw=none]\phantom{\ch{NO2}})-)}}
					{2-directed, \textit{ortho} (major)}	\hspace{15mm}

					\chemname{\chemfig{**6(-(-[:270]!\molNitro)---(-[:90]!\molMe)--)}}
					{4-directed, \textit{para} (major)}		\hspace{15mm}

					\chemname{\chemfig{**6((-[:210,,,,draw=none]\phantom{\ch{NO2}})--(-[:-30]!\molNitro)--(-[:90]!\molMe)--)}}
					{3-directed, \textit{meta} (minor)}
				}

			% end subsubsection

			\pagebreak
			\subsubsection{Free Radical Substitution}

				In the absence of Lewis-acid catalysts, halogens will not react with the benzene ring, due to the high unfavourability of
				that reaction. However, remember that reactants can, and will, react with \textit{any} functional group on the molecule. In
				this case, the halogens will react with the alkyl side-chain of the alkylbenzene. The conditions are identical to that of
				regular free radical substitution.

				\vspace{1.5em}

				\vbox{\textbf{Conditions:} \tabto{35mm}UV Light, \ch{Br2} / \ch{\chlorine2} gas}	% single line, not needed \vspace{0.5em}
				\vbox{\textbf{Observations:} \tabto{35mm}\textit{\color{Mahogany}Reddish-brown} \ch{Br2} / \textit{\color{YellowGreen}yellowish-green} \ch{\chlorine2} decolourises.}

				\diagram[1.0]{
					\schemestart[0, 1.5, thick]
					\chemfig[yshift=-1.5em]{**6(----(-[:90]!\molMe)--)}
					\hspace{2mm} + \hspace{2mm}
					\chemfig{\ch{\chlorine2}}
					\arrow(.base east--.base west){->[UV Light]}
					\chemfig{\ch{H\chlorine}}
					\hspace{2mm} + \hspace{2mm}
					\chemfig[yshift=-1.5em]{**6(----(-[:90]\ch{C}|\ch{H2}\ch{\chlorine})--)}
					\schemestop
				}

				Naturally, the alkyl side-chain can also undergo multiple substitutions, giving a mix of products and isomers.


			% end subsubsection

			\subsubsection{Side-chain Oxidation}

				When reacted with the strong oxidising agent, \ch{KMnO4}, and heated, the alkyl chain attached to the benzene will be oxidised.
				Regardless of the length of the chain, benzoic acid is always formed.

				\vspace{1.5em}

				\vbox{\textbf{Conditions:}\tabto{35mm}Heat, \ch{KMnO4}, dilute acid or alkali.}\vspace{0.5em}
				\vbox{\textbf{Observations:}\tabto{35mm}\textit{\color{Plum}Purple} \ch{KMnO4} decolourises (\textit{acid}), or
											\tabto{35mm}forms \textit{\color{Brown}brown} precipitate of \ch{MnO2} (\textit{alkali}).}


				\diagram[1.0]{
					\schemestart[0, 1.5, thick]
					\chemfig[yshift=-1.5em]{**6(----(-[:90]!\molMe)--)}
					\arrow(.base east--.base west){->[\ch{KMnO4}, \ch{H+}][heat]}
					\chemfig[yshift=-1.5em]{**6(----(-[:90]C(=[:150]!\molO)(-[:30]!\molOH))--)}
					\schemestop
				}{In this case, an acidic medium is used, hence \ch{H+}.}

				\pagebreak
				Alternatively, an alkali medium can be used, for instance with \ch{NaOH \stAq}. Instead of forming benzoic acid however, the
				benzoate ion is formed, which would form an ionic bond with \ch{Na}.


				\diagram[1.0]{
					\schemestart[0, 1.5, thick]
					\chemfig[yshift=-1.5em]{**6(----(-[:90]!\molMe)--)}
					\arrow(.base east--.base west){->[\ch{KMnO4}, \ch{OH-}][heat]}
					\chemfig[yshift=-1.5em]{**6(----(-[:90]C(=[:150]!\molO)(-[:30]!\molOH))--)}
					\schemestop
				}



				If the alkyl chain is 2-long, (ie. ethylbenzene), then \ch{CO2} will be formed from the oxidation of the second carbon,
				in addition to benzoic acid.

				\diagram[1.0]{
					\schemestart[0, 1.5, thick]
					\chemfig[yshift=-1.5em]{**6(----(-[:90]!\molMe)--)}
					\arrow(.base east--.base west){->[\ch{KMnO4}, \ch{H+}][heat]}
					\chemfig[yshift=-1.5em]{**6(----(-[:90]C(=[:150]!\molO)(-[:30]!\molOH))--)}
					\hspace{2mm} + \hspace{2mm}
					\chemfig{\ch{CO2}}
					\schemestop
				}

				If the chain is 3 or longer, then the rest of the chain (apart from the first) will be oxidised to form a carboxylic acid.

				\diagram[1.0]{
					\schemestart[0, 1.5, thick]
					\chemfig[yshift=-1.5em]{**6(----(-[:90]\ch{C}|\ch{H2}-[:0]!\molR)--)}
					\arrow(.base east--.base west){->[\ch{KMnO4}, \ch{H+}][heat]}
					\chemfig[yshift=-1.5em]{**6(----(-[:90]C(=[:150]!\molO)(-[:30]!\molOH))--)}
					\hspace{5mm} + \hspace{5mm}
					\chemfig{!\molR-[:0]C(=[:60]!\molO)(-[:300]!\molOH)}
					\schemestop
				}


			% end subsubsection
		% end subsection
	% end section
% end part










































