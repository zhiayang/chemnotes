% Chapter 14 - Esters.tex
% Copyright (c) 2014 - 2016, zhiayang@gmail.com
% Licensed under the Apache License Version 2.0.


\pagebreak
\hypertarget{ChapterEsters}{}
\part{Esters}

	\section{Structure}

		Esters are also an acyl derivative, where the \ch{OH} group of a carboxylic acid is instead replaced with an oxygen atom
		bonded to an R group.

		\diagram[1.0]{
			\chemfig{C(-[:180]!\molR)(=[:60]!\molO)(-[:300]!\molO-[:0]!\molR)}
		}{The structure of an ester.}

	% end section



	\section{Physical Properties}

		\paragraph{Melting and Boiling Points}

		Esters, like acyl chlorides, have far lower melting and boiling points than their equivalent carboxylic acid, due to the lack of
		intermolecular hydrogen bonding. Thus, they rely only on permanent dipole interactions, leading to low melting and boiling points.


		\paragraph{Solubility}

		Esters are mostly insoluble in water once the carbon chain reaches any appreciable length, due again to the lack of hydrogen bonding.
		Thus, only permanent dipole interactions can be formed between water and the ester.

		Of course, the carbon chain makes esters highly soluble in non-polar solvents.

	% end section



	\section{Creation of Esters}

		There are two main methods of creating esters, one of which is superior to the other. Both have been discussed in detail before.

		\subsection{From Carboxylic Acids}

			\vspace{1.5em}
			\vbox{\textbf{Conditions:}	\tabto{35mm}Carboxylic acid and alcohol,
										\tabto{35mm}Several drops of concentrated \ch{H2SO4}, heated under reflux.}

			\diagram[1.0]{
				\schemestart[0, 2.0, thick]
					\chemfig{!\molRon-[:0]C(=[:45]!\molO)(-[:315]@{oh}{\color{Red}O}|{\color{Red}H})}
					\hspace{2mm} + \hspace{2mm}
					\chemfig{!\molRtw(-[:180]!\molO-[:180,0.75]@{hyd}H)}
					\arrow{<=>[conc. \ch{H2SO4}][reflux]}
					\chemfig{C(-[:180]!\molRon)(=[:45]!\molO)(-[:315]!\molO-[:0]!\molRtw)}
					\hspace{2mm} + \hspace{2mm}
					\ch{H2O}
				\schemestop
			}


		\subsection{From Acyl Chlorides}

			Again, note that only acyl chlorides can react with phenols (or more effectively with phenoxide) to form an ester.

			\vspace{1.5em}
			\vbox{\textbf{Conditions:}	\tabto{35mm}Acyl chloride and alcohol,
										\tabto{35mm}Room temperature.}

			\diagram[1.0]{
				\schemestart[0, 1.5, thick]
					\chemfig{!\molRon-[:0]C(=[:45]!\molO)(-[:315]@{cl}\color{OliveGreen}\chlorine)}
					\hspace{2mm} + \hspace{2mm}
					\chemfig{!\molRtw(-[:180]!\molO-[:180,0.75]@{hyd}H)}
					\arrow
					\chemfig{C(-[:180]!\molRon)(=[:45]!\molO)(-[:315]!\molO-[:0]!\molRtw)}
					\hspace{2mm} + \hspace{2mm}
					\ch{H\chlorine}
				\schemestop
			}

		% end subsection

	% end section



	\pagebreak
	\section{Ester Reactions}

		\subsection{Hydrolysis}

			Like most hydrolysis reactions, esters can be hydrolysed in either an acidic or alkaline medium. However, acid hydrolysis is
			the same equilibrium reaction as esterification with a carboxylic acid, only this time an excess of water is used to move
			the position of equilibrium --- thus the reaction is slow and the yield is low.

			\subsubsection{Acid Hydrolysis}

				\vspace{1.5em}
				\vbox{\textbf{Conditions:}	\tabto{35mm}Dilute \ch{H2SO4}, heat under reflux.}

				\diagram[1.0]{
					\schemestart[0, 2.0, thick]
						\chemfig{C(-[:180]!\molR)(=[:45]!\molO)(-[:315]!\molO-[:0]!\molRon)}
						\hspace{2mm} + \hspace{2mm}
						\chemfig{\ch{H2O}}
						\arrow{<=>[dil. \ch{H2SO4}][reflux]}
						\chemfig{C(-[:180]!\molR)(=[:60]!\molO)(-[:300]!\molOH)}
						\hspace{2mm} + \hspace{2mm}
						\chemfig{!\molRon-[:0]!\molOH}
					\schemestop
				}

			% end subsubsection


			\subsubsection{Alkaline Hydrolysis (Saponification)}

				Alternatively, using an alkaline medium results in a fast and irreversible reaction, creating a carboxylate salt. Dilute
				acid can be used to protonate the salt afterwards.

				This reaction is also known as saponification, and is preferred for its higher yield and faster reaction.

				\vspace{1.5em}
				\vbox{\textbf{Conditions:}	\tabto{35mm}Dilute \ch{NaOH}, heat under reflux.}

				\diagram[1.0]{
					\schemestart[0, 2.0, thick]
						\chemfig{C(-[:180]!\molR)(=[:45]!\molO)(-[:315]!\molO-[:0]!\molR)}
						\hspace{2mm} + \hspace{2mm}
						\chemfig{\ch{H2O}}
						\arrow{->[dil. \ch{NaOH}][reflux]}
						\chemfig{C(-[:180]!\molR)(=[:60]!\molO)(-[:300]!\molO\mch Na\pch)}
						\hspace{2mm} + \hspace{2mm}
						\chemfig{!\molRon-[:0]!\molOH}
					\schemestop
				}

			% end subsubsection

		% end subsection

		% end subsection

	% end section

% end part
