% Chapter 15 - Amides.tex
% Copyright (c) 2014 - 2016, zhiayang@gmail.com
% Licensed under the Apache License Version 2.0.



\pagebreak
\hypertarget{ChapterAmides}{}
\part{Amides}

	\section{Structure}

		Amides are also a carboxylic acid derivative, this time where the \ch{OH} group has been replaced with an \ch{NH2} group that
		may or may not be substituted.

		Naturally, amides can also be classified into primary, secondary and tertiary amides.

		\begin{center}\begin{table}[ht]\renewcommand{\arraystretch}{1.4}
		\begin{tabu} to \textwidth {| X[c,m] | X[c,m] | X[c,m] |}

			\hline
			% row one: molecules
			\vspace{2mm}\chemfig{C(-[:180]!\molR)(=[:60]!\molO)(-[:300]!\molN(-[:0]H)(-[:240]H))}			\vspace{2mm}	&
			\vspace{2mm}\chemfig{C(-[:180]!\molR)(=[:60]!\molO)(-[:300]!\molN(-[:0]H)(-[:240]!\molR))}		\vspace{2mm}	&
			\vspace{2mm}\chemfig{C(-[:180]!\molR)(=[:60]!\molO)(-[:300]!\molN(-[:0]!\molR)(-[:240]!\molR))}	\vspace{2mm}	\\

			\hline
			\vspace{2mm}Primary (\SI{1}{\degree})		\vspace{2mm} &
			\vspace{2mm}Secondary (\SI{2}{\degree})		\vspace{2mm} &
			\vspace{2mm}Tertiary (\SI{3}{\degree})		\vspace{2mm} \\
			\hline

		\end{tabu}
		\end{table}\end{center}\vspace{-10mm}

	% end section



	\section{Physical Properties}

		\paragraph{Melting and Boiling Points}

		Amides are generally crystalline solids at room temperature, owing to their ability to form hydrogen bonds. The electronegative
		nitrogen atom has two hydrogen atoms (in unsubstituted amides), allowing two hydrogen bonds per molecule. Thus, they have
		higher melting and boiling points than carboxylic acids.


		\paragraph{Solubility}

		Amides are generally quite soluble in water, again due to the ability to form hydrogen bonds with the water molecules. However,
		the presence of long alkyl chains can hinder the formation of these bonds, leading to decreased solubility at higher carbon lengths.

	% end section



	\pagebreak
	\section{Creation of Amides}

		\subsection{From Acyl Chlorides}

			Amides are generally created by reacting an acyl chloride with ammonia or an amide, replacing the \ch{\Cl} group. The amine is
			used in excess to compensate for the reaction with \ch{H\Cl} produced during the reaction.

			Furthermore, carboxylic acids cannot be used to create amides, since they will react in an acid-base reaction with the amine instead.
			Tertiary amines also cannot be used since they do not have a hydrogen atom to replace.

			\vspace{1.5em}
			\vbox{\textbf{Conditions:}\tabto{35mm}Acyl chloride, amine of choice in excess, room temperature.}
			\vbox{\textbf{Observations:}\tabto{35mm}Formation of white fumes of \ch{H\Cl} gas.}

			\diagram[1.0]{
				\schemestart[0, 1.5, thick]
					\chemfig{C(-[:180]!\molR)(=[:60]!\molO)(-[:300]!\molCl)}
					\hspace{5mm} + \hspace{5mm}
					\chemfig{!\molN(-[:180]H)(-[:60]!\molRon)(-[:300]!\molRtw)}
					\arrow(.mid east--.mid west)
					\chemfig{C(-[:180]!\molR)(=[:60]!\molO)(-[:300]!\molN(-[:0]!\molRon)(-[:240]!\molRtw))}
					\hspace{5mm} + \hspace{5mm}
					\chemfig{H\Cl}
				\schemestop
			}

		% end subsection

	% end section



	\pagebreak
	\section{Amide Reactions}

		\subsection{General Reactivity}

			Amides are generally unreactive compared to amides and carboxylic acids. This is due to the delocalisation of the lone pair
			on nitrogen across the carboxyl group; the carbon is \sptwo hybridised, allowing for p-orbital overlaps between the nitrogen,
			carbon and oxygen atoms.

			There is reduced electron density on the nitrogen atom, and thus amides are neutral in solution since the lone pair is unavailable
			to accept a proton to act as a Brønsted base.

			\diagram[1.0]{
				% here's the deal: we draw the carboxylic acid as normal.
				% then, we draw a benzene ring attached to the carbon.
				% it is attached with a negative bond length to bring the circle closer to the carbon
				% the circle is cut off "**[130,230,dashed]6" to certain angles
				% none of the benzene ring bonds are actually drawn (hence 6(), without ------ inside)
				% a bond is drawn from the first position to the centre (using bond angle = :0), for the plus.
				% this is stupid.

				\chemfig{C(-[:60]!\molO)(-[:300]!\molAmine)(-[:180]!\molR)(
					-[:0,-0.10,,,draw=none](**[130,230,dashed]6([,,,,draw=none](-[:-5,0.60,,,draw=none]\fscrm))))}

			}{The delocalisation of the lone pair across the entire carboxyl group.}

		% end subsection



		\subsection{Hydrolysis}

			Amides can be hydrolysed with either acids or bases to give a carboxylic acid, and an amine. If the original amide is unsubstituted,
			then the result will be \ch{NH3}, or, in an acidic medium, \ch{NH4+}. The products illustrated below can have \ch{R} be an alkyl
			substituent or a hydrogen atom.

			Note that the formation of \ch{NH4+} is due to the fact that \ch{NH3} is a base, and will react with \ch{H+} ions in the acidic
			medium.

			\subsubsection{Acid Hydrolysis}

				The hydrolysis of an amide in an acidic medium yields the carboxylic acid directly.

				\vspace{1.5em}
				\vbox{\textbf{Conditions:}	\tabto{35mm}Dilute \ch{H2SO4} or \ch{H\Cl}, heat under reflux.}

				\diagram[1.0]{
					\schemestart[0, 2.0, thick]
						\chemfig{C(-[:180]!\molR)(=[:60]!\molO)(-[:300]N|R\sbs{2})}
						\arrow{->[dil. \ch{H2SO4}][heat with reflux]}
						\chemfig{C(-[:180]!\molR)(=[:60]!\molO)(-[:300]!\molOH)}
						\hspace{5mm} + \hspace{5mm}
						\ch{NH2R2+}
					\schemestop
				}
			% end subsubsection


			\subsubsection{Alkaline Hydrolysis}

				The alternative is to conduct the hydrolysis in an alkaline medium using \ch{NaOH}, which gives a carboxylate salt.
				This salt can be acidified using a dilute acid to yield the carboxylic acid.

				\vspace{1.5em}
				\vbox{\textbf{Conditions:}	\tabto{35mm}Dilute \ch{NaOH}, heat under reflux.}

				\diagram[1.0]{
					\schemestart[0, 2.0, thick]
						\chemfig{C(-[:180]!\molR)(=[:60]!\molO)(-[:300]N|R\sbs{2})}
						\arrow{->[dil. \ch{H2SO4}][heat with reflux]}
						\chemfig{C(-[:180]!\molR)(=[:60]!\molO)(-[:300]!\molO\mch Na\pch)}
						\hspace{5mm} + \hspace{5mm}
						\ch{NHR2}
					\schemestop
				}{The carboxylate salt can be acidified later.}

			% end subsubsection

		% end subsection






	% end section


% end part





























