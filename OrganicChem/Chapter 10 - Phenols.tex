% Chapter 10 - Phenols.tex
% Copyright (c) 2014 - 2016, zhiayang@gmail.com
% Licensed under the Apache License Version 2.0.


\pagebreak
\hypertarget{ChapterPhenols}{}
\part{Phenols}

	\section{Structure}

		Phenols are basically benzene rings with an \ch{OH} group substituted on the ring. However, their chemical behaviour is quite
		different from that of regular aliphatic alcohols --- hence this separation of topic.

		\diagram[1.0]{
			\chemfig{**6(----(-[:90]!\molOH)--)}
		}{The structure of phenol.}


		\subsection{Physical Properties}

			\paragraph{Melting and Boiling Points}

			Most phenols are solids at room temperature, due to the existence of hydrogen bonds facilitated by the \ch{OH} group. As such,
			their melting points are much higher than simple benzene rings.


			\paragraph{Solubility}

			Phenols are moderately soluble in water compared to benzenes, due to the ability to form favourable solvent-solute interactions,
			again due to the hydrogen bonding provided by the \ch{OH} group. This is despite the large hydrophobic benzene ring.

	% end section


	\section{Phenol Reactions}

		\subsection{Acidity of Phenols}

			Phenols are far more acidic than alcohols, due to the ability for the π-electron cloud of the benzene ring to delocalise the
			negative charge on the oxygen atom after \ch{H+} dissociates. Thus, the conjugate base of phenol is much more stable than
			that of alcohols, and so the \ch{H+} ion will be more likely to dissociate, increasing acidity.

			\subsubsection{Effect of Substituents}

				Similar to regular alcohols, electron-withdrawing substituents attached to the benzene ring can delocalise the negative
				charge of the phenoxide, increasing its stability and hence acidity.

				Conversely, electron-donating groups (mainly through resonance) attached to the benzene ring have the opposite effect,
				intensifying the negative charge and further destabilising the phenoxide ion. Thus, acidity is reduced in this case.

			% end subsection

		% end subsection


		\subsection{Reactions as an Acid}

			While phenols can react with bases and metals, they are still too weak of an acid to react with carbonates.

			\subsubsection{Reaction with Metals}
				In a manner similar to acids and alcohols, phenols can react with \textit{reactive} metals to form ionic salts.

				The \ch{O-H} bond is broken, and the \ch{H} atom is replaced with the metal cation.

				\vspace{1.5em}
				\vbox{\textbf{Conditions:}\tabto{35mm}Solid metal (eg. \ch{Na}, \ch{K}, etc.), room temperature.}
				\vbox{\textbf{Observations:}\tabto{35mm}Slow effervescence of \ch{H2} gas.}

				\diagram[1.0]{
					\schemestart[0,1.5,thick]
						\ch{Na \stS}
						\hspace{5mm} + \hspace{5mm}
						\chemfig[yshift=-1.5em]{**6(----(-[:90]!\molOH)--)}
						\arrow(.base east--.base west)
						\chemfig[yshift=-1.5em]{**6(----(-[:90]!\molO|\mch Na\pch)--)}
						\hspace{5mm} + \hspace{5mm}
						\ch{\fracHalf H2 \stG}
					\schemestop
				}

			% end subsubsection

			\pagebreak
			\subsubsection{Reaction with Bases}

				Unlike alcohols, phenols are strong enough to react with bases, also forming ionic salts in a similar manner.


				\vspace{1.5em}
				\vbox{\textbf{Conditions:}\tabto{35mm}Aqueous base (eg. \ch{NaOH}, \ch{KOH}, etc.), room temperature.}
				\vbox{\textbf{Observations:}\tabto{35mm}Phenol dissolves.}

				\diagram[1.0]{
					\schemestart[0,1.5,thick]
						\chemfig{\ch{NaOH \stAq}}
						\hspace{5mm} + \hspace{5mm}
						\chemfig[yshift=-1.5em]{**6(----(-[:90]!\molOH)--)}
						\arrow(.base east--.base west)
						\chemfig[yshift=-1.5em]{**6(----(-[:90]!\molO|\mch Na\pch)--)}
						\hspace{5mm} + \hspace{5mm}
						\chemfig{\ch{H2O \stG}}
					\schemestop
				}


			% end subsubsection

		% end subsection


		\subsection{Oxidation of Phenols}

			Another trick --- the carbon atom attached to the \ch{OH} group has no hydrogen atom, and so phenols \textit{cannot} be
			oxidised.

		% end subsection


		\subsection{Electrophilic Substitution of Phenols}

			Phenols can undergo electrophilic substitution on the other, non \ch{OH}-substituted positions. This is more a characteristic
			of the underlying benzene ring than it is of the \ch{OH} group, however.

			Still, the \ch{OH} group is electron-donating via the resonance effect (due to the overlap of the p-orbital of the \ch{O} atom
			with the π-electron cloud of the benzene ring), and as such is an \textit{activating} substituent. Thus, the electron density
			in the ring is increased, and it undergoes electrophilic substitution reactions with greater ease.

			Note that the \ch{OH} group is \textit{ortho/para}, or 2,4-directing.

			\pagebreak
			\subsubsection{Halogenation of Phenols}

				Due to the activating nature of the \ch{OH} group, it facilitate electrophilic substitution of hydrogens on the benzene ring
				without the use of a catalyst (eg. anhydrous \ch{FeBr3} like for normal benzene), and at room temperature as well.


				\vspace{1.5em}
				\vbox{\textbf{Conditions:}	\tabto{35mm}\ch{Br2 \stAq} / \ch{\chlorine 2 \stAq}, room temperature.}

				\vspace{0.75em}
				\vbox{\textbf{Observations:}\tabto{35mm}\boit{\color{Dandelion}Yellow} \ch{Br2 \stAq} / \boit{\color{Goldenrod}pale yellow} \ch{\chlorine2 \stAq} decolourises,			\tabto{35mm}white precipitate is formed.}

				\diagram[1.0]{
					\schemestart[0, 1.5, thick]
					\chemfig[yshift=-1.15em]{**6(-(-[:270,,,,draw=none]\phantom{Br})---(-[:90]!\molOH)--)}
					\hspace{5mm} + \hspace{5mm}
					\ch{X2 \stAq}
					\arrow
					\ch{HX}
					\hspace{5mm} + \hspace{5mm}
					\chemfig[yshift=-1.15em]
					{**6(-(-[:270]!\molX)--(-[:30]!\molX)-(-[:90]!\molOH)-(-[:150]!\molX)-)}
					\schemestop
				}


				Alternatively, to prevent tri-substitution, methods can be employed to reduce the electrophilic reactivity of bromine, including
				using a non-polar inert solvent to prevent instantaneous dipole moments (or induced dipole moments in fact), and reducing
				the temperature.



				\vspace{1.5em}
				\vbox{\textbf{Conditions:}	\tabto{35mm}\ch{Br2 \stL} / \ch{\chlorine 2 \stG} in inert, non-polar solvent (eg. \ch{C\chlorine 4}), room temperature.}
				\vbox{\textbf{Observations:}\tabto{35mm}\boit{\color{Mahogany}Reddish-brown} \ch{Br2} / \boit{\color{YellowGreen}yellowish-green} \ch{\chlorine2} decolourises.}

				\diagram[0.9]{
					\schemestart[0, 1.5, thick]
					\chemfig[yshift=-1.15em]{**6(-(-[:270,,,,draw=none]\phantom{Br})---(-[:90]!\molOH)--)}
					\hspace{5mm} + \hspace{5mm}
					\ch{Br2 \stL}
					\arrow
					\ch{HX}
					\hspace{5mm} + \hspace{5mm}
					\chemfig[yshift=-1.15em]{**6(-(-[:270,,,,draw=none]\phantom{X})--(-[:30]!\molX)-(-[:90]!\molOH)--)}
					\hspace{5mm} + \hspace{5mm}
					\chemfig[yshift=-1.15em]{**6(-(-[:270]!\molX)---(-[:90]!\molOH)--)}
					\schemestop
				}



			% end subsubsection

			\pagebreak

			\subsubsection{Nitration of Phenols}

				Similar to benzene, phenols can also be nitrated, but this time without requiring concentrated \ch{H2SO4} as a catalyst.
				The degree of substitution can be controlled using the concentration of \ch{HNO3} used.

				Concentrated \ch{HNO3} can be used to achieve tri-substitution, \enquote{moderately concentrated} for di-substitution, and
				dilute \ch{HNO3} for mono-substitution.

				\vspace{1.5em}
				\vbox{\textbf{Conditions:}	\tabto{35mm}Dilute \ch{HNO3 \stAq}, room temperature.}

				\diagram[1.0]{
					\schemestart[0, 2.0, thick]
					\chemfig[yshift=-1.15em]{**6(-(-[:270,,,,draw=none]\phantom{!\molNitro})---(-[:90]!\molOH)--)}
					\arrow{->[dil. \ch{HNO3}]}
					\chemfig[yshift=-1.15em]{**6(-(-[:270,,,,draw=none]\phantom{!\molNitro})--(-[:30]!\molNitro)-(-[:90]!\molOH)--)}
					\hspace{5mm} + \hspace{5mm}
					\chemfig[yshift=-1.15em]{**6(-(-[:270]!\molNitro)---(-[:90]!\molOH)--)}
					\schemestop
				}

				\vspace{1.5em}
				\vbox{\textbf{Conditions:}	\tabto{35mm}Concentrated \ch{HNO3 \stAq}, room temperature.}

				\diagram[1.0]{
					\schemestart[0, 2.0, thick]
					\chemfig[yshift=-1.15em]{**6(-(-[:270,,,,draw=none]\phantom{!\molNitro})---(-[:90]!\molOH)--)}
					\arrow{->[conc. \ch{HNO3}]}
					\chemfig[yshift=-1.15em]{**6(-(-[:270]!\molNitro)--(-[:30]!\molNitro)-(-[:90]!\molOH)-(-[:150]!\molNitro)-)}
					\schemestop
				}


			% end subsubsection

		% end subsection

		\pagebreak
		\subsection{Formation of Complex with \ch{Fe\chlorine3}}

			Phenol can form a complex with neutral \ch{Fe\chlorine3 \stAq}, forming a violet complex. This reaction is mainly
			used as a distinguishing test for the presence of phenol, due to the very obvious colour.

			However, the violet only applies to unsubstituted phenol --- the colour might change slightly if the phenol is
			indeed substituted.

			\vspace{1.5em}
			\vbox{\textbf{Conditions:}	\tabto{35mm}Neutral \ch{Fe\chlorine3 \stAq}, room temperature.}
			\vbox{\textbf{Observations:}\tabto{35mm}\boit{\color{Violet}Violet} complex formed.}


			\diagram[1.0]{
				\schemestart[0, 2.0, thick]
				6 \hspace{2mm} \arrow{0}[,0]
				\chemfig{[:270]**6(----(-[:0]!\molOH)--)}
				\arrow{->[neutral \ch{Fe^3+ \stAq}]}
				\chemleft[
				\chemfig{[:270]**6(-(-[:180,0.3,,,draw=none]@{left}\phantom{X})---(-[:0]{\color{Red}O}(-[@{right,0.25}:0,1.5,,,-Stealth]Fe))--)}
				\chemright]
				\schemestop

				\chemmove{\node[xshift=3mm] at (c3.north east) {3–};}
				\makebraces[2.5em, 2.5em]{6}{left}{right}
			}

		% end subsection



		\subsection{Esterification (Nucleophilic Acyl Substitution)}

			Similar to how alcohols can undergo esterification with carboxylic acids, phenols too can esterify, but only with acyl
			chlorides. This is because carboxylic acids are too unreactive, and phenols are too poor of a nucleophile to react with any
			significant yield.

			The poor nucleophile character is caused by the delocalisation of the lone pair on the \ch{O} atom into the π-system. Thus,
			it is possible to increase the effectiveness of this reaction by deprotonating the phenol, exposing a negatively charged oxygen
			atom. Note that this is still insufficient to enable a reaction with carboxylic acids.


			\vspace{1.5em}
			\vbox{\textbf{Conditions:}\tabto{35mm}Acyl chloride, \ch{NaOH \stAq}, room temperature.}
			\vbox{\textbf{Observations:}\tabto{35mm}Formation of white fumes of \ch{H\chlorine} gas.}

			\diagram[0.9]{
				\schemestart[0, 1.5, thick]
					\chemfig{C(-[:180]!\molR)(=[:60]!\molO)(-[:300]!\molCl)}
					\hspace{5mm} + \hspace{5mm}
					\chemfig[yshift=-1.15em]{**6(----(-[:90]!\molO\mch)--)}
					\arrow(.base east--.base west)
					\chemfig{C(-[:180]!\molR)(=[:60]!\molO)(-[:300]!\molO-[:0]**6(------))}
					\hspace{5mm} + \hspace{5mm}
					\chemfig{H\chlorine}
				\schemestop
			}

		% end subsection

	% end section

% end part
