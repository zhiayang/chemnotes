% Chapter 03 - Reaction Mechanisms.tex
% Copyright (c) 2014 - 2016, zhiayang@gmail.com


\pagebreak
\hypertarget{ChapterReactionMechanisms}{}
\part{Reaction Mechanisms}

	\section{Bond Breaking}
		\subsection{Homolytic Fission}

			Homolytic fission involves the splitting of a single bond, with an equal (\textit{homo}) distribution of the two electrons
			of the aforementioned bond. This results in the formation of radicals, since now both atoms from the bond will have
			unpaired electrons.

			The movement of these single electrons is represented with single-hooked arrows. The arrow starts from the bond line, and points
			towards the target atom receiving the electron.

			\diagram{
				\schemestart[0, 1.5, thick]
				\chemfig{@{carb}C(-[@{bond}:0]@{hyd}H)(-[:90]H)(-[:180]H)(-[:270]H)}
				\arrow{->}
				\chemfig{\lewis{0.,C}(-[:90]H)(-[:180]H)(-[:270]H)} \hspace{5mm} + \hspace{5mm} \chemfig{\lewis{4.,H}}
				\schemestop

				% draw arrows
				\chemmove{\draw[-{Stealth[left]},line width=0.4mm,shorten <=1mm,shorten >=1mm](bond).. controls +(90:5mm) and +(120:8mm).. (hyd);}
				\chemmove{\draw[-{Stealth[left]},line width=0.4mm,shorten <=1mm,shorten >=1mm](bond).. controls +(270:5mm) and +(285:8mm).. (carb);}

			}{The homolytic fission of \ch{CH4}, forming \ch{^.CH3} and \ch{^.H} radicals.}

		% end subsection

		\subsection{Heterolytic Fission}

			As the name implies, heterolytic fission is the opposite; it distributes both electrons of the bond to a single atom, which
			usually results in the formation of ions. The transfer here is represented using a full (double-hooked) arrow.


			\diagram{
				\schemestart[0, 1.5, thick]
				% note: chlorine has to be coloured manually here
				% because somewhere in chemfig it fucks up, you can't name a submolecule.
				\chemfig{@{carb}C(-[@{bond}:0]@{chl}\color{OliveGreen}\chlorine)(-[:90]H)(-[:180]H)(-[:270]H)}
				\arrow{->}
				\chemfig{C|\sps{+}(-[:90]H)(-[:180]H)(-[:270]H)}
				\hspace{5mm} + \hspace{5mm}
				\chemfig{\lewis{4:,\color{OliveGreen}\chlorine}}
				\schemestop

				% draw arrows
				\chemmove{\draw[-{Stealth},line width=0.4mm,shorten <=1mm,shorten >=1mm](bond).. controls +(90:5mm) and +(120:8mm).. (chl);}


			}{The heterolytic fission of \ch{CH3\chlorine}, resulting in \ch{CH3+} and \ch{\chlorine-} ions}

		% end subsection

	\pagebreak
	\section{Bond Forming}
		\subsection{Single Electrons}

			Bond forming is essentially the reverse of bond breaking, and hence the same notations apply --- single-hooked arrows for the
			movement of single electrons, and double-hooked arrows for the movement of an electron pair.

			In this case, two single electrons from each atom contribute to the new bond.


			\diagram{
				\schemestart[0, 1.5, thick]
				\chemfig{@{carb}\lewis{1.,C}(-[:90]H)(-[:180]H)(-[:270]H)}
				\hspace{5mm} \chemfig{@{plus}+} \hspace{5mm}
				\chemfig{@{hyd}\lewis{5.,H}}
				\arrow{->}
				\chemfig{C(-[:0]H)(-[:90]H)(-[:180]H)(-[:270]H)}
				\schemestop

				% draw arrows
				\chemmove{\draw[-{Stealth[left]},line width=0.4mm,shorten <=2mm,shorten >=1mm](carb).. controls +(45:5mm) and +(120:8mm).. (plus);}
				\chemmove{\draw[-{Stealth[left]},line width=0.4mm,shorten <=2mm,shorten >=1mm](hyd).. controls +(225:5mm) and +(285:8mm).. (plus);}

			}{The formation of a bond between \ch{^.CH3} and \ch{^.H} radicals to form \ch{CH4}.}


		% end subsection


		\subsection{Electron Pairs}

			Electron pairs usually come from negatively charged radicals or lone pars, although this is not a rule.
			Both electrons come from a single source to form a bond, but this is \textit{not} a dative bond --- it is simply a normal
			bond.

			\diagram{
				\schemestart[0, 1.5, thick]
				\chemfig{@{carb}C|\sps{+}(-[:90]H)(-[:180]H)(-[:270]H)}
				\hspace{5mm} + \hspace{5mm}
				\chemfig{@{chl}\lewis{3:,\color{OliveGreen}\ch{\chlorine-}}}
				\arrow{->}
				\chemfig{C(-[:0]!\molCl)(-[:90]H)(-[:180]H)(-[:270]H)}
				\schemestop

				% draw arrows
				\chemmove{\draw[-{Stealth},line width=0.4mm,shorten <=3mm,shorten >=1mm](chl).. controls +(150:12mm) and +(45:12mm).. (carb);}

			}{The formation of a bond between \ch{^.CH3} and \ch{\chlorine-} radicals, to form \ch{CH3\chlorine}.}

		% end subsection
	% end section


	\pagebreak
	\section{Electrophiles and Nucleophiles}
		\subsection{Electrophiles}

			Electrophiles are electron-deficient species that accept an electron pair from a nucleophile donor. Most electrophiles
			either have a positive charge, or contain an atom that is polarised and thus has a partial positive charge.

			Examples of electrophiles include \ch{CH3+}, \ch{Br+}, \ch{NO2+}, polarised \ch{Br2}, and \ch{HBr}.

		% end subsection

		\subsection{Nucleophiles}

			Nucleophiles are electron-rich species that donate electron pairs to electrophiles. This process typically results
			in the formation of a new covalent bond. Nucleophiles usually contain atoms that are either negatively charged, or,
			more frequently, contain lone electron pairs that are not bonded.

			Molecules with a π-bond, such as ethene or benzene, can also act as nucleophiles, due to the high electron
			density of the π-system.

			Examples of nucleophiles include \ch{H2O}, \ch{NH3}, and \ch{OH-}.

		% end subsection
	% end section
% end part

















