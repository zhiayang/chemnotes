% Chapter 01 - Functional Groups.tex
% Copyright (c) 2014 - 2016, zhiayang@gmail.com


\pagebreak
\part{Functional Groups}

	\section{Overview}

		Functional groups are the main determinant of the chemical properties of a molecule. Molecules with the
		same functional group are of the same \textit{family} and similar \textit{chemical properties}.

		Functional groups can be any size –– the ketone functional group is just an oxygen double-bonded to a carbon atom
		in any position, while the carboxylic acid functional group has a more complex structure.

	% end section

	\section{Basic Functional Groups}

		Note that this table is ordered based on the \textit{priority} of the functional group, in terms of where it appears in
		an IUPAC-named chemical compound.

		\textit{R}, as used below, represents a substituent alkyl or aryl group, the former being an arbitrary carbon chain, and the
		latter being some aromatic ring derivative.

		\begin{center}\renewcommand{\arraystretch}{1.4}
		\begin{longtabu} to \textwidth {| X[c,m] | X[c,m] | X[c,m] | X[c,m] |}

			\hline

		Name			&	Prefix Form		&	Suffix Form		&	Structure \\
		\hline
		Carboxylic Acid	&	carboxy–*		&	*–oic acid		&
			\vspace{2mm}
			\chemfig{!\molO=C(-[:315]!\molOH)(-[:45]!\molR)}
			\vspace{2mm}	\\


		\hline
		Ester			&		–			&	*–oate			&
			\vspace{2mm}
			\chemfig[][scale=0.9]{!\molO=C(-[:315]!\molO(-[:0]!\molR))(-[:45]!\molR)}
			\vspace{2mm}	\\


		\hline
		Acyl Halide		&	halocarbonyl–*	&	*–oyl halide	&
			\vspace{2mm}
			\chemfig{!\molO=C(-[:315]!\molX)(-[:45]!\molR)}
			\vspace{2mm}	\\


		\hline
		Amide			&	carbamoyl–*		&	*–amide			&
			\vspace{2mm}
			\chemfig[][scale=0.9]{!\molO=C(-[:315]!\molR)(-[:45]!\molN(-[:0]!\molStar)(-[:135]!\molStar))}
			\vspace{2mm}	\\


		\hline
		Nitrile			&	cyano–*			&	*–nitrile		&
			\vspace{10mm}	% extra padding
			\chemfig{!\molN~C-!\molR}
			\vspace{10mm}	\\


		\hline
		Aldehyde		&	formyl–*		&	*–al			&
			\vspace{2mm}
			\chemfig{!\molO=C(-[:315]H)(-[:45]!\molR)}
			\vspace{2mm}	\\


		\hline
		Ketone			&	oxo–*			&	*–one			&
			\vspace{2mm}
			\chemfig{!\molO=C(-[:315]!\molR)(-[:45]!\molR)}
			\vspace{2mm}	\\


		\hline
		Alcohol			&	hydroxy–*		&	*–ol			&
			\vspace{2mm}
			\chemfig{C(-[:0]!\molOH)(-[:90]!\molStar)(-[:180]!\molStar)(-[:270]!\molStar)}
			\vspace{2mm}	\\


		\hline
		Amine			&	amino–*			&	*–amine			&
			\vspace{2mm}
			\chemfig{!\molR-[:0]!\molN(-[:45]!\molStar)(-[:315]!\molStar)}
			\vspace{2mm}	\\


		\hline
		Alkene			&	alkenyl–*		&	*–ene			&
			\vspace{2mm}
			\chemfig{C(-[:135]!\molStar)(-[:225]!\molStar)=[:0]C(-[:45]!\molStar)(-[:315]!\molStar)}
			\vspace{2mm}	\\


		\hline
		Ether			&	alkoxy–*		&		–			&
			\vspace{10mm}	% extra padding again
			\chemfig{!\molO(-[:330]!\molR)(-[:210]!\molR)}
			\vspace{10mm}	\\


		\hline
		Alkyl Halide	&	halogen–*		&		–			&
			\vspace{2mm}
			\chemfig{C(-[:0]!\molX)(-[:90]!\molStar)(-[:180]!\molStar)(-[:270]!\molStar)}
			\vspace{2mm}	\\


		\hline

		\end{longtabu}
		\end{center}

	% end section
% end part


