% Chapter 02 - Isomers.tex
% Copyright (c) 2014 - 2016, zhiayang@gmail.com


\pagebreak
\part{Isomers}

	\section{Overview}

	Isomers are molecules that share the same structural formula, yet exist in different forms. The two main
	forms of isomerism are structural isomerism and stereoisomerism. In both cases, the molecular formula of the
	compounds the same, while the chemical and physical properties may differ greatly.

	% end section

	\section{Structural Isomerism}

		\subsection{Chain Isomerism}

			In chain isomerism, molecules have the same functional groups, except they are placed in different positions
			along the carbon chain. Chain isomers typically have similar chemical properties, but different physical
			properties (since the shape of the carbon chain can determine the strength of id-id interactions).

			\diagram[1.0]{
				\chemfig{!\molMeR-[:30]-[:-30]-[:30]-[:-30]-[:30]!\molOH}\hspace{10mm}
				\chemfig{!\molMeR-[:30]-[:-30](-[:270]!\molMe)-[:30]-[:-30]!\molOH}
			}{Pentanol and 3-methylbutan-1-ol are chain isomers.}

		% end subsection



		\subsection{Positional Isomerism}
			Positional isomers have the same arrangement their of carbon chains, and the same functional groups. However, said
			functional groups are placed at different positions.


			\diagram[1.0]{
				\chemfig{!\molMeR-[:-30](-[:270]!\molOH)-[:30](-[:90]!\molOH)-[:-30]-[:30]!\molMe}\hspace{10mm}
				\chemfig{!\molMeR-[:-30](-[:270]!\molOH)-[:30]-[:-30](-[:270]!\molOH)-[:30]!\molMe}
			}{Butan-2,3-ol and butan-2,4-ol are positional isomers.}

		% end subsection



		\pagebreak
		\subsection{Functional Group Isomerism}
			This is somewhat a misnomer, since it is exactly the opposite of what it seems to intuit. Functional group isomers
			share only their molecular formulas; the carbon chain, the functional groups, etc. are all different.


			\diagram[1.0]{
				% note: this is drawn in this weird fashion (starting with the oxygen) because
				% chemfig starts the molecule's baseline at the first atom.

				\chemfig{!\molO=^[:90]-[:150]-[:210]!\molMeR}			\hspace{10mm}
				\chemfig{!\molMeR-[:30](=[:90]!\molO)-[:-30]!\molMe}
			}{Propanal and acetone are functional group isomers.}

		% end subsection
	% end section



	\section{Stereoisomerism}

		Stereoisomers of molecules have the same structure and functional groups, but those groups are arranged in a spatially
		differing way. The main forms of stereoisomerism are cis/trans (or E/Z) isomerism, and optical isomersim.

		Note that for certain cis/trans (E/Z) isomers, the trans (or E) variant can 'pack' better, resulting in a higher melting
		point. This doesn't affect the boiling point, since the molecules are too far apart in the gaseous phase for it to matter.

		Additionally, the presence of \textit{intramolecular} (within the same molecule) hydrogen bonds can also be a factor, since
		it reduces the strength of \textit{intermolecular} hydrogen bonds. Other than these, the chemical and physical properties of
		stereoisomers are mostly the same.


		% this puts the "cis/trans isomerism" chapter on the next page, so as not to split the image.
		\pagebreak
		\subsection{Cis/Trans Isomerism}

			Only alkenes can exhibit cis-trans isomerism, due to the \ch{C=C} bond that restricts rotation along its axis. This is because
			of the π-bonds that only bond at 180° intervals. At these positions, it forms the cis and trans isomers.

			\diagram[1.2]{
				\chemfig{C(-[:135]H)(-[:225]!\molRon)=C(-[:45]!\molRtw)(-[:315]H)}		\hspace{15mm}
				\chemfig{C(-[:135]H)(-[:225]!\molRon)=C(-[:45]H)(-[:315]!\molRtw)}
			}

			The molecule on the left is the \textit{trans} isomer, while the one on the right is the \textit{cis} isomer. \textit{trans}
			is from Latin, meaning \textit{other side}, while \textit{cis} obviously means \textit{same side}. In the former,
			the larger groups are on opposite sides of the alkene double bond, while in the latter, the larger groups are on the same
			side. Note that these \textit{groups} do not necessarily have to be functional groups –– they can be as large or as simple
			as one wishes.

			When naming cis/trans isomers, the qualifier \textit{cis} or \textit{trans} is placed before the stem name, like so:
			\begin{bulletlist}
				& cis-but-2-ene
				& trans-but-2-ene
			\end{bulletlist}

			Note that cis/trans isomerism is not applicable to molecules where identical groups are attached to the any one carbon atom,
			since there is a line of symmetry going along the plane of the double bond.


			\diagram[1.2]{
				\chemfig{C(-[:135]H)(-[:225]H)=C(-[:45]!\molRon)(-[:315]!\molRtw)}
			}


		% end subsection


		% again, this is to prevent hanging section titles with only a few lines of text before a page break.
		\pagebreak
		\subsection{E/Z Isomerism}

			E/Z is thought of as the more comprehensive, general form of cis/trans. Consider the alkene below, with two variants
			due to the lack of a plane of symmetry:

			\diagram[1.2]{
				\chemfig{C(-[:135]!\molCl)(-[:225]!\molF)=C(-[:45]!\molMe)(-[:315]!\molBr)}		\hspace{15mm}
				\chemfig{C(-[:135]!\molCl)(-[:225]!\molF)=C(-[:45]!\molBr)(-[:315]!\molMe)}
			}

			It has 4 distinct substituents, and thus a cis/trans designation cannot be used. Behold, E/Z isomerism. At its core it
			is simply another method of handling alkene isomers that works for more general forms of molecules.

			It follows a scheme of priorities (Cahn-Ingold-Prelog, or CIP), which are determined as such:
			\begin{romanlist}
				& The group with the largest atomic mass \textit{directly attached} to the alkene's carbon has the highest priority.
				& In the event of a tie, look at the atoms directly attached to \textit{that} offending atom. Use the previous
				  rule, favouring higher priority atoms, to tiebreak. Double bonds are counted twice.
				& Recursively evaluate the previous rule with attached atoms, to continue the tiebreaking process.
			\end{romanlist}

			The above scheme should be used to find the two highest-priority groups attached to the alkene, after which, if the two
			groups are attached on the same side of the alkene, then the molecule is the \textit{Z} isomer, and it is known as the
			\textit{E} isomer if the reverse is true.

			E and Z come from the German \textit{entgegen} and \textit{zusammen}, \textit{opposite} and \textit{together}
			respectively.

		% end subsection


		\pagebreak
		\subsection{Optical Isomerism}

			Due to the 3-dimensional tetrahedral nature of a carbon atom, if there are 4 \textit{distinct} substituents, two
			non-superimposable mirror images can exist, resulting from a lack of any plane of symmetry. Such carbons are called
			\textit{chiral carbons}, and are the \textit{chiral centres} of the molecule. A molecule can have multiple chiral
			centres.

			Consider butan-2-ol below. Carbon 2, with the OH group, has 4 different groups attached to it. As such, it is a chiral carbon.

			\diagram{
				\chemfig{H-C(-[:90]H)(-[:270]H)-C(-[:90]H)(-[:270]!\molOH)-C(-[:90]H)(-[:270]H)-C(-[:90]H)(-[:270]H)-H}
			}

			Indeed, it can take the form of 2 non-superimposable mirror images, that are differentiated by the adjacency of 2 of
			the 4 groups bonded to the chiral carbon. As can be seen, these two mirror images cannot be rearranged into the other,
			and are distinct –– these are called \textit{enantiomers}.

			\diagram[1.2]{
				\chemfig{C(-[:225]H)(-[:90]!\molEthyl)(<:[:315]!\molOH)(<[:350]!\molMe)}		\hspace{7.5mm}\vdashrule{30mm}\hspace{7.5mm}
				\chemfig{C(-[:315]H)(-[:90]!\molEthyl)(<:[:225]!\molOH)(<[:190]!\molMe)}

			}{The solid wedge indicates a bond coming out of the plane of the paper, while the
				dashed wedge indicates a bond going into the plane.}


			Optical isomers have identical physical properties, except for their treatment of plane-polarised light (this is
			where the light wave is only oscillating along one plane). One of the isomers will rotate this light clockwise, while the
			other will rotate it counterclockwise. The magnitude of both rotations are the same.

			If equal amounts of each enantiomer are present in a solution, it is called a \textit{racemic mixture}, and the net
			rotation of plane polarised light will be 0, for obvious reasons.

			Enantiomers usually have identical chemical properties, except for certain interactions that require a certain spatial
			orientation to work, such as proteins, that can only work with one of the optical isomers.

		% end subsection
	% end section
% end part
















