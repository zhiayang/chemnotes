% Chapter 09 - Alcohols.tex
% Copyright (c) 2014 - 2016, zhiayang@gmail.com
% Licensed under the Apache License Version 2.0.


\pagebreak
\hypertarget{ChapterAlcohols}{}
\part{Alcohols}

	\section{Aliphatic Alcohols}
		Aliphatic alcohols generally have the formula \ch{R-OH} –– the \ch{OH} group dictates the reactions of the alcohol. In a way they
		are basically carbon chains with one or more \ch{OH} groups replacing a hydrogen atom. Indeed, they are classified similarly to
		halogenoalkanes in this respect:

		\begin{center}\begin{table}[ht]\renewcommand{\arraystretch}{1.4}
		\begin{tabu} to \textwidth {| X[c,m] | X[c,m] | X[c,m] |}

			\hline
			% row one: molecules
			\vspace{2mm}	\chemfig{C(-[:0]!\molOH)(-[:90]H)(-[:180]!\molR)(-[:270]H)}				\vspace{2mm}	&
			\vspace{2mm}	\chemfig{C(-[:0]!\molOH)(-[:90]!\molR)(-[:180]!\molR)(-[:270]H)}		\vspace{2mm}	&
			\vspace{2mm}	\chemfig{C(-[:0]!\molOH)(-[:90]!\molR)(-[:180]!\molR)(-[:270]!\molR)}	\vspace{2mm}	\\

			\hline
			Primary (\SI{1}{\degree})		&
			Secondary (\SI{2}{\degree})		&
			Tertiary (\SI{3}{\degree})		\\
			\hline

		\end{tabu}
		\end{table}\end{center}\vspace{-10mm}

		Methanol, \ch{CH3OH}, is classified as a primary alcohol.

		As a sidenote, progression beyond this point requires basic understanding of acid-base equilibrium concepts, such as \Ka constants
		and acid/base conjugation.

		\subsection{Physical Properties}

			Compared to an alkane of a similar length, alcohols generally have much higher melting and boiling points. This is due to
			the formation of intermolecular hydrogen bonds (note that hydrogen bonds are a special kind of permanent dipole interaction),
			due to the difference in electronegativity in the \ch{O} and \ch{H} atoms. However, like other aliphatic carbon chains in
			general, instantaneous dipole-induced dipole (id-id) interactions are still present.

			Alcohols are also much more soluble in water and other polar solvents than alkanes, due again to their ability to form hydrogen
			bonds, this time resulting in favourable solvent-solute interactions that enable dissolution. They also retain their
			solubility in non-polar solvents due to the carbon backbone.

	% end section

	\pagebreak
	\section{Phenols}

		Phenols are benzene rings with one or more \ch{OH} groups directly bonded to the ring itself. They have markedly different properties
		from normal alcohols, and are covered in the \hyperlink{ChapterPhenols}{\boit{following chapter}}.

		\diagram[1.0]{
			\chemfig{**6(----(-[:90]!\molOH)--)}
		}{Phenol in all its glory.}

	% end section


	\section{Creation of Alcohols}

		Alcohols are versatile and crucial in organic synthesis, in that they can be transformed to and from a wide variety of other
		compounds and functional groups.

		\subsection{Electrophilic Addition (Hydration) of Alkenes}

			As previously mentioned, alkenes can undergo electrophilic addition with steam or water, forming a mono-alcohol.


			\vspace{1.5em}
			\vbox{\textbf{Conditions:}	\tabto{35mm}\SI{300}{\celsius}, at \SI{70}{atm}, \ch{H3PO4} catalyst, \textit{OR}
										\tabto{35mm}Concentrated \ch{H2SO4}, \ch{H2O}, warming.}

			\diagram[1.0]{
				\schemestart[0, 1.5, thick]
				\chemfig{C(-[:135]H)(-[:225]!\molMeR)=[:0]C(-[:45]H)(-[:315]H)}
				\hspace{5mm} + \hspace{5mm}
				\chemfig{\water}
				\arrow
				\chemfig{C(-[:270]!\molMe)(-[:180]H)(-[:90]!\molOH)-C(-[:90]H)(-[:0]H)(-[:270]H)}
				\schemestop
			}

		% end subsection


		\pagebreak
		\subsection{Nucleophilic Substitution of Halogenoalkanes}

			Alkyl halides can undergo nucleophilic substitution to create alcohols as well –– the \ch{OH-} ion acts as the
			nucleophile here, attacking the partial-positively charged \ch{C} atom attached to the halogen.

			\vspace{1.5em}
			\vbox{\textbf{Conditions:}\tabto{35mm}Aqueous \ch{NaOH} or \ch{KOH}, heat.}

			\diagram[1.0]{
				\schemestart[0,1.5,thick]
					\chemfig{!\molR-[:0]!\molX}
					\hspace{2mm} + \hspace{2mm}
					\chemfig{!\molOH\mch}
					\arrow
					\chemfig{!\molR-[:0]!\molOH}
					\hspace{2mm} + \hspace{2mm}
					\chemfig{!\molX\mch}
				\schemestop
			}
		% end subsection


		\subsection{Reduction of Aldehydes and Ketones}

			Carbonyl compounds (covered \hyperlink{ChapterAldehydesAndKetones}{\boit{later}}) can be reduced, forming either primary or secondary
			alcohols. Tertiary alcohols cannot be created through this method.

			All 3 common reducing agents can be used.

			\vspace{1.5em}
			\vbox{\textbf{Conditions:}	\tabto{35mm}\ch{Li\aluminium H4} in dry ether (diethyl ether), \textit{OR}
  										\tabto{35mm}\ch{NaBH4} in methanol, \textit{OR}
  										\tabto{35mm}\ch{H2 \stG}, \ch{Ni} catalyst, high temperature and pressure.}

			\diagram[1.0]{
				\schemestart[0,1.5,thick]
					\chemfig{C(=[:90]!\molO)(-[:210]!\molR)(-[:330]H)}
					\arrow
					\chemfig{C(-[:0]!\molOH)(-[:90]H)(-[:270]H)(-[:180]!\molR)}
				\schemestop
			}{Aldehydes (above) are reduced to primary alcohols, while ketones (below) are reduced to secondary alcohols.}

			\diagram[1.0]{
				\schemestart[0,1.5,thick]
					\chemfig{C(=[:90]!\molO)(-[:210]!\molRon)(-[:330]!\molRtw)}
					\arrow
					\chemfig{C(-[:0]!\molOH)(-[:90]!\molRon)(-[:270]!\molRtw)(-[:180]H)}
				\schemestop
			}


		% end subsection


		\pagebreak
		\subsection{Reduction of Carboxylic Acids}

			Carboxylic acids, also covered \hyperlink{ChapterCarboxylicAcids}{\boit{later}}, can also be reduced, forming primary alcohols.
			This reduction requires a strong reducing agent, so only \ch{Li\aluminium H4} can be used. See the
			\hyperlink{AppendixReducingAgents}{\boit{appendix}} for a list of reducing agents and their applicable uses.

			\vspace{1.5em}
			\vbox{\textbf{Conditions:}\tabto{35mm}\ch{Li\aluminium H4} in dry ether (diethyl ether).}

			\diagram[1.0]{
				\schemestart[0,1.5,thick]
					\chemfig{C(-[:180]!\molR)(=[:45]!\molO)(-[:315]!\molOH)}
					\arrow
					\chemfig{C(-[:0]!\molOH)(-[:90]H)(-[:270]H)(-[:180]!\molR)}
				\schemestop
			}

		% end subsection


		\subsection{Hydrolysis of Esters}

			Finally, esters can be hydrolysed to form an alcohol as one of its products. Note that this hydrolysis is preferably conducted
			in an alkaline medium, since the reaction is slow and reversible in an acidic medium, but fast and irreversible under
			alkaline conditions. All that is required is a dilute acid to protonate the carboxylate salt after.


			\vspace{1.5em}
			\vbox{\textbf{Conditions:}	\tabto{35mm}Dilute \ch{H2SO4}, heat under reflux, \textit{OR}
										\tabto{35mm}Dilute \ch{NaOH}, heat under reflux.}

			\diagram[1.0]{
				\schemestart[0, 2.0, thick]
					\chemfig{C(-[:180]!\molR)(=[:45]!\molO)(-[:315]!\molO-[:0]!\molRon)}
					\hspace{2mm} + \hspace{2mm}
					\chemfig{\ch{H2O}}
					\arrow{<=>[dil. \ch{H2SO4}][reflux]}
					\chemfig{C(-[:180]!\molR)(=[:30]!\molO)(-[:330]!\molOH)}
					\hspace{2mm} + \hspace{2mm}
					\chemfig{!\molRon-[:0]!\molOH}
				\schemestop
			}{A carboxylic acid is also formed, but for our purposes the alcohol is the main product.}
			\diagram[1.0]{
				\schemestart[0, 2.0, thick]
					\chemfig{C(-[:180]!\molR)(=[:45]!\molO)(-[:315]!\molO-[:0]!\molR)}
					\hspace{2mm} + \hspace{2mm}
					\chemfig{\ch{H2O}}
					\arrow{->[dil. \ch{NaOH}][reflux]}
					\chemfig{C(-[:180]!\molR)(=[:30]!\molO)(-[:330]!\molO\mch Na\pch)}
					\hspace{2mm} + \hspace{2mm}
					\chemfig{!\molRon-[:0]!\molOH}
				\schemestop
			}

		% end subsection

	% end section

	\pagebreak
	\section{Alcohol Reactions}

		\subsection{Acidity of Alcohols}

			Firstly, alcohols do have some weak acidic properties, as the \ch{OH} group is able to allow \ch{H+} to dissociate, albeit
			only slightly. This effect is so weak that alcohols react with neither bases (eg. \ch{NaOH}) nor turn blue litmus paper red.

			The \ch{H+} ion dissociates to a smaller extent than water, due to the presence of electron-donating alkyl groups that are
			not present in water (water only has \ch{H} atoms bonded to the oxygen). These alkyl groups \textit{intensify} the negative
			charge on the alkoxide ion, destabilising it. Thus, the \ch{H+} ion is unlikely to dissociate, as things like to remain stable.

			\subsubsection{Effect of Substituents}

				Substituents along the alcohol's carbon chain can have an effect on its acidity. Of course, groups closer to the \ch{OH}
				itself (better yet, attached to the same carbon atom) have a greater effect. Note that the length of the alkyl group
				(or the carbon chain) of the alcohol does not have a significant impact on its acidity.

				Electron withdrawing groups will \textit{disperse} the negative charge on the alkoxide anion, thus resulting in a higher
				likelihood for the \ch{H+} ion to dissociate, meaning a higher acidity (and thus higher \Ka). Conversely, electron-donating
				groups (basically alkyl groups) \textit{intensify} the negative charge, destabilising the anion and decreasing acidity.

				Note that the classification of electron-donating/withdrawing groups is different from that of arenes, as the carbon
				attached to the alcohol is \spthree hybridised. Refer to \hyperlink{CaveatResonanceTable}{\boit{this explanation}} for an
				elaboration.

			% end subsection

		% end subsection

		\pagebreak
		\subsection{Reactions as an Acid}

			\subsubsection{Reaction with Metals}
				In a manner similar to acids, alcohols can react with \textit{reactive} metals to form ionic salts. However, these reactions
				will be slower than that of mineral acids or even carboxylic acids.

				The \ch{O-H} bond is broken, and the \ch{H} atom is replaced with the metal cation.


				\vspace{1.5em}
				\vbox{\textbf{Conditions:}\tabto{35mm}Solid metal (eg. \ch{Na}, \ch{K}, etc.), room temperature.}
				\vbox{\textbf{Observations:}\tabto{35mm}Slow effervescence of \ch{H2} gas.}

				\diagram[1.0]{
					\schemestart[0,1.5,thick]
						2 \chemfig{!\molR-[:0]!\molOH}
						\hspace{2mm} + \hspace{2mm}
						\ch{2 Na \stS}
						\arrow
						2 \chemfig{!\molR-[:0]!\molO\mch Na\pch}
						\hspace{2mm} + \hspace{2mm}
						\ch{H2 \stG}
					\schemestop
				}

			% end subsubsection

			\subsubsection{Reaction with Bases and Carbonates}

				Alcohols are far too weak of an acid to react react with carbonates (\ch{CO3^2-}) to form \ch{CO2}, and also do not
				react with bases (eg. \ch{NaOH}) either.

			% end subsubsection

		% end subsection

		\subsection{Esterification (Nucleophilic Acyl Substitution)}

			\subsubsection{Carboxylic Acids}

				While this is more a property of carboxylic acids than alcohols, alcohols are still a main reagent, and so it is included here.
				These are condensation reactions, since \ch{H2O} or \ch{H\chlorine} is removed and the carbon chains joined.

				There are two main methods of creating esters, one of which is vastly more effective than the other. But first, the ineffective
				method. Carboxylic acids and alcohols can react together in a slow and reversible reaction, with the use of heat and a
				catalytic dehydrating agent –– a few drops of concentrated \ch{H2SO4} is often used.

				An additional limitation of this method is that phenols cannot be used to esterify, as it is too weak a nucleophile to
				perform the necessary substitution of the \ch{OH} on the carboxylic acid.

				To note, the \ch{C-O} bond in the carboxylic acid is broken, while only the \ch{O-H} bond in the alcohol is broken.

				\vspace{1.5em}
				\vbox{\textbf{Conditions:}	\tabto{35mm}Carboxylic acid and alcohol,
											\tabto{35mm}Several drops of concentrated \ch{H2SO4}, heated under reflux.}

				\diagram[1.0]{
					\schemestart[0, 2.0, thick]
						\chemfig{!\molRon-[:0]C(=[:45]!\molO)(-[:315]@{oh}{\color{Red}O}|{\color{Red}H})}
						\hspace{2mm} + \hspace{2mm}
						\chemfig{!\molRtw(-[:180]!\molO-[:180,0.75]@{hyd}H)}
						\arrow{<=>[conc. \ch{H2SO4}][reflux]}
						\chemfig{C(-[:180]!\molRon)(=[:45]!\molO)(-[:315]!\molO-[:0]!\molRtw)}
						\hspace{2mm} + \hspace{2mm}
						\ch{H2O}
					\schemestop

					% circle the groups
					\chemmove{
						\draw[-latex, red, thick]
						(oh.east) circle(5mm)
						(hyd.center) circle(4mm);
					}
				}{Note that it is the circled groups that form the water that is removed –– this is important.}

			% end subsubsection

			\subsubsection{Acyl Chlorides}

				Alternatively, acyl chlorides (which are themselves derivatives of carboxylic acids) can be used instead. This reaction
				is far superior, allowing for the esterification with phenols (which carboxylic acids cannot do).

				It is also conducted at room temperature and is an irreversible reaction requiring no catalysts, due to the high
				reactivity of the acyl chloride.

				\vspace{1.5em}
				\vbox{\textbf{Conditions:}	\tabto{35mm}Acyl chloride and alcohol,
											\tabto{35mm}Room temperature.}

				\diagram[1.0]{
					\schemestart[0, 1.5, thick]
						\chemfig{!\molRon-[:0]C(=[:45]!\molO)(-[:315]@{cl}\color{OliveGreen}\chlorine)}
						\hspace{2mm} + \hspace{2mm}
						\chemfig{!\molRtw(-[:180]!\molO-[:180,0.75]@{hyd}H)}
						\arrow
						\chemfig{C(-[:180]!\molRon)(=[:45]!\molO)(-[:315]!\molO-[:0]!\molRtw)}
						\hspace{2mm} + \hspace{2mm}
						\ch{H\chlorine}
					\schemestop

					% circle the groups
					\chemmove{
						\draw[-latex, red, thick]
						(cl.center) circle(4mm)
						(hyd.center) circle(4mm);
					}
				}{Again, note that it is the circled groups that form the water that is removed –– this is important.}


			% end subsubsection


		% end subsection


		\subsection{Nucleophilic Substitution with Halogens}

			Alcohols can undergo nucleophilic substitution, where the \ch{OH} group is substituted by a halogen atom. However, this
			has been covered in a previous chapter, under the \hyperlink{NSubAlcohols}{\boit{formation of halogenoalkanes}}, and will not
			be reproduced here for brevity.

			(click the bolded bit, this PDF has links!)

		% end subsection


		\pagebreak
		\subsection{Dehydration (Elimination of Water)}

			Alcohols can undergo dehydration, where an \ch{OH} group combines with the \ch{H} atom on a \textit{neighbouring} carbon, forming
			water and an alkene, under the right conditions.

			\vspace{1.5em}
			\vbox{\textbf{Conditions:}	\tabto{35mm}Excess concentrated \ch{H2SO4}, \SI{170}{\celsius}, \textit{OR}
										\tabto{35mm}\ch{\aluminium2O3}, heat.}

			\diagram[1.0]{
				\schemestart[0,1.5,thick]
					\chemfig{C(-[:90]H)(-[:270]@{hyd}H)(-[:180]H)-[:0]C(-[:90]H)(-[:0]H)(-[:270]@{oh}{\color{Red}O}|{\color{Red}H})}
					\arrow
					\chemfig{C(-[:135]H)(-[:225]H)=[:0]C(-[:45]H)(-[:315]H)}
					\hspace{2mm} + \hspace{2mm}
					\ch{H2O}
				\schemestop

				% circle the groups
				\chemmove{
					\draw[-latex, red, thick]
					(hyd.center) circle(4mm)
					(oh.east) circle(5mm);
				}
			}{Note that it is the circled groups that form the water that is removed.}

			Naturally, this reaction cannot take place when there is no hydrogen atom on any adjacent carbon atom.

			\subsubsection{Zaitsev's Rule}

				It is obvious that in certain cases, there will be ambiguity regarding which hydrogen atom is removed together with the \ch{OH}
				group. This is resolved with the Rule of Zaitsev, which states that the major product will be the alkene with more alkyl
				substituents on the double-bonded carbons.

				For example:

				\diagram[0.75]{
					\chemnameinit{\chemfig{C(-[:90]H)(-[:270]H)}}

					\schemestart[0, 2.0, thick]
						\chemname{\chemfig{C(-[:90]!\molMe)(-[:270]@{hyd1}H)(-[:180]!\molMe)
							-[:0]C(-[:90]H)(-[:270]@{oh}{\color{Red}O}|{\color{Red}H})-[:0]C(-[:90]H)(-[:0]H)(-[:270]@{hyd2}H)}}{}
						\arrow{->[][- \ch{H2O}]}
						\chemname{\chemfig{C(-[:135]!\molMe)(-[:225]!\molMe)=[:0]C(-[:45]H)(-[:315]!\molMe)}}{(major)}
						\hspace{5mm} + \hspace{5mm}
						\chemname{\chemfig{C(-[:90]!\molMe)(-[:180]H)(-[:270]!\molMe)-[:0]C(-[:90]H)=[:0]C(-[:45]H)(-[:315]H)}}{(minor)}
					\schemestop

					% circle the groups
					\chemmove{
						\draw[-latex, red, thick]
						(hyd1.center) circle(4mm)
						(hyd2.center) circle(4mm)
						(oh.east) circle(5mm);
					}
				}

			% end subsubsection

			\pagebreak
			\subsubsection{Elimination of Water in Gem-diols}

				Gem-diols, or carbons with two \ch{OH} groups attached at once, will generally spontaneously eliminate \ch{H2O} to form a
				ketone or aldehyde. The reason for this reaction is generally understood to be the much higher stability of the resulting
				carbonyl.

				Strongly electron-withdrawing substituents can result in the preference of the gem-diol, since the resulting carbonyl
				would be destabilised due to the very large partial-positive charge on the central carbon (with 2 or 3 electron-withdrawing groups).

				In general however, water is eliminated.


				\diagram[1.0]{

					\schemestart[0,1.5,thick]
						\chemfig{C(-[:90]!\molRon)(-[:180]!\molRtw)(-[:0]@{oh1}{\color{Red}O}|{\color{Red}H})(
							-[:270]@{oh2}{\color{Red}O}|{\color{Red}H})}
						\arrow
						\chemfig{C(-[:210]!\molRon)(-[:330]!\molRtw)(=[:90]!\molO)}
						\hspace{2mm} + \hspace{2mm}
						\ch{H2O}
					\schemestop

					% circle the groups
					\chemmove{
						\draw[-latex, red, thick]
						(oh1.east) circle(5mm)
						(oh2.east) circle(5mm);
					}
				}
			% end subsubsection
		% end subsection

		\pagebreak
		\subsection{Combustion}

			Alcohols such as ethanol can be burned as fuel, reacting with oxygen in the air to form \ch{CO2} and \ch{H2O}.

			\diagram[1.0]{
				\schemestart[0,1.0,thick]
					\ch{CH3CH2OH \stL}\hspace{2mm} + \hspace{2mm}\ch{3 O2 \stG}
					\arrow
					\ch{2 CO2 \stG}\hspace{2mm} + \hspace{2mm}\ch{3 H2O \stL}
				\schemestop
			}

		% end subsection

		\subsection{Oxidation of Primary Alcohols}

			Here again, a distinction must be made between primary, secondary and tertiary alcohols. For primary alcohols, there
			are two stages of oxidation –– first to an aldehyde, then further oxidation to a carboxylic acid.

			\hypertarget{OxidationOfPrimaryAlcohols}{}
			\subsubsection{Controlled Oxidation to Aldehydes}

				The \textit{controlled} part of this reaction lies in the immediate distillation of the aldehyde that is formed, using
				a fractionating column. Since aldehydes lack hydrogen bonding, they have a lower boiling point than their parent alcohol,
				and can be distilled away before they are further oxidised.

				Note that, in this case, the \textit{only} suitable oxidising agent is \ch{Cr2O7^2-}, as the alternative, \ch{MnO4-}, is
				too strong and will immediately oxidise the alcohol to a carboxylic acid. (note: this may be desired under some circumstances)


				\vspace{1.5em}
				\vbox{\textbf{Conditions:}	\tabto{35mm}\ch{K2Cr2O7} with dilute \ch{H2SO4},
											\tabto{35mm}heat with immediate distillation.}
				\vspace{0.75em}
				\vbox{\textbf{Observations:}\tabto{35mm}\boit{\color{BurntOrange}Orange} \ch{Cr2O7^2-}, turns \boit{\color{LimeGreen}green}
														(\ch{Cr^3+} formed).}

				\diagram[1.0]{
					\schemestart[0,1.5,thick]
						\chemfig{C(-[:90]H)(-[:270]H)(-[:180]!\molR)(-[:0]!\molOH)}
						\arrow
						\chemfig{C(-[:210]H)(-[:330]!\molR)(=[:90]!\molO)}
					\schemestop
				}

			% end subsubsection


			\pagebreak
			\hypertarget{CompleteOxidationOfPrimaryAlcohols}{}
			\subsubsection{Complete Oxidation to Carboxylic Acids}

				When the aldehyde is not distilled immediately and instead left to interact with the remaining oxidising agent, it will
				be further oxidised to a carboxylic acid. Alternatively, the strong oxidising agent \ch{KMnO4} can be used to immediately
				create a carboxylic acid.

				\vspace{1.5em}
				\vbox{\textbf{Conditions:}	\tabto{35mm}\ch{K2Cr2O7} with dilute \ch{H2SO4}, \textit{OR} \ch{KMnO4} with dilute \ch{H2SO4},
											\tabto{35mm}heat under reflux.}
				\vspace{0.75em}
				\vbox{\textbf{Observations:}\tabto{35mm}\boit{\color{BurntOrange}Orange} \ch{Cr2O7^2-}, turns \boit{\color{LimeGreen}green}
														(\ch{Cr^3+} formed), \textit{OR}
											\tabto{35mm}\boit{\color{Plum}Purple} \ch{MnO4-} decolourises (\ch{Mn^2+} formed)}

				\diagram[1.0]{
					\schemestart[0, 1.5, thick]
						\chemfig{C(-[:90]H)(-[:270]H)(-[:180]!\molR)(-[:0]!\molOH)}
						\arrow
						\chemfig{C(-[:180]!\molR)(=[:45]!\molO)(-[:315]!\molOH)}
					\schemestop
				}


				\diagram[1.0]{
					\schemestart[0, 1.5, thick]
						\chemfig{C(-[:210]H)(-[:330]!\molR)(=[:90]!\molO)}
						\arrow
						\chemfig{C(-[:180]!\molR)(=[:45]!\molO)(-[:315]!\molOH)}
					\schemestop
				}{It is also possible to start with aldehydes, and oxidise them to carboxylic acids.}


				Note that, with \ch{KMnO4}, two special carboxylic acids can be \textit{even further oxidised} to give \ch{CO2} and \ch{H2O};
				they are ethanedioic acid and methanoic acid.


				\diagram[1.0]{
					\schemestart[0,1.5,thick]
						\chemfig{C(-[:180]H)(=[:45]!\molO)(-[:315]!\molOH)}
						\arrow
						\ch{CO2}
						\hspace{2mm} + \hspace{2mm}
						\ch{H2O}
						\arrow(@c1.south east--.north east){0}[-90,.25]
						\chemfig{C(-[:135]!\molHO)(=[:225]!\molO)-C(=[:45]!\molO)(-[:315]!\molOH)}
						\arrow
						\ch{2 CO2}
						\hspace{2mm} + \hspace{2mm}
						\ch{H2O}
					\schemestop
				}

			% end subsubsection

		% end subsection

		\pagebreak
		\subsection{Oxidation of Secondary Alcohols}

			Unlike primary alcohols, secondary alcohols only have a single stage of oxidation, and thus either \ch{K2Cr2O7} or \ch{KMnO4} can be
			used. Secondary alcohols are oxidised to ketones, which cannot be further oxidised.

			\vspace{1.5em}
			\vbox{\textbf{Conditions:}	\tabto{35mm}\ch{K2Cr2O7} with dilute \ch{H2SO4}, \textit{OR} \ch{KMnO4} with dilute \ch{H2SO4},
										\tabto{35mm}heat under reflux.}
			\vspace{0.75em}
			\vbox{\textbf{Observations:}\tabto{35mm}\boit{\color{BurntOrange}Orange} \ch{Cr2O7^2-}, turns \boit{\color{LimeGreen}green}
													(\ch{Cr^3+} formed), \textit{OR}
										\tabto{35mm}\boit{\color{Plum}Purple} \ch{MnO4-} decolourises (\ch{Mn^2+} formed)}

			\diagram[1.0]{
				\schemestart[0,1.5,thick]
					\chemfig{C(-[:90]!\molR)(-[:270]H)(-[:180]!\molR)(-[:0]!\molOH)}
					\arrow
					\chemfig{C(=[:90]!\molO)(-[:210]!\molR)(-[:330]!\molR)}
				\schemestop
			}

		% end subsection

		\subsection{Oxidation of Tertiary Alcohols}

			This is really a pointless section, because tertiary alcohols cannot be oxidised under normal conditions. They do not have a hydrogen
			atom on the central carbon, thus preventing the addition of oxygen atoms.

		% end subsection

		\pagebreak
		\subsection{Tri-iodomethane (Iodoform) Formation}

			This reaction is useful both in determining the structure of a given compound, and in reducing the length of the carbon chain by
			forming a carboxylate ion.

			It can only occur in the presence of a specific arrangement of groups, which allows for the fingerprinting of a compound. Note that
			this is actually a redox reaction, as the alcohol is oxidised and the iodine is reduced to \ch{I-}.


			\vspace{1.5em}
			\vbox{\textbf{Conditions:}	\tabto{35mm}\ch{I2 \stAq}, \ch{NaOH \stAq}, warmed.}
			\vbox{\textbf{Observations:}\tabto{35mm}\boit{\color{Dandelion}Yellow} precipitate of \ch{CHI3} is formed.}

			\diagram[1.0]{
				\schemestart[0, 2.0, thick]
					\chemfig{C(-[:90]H)(-[:270]!\molMe)(-[:180]!\molR)(-[:0]!\molOH)}
					\arrow{->[\ch{I2 \stAq}, \ch{NaOH \stAq}][warm]}
					\chemfig{C(-[:180]!\molR)(=[:45]!\molO)(-[:315]!\molO\mch)}
				\schemestop
			}{The group to the right of the R group is the necessary structure for a positive reaction.}

			Note that ethanol, where the R group is simply a hydrogen atom, is the only primary alcohol to have a reaction, since there will
			be no \ch{CH3} group on any other primary alcohol.

		% end subsection

	% end section

%end part



















