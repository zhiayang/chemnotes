% Chapter 11 - Aldehydes and Ketones.tex
% Copyright (c) 2014 - 2016, zhiayang@gmail.com
% Licensed under the Apache License Version 2.0.

\pagebreak
\hypertarget{ChapterAldehydesAndKetones}{}
\part{Aldehydes and Ketones}

	\section{Structure}

		Aldehydes and ketones have a very similar structure, both revolving around the carbonyl-bearing central carbon atom, with an
		oxygen atom double-bonded to it.

		\subsection{Ketones}

			Ketones are the more general form of carbonyls. There exists a central carbon atom, with one oxygen atom double-bonded to it, and
			two alkyl groups on either side. For ketones specifically, R in this case \itl{cannot} be a hydrogen atom.

			\diagram[1.0]{
				\chemfig{C(-[:210]!\molR)(-[:330]!\molR)=[:90]!\molO}
			}{The structure of a ketone.}

		% end subsection

		\subsection{Aldehydes}

			Aldehydes are ketones where one or both of the R groups on the central carbon are instead replaced with hydrogen atoms. Methanal,
			where both are hydrogen atoms, is the simplest aldehyde and the only case (duh) where there are two hydrogens.

			\diagram[1.0]{
				\chemfig{C(-[:210]H)(-[:330]!\molR)=[:90]!\molO}
			}{The structure of an aldehyde}

		% end subsection

		\pagebreak
		\subsection{Hybridisation}

			The central carbon in both aldehydes and ketones is \sptwo hybridised, resulting in a planar construction with a bond angle
			of \SI{120}{\degree}. As such, the unhybridised p-orbital of the carbon atom is able to overlap with the p-orbital the neighbouring
			oxygen atom, forming a π-bond.

			Furthermore, the high electronegativity of the oxygen atom results in it having a partial negative charge, \deltam{}, while the
			carbon atom has a partial positive charge, \deltap{}. It is this polarisation and permanent dipole that lends aldehydes and ketones
			their chemical and physical properties.

		% end subsection

	% end section

	\section{Physical Properties}

		\paragraph{Melting and Boiling Points}

		Both aldehydes and ketones can form intermolecular permanent dipole interactions, thus resulting in stronger electrostatic forces
		of attraction, and thus higher melting and boiling points than their alkane counterparts.

		However, the lack of an \ch{H} atom attached to the \ch{O} atom means that there is no possibility for intermolecular hydrogen
		bonds to be formed. Thus, aldehydes and ketones have lower melting and boiling points than their alcohol and carboxylic acid
		counterparts.


		\paragraph{Solubility}

		Furthermore, due to the existence of both a polar and non-polar region, carbonyls are soluble in both polar solvents, like water,
		and non-polar solvents, like \ch{C\Cl4}. Their solubility in the former is due to their ability to form favourable
		solvent-solute interactions in the form of hydrogen bonds, between the partial negatively-charged \ch{O} atom and the partial
		positively-charged hydrogen atom in \ch{H2O}.

		Beyond 5 carbons however, the bulkiness of the alkyl chain prevents further solubility of both aldehydes and ketones.

	% end section

	\pagebreak
	\section{Creation of Aldehydes and Ketones}

		\subsection{Oxidation of Alcohols}

			\subsubsection{Oxidation of Primary Alcohols}

				The details on the mechanics of this reaction can be found \hyperlink{OxidationOfPrimaryAlcohols}{here}, including
				the use of \ch{K2Cr2O7} instead of \ch{KMnO4}.

				\vspace{1.5em}
				\vbox{\textbf{Conditions:}	\tabto{35mm}\ch{K2Cr2O7} with dilute \ch{H2SO4},
											\tabto{35mm}heat with immediate distillation.}
				\vspace{0.75em}
				\vbox{\textbf{Observations:}\tabto{35mm}\boit{\color{BurntOrange}Orange} \ch{Cr2O7^2-}, turns \boit{\color{LimeGreen}green}
														(\ch{Cr^3+} formed).}

				\diagram[1.0]{
					\schemestart[0,1.5,thick]
						\chemfig{C(-[:90]H)(-[:270]H)(-[:180]!\molR)(-[:0]!\molOH)}
						\arrow
						\chemfig{C(-[:210]H)(-[:330]!\molR)(=[:90]!\molO)}
						\hspace{2mm} + \hspace{2mm}
						\ch{H2O}
					\schemestop
				}

			% end subsubsection

			\subsection{Oxidation of Secondary Alcohols}

				Secondary alcohols can be oxidised to ketones directly, using either \ch{K2Cr2O7} or \ch{KMnO4}.

				\vspace{1.5em}
				\vbox{\textbf{Conditions:}	\tabto{35mm}\ch{K2Cr2O7} with dilute \ch{H2SO4}, \itl{OR} \ch{KMnO4} with dilute \ch{H2SO4},
											\tabto{35mm}heat under reflux.}
				\vspace{0.75em}
				\vbox{\textbf{Observations:}\tabto{35mm}\boit{\color{BurntOrange}Orange} \ch{Cr2O7^2-}, turns \boit{\color{LimeGreen}green}
														(\ch{Cr^3+} formed), \itl{OR}
											\tabto{35mm}\boit{\color{Plum}Purple} \ch{MnO4-} decolourises (\ch{Mn^2+} formed)}

				\diagram[1.0]{
					\schemestart[0,1.5,thick]
						\chemfig{C(-[:90]!\molR)(-[:270]H)(-[:180]!\molR)(-[:0]!\molOH)}
						\arrow
						\chemfig{C(=[:90]!\molO)(-[:210]!\molR)(-[:330]!\molR)}
						\hspace{2mm} + \hspace{2mm}
						\ch{H2O}
					\schemestop
				}

			% end subsection

		% end subsection

		\pagebreak
		\subsection{Oxidative Cleavage of Alkenes}

			Certain alkenes can undergo strong oxidation, cleaving the double bond. The details are elaborated on
			\hyperlink{OxidativeCleavageOfAlkenes}{\boit{here}}, but only one kind of alkene can form a ketone, where the carbon
			atom has two alkyl groups substituted, and no hydrogens.

			\vspace{1.5em}
			\vbox{\textbf{Conditions:}	\tabto{35mm}\ch{KMnO4}, dilute \ch{H2SO4}, heat.}
			\vbox{\textbf{Observations:}\tabto{35mm}{\boit{\color{Plum}Purple}} \ch{KMnO4} decolourises, forming colourless \ch{Mn^2+}.}


			\diagram[1.0]{
				\schemestart[0, 2.0, thick]
					\chemfig{C(-[:135]!\molMe)(-[:225]!\molMe)=[:0]C(-[:45]!\molMe)(-[:315]!\molMe)}
					\arrow{->[\ch{MnO4-}][heat]}
					\chemfig{C(-[:120]!\molMe)(-[:240]!\molMe)=[:0]!\molO}
				\schemestop
			}

			Note that alkenes with one hydrogen substituent and one alkyl substituent \itl{will not} form aldehydes, since the hydrogen
			atom will be oxidised, and instead give a carboxylic acid.

		% end subsection

	% end section

	\pagebreak
	\section{Aldehyde and Ketone Reactions}

		\subsection{Nucleophilic Addition}

			Like all other mechanisms, the reaction mechanism for this can be found in
			\hyperlink{AppendixNucleophilicAddition}{\boit{the appendix}}. Note that the trigonal-planar nature of the carbonyl makes it
			open to attack from both sides, which can result in stereoisomers and chirality.


			\diagram[1.0]{
				\schemestart[0,1.5,thick]
					\chemfig{C(-[:120]!\molR)(-[:240]!\molR)(=[:0]!\molO)}
					\hspace{5mm} + \hspace{5mm}
					\chemfig{H-!\molCN}
					\arrow
					\chemfig{C(-[:90]!\molR)(-[:180]!\molR)(-[:0]!\molOH)(-[:270]!\molCN)}
				\schemestop
			}{This is the basic scheme for nucleophilic addition.}

			\subsubsection{Formation of Nitriles (Cyanohydrins)}

				The nucleophilic addition of \ch{CN} to a carbonyl is perhaps one of the more common forms of nucleophilic addition.
				Since \ch{HCN} is a toxic gas, it is typically generated \itl{in-situ} by reacting \ch{KCN} with dilute \ch{H2SO4}.

				\diagram[1.0]{
					\schemestart[0,1.0,thick]
					\chemfig{\ch{2 KCN}} \hspace{2mm} + \hspace{2mm} \chemfig{\ch{H2SO4}}
					\arrow
					\chemfig{\ch{K2SO4}} \hspace{2mm} + \hspace{2mm} \chemfig{\ch{2 HCN}}
					\schemestop
				}

				\vspace{-5mm}


				\vspace{1.5em}
				\vbox{\textbf{Conditions:}	\tabto{35mm}Cold \ch{HCN}, trace \ch{KCN \stAq}.}

				\diagram[1.0]{
					\schemestart[0,2.0,thick]
						\chemfig{C(-[:120]!\molR)(-[:240]!\molR)(=[:0]!\molO)}
						\hspace{5mm} + \hspace{5mm}
						\chemfig{H-!\molCN}
						\arrow{->[trace \ch{KCN \stAq}][cold]}
						\chemfig{C(-[:90]!\molR)(-[:180]!\molR)(-[:0]!\molOH)(-[:270]!\molCN)}
					\schemestop
				}

				Even though the stated reactant of the reaction above is indeed \ch{HCN}, simply using pure \ch{HCN} will result in a dismal
				rate of reaction --- \ch{HCN} is a weak acid, and as such only partially dissociates in water to give \ch{CN-} ions, which is
				the actual electrophile.

				Thus, a trace amount of \ch{KCN} is added, which, as a salt, will completely dissociate in water to give \ch{CN-} ions,
				increasing the rate of reaction. Since \ch{CN-} is regenerated at the end of the reaction, the \ch{KCN} added actually
				acts as a catalyst.



				The resulting nitrile can be used to create carboxylic acids through hydrolysis, as well as amines through
				reduction. The details can be found \hyperlink{NitrileUses}{\boit{here}}.

			% end subsubsection

		% end subsection

		\subsection{Condensation}

			Primary amines, which have a nucleophilic nitrogen atom with two attached hydrogen atoms, can undergo an elimination
			reaction with both aldehydes and ketones, with the removal of \ch{H2O}.

			\diagram[1.0]{
				\schemestart[0, 1.5, thick]
					\chemfig{C(-[:120]!\molR)(-[:240]!\molR)(=[:0]@{oxy}{\color{Red}O})}
					\hspace{5mm} + \hspace{5mm}
					\chemfig{N(-[:120]@{h1}H)(-[:240]@{h2}H)-!\molRon}
					\arrow
					\chemfig{C(-[:120]!\molR)(-[:240]!\molR)=[:0]!\molN-[:0]!\molRon}
					\hspace{5mm} + \hspace{5mm}
					\chemfig{\ch{H2O}}
				\schemestop

				\chemmove{
					\draw[-latex, red, thick]
					(oxy.center) circle(4mm)
					(h1.center) circle(4mm)
					(h2.center) circle(4mm);
				}
			}{The circled atoms form the water that is eventually eliminated.}

			\subsubsection{Distinguishing test with 2,4-DNPH}

				One of the more useful reactions involving this mechanism is the distinguishing test with 2,4-dinitrophenylhydrazine, which
				is a relatively large molecule with a distinct orange precipitate. This allows for the verification of the presence of
				an aldehyde or a ketone --- both aliphatic and aromatic carbonyls will have a positive result.

				The molecule is somewhat complex, but reading it right-to-left is sufficient to deduce its structure --- hydrazine is the parent
				structure, and it has a substituent of 2,4-dinitrophenyl.

				\vspace{1.5em}
				\vbox{\textbf{Conditions:}	\tabto{35mm}2,4-DNPH, room temperature.}
				\vbox{\textbf{Observations:}\tabto{35mm}{\boit{\color{BurntOrange}Orange}} precipitate forms with a carbonyl.}

				\diagram[0.65]{
					\schemestart[0, 1.5, thick]
						\chemfig{C(-[:120]!\molR)(-[:240]!\molR)(=[:0]!\molO)}
						\hspace{2mm} + \hspace{2mm}
						\chemfig{!\molN(-[:120]H)(-[:240]H)-[:0]!\molN(-[:270]H)(-[:0]**6(-(-[:240]!\molNitro)--(-[:0]!\molNitro)---))}
						\arrow(.mid east--.mid west)
						\chemfig{C(-[:120]!\molR)(-[:240]!\molR)=[:0]!\molN-[:0]!\molN(-[:270]H)(-[:0]**6(-(-[:240]!\molNitro)--(-[:0]!\molNitro)---))}
						\hspace{2mm} + \hspace{2mm}
						\chemfig{\ch{H2O}}
					\schemestop
				}

			% end subsubsection

		% end subsection

		\pagebreak
		\subsection{Reduction}

			In what is basically the reverse of the oxidation of alcohols, aldehydes can be reduced into primary alcohols, and
			ketones can be reduced into secondary alcohols.

			Clearly, tertiary alcohols cannot be formed like this.

			\subsubsection{Reduction of Aldehydes}

				Aldehydes are reduced to primary alcohols. Any of the three reducing agents can be used.

				\vspace{1.5em}
				\vbox{\textbf{Conditions:}	\tabto{35mm}\ch{Li\Al H4} in dry ether (diethyl ether), \itl{OR}
	  										\tabto{35mm}\ch{NaBH4} in methanol, \itl{OR}
	  										\tabto{35mm}\ch{H2 \stG}, \ch{Ni} catalyst, high temperature and pressure.}

				\diagram[1.0]{
					\schemestart[0,1.5,thick]
						\chemfig{C(=[:90]!\molO)(-[:210]!\molR)(-[:330]H)}
						\arrow
						\chemfig{C(-[:0]!\molOH)(-[:90]H)(-[:270]H)(-[:180]!\molR)}
					\schemestop
				}

			% end subsubsection

			\subsubsection{Reduction of Ketones}

				Ketones are reduced to secondary alcohols. Any of the three reducing agents can be used.

				\vspace{1.5em}
				\vbox{\textbf{Conditions:}	\tabto{35mm}\ch{Li\Al H4} in dry ether (diethyl ether), \itl{OR}
	  										\tabto{35mm}\ch{NaBH4} in methanol, \itl{OR}
	  										\tabto{35mm}\ch{H2 \stG}, \ch{Ni} catalyst, high temperature and pressure.}

				\diagram[1.0]{
					\schemestart[0,1.5,thick]
						\chemfig{C(=[:90]!\molO)(-[:210]!\molRon)(-[:330]!\molRtw)}
						\arrow
						\chemfig{C(-[:0]!\molOH)(-[:90]!\molRon)(-[:270]!\molRtw)(-[:180]H)}
					\schemestop
				}

			% end subsubsection

		% end subsection

		\subsection{Oxidation}

			\subsubsection{Oxidation of Aldehydes to Carboxylic Acids}

				As briefly covered in the chapter on \hyperlink{CompleteOxidationOfPrimaryAlcohols}{\boit{alcohols}}, aldehydes can be
				further oxidised into carboxylic acids.

				Also, two special carboxylic acids can be further oxidised, and they are covered above.

				\vspace{1.5em}
				\vbox{\textbf{Conditions:}	\tabto{35mm}\ch{K2Cr2O7} with dilute \ch{H2SO4}, \itl{OR} \ch{KMnO4} with dilute \ch{H2SO4},
											\tabto{35mm}heat under reflux.}
				\vspace{0.75em}
				\vbox{\textbf{Observations:}\tabto{35mm}\boit{\color{BurntOrange}Orange} \ch{Cr2O7^2-}, turns \boit{\color{LimeGreen}green}
														(\ch{Cr^3+} formed), \itl{OR}
											\tabto{35mm}\boit{\color{Plum}Purple} \ch{MnO4-} decolourises (\ch{Mn^2+} formed)}

				\diagram[1.0]{
					\schemestart[0, 1.5, thick]
						\chemfig{C(-[:210]H)(-[:330]!\molR)(=[:90]!\molO)}
						\arrow
						\chemfig{C(-[:180]!\molR)(=[:45]!\molO)(-[:315]!\molOH)}
					\schemestop
				}

				This reaction can also be used to distinguish aldehydes from ketones, as ketones will \itl{not} undergo further oxidation
				due to their lack of a hydrogen atom on the central carbon.


			\subsubsection{Tollens' Reagent}

				Tollens' Reagent, or the \itl{silver mirror test}, can also be used to distinguish aldehydes and ketones. Only aldehydes,
				both aliphatic and aromatic, can react reduce the silver complex to silver metal, while ketones cannot --- this is due to the
				fact that ketones cannot be oxidised.

				A positive test results in the depositing of silver metal on the surface of the reaction vessel, hence the silver mirror.

				\vspace{1.5em}
				\vbox{\textbf{Conditions:}	\tabto{35mm}Tollens' Reagent, heat.}
				\vbox{\textbf{Observations:}\tabto{35mm}\boit{\color{Gray}Silver} metal coats the reaction vessel.}

				\diagram[0.80]{
					\schemestart[0,1.5,thick]
						\ch{2 [Ag(NH3)2]+}
						\hspace{2mm} + \hspace{2mm}
						\ch{2 OH-}
						\hspace{2mm} + \hspace{2mm}
						\chemfig{C(-[:210]!\molR)(-[:330]H)(=[:90]!\molO)}
						\arrow(.mid east--.mid west){->[heat]}
						\chemfig{C(-[:210]!\molR)(-[:330]!\molO\mch)(=[:90]!\molO)}
						\hspace{2mm} + \hspace{2mm}
						\ch{2 Ag}
						\hspace{2mm} + \hspace{2mm}
						\ch{4 NH3}
						\hspace{2mm} + \hspace{2mm}
						\ch{2 H2O}
					\schemestop
				}


			% end subsubection


			\subsubsection{Fehling's Solution}

				Fehling's Solution, like Tollens' Reagent, relies on the oxidation of the aldehyde, and hence the reduction of itself. In
				this case, \ch{Cu^2+} is reduced to \ch{Cu+} in the form of \ch{Cu2O}, which gives a distinct, brick-red precipitate.

				Fehling's Solution is used to distinguish between aliphatic and aromatic aldehydes. Again, since ketones cannot be oxidised,
				they do not give a positive test for Fehling's Solution.


				\vspace{1.5em}
				\vbox{\textbf{Conditions:}	\tabto{35mm}Fehling's Solution, heat.}
				\vbox{\textbf{Observations:}\tabto{35mm}\boit{\color{OrangeRed}Brick-red}, or \boit{\color{Mahogany}reddish-brown} precipitate is formed.}

				\diagram[0.80]{
					\schemestart[0,1.5,thick]
						\ch{2 Cu^2+}
						\hspace{2mm} + \hspace{2mm}
						\ch{5 OH-}
						\hspace{2mm} + \hspace{2mm}
						\chemfig{C(-[:210]!\molR)(-[:330]H)(=[:90]!\molO)}
						\arrow(.mid east--.mid west){->[heat]}
						\chemfig{C(-[:210]!\molR)(-[:330]!\molO\mch)(=[:90]!\molO)}
						\hspace{2mm} + \hspace{2mm}
						\ch{Cu2O}
						\hspace{2mm} + \hspace{2mm}
						\ch{3 H2O}
					\schemestop
				}{The aldehyde here cannot be aromatic (aka the R group is a benzene ring). Only normal alkyl-based aldehydes will work.}


			% end subsubsection


			\subsubsection{Tri-iodomethane (Iodoform) Formation}

				Similar to alcohols, ketones and aldehydes can be oxidised by aqueous iodine to give the distinct yellow precipitate, which
				is useful as a fingerprinting, structural test.

				Due to the requirements of having a \ch{CH3} group, ethanal is the only aldehyde that gives a positive result for this test,
				since the R group will be a hydrogen atom. Otherwise, only terminal ketones will test positive.

				\vspace{1.5em}
				\vbox{\textbf{Conditions:}	\tabto{35mm}\ch{I2 \stAq}, \ch{NaOH \stAq}, warmed.}
				\vbox{\textbf{Observations:}\tabto{35mm}\boit{\color{Dandelion}Yellow} precipitate of \ch{CHI3} is formed.}

				\diagram[1.0]{
					\schemestart[0, 2.0, thick]
						\chemfig{C(-[:210]!\molR)(-[:330]!\molMe)(=[:90]!\molO)}
						\arrow{->[\ch{I2 \stAq}, \ch{NaOH \stAq}][warm]}
						\chemfig{C(-[:210]!\molR)(-[:330]!\molO\mch)(=[:90]!\molO)}
					\schemestop
				}

			% end subsubsection

		% end subsection

	% end section

% end part
















































