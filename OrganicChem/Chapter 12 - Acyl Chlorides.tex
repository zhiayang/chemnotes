% Chapter 13 - Acyl Chlorides.tex
% Copyright (c) 2014 - 2016, zhiayang@gmail.com
% Licensed under the Apache License Version 2.0.


\pagebreak
\hypertarget{ChapterAcylChlorides}{}
\part{Acyl Chlorides}

	\section{Structure}

		Acyl chlorides, being a derivative of carboxylic acids, naturally share a similar structure. Indeed, they are basically identical
		save the replacement of the \ch{OH} group with a \ch{\Cl} atom.

		\diagram[1.0]{
			\chemfig{C(-[:180]!\molR)(=[:60]!\molO)(-[:300]!\molCl)}
		}{An acyl chloride.}

	% end section



	\section{Physical Properties}

		\paragraph{Melting and Boiling Points}

		Acyl chlorides have much lower melting and boiling points compared to their carboxylic acid brethren, given that there is no hydrogen
		atom directly attached to an electronegative atom. Thus, intermolecular hydrogen bonds cannot be formed, and acyl chlorides rely only
		on permanent dipole interactions, leading to lower melting and boiling points.

		They also do not dimerise, owing to this lack of hydrogen bonding.


		\paragraph{Solubility}

		The solubility of acyl chlorides in non-polar solvents depends on the length of the carbon chain, and hence its ability to form
		\idid{} interactions with the solvent, and is not greatly dependent on the acyl group.

		However, acyl chlorides cannot be said to \itl{dissolve} in water --- it reacts violently in a hydrolysis reaction, destroying the
		acyl chloride. Thus, it cannot exist in an aqueous state.

	% end section



	\pagebreak
	\section{Acyl Chloride Formation}

		Acyl chlorides are most commonly created through the nucleophilic acyl substitution of a carboxylic acid, as covered previously.
		Either \ch{P\Cl5}, \ch{P\Cl3}, or \ch{SO\Cl2} can be used.


		\paragraph{Phosphorous Pentachloride (\ch{P\Cl5})}

		\vspace{1.5em}
		\vbox{\textbf{Conditions:}\tabto{35mm}Solid \ch{P\Cl5}, room temperature.}
		\vbox{\textbf{Observations:}\tabto{35mm}Formation of white fumes of \ch{H\Cl} gas.}

		\diagram[1.0]{
			\schemestart[0,1.5,thick]
				\chemfig{C(-[:180]!\molR)(=[:60]!\molO)(-[:300]!\molOH)}
				\hspace{5mm} + \hspace{5mm}
				\chemfig{P\Cl\sbs{5}}
				\arrow
				\chemfig{C(-[:180]!\molR)(=[:60]!\molO)(-[:300]!\molCl)}
				\hspace{2mm} + \hspace{2mm}
				\chemfig{PO\Cl\sbs{3}}
				\hspace{2mm} + \hspace{2mm}
				\chemfig{H\Cl}
			\schemestop
		}


		\paragraph{Phosphorous Trichloride (\ch{P\Cl3})}

		\vspace{1.5em}
		\vbox{\textbf{Conditions:}\tabto{35mm}Solid \ch{P\Cl3}, room temperature.}

		\diagram[1.0]{
			\schemestart[0,1.5,thick]
				\chemfig{3}
				\chemfig{C(-[:180]!\molR)(=[:60]!\molO)(-[:300]!\molOH)}
				\hspace{5mm} + \hspace{5mm}
				\chemfig{P\Cl\sbs{3}}
				\arrow
				\chemfig{3}
				\chemfig{C(-[:180]!\molR)(=[:60]!\molO)(-[:300]!\molCl)}
				\hspace{2mm} + \hspace{2mm}
				\chemfig{H\sbs{3}PO\sbs{3}}
			\schemestop
		}


		\paragraph{Thionyl Chloride (\ch{SO\Cl2})}

		This reaction is slightly preferred over the others, since both by-products (\ch{SO2} and \ch{H\Cl}) are
		gaseous, and would bubble out of the solution, leaving mainly the halogenoalkane in the reaction mixture.

		\vspace{1.5em}
		\vbox{\textbf{Conditions:}\tabto{35mm}Warm, liquid \ch{SO\Cl2}.}

		\vspace{0.75em}
		\vbox{\textbf{Observations:}\tabto{35mm}Formation of colourless, pungent \ch{SO2} gas,
									\tabto{35mm}white fumes of \ch{H\Cl} gas.}

		\diagram[1.0]{
			\schemestart[0,1.5,thick]
				\chemfig{C(-[:180]!\molR)(=[:60]!\molO)(-[:300]!\molOH)}
				\hspace{5mm} + \hspace{5mm}
				\chemfig{SO\Cl\sbs{2}}
				\arrow
				\chemfig{C(-[:180]!\molR)(=[:60]!\molO)(-[:300]!\molCl)}
				\hspace{2mm} + \hspace{2mm}
				\chemfig{SO\sbs{2}}
				\hspace{2mm} + \hspace{2mm}
				\chemfig{H\Cl}
			\schemestop
		}

	% end section



	\section{Acyl Chloride Reactions}

		\subsection{Hydrolysis}

			Acyl chlorides can be hydrolysed, undergoing a rather violent reaction to form \ch{H\Cl} and a carboxylic acid.

			\vspace{1.5em}
			\vbox{\textbf{Conditions:}\tabto{35mm}\ch{H2O}, room temperature.}
			\vbox{\textbf{Observations:}\tabto{35mm}Formation of white fumes of \ch{H\Cl} gas.}

			\diagram[1.0]{
				\schemestart[0, 1.5, thick]
					\chemfig{C(-[:180]!\molR)(=[:60]!\molO)(-[:300]!\molCl)}
					\hspace{5mm} + \hspace{5mm}
					\ch{H2O}
					\arrow
					\chemfig{C(-[:180]!\molR)(=[:60]!\molO)(-[:300]!\molOH)}
					\hspace{5mm} + \hspace{5mm}
					\ch{H\Cl}
				\schemestop
			}

		% end subsection



		\subsection{Nucleophilic Acyl Substitution}

			Acyl chlorides themselves can undergo a substitution, replacing the \ch{\Cl} atom with another group.

			\subsubsection{Esterification}

				Acyl chlorides are the preferred path to esters, due to their high reactivity and yield, plus the ability to form
				phenyl esters.

				\vspace{1.5em}
				\vbox{\textbf{Conditions:}\tabto{35mm}Acyl chloride, \ch{NaOH \stAq}, room temperature.}
				\vbox{\textbf{Observations:}\tabto{35mm}Formation of white fumes of \ch{H\Cl} gas.}

				\diagram[1.0]{
					\schemestart[0, 1.5, thick]
						\chemfig{C(-[:180]!\molR)(=[:60]!\molO)(-[:300]!\molCl)}
						\hspace{5mm} + \hspace{5mm}
						\chemfig{!\molRon-!\molOH}
						\arrow
						\chemfig{C(-[:180]!\molR)(=[:60]!\molO)(-[:300]!\molO-!\molRon)}
						\hspace{5mm} + \hspace{5mm}
						\chemfig{H\Cl}
					\schemestop
				}


			% end subsubsection


			\pagebreak
			\hypertarget{AcylChloridesReactionWithAmines}{}
			\subsubsection{Formation of Amides}

				Amides, which are covered \hyperlink{ChapterAmides}{\boit{later}}, are an acyl derivative where the \ch{OH} group
				is replaced with an \ch{NH2} group --- the hydrogen atoms on the nitrogen may also be substituted for R groups, creating
				substituted amides.

				Acyl chlorides react readily with ammonia and amines to form primary amides and substituted amides respectively. Note that an
				excess of the amine (or ammonia) is used to compensate for the acid-base reaction between the \ch{H\Cl} produced and the
				amine.

				Finally, tertiary amines (where all the hydrogen atoms have been substituted) cannot be used, since they do not have a
				substitutable hydrogen atom.


				\vspace{1.5em}
				\vbox{\textbf{Conditions:}\tabto{35mm}Acyl chloride, amine of choice in excess, room temperature.}
				\vbox{\textbf{Observations:}\tabto{35mm}Formation of white fumes of \ch{H\Cl} gas.}

				\diagram[1.0]{
					\schemestart[0, 1.5, thick]
						\chemfig{C(-[:180]!\molR)(=[:60]!\molO)(-[:300]!\molCl)}
						\hspace{5mm} + \hspace{5mm}
						\chemfig{!\molN(-[:180]H)(-[:60]!\molRon)(-[:300]!\molRtw)}
						\arrow(.mid east--.mid west)
						\chemfig{C(-[:180]!\molR)(=[:60]!\molO)(-[:300]!\molN(-[:0]!\molRon)(-[:240]!\molRtw))}
						\hspace{5mm} + \hspace{5mm}
						\chemfig{H\Cl}
					\schemestop
				}

			% end subsubsection

		% end subsection


	% end section

% end part




































