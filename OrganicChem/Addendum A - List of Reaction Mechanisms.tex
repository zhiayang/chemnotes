% Addendum 1 - List of Reaction Mechanisms.tex
% Copyright (c) 2014 - 2016, zhiayang@gmail.com
% Licensed under the Apache License Version 2.0.

\pagebreak
\hypertarget{AddendumReactionMechanisms}{}
\part{List of Reaction Mechanisms}

	\section{Electrophilic Addition}

		Electrophilic Addition the main reaction mechanism for alkenes, and involves an electrophile attacking the electron-rich
		π-bond of the alkene.

		\subsection{Markovnikov's Rule}

			This rule governs the major product that is formed when asymmetrical compounds are electrophilically added to alkenes,
			such as hydrogen halides. It does not apply to symmetrical reactants like \ch{\chlorine2} or \ch{Br2}.

			The rule states basically states that when a hydrogen compound with the general form \ch{H-X} is added to an
			alkene that is asymmetrical about the double-bond, the hydrogen atom will be added to the carbon with \textit{more}
			existing hydrogen atoms. Note that \textit{X} can be a halogen, a hydroxide (ie. \ch{H-X} is \ch{H2O}) or some other
			electronegative species.

			The mirror of the rule, used more with the cyclic halonium ion, is that the nucleophile will be added to the more substituted
			carbon atom.

		% end subsection




		\pagebreak
		\subsection{Electrophilic Addition of HX}

			In this reaction, the hydrogen and halogen atom are added across the double bond of the alkene. The hydrogen halide
			should be in a gaseous state for this reaction.

			\vspace{1.5em}
			\vbox{\textbf{Conditions:}\tabto{35mm}Gaseous HX (usually \ch{H\chlorine} or \ch{HBr}).}

			\paragraph{Step 1}


			\diagram[0.95]{
				\schemestart[0, 1.5, thick]
				\chemfig{C(-[:135]H)(-[:225]H)=[@{b1}:0]C(-[:45]H)(-[:315]H)}
				\hspace{5mm} + \hspace{5mm}
				\chemfig{@{hyd}\chembelow{H}{\smdeltap}-[@{b2}:0]@{hal}\chembelow{{\color{OliveGreen}X}}{\smdeltam}}
				\arrow{->[slow]}
				\chemfig{C(-[:90]H)(-[:180]H)(-[:270]H)-\chemabove{C}{\smfplus}(-[:45,,1]H)(-[:315,,1]H)}
				\hspace{5mm} + \hspace{5mm}
				\chemfig{{{\color{OliveGreen}X}}\mch}
				\schemestop

				\chemmove{\draw[-Stealth,line width=0.4mm,shorten <=2mm,shorten >=1mm](b1) .. controls +(90:20mm) and +(90:20mm) .. (hyd);}
				\chemmove{\draw[-Stealth,line width=0.4mm,shorten <=2mm,shorten >=1mm](b2) .. controls +(90:8mm) and +(90:8mm) .. (hal);}
			}

			This is the rate-determining step, which involves the breaking of the π-bond. The polar \ch{HX} molecule has to
			approach the electron cloud of the π-bond in the correct orientation (hydrogen-facing), where the π-bond electrons
			attack the electron deficient, \ch{"\ox[parse=false]{\delp,H}"} atom.

			The \ch{H-X} bond then undergoes heterolytic fission, producing a carbocation intermediate and a halide ion.


			% look better

			\paragraph{Step 2}


			\diagram[1.0]{
				\schemestart[0, 1.5, thick]
				\chemfig{C(-[:90]H)(-[:180]H)(-[:270]H)(-C|@{pl}\smfplus(-[:45,,1]H)(-[:315,,1]H))}
				\hspace{5mm} + \hspace{5mm}
				\chemfig{@{hal}\lewis{2:,\color{OliveGreen}X}|\mch}
				\arrow
				\chemfig{C(-[:90]H)(-[:180]H)(-[:270]H)-C(-[:90]{{\color{OliveGreen}X}})(-[:0]H)(-[:270]H)}
				\schemestop

				\chemmove{\draw[-Stealth,line width=0.4mm,shorten <=1.5mm,shorten >=1mm](hal) .. controls +(90:15mm) and +(35:15mm) .. (pl);}

			}{The electron pair arrow points to the plus charge on the carbon, \textit{not} the carbon atom itself.}

			This is the fast step; the bromide anion acts as a nucleophile, attacking the positively-charged carbocation.


			\paragraph{Overall Reaction}

			\diagram[1.0]{
				\schemestart[0, 1.5, thick]
				\chemfig{C(-[:135]H)(-[:225]H)=C(-[:45]H)(-[:315]H)}
				\hspace{5mm} + \hspace{5mm}
				\ch{HX}
				\arrow
				\chemfig{C(-[:90]H)(-[:180]H)(-[:270]H)-C(-[:90]{{\color{OliveGreen}X}})(-[:0]H)(-[:270]H)}
				\schemestop
			}

		% end subsection


		\pagebreak

		\subsection{Electrophilic Addition of \ch{X2}}

			The electrophilic addition of halogens to alkanes does not involve Markovnikov's rule, since it is symmetrical.
			As the non-polar halogen molecule approaches the high-electron-density π-bond, it is polarised, forming
			\deltap and \deltam partial charges. The reaction mechanism involves the formation of a \textit{cyclic halonium ion} –– the double bond
			breaks, and each carbon forms a single bond with one positive halide ion.

			\vspace{1.5em}

			\vbox{\textbf{Conditions:}	\tabto{35mm}No UV Light, gaseous \ch{X2}.}	% again, single line, no need for \vspace{0.5em}
			\vbox{\textbf{Observations:}\tabto{35mm}\boit{\color{Mahogany}Reddish-brown} \ch{Br2} / \boit{\color{YellowGreen}yellowish-green} \ch{\chlorine2} decolourises.}

			\paragraph{Step 1}

			This initial step is the slow, rate-determining one. The resulting cyclic halonium ion is highly unstable, due to the
			geometric constraints of having a three-membered ring.

			\diagram[0.90]{
				\schemestart[0, 1.5, thick]
				\chemfig{C(-[:135]H)(-[:225]H)=[@{b1}:0]C(-[:45]H)(-[:315]H)}
				\hspace{5mm} + \hspace{5mm}
				\chemfig{@{x1}\chembelow{{\color{OliveGreen}X}}{\smdeltap}-[@{b2}:0]@{x2}\chembelow{{\color{OliveGreen}X}}{\smdeltam}}
				\arrow{->[slow]}
				\chemfig{{\color{OliveGreen}X}\mch}
				\hspace{5mm} + \hspace{5mm}
				\chemfig{C?(-[:90]H)(-[:180]H)-C?(-[:90]H)(-[:0]H)(-[:240,,,1]{\color{OliveGreen}X}?|\pch)}
				\schemestop

				\chemmove{\draw[-Stealth,line width=0.4mm,shorten <=2mm,shorten >=1mm](b1) .. controls +(90:15mm) and +(90:15mm) .. (x1);}
				\chemmove{\draw[-Stealth,line width=0.4mm,shorten <=2mm,shorten >=1mm](b2) .. controls +(90:8mm) and +(90:8mm) .. (x2);}
			}


			\paragraph{Step 2}

			In the second, fast step, the negatively-charged bromide ion from the first step attacks the overall
			positively-charged cyclic bromonium ion. The bromide ion attacks one of the carbons attached to the positive bromide ion,
			breaking that bond.

			\diagram[1.0]{
				\schemestart[0, 1.5, thick]
				\chemfig{C?(-[:90]H)(-[:180]H)-@{carb}C?(-[:90]H)(-[:0]H)(-[@{b1}:240,,,1]@{x}{\color{OliveGreen}X}?|\pch)}
				\hspace{5mm} + \hspace{5mm}
				\chemfig{@{x1}\lewis{5:,\color{OliveGreen}X}\mch}
				\arrow{->[fast]}
				\chemfig{C(-[:90]H)(-[:180]H)(-[:270]{{\color{OliveGreen}X}})-C(-[:90]H)(-[:0]H)(-[:270]{{\color{OliveGreen}X}})}
				\schemestop

				\chemmove{\draw[-Stealth,line width=0.4mm,shorten <=3mm,shorten >=1mm](x1) .. controls +(215:15mm) and +(315:15mm) .. (carb);}
				\chemmove{\draw[-Stealth,line width=0.4mm,shorten <=2mm,shorten >=1mm](b1) .. controls +(330:8mm) and +(350:8mm) .. (x);}
			}

			\paragraph{Overall Reaction}

			\diagram[1.0]{
				\schemestart[0, 1.5, thick]
				\chemfig{C(-[:135]H)(-[:225]H)=C(-[:45]H)(-[:315]H)}
				\hspace{5mm} + \hspace{5mm}
				\ch{X2}
				\arrow
				\chemfig{C(-[:270]!\molX)(-[:90]H)(-[:180]H)-C(-[:270]!\molX)(-[:90]H)(-[:0]H)}
				\schemestop
			}

		% end subsection



		\pagebreak
		\subsection{Electrophilic Addition of Aqueous \ch{X2}}

			The mechanics of this reaction are basically the same as that of the electrophilic addition of \ch{Br2}, and the conditions and
			observations are fairly similar as well.

			Also, since the \ch{Br} atoms are not both added across the double bond in a symmetrical manner,
			Markovnikov's Rule applies –– the OH group will preferentially attack the carbon that has more stabilising alkyl groups.

			\vspace{1.5em}
			\vbox{\textbf{Conditions:}	\tabto{35mm}No UV Light, aqueous \ch{Br2}.}	% again, single line, no need for \vspace{0.5em}
			\vbox{\textbf{Observations:}\tabto{35mm}\boit{\color{Dandelion}Yellow} \ch{Br2 \stAq} decolourises.}



			\paragraph{Step 1}
			\diagram[0.90]{
				\schemestart[0, 1.5, thick]
				\chemfig{C(-[:135]H)(-[:225]H)=[@{b1}:0]C(-[:45]H)(-[:315]H)}
				\hspace{5mm} + \hspace{5mm}
				\chemfig{@{br1}\chembelow{{\color{Mahogany}Br}}{\smdeltap}-[@{b2}:0]@{br2}\chembelow{{\color{Mahogany}Br}}{\smdeltam}}
				\arrow{->[slow]}
				\chemfig{{\color{Mahogany}Br}\mch}
				\hspace{5mm} + \hspace{5mm}
				\chemfig{C?(-[:90]H)(-[:180]H)-C?(-[:90]H)(-[:0]H)(-[:240,,,1]{\color{Mahogany}Br}?|\pch)}
				\schemestop

				\chemmove{\draw[-Stealth,line width=0.4mm,shorten <=2mm,shorten >=1mm](b1) .. controls +(90:15mm) and +(90:15mm) .. (br1);}
				\chemmove{\draw[-Stealth,line width=0.4mm,shorten <=2mm,shorten >=1mm](b2) .. controls +(90:8mm) and +(90:8mm) .. (br2);}
			}{The cyclic bromonium ion is formed.}



			\paragraph{Step 2}

			Once the cyclic bromonium ion is formed however, it is susceptible to attack from
			any and all nucleophiles –– including water, which has lone pairs and is a stronger nucleophile \ch{Br-}. Since it is in much higher
			concentrations than \ch{Br-}, the primary product will now have an \ch{OH} group.

			\diagram[0.825]{
				\schemestart[0, 1.5, thick]
				\chemfig{C?(-[:90]H)(-[:180]H)-@{carb}C?(-[:90]H)(-[:0]H)(-[@{b1}:240,,,1]@{br}{\color{Mahogany}Br}?|\pch)}
				\hspace{5mm} + \hspace{5mm}
				\chemfig{@{oxy}\lewis{1:3:,\color{Red}O}(-[:322]H)(-[:218]H)}
				\arrow{->[fast]}
				\chemfig{C(-[:180]H)(-[:90]H)(-[:270]!\molOH)-C(-[:90]H)(-[:0]H)(-[:270]!\molBr)}
				\hspace{5mm} + \hspace{5mm}
				\chemfig{H-!\molBr}
				\schemestop

				\chemmove{\draw[-Stealth,line width=0.4mm,shorten <=2mm,shorten >=1mm](b1) .. controls +(330:8mm) and +(350:8mm) .. (br);}
				\chemmove{\draw[-Stealth,line width=0.4mm,shorten <=2mm,shorten >=1mm](oxy) .. controls +(135:15mm) and +(45:15mm) .. (carb);}
			}{Following Markovnikov's Rule, the \ch{OH-} group will attach to the more substituted carbon.}


			\paragraph{Overall Reaction}

			\diagram[0.75]{
				\schemestart[0, 1.5, thick]
				\chemfig{C(-[:135]H)(-[:225]H)=[@{b1}:0]C(-[:45]H)(-[:315]H)}
				\hspace{5mm} + \hspace{5mm}
				\ch{Br2 \stAq}
				\arrow(.base east--.base west)
				\chemname{\chemfig{C(-[:180]H)(-[:90]H)(-[:270]!\molOH)-C(-[:90]H)(-[:0]H)(-[:270]!\molBr)}}{(major)}
				\hspace{5mm} + \hspace{5mm}
				\chemname{\chemfig{C(-[:180]H)(-[:90]H)(-[:270]!\molBr)-C(-[:90]H)(-[:0]H)(-[:270]!\molBr)}}{(minor)}
				\schemestop

			}

		% end subsection







	% end section


	\section{Electrophilic Substitution}

	% end section


	\section{Nucleophilic Addition}

	% end section


	\section{Nucleophilic Substitution (\snone)}

	% end section


	\section{Nucleophilic Substitution (\sntwo)}

	% end section

% end part
