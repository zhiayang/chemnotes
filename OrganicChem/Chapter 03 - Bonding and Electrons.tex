% Chapter 03 - Bonding and Electrons.tex
% Copyright (c) 2014 - 2016, zhiayang@gmail.com


\pagebreak
\hypertarget{ChapterBondingAndElectrons}{}
\part{Bonding and Electrons}

	\section{Bond Breaking}
		\subsection{Homolytic Fission}

			Homolytic fission involves the splitting of a single bond, with an equal (\itl{homo}) distribution of the two electrons
			in the aforementioned bond. This results in the formation of radicals, since now both atoms from the bond will have
			unpaired electrons.

			The movement of these single electrons is represented with single-hooked arrows. The arrow starts from the bond line, and points
			towards the target atom receiving the electron.

			\diagram{
				\schemestart[0, 1.5, thick]
				\chemfig{@{carb}C(-[@{bond}:0]@{hyd}H)(-[:90]H)(-[:180]H)(-[:270]H)}
				\arrow{->}
				\chemfig{\lewis{0.,C}(-[:90]H)(-[:180]H)(-[:270]H)} \hspace{5mm} + \hspace{5mm} \chemfig{\lewis{4.,H}}
				\schemestop

				% draw arrows
				\chemmove{\draw[-{Stealth[left]},line width=0.4mm,shorten <=1mm,shorten >=1mm](bond).. controls +(90:5mm) and +(120:8mm).. (hyd);}
				\chemmove{\draw[-{Stealth[left]},line width=0.4mm,shorten <=1mm,shorten >=1mm](bond).. controls +(270:5mm) and +(285:8mm).. (carb);}

			}[The homolytic fission of \ch{CH4}, forming \ch{^.CH3} and \ch{^.H} radicals.]

		% end subsection

		\subsection{Heterolytic Fission}

			As the name implies, heterolytic fission is the opposite; it distributes both electrons of the bond to a single atom, which
			usually results in the formation of ions. The transfer here is represented using a full (double-hooked) arrow.


			\diagram{
				\schemestart[0, 1.5, thick]
				% note: chlorine has to be coloured manually here
				% because somewhere in chemfig it fucks up, you can't name a submolecule.
				\chemfig{@{carb}C(-[@{bond}:0]@{chl}\color{OliveGreen}\Cl)(-[:90]H)(-[:180]H)(-[:270]H)}
				\arrow{->}
				\chemfig{C|\sps{+}(-[:90]H)(-[:180]H)(-[:270]H)}
				\hspace{5mm} + \hspace{5mm}
				\chemfig{\lewis{4:,\color{OliveGreen}\Cl}}
				\schemestop

				% draw arrows
				\chemmove{\draw[-{Stealth},line width=0.4mm,shorten <=1mm,shorten >=1mm](bond).. controls +(90:5mm) and +(120:8mm).. (chl);}


			}[The heterolytic fission of \ch{CH3\Cl}, resulting in \ch{CH3+} and \ch{\Cl-} ions]

		% end subsection

	\pagebreak
	\section{Bond Forming}
		\subsection{Single Electrons}

			Bond forming is essentially the reverse of bond breaking, and hence the same notations apply --- single-hooked arrows for the
			movement of single electrons, and double-hooked arrows for the movement of an electron pair.

			In this case, two single electrons from each atom contribute to the new bond.


			\diagram{
				\schemestart[0, 1.5, thick]
				\chemfig{@{carb}\lewis{1.,C}(-[:90]H)(-[:180]H)(-[:270]H)}
				\hspace{5mm} \chemfig{@{plus}+} \hspace{5mm}
				\chemfig{@{hyd}\lewis{5.,H}}
				\arrow{->}
				\chemfig{C(-[:0]H)(-[:90]H)(-[:180]H)(-[:270]H)}
				\schemestop

				% draw arrows
				\chemmove{\draw[-{Stealth[left]},line width=0.4mm,shorten <=2mm,shorten >=1mm](carb).. controls +(45:5mm) and +(120:8mm).. (plus);}
				\chemmove{\draw[-{Stealth[left]},line width=0.4mm,shorten <=2mm,shorten >=1mm](hyd).. controls +(225:5mm) and +(285:8mm).. (plus);}

			}[The formation of a bond between \ch{^.CH3} and \ch{^.H} radicals to form \ch{CH4}.]


		% end subsection


		\subsection{Electron Pairs}

			Electron pairs usually come from negatively charged radicals or lone pars, although this is not a rule.
			Both electrons come from a single source to form a bond, but this is \itl{not} a dative bond --- it is simply a normal
			bond.

			\diagram{
				\schemestart[0, 1.5, thick]
				\chemfig{@{carb}C|\sps{+}(-[:90]H)(-[:180]H)(-[:270]H)}
				\hspace{5mm} + \hspace{5mm}
				\chemfig{@{chl}\lewis{3:,\color{OliveGreen}\ch{\Cl-}}}
				\arrow{->}
				\chemfig{C(-[:0]!\molCl)(-[:90]H)(-[:180]H)(-[:270]H)}
				\schemestop

				% draw arrows
				\chemmove{\draw[-{Stealth},line width=0.4mm,shorten <=3mm,shorten >=1mm](chl).. controls +(150:12mm) and +(45:12mm).. (carb);}

			}[The formation of a bond between \ch{^.CH3} and \ch{\Cl-} radicals, to form \ch{CH3\Cl}.]

		% end subsection
	% end section


	\pagebreak
	\section{Electrophiles and Nucleophiles}
		\subsection{Electrophiles}

			Electrophiles are electron-deficient species that accept an electron pair from a nucleophile donor. Most electrophiles
			either have a positive charge, or contain an atom that is polarised and thus has a partial positive charge.

			Examples of electrophiles include \ch{CH3+}, \ch{Br+}, \ch{NO2+}, polarised \ch{Br2}, and \ch{HBr}.

		% end subsection

		\subsection{Nucleophiles}

			Nucleophiles are electron-rich species that donate electron pairs to electrophiles. This process typically results
			in the formation of a new covalent bond. Nucleophiles usually contain atoms that are either negatively charged, or,
			more frequently, contain lone electron pairs that are not bonded.

			Molecules with a $\pi$-bond, such as ethene or benzene, can also act as nucleophiles, due to the high electron
			density of the $\pi$-system.

			Examples of nucleophiles include \ch{H2O}, \ch{NH3}, and \ch{OH-}.

		% end subsection
	% end section


	\section{Inductive Effect}

		The inductive effect occurs through covalent bonds, where there is a significant difference in the electronegativity of
		participating atoms. Electrons are either withdrawn or donated through a $\sigma$-bond, due to the polarity of the molecule.

		\subsection{Withdrawal}

			Below, \ch{\Cl} is more electronegative than the carbon it is bonded to. As such, it
			\itl{inductively withdraws} electrons through the $\sigma$-bond.

			\diagram{
				\chemfig{C(-[:0,,,,-Stealth]!\molCl)(-[:90]!\molStar)(-[:180]!\molStar)(-[:270]!\molStar)}

			}[Note that the arrow represents a withdrawal of electrons, \itl{not} a dative bond.]

		% end subsubsection

		\pagebreak
		\subsection{Donation}

			Alkyl groups, or groups with the general formula \ch{C_nH_{2n+1}}, \itl{inductively donate} electrons. This behaviour
			is due to hyperconjugation, which lowers the total energy of the system, through an interaction between electrons in a
			$\sigma$-bond of the alkyl group, with a partially-filled or empty p-orbital in the adjacent atom.


			\diagram[1.0]{
				\chemfig{C(-[:0,,,,Stealth-]!\molMe)(-[:90]!\molStar)(-[:180]!\molStar)(-[:270]!\molStar)}

			}[Again, the arrow \itl{does not} represent a dative bond.]

		% end subsection
	% end section


	\section{Resonance Effect}

		The resonance effect is the withdrawal or donation of electrons through the side-on overlap of \itl{unhybridised} p-orbitals.
		Thus, the resonance effect can only occur when the central atom is \spone{} or \sptwo{} hybridised, since only those
		configurations have unhybridised p-orbitals.


		\subsection{Withdrawal}

			In the case below, electrons flow from the double-bonds to the substituent, via the resonance effect. In general, substituents
			that exhibit an electron-withdrawing resonance effect usually take the form of \ch{–Y=Z}, where Z is more electronegative than Y.
			Examples include carbonyls and nitriles.

			\diagram{
				\schemestart[0, 1.5, thick]
				\chemfig{C(-[:180]H)(=[@{b1}:270]@{oxy}{{\color{Red}O}})(-[@{b2}:0]C(-[:90]H)(=[@{b3}:0]C(-[:45]H)(-[:315]H)))}
				\arrow{->}
				\chemfig{C(-[:180]H)(-[:270]!\molO|\sps{-})(-[:0]C(-[:90]H)(=[:0]C|\sps{+}(-[:45,,1]H)(-[:315,,1]H)))}
				\schemestop

				% draw arrows
				\chemmove{\draw[-Stealth,line width=0.4mm,shorten <=2mm,shorten >=1mm](b3).. controls +(90:8mm) and +(90:8mm).. (b2);}
				\chemmove{\draw[-Stealth,line width=0.4mm,shorten <=2mm,shorten >=1mm](b1).. controls +(180:8mm) and +(180:8mm).. (oxy);}

			}[Electrons move from the electron-rich double-bonds, which have unhybridised p-orbitals, to an adjacent atom.]

		% end subection


		\pagebreak
		\subsection{Donation}

			On the other hand, groups can donate electrons through resonance, flowing from the substituent to a single-bond, forming
			a double-bond. Substituents such as halogens (\ch{F}, \ch{\Cl}, etc.), hydroxyls (\ch{-OH}), and amines (\ch{-NH2})
			are examples. They usually contain a lone pair of electrons that are free for donation.


			\diagram{
				\schemestart[0, 1.5, thick]
				\chemfig{C(-[:225]H)(-[@{b1}:90]@{oh}\color{Red}\lewis{4:,O}|{\color{Red}H})(=[@{b2}:315]@{oxy}{\color{Red}O})}
				\arrow{<->}
				\chemfig{C(-[:225]H)(=[:90,,,2]!\molHO|\sps{+})(-[:315]!\molO\sps{-})}
				\schemestop

				% draw arrows
				\chemmove{\draw[-Stealth,line width=0.4mm,shorten <=2mm,shorten >=1mm](oh).. controls +(180:8mm) and +(180:8mm).. (b1);}
				\chemmove{\draw[-Stealth,line width=0.4mm,shorten <=2mm,shorten >=1mm](b2).. controls +(45:8mm) and +(45:8mm).. (oxy);}

			}[This example shows both resonant withdrawal and donation.]


		% end subsection


		\subsection{Overall Effect}

			Since it is possible for a substituent group to simultaneously withdraw and donate electrons through different
			mechanisms, it can be difficult to determine whether the overall effect serves to withdraw or donate electrons.

			\begin{table}[htb]\renewcommand{\arraystretch}{1.5}\begin{center}
			\begin{tabu} to 0.9\textwidth { X[4,c,m] | X[c,m] | X[c,m] }

				Substituent Group                           &   Strength    &   Overall Effect  \\ \hline

				Alkyl/aryl groups (eg. \ch{-CH3})           &   Weak        &   Donating        \\
				\ch{-OH}, \ch{-NH2}, \ch{-OCH3}             &   Strong      &   Donating        \\
				\ch{-\Cl}, \ch{-Br}                         &   Weak        &   Withdrawing     \\
				\ch{-CHO}, \ch{-NO2}, \ch{-CN}, \ch{-CO2H}  &   Strong      &   Withdrawing     \\

			\end{tabu}\end{center}
			\end{table}\vspace{-1em}

			\subsubsection{Caveats}
			\hypertarget{CaveatResonanceTable}{}

				Note that the table above only applies to benzene rings, which will be covered \hyperlink{ChapterArenes}{\boit{later}}. This is because
				the resonance effect (overlapping of p-orbitals with $\pi$-electron clouds) is only commonly applicable to arenes, or where
				the atom is \sptwo{} hybridised.

				In all other cases, where the substituent is bonded to an \spthree{} hybridised atom, the electron-donating or withdrawing
				behaviour is determined mostly by the electronegativity, since there is no unhybridised p-orbital to overlap with. Hence,
				alkyl groups are basically the only electron-donating species in this case.

			% end subsubsection
		% end subsection
	% end section
% end part

















