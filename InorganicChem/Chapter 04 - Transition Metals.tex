% Chapter 04 - Transition Metals.tex
% Copyright (c) 2014 - 2016, zhiayang@gmail.com
% Licensed under the Apache License Version 2.0.



\pagebreak
\part{Transition Metals}

	\section{Overview}

		Transition metals are d-block elements with atoms that have a \itl{partially-filled} d-subshell, or form cations with incomplete
		d-subshells.

		\imgdiagram{100mm}{../figures/inorganic/ch04/periodic_table_blocks.png}{The d-block is in green.}

		Even though zinc and scandium are in the d-block, they are generally not considered transition metals, and do not exhibit many
		of the typical properties of transition metals.


		\subsection{Electronic Configuration}

			The electronic configurations of the period 4 d-block elements are laid out below. For reference, the electronic configuration
			of argon is 1s\sps{2}2s\sps{2}2p\sps{6}3s\sps{2}3p\sps{6}.

			Moving down the periodic table, most of the "rules" and "trends" that govern electrons start to break down --- especially when involving
			transition metals. It becomes more useful to observe experimental data and come up with explanations, rather than creating predictions
			and making exceptions.

			\pagebreak
			\subsubsection{Atomic Configuration}

				These are the atomic configurations of the first period of transition metals.

				\begin{table}[htb]\renewcommand{\arraystretch}{1.3}\begin{center}
				\begin{tabu} to 0.9\textwidth { X[c,m] | X[c,m] }

					% headings
					Element &   Atomic Configuration            \\ \hline
					\ch{Sc} &   [\ch{Ar}] 3d\sps{1}4s\sps{2}    \\
					\ch{Ti} &   [\ch{Ar}] 3d\sps{2}4s\sps{2}    \\
					\ch{V}  &   [\ch{Ar}] 3d\sps{3}4s\sps{2}    \\
					\ch{Cr} &   [\ch{Ar}] 3d\sps{5}4s\sps{1}    \\
					\ch{Mn} &   [\ch{Ar}] 3d\sps{5}4s\sps{2}    \\
					\ch{Fe} &   [\ch{Ar}] 3d\sps{6}4s\sps{2}    \\
					\ch{Co} &   [\ch{Ar}] 3d\sps{7}4s\sps{2}    \\
					\ch{Ni} &   [\ch{Ar}] 3d\sps{8}4s\sps{2}    \\
					\ch{Cu} &   [\ch{Ar}] 3d\sps{10}4s\sps{1}   \\
					\ch{Zn} &   [\ch{Ar}] 3d\sps{10}4s\sps{2}   \\

				\end{tabu}\end{center}
				\end{table}\vspace{-1em}

				Notably, Chromium and Copper seem out of place with their 4s\sps{1} configurations. While the \itl{generally accepted} explanation
				is that half-full and fully-filled subshells (3d\sps{5} and 3d\sps{10} respectively) are more stable, in reality there is no extra
				stability conferred to this configuration.

				A more complete explanation can be found in the appendix, but the TL;DR is that the \itl{alternative} configurations are less stable
				(due to a number of factors --- electronic repulsion, exchange energy, atomic radius), so the electrons end up in that arrangement.

				% parallel spin avoidance factor (http://www.quimica.ufpr.br/edulsa/cq115/artigos/Ionization_Energies_Parallel_spins_and_the_stability_of_half_filled_shells.pdf)
				% exchange energy
				% https://socratic.org/questions/why-is-the-electron-configuration-of-chromium-1s-2-2s-2-2p-6-3s-2-3p-6-3d-5-4s-1
				% https://www.quora.com/Why-chromium-has-electron-configuration-4s1-3d5/answer/Michael-Flynn-2

			% end subsubsection


			\subsubsection{Ionic Configuration}

				As previously mentioned, transition metals exist in multiple oxidation states, so it is futile to list them all. However, it is
				important to note that the 4s electrons are removed first, before the 3d electrons.

				\begin{table}[htb]\renewcommand{\arraystretch}{1.3}\begin{center}
				\begin{tabu} to 0.9\textwidth { X[c,m] | X[c,m] }

					% headings
					Ion         &   Ionic Configuration     \\ \hline
					\ch{Sc^3+}  &   [\ch{Ar}] 3d\sps{0}     \\
					\ch{Cr^3+}  &   [\ch{Ar}] 3d\sps{3}     \\
					\ch{Co^2+}  &   [\ch{Ar}] 3d\sps{7}     \\
					\ch{Cu^2+}  &   [\ch{Ar}] 3d\sps{9}     \\
					\ch{Zn^2+}  &   [\ch{Ar}] 3d\sps{10}    \\

				\end{tabu}\end{center}
				\end{table}\vspace{-1em}

				\pagebreak
				The electron shells are filled in this order (commonly known as the Madelung rule):

				\imgdiagram{70mm}{../figures/inorganic/ch04/electron_filling.png}{\vspace{-1em}}

				However, there is nothing that implies electrons must be removed in the corresponding order (like a stack). It must be noted
				that the electron state of an \ch{Fe^2+} is not simply that of an \ch{Fe^3+} ion with one extra electron. In the ground state,
				an empty 4s subshell does have a lower energy than an empty 3d subshell, so the former is filled first, across the period.

				However, the subsequent addition of electrons to the 3d subshell lowers its energy below that of the 4s (but it's already full,
				so it doesn't matter). Thus, when electrons are removed, they are pulled from the higher-energy orbitals --- so the 4s subshell is
				the first to go.

				The key appreciation is that the 3d and 4s subshells are very close together in terms of energy levels, such that certain
				stabilising or destabilising effects can have enough impact to move an electron across a subshell.

			% end subsubsection

		% end subsection

	% end section

	\pagebreak
	\section{Physical Properties}

		\subsection{Atomic Radii}

			Generally speaking, the atomic radii across the transition metals remain relatively constant. Physically, the 4s electrons
			are the valence electrons; hence, since subsequent electrons are added to the 3d subshell, the increase in the shielding effect
			somewhat mitigates the increase in nuclear charge.

			\imgdiagram{110mm}{../figures/inorganic/ch04/atomic_radii.png}{\vspace{-1em}}

			Hence, the atomic radius remains relatively constant across the d-block elements. In comparison with potassium and calcium,
			the radii of the d-block elements are \itl{much smaller}; this is due to the fact that the shielding effect of the 3d-subshell
			is quite poor, due to its \itl{diffuse nature}.

			The increase in nuclear charge from \ch{Ca} to \ch{Sc} without a correspondingly significant change in the shielding effect
			leads to a sharp decrease in atomic radius, by virtue of being able to attract the valence electrons closer to the nucleus.

		% end subsection


		\subsection{Melting and Boiling Points}

			There is no singular trend, but generally speaking the melting and boiling points for the transition elements are all higher
			than the s-block metals. This is due to the increased number of electrons available for metallic bonding --- transition
			metals can use both the 3d and 4s electrons, due to the small difference in their energy levels.

			From vanadium to manganese however, there is a decrease in melting and boiling points, which is due to the stability of the
			singly-filled 3d-subshell, which reduces their availability for metallic bonding.

			Somewhat similarly, from iron to zinc, the melting and boiling points start to decrease, as paired electrons are again less
			available for bonding.

		% end subsection


		\subsection{First Ionisation Energy}

			The trend for the first IE is somewhat similar to the trend for atomic radii. For s- and p-block elements, additional electrons
			do not contribute significantly to the shielding effect as they are added to the valence shell. Hence, with an increasing nuclear
			charge, it becomes more difficult to remove a valence electron.

			\imgdiagram{120mm}{../figures/inorganic/ch04/first_ie.png}{\vspace{-1em}}

			By contrast, electrons are added to the inner 3d-subshell across the transition elements, leading to an increase in the shielding
			effect. Thus mostly negates the increasing nuclear charge, since the valence electron, which is the one being removed, is in the
			4s-subshell.

			Hence, first IE increases only \itl{slightly} across the transition metals.

		% end subsection


		\subsection{Other things}

			Other than the 3 main trends, transition metals are generally denser than s-block metals, mostly due to their increased atomic
			mass and smaller atomic radii.

			Furthermore, they are generally better conductors (lower resistance) due to the larger number of delocalised electrons.

		% end subsection

	% end section



	\pagebreak
	\section{Chemical Properties}

		\subsection{Variable Oxidation States}

			The primary reason for transition metals having variable oxidation states is the relatively small difference in energy levels
			of the 3d and 4s subshells. Thus, once the 4s electrons are removed, not much more energy is needed to start removing electrons
			from the 3d subshell.

			Conversely, for s-block metals, the next subshell to remove electrons from would be the p-subshell, which is at a far higher
			energy level, and hence is not easily removed --- thus s-block metals tend to only have one oxidation state, the one where all the
			4s electrons are removed.

			Generally speaking, the 2+ ion is usually more stable than the 3+ ion, due to the need to remove an extra electron. Iron is an
			exception here, due to the existence of 6 electrons in its 3d subshell, with 1 electron pair and 4 unpaired electrons, which
			is less stable than having 5 unpaired electrons; hence \ch{Fe^3+} is more stable than \ch{Fe^2+}.

			\subsubsection{Trends in Oxidation States}

				Across the period, the number of possible oxidation states increases, and peaks at manganese (having up to +7), before
				decreasing back to +2 in zinc.

				The reason for the first part in the trend is attributable to the increasing number of 3d electrons, all of which can be
				removed or involved in bonding to increase the oxidation state. Hence, manganese, with 5 electrons in the 3d-subshell and 2
				in the 4s-subshell, can have 7 electrons removed, or involved in bonding (typically the latter)

				In the second half, electrons in the d-subshell start pairing up, reducing the possible number of electrons that can participate
				in covalent bonding, hence the number of possible oxidation states decrease.

			% end subsubsection

			\subsubsection{Bonding in Various Oxidation States}

				Transition metals at relatively low oxidation states, from +1 to +3, typically exist as ions, and form ionic solids. In this
				respect they behave like normal s-block metals. For example, \ch{Fe\Cl3}, \ch{CuSO4} have \ch{Fe^3+} and \ch{Cu^2+} ions.

				However, at higher oxidation states, it is unstable to have exposed ions with high charge, so the transition metals at such
				high oxidation states are typically \itl{covalently bonded} (yes, that's right) with oxygen, forming covalent molecules or
				oxy-anions. Examples include \ch{MnO4-} and \ch{Cr2O7^2-}.

			% end subsubsection

		% end subsection


		\subsection{Catalytic Properties}

			\subsubsection{Heterogeneous Catalysis}

				The main reason transition metals are able to act as heterogeneous catalysts is due to the presence of a \itl{partially-filled}
				d-subshell. This lets them either \itl{accept} or \itl{donate} electron pairs, allowing for the adsorption of reactant molecules.

				The detailed mechanism is covered elsewhere, but suffice to say this ability to accept or donate electrons can weaken the bonds
				in the reactant, lowering the activation energy.

				Some examples of heterogeneous catalysts include:

				\begin{bulletlist}
					& Finely divided \ch{Fe} powder, \itl{Haber Process}:	\tabto{70mm}\ch{3 H2 + N2 -> 2 NH3}
					& \ch{V2O5}, \itl{Contact Process}:						\tabto{70mm}\ch{2 SO2 + O2 -> 2 SO3}
					& Nickel, \itl{Reduction of Alkenes}:					\tabto{70mm}\ch{RCH=CH2 + H2 -> RCH2CH3}
				\end{bulletlist}

			% end subsubsection



			\subsubsection{Homogeneous Catalysis}

				The ability for transition metals to act as homogeneous catalysts is due to an entirely different reason; it is a result
				of the ease of converting between different oxidation states. This is often useful when there is a need to react
				two negatively-charged ions --- the transition metal ion can act as an intermediary.

				A common example is the oxidation of \ch{I-} to \ch{I2} by \ch{S2O8^2-}.

				\txtreactioneqn{
					\ch{2 Fe^2+} + \ch{S2O8^2-}\arrow{->[\tinytext{step 1}]}\ch{2 Fe^3+} + \ch{2 SO4^2-}
					\arrow(@c1.south east--.north east){0}[-90,.4]
					\ch{2 Fe^3+} + \ch{2 I-}\arrow{->[\tinytext{step 2}]}\ch{I2} + \ch{Fe^2+}
					\arrow(@c3.south east--.north east){0}[-90,.4]
					\ch{S2O8^2-} + \ch{2 I-}\arrow{->[\tinytext{overall}]}\ch{I2} + \ch{2 SO4^2-}
				}

				Here, the \ch{Fe^2+} ion and the \ch{S2O8^2-} ion can easily react, because they are oppositely charged. \ch{Fe^2+} is
				oxidised by the peroxodisulphate ion to \ch{Fe^3+}, forming one of the products, \ch{SO4^2-}.

				The \ch{Fe^3+} produced then proceeds to oxidise the \ch{I-} ion, forming the other product, \ch{I2}, and regenerating
				the \ch{Fe^2+} catalyst.

				Without the catalyst, the need to react two negatively charged ions directly (\ch{S2O8^2-} and \ch{I-}) would lead to an
				abysmal rate of reaction.

				Note that the order of the two sub-reactions is irrelevant, so \ch{Fe^3+} can be used as the catalyst as well.

			% end subsubsection

		% end subsection


	% end section



	\section{Complexes}

		\subsection{Overview}

			A complex consists of a central metal atom or cation, datively-bonded to surrounding ligands, which are anions or molecules with
			at least one lone pair.

			While all metal cations can form complexes, transition metals form them more readily than other metals. This is due to their
			relatively higher charge density than s-block metals (smaller atomic radius, with similar or higher nuclear charge), as well
			as empty orbitals for dative bonding.

			Note that aluminium fulfils both of these criteria, and expectedly forms complex ions; \ch{[\Al(OH)4]-} and the hydrated ion
			\ch{[\Al(H2O)6]^3+} are examples. The \ch{\Al^3+} ion has entirely empty 3s-, 3p-, and 3d-subshells, which can accept electron
			pairs from ligands.

			In transition metals as well as aluminium for instance, empty subshells are typically \itl{hybridised} into identical orbitals with
			the same energy levels; aluminium hybridises the 3s, 3p, and 3d subshells, while transition metals typically can hybridise the
			4s, 4p, and 4d subshells as well.

		% end subsection


		\subsection{Ligand Dentateness}

			While a fair number of ligands are monodentate, ie. they form a single dative bond with the central atom, there are also a number
			of \itl{polydentate} ligands, forming more than one dative bond.

			\paragraph{Monodentate Ligands}

			\diagram[1.0]{
				\begin{tabular}{c@{\hspace{20mm}}c@{\hspace{20mm}}c}

					\chemfig{\dotlewis{\color{Red}O}{45,135}(-[:210]H)(-[:330]H)}
					&
					\chemfig{\dotlewis{\color{RoyalBlue}N}{90}(-[:210]H)(<:[:340]H)(<[:300]H)}
					&
					\chemfig{\mch\dotlewis{\color{Red}O}{90,180,270}-[:0]H}

					\\ \\

					water	&	cyanide		&	hydroxide

				\end{tabular}
			}

			\paragraph{Bidentate Ligands}

			\diagram[1.0]{
				\begin{tabular}{c@{\hspace{20mm}}c}

					\chemfig{C(-[:190]H)(-[:240]H)(-[:110]H\sbs{2}\dotlewis{\color{RoyalBlue}N}{90})-C(-[:70]\dotlewis{\color{RoyalBlue}N}{90}H\sbs{2})(-[:350]H)(-[:300]H)}
					&
					\chemfig{C(-[:120]\mch\dotlewis{\color{Red}O}{30,120,210})(=[:240]!\molO)-C(=[:300]!\molO)(-[:60]\dotlewis{\color{Red}O}{60,150,240}\mch)}

					\\ \\

					ethane-1,2-diamine (abbrv. en)	& 	ethanedioate

				\end{tabular}
			}

			These bidentate ligands form two dative bonds per molecule of ligand.


			\pagebreak
			\paragraph{Hexadentate Ligands}

			The prime example of a hexdentate ligand is ethylenediaminetetraacetate, otherwise known as EDTA. It is on the cover page.

			\diagram[1.0]{
				\chemname{\chemfig{M?[a,,dashed]?[b,,dashed]?[c,,dashed]?[d,,dashed]?[e,,dashed](-[:160,1.5,,,dashed]\dotlewis{\color{RoyalBlue}N}{340}(-[:35]-[:0]C(=[:60]!\molO)(-[:325]\dotlewis{\color{Red}O}{205}?[a,,dashed]))(-[:112]-[:30]C(=[:95]!\molO)(-[:340]\lewis{6:,\color{Red}O}?[b,,dashed]))(-[:190]-[:260,0.9]-[:350]\dotlewis{\color{RoyalBlue}N}{30}?[c,,dashed](-[:330]-[:0,1.2]C(=[:300]!\molO)(-[:30]\dotlewis{\color{Red}O}{150}?[d,,dashed]-[:0,,,,draw=none]))(-[:260]-[:322.5]C(-[:20]\lewis{2:,\color{Red}O}?[e,,dashed])(=[:260]!\molO))))}}{ethylenediaminetetraacetate (abbrv. EDTA)}
			}

		% end subsection


		\subsection{Complex Charges}

			Ironically it is not very complex to determine the charge of a complex; simply sum the oxidation number of the central metal atom and
			the ionic charges of the ligands. It is possible to have neutral complexes, either by having balanced charges, or with a neutral
			metal atom and neutral ligands.

			A compound containing a complex ion typically has \itl{counter ions} that serve to balance the charge. In an aqueous solution, the
			counter ions dissociate from the complex ion as expected. The complex ion is treated as a \itl{single, polyatomic ion}.

		% end subsection


		\pagebreak
		\subsection{Coordination Number}

			The coordination number of a complex determines the number of ligands around the central atom, as well as their arrangement around
			the central atom, in a manner similar to VSEPR theory.

			\diagram[1.0]{
				\begin{tabular}{c@{\hspace{20mm}}c}
					\chemfig{Ag\pch
					(-[:0,1.5,,,Stealth-]{\color{RoyalBlue}N}H\sbs{3})
					(-[:180,1.5,,,Stealth-]H\sbs{3}{\color{RoyalBlue}N})}
					&
					\chemfig{Ni\sps{2}\pch
					(-[:210,1.8,,,Stealth-]\mch{\color{RoyalBlue}N}|C?[a,,dashed])
					(-[:30,1.8,,,Stealth-]C?[b,,dashed]?[c,,dashed]|{\color{RoyalBlue}N}\mch)
					(-[:135,1.26,,,Stealth-]\mch{\color{RoyalBlue}N}|C?[a,,dashed]?[b,,dashed])
					(-[:315,1.26,,,Stealth-]C?[a,,dashed]?[c,,dashed]|{\color{RoyalBlue}N}\mch)
					}

					\\ \\

					\ch{[Ag(NH3)2]+}, linear, CN: 2		&		\ch{[Ni(CN)4]^2-}, square planar, CN: 4

					\\ \\ \\ \\

					\chemfig{Cu\sps{2}\pch
					(-[:330,1.3,,,Stealth-]{\color{OliveGreen}\Cl}?[a,,dashed]\mch)
					(-[:255,1.25,,,Stealth-]{\color{OliveGreen}\Cl}?[a,,dashed]?[b,,dashed]?[c,,dashed]\mch)
					(-[:200,1.2,,,Stealth-]{\color{OliveGreen}\Cl}?[a,,dashed]?[b,,dashed]?[c,,dashed]?[d,,dashed]\mch)
					(-[:90,1.5,,,Stealth-]{\color{OliveGreen}\Cl}?[a,,dashed]?[b,,dashed]?[d,,dashed]\mch)
					}
					&
					\chemfig{Fe\sps{3}\pch
					(-[:90,2.1,,,Stealth-]C?[top,,dashed]|{\color{RoyalBlue}N}\mch)
					(-[:270,2.1,,,Stealth-]C?[bottom,,dashed]|{\color{RoyalBlue}N}\mch)
					(-[:240,1.1,,,Stealth-]\mch{\color{RoyalBlue}N}|C?[top,,dashed]?[bottom,,dashed]?[frontl,,dashed]?[frontr,,dashed])
					(-[:0,1.7,,,Stealth-]C?[top,,dashed]?[bottom,,dashed]?[frontr,,dashed]?[backr,,dashed]|{\color{RoyalBlue}N}\mch)
					(-[:180,1.7,,,Stealth-]\mch{\color{RoyalBlue}N}|C?[top,,dashed]?[bottom,,dashed]?[frontl,,dashed]?[backl,,dashed])
					(-[:60,,,,Stealth-]C?[top,,dashed]?[bottom,,dashed]?[back,,dashed]?[backl,,dashed]?[backr,,dashed]|{\color{RoyalBlue}N}\mch)
					}

					\\ \\

					\ch{[Cu\Cl4]^2-}, tetrahedral, CN: 4	&	\ch{[Fe(CN)6]^3-}, square planar, CN: 4

				\end{tabular}
			}

		% end subsection



		\subsection{Hydrated Complex Ions}

			Otherwise known as aqua complexes, these ions are formed when a transition metal, or any metal with a sufficiently high charge
			density, is dissolved in water. Essentially these are complexes where the water molecules act as ligands. This is because
			transition metal cations have high charge density and thus polarising power, preferring the formation of covalent bonds as
			opposed to ionic interactions.

			By contrast, most s-block metals (eg. sodium) do not form hydrated complexes, only ion-dipole interactions.


			\pagebreak
			\subsubsection{Hydrolysis of Hydrated Complexes}

				As discussed to death in previous sections, these hydrated complex ions are able to hydrolyse in water to some extent, producing
				\ch{H3O+ \stAq} ions that make the solution acidic.

				This ability is due to the high polarising power of the central cation, which can polarise and weaken \ch{O-H} bonds in the
				water ligand, eventually allowing another water molecule to attack the weakly-attached H atom, forming \ch{H3O+}.


				\txtreactioneqn{
					\ch{[Fe(H2O)6]^3+ \stAq}\hspace{2mm} + \hspace{2mm}\ch{H2O \stL}\arrow{<=>}\ch{[Fe(H2O)5OH]^2+ \stAq}\hspace{2mm} + \hspace{2mm}\ch{H3O+ \stAq}
					\arrow(@c1.south east--.north east){0}[-90,.20]
					\ch{[Cr(H2O)6]^3+ \stAq}\hspace{2mm} + \hspace{2mm}\ch{H2O \stL}\arrow{<=>}\ch{[Cr(H2O)5OH]^2+ \stAq}\hspace{2mm} + \hspace{2mm}\ch{H3O+ \stAq}
				}

			% end subsubsection

			\subsubsection{Cations with High Oxidation States}

				Transition metal cations with very high oxidation states, typically above +3, do not form hydrated complex ions. This is due
				to the extreme polarising power of the ion, which instead completely breaks the \ch{O-H} bonds, favouring the formation of
				oxy-anions.

				As an example, aqueous chromium ions in the +6 oxidation state \itl{do not} exist as \ch{[Cr(H2O)6]^6+}. Instead, it
				forms \ch{CrO4^2-} and \ch{Cr2O7^2-}, chromate and dichromate, which exist in an equilibrium.

				\txtreactioneqn{
					\ch{2 CrO4^2- \stAq} + \ch{2 H+ \stAq}\arrow{<=>}\ch{Cr2O7^2- \stAq} + \ch{H2O \stL}
				}

				In acidic conditions, [\ch{H+}] is high, so by Le Châtelier's Principle, the position of equilibrium moves to the right,
				favouring the production of orange dichromate. Conversely in an alkaline medium, the position of equilibrium moves to the left
				to favour yellow chromate.

			% end subsubsection

		% end subsection


		\pagebreak
		\subsection{Ligand Exchange}

			Ligand exchange reactions are simply reactions where one or more of the ligands of a complex are replaced by other ligands. That
			is all. The new complex can often have different properties, including reduction potential and colour.

			\subsubsection{Stability Constant, \K{stab}}

				\K{stab} is essentially another form of \Kc{} for complex ions. For example, \ch{[Cu(H2O)6]^2+} were to replace its \ch{H2O} ligands
				with \ch{\Cl-} ligands:

				\txtreactioneqn{
					\ch{[Cu(H2O)6]^2+ \stAq} + \ch{4 \Cl- \stAq}\arrow{<=>}\ch{[Cu\Cl4]^2- \stAq} + \ch{6 H2O \stL}
				}

				The \K{stab} equation would be as such:

				\mathdiagram{
					\[ K_{stab} = \frac{[[CuCl_{4}]^{2-}]}{[[Cu(H_{2}O)_{6}]^{2+}][Cl-]^{4}} \]
				}

				The higher the \K{stab} value, then, the greater the probability that ligand exchange will take place.

			% end subsubsection



			\subsubsection{Effect on Reduction Potential}

				Ligands can affect the reduction potential of the central metal ion. For instance, \ch{[Fe(H2O)6]^3+} has a reduction
				potential of \SI[retain-explicit-plus]{+0.77}{\volt}, while \ch{[Fe(CN)6]^3-} only has a reduction potential of
				\SI[retain-explicit-plus]{+0.36}{\volt}.

				The primary difference lies in the stability of the complex formed, which affects the concentration of free metal ions in
				solution that can undergo reduction. Anionic ligands also have a stabilising effect on the metal cation, reducing their
				reduction potential electrostatically.

			% end subsubsection


			\subsubsection{Ligand Exchange in Haemoglobin}

				Ligand exchange underlies the function of haemoglobin molecules in transporting oxygen to cells. Under normal
				circumstances, the haemoglobin molecule has a central \ch{Fe^2+} ion, and can \itl{reversibly} form dative bonds with \ch{O2}
				molecules, creating \itl{oxyhaemoglobin}.

				When exposed to molecules of \ch{CO} however, oxyhaemoglobin undergoes a ligand exchange reaction, replacing the \ch{O2} ligand
				with a \ch{CO} ligand. Unfortunately, the \K{stab} value of \itl{carboxyhaemoglobin} is around \itl{\num{210} times} greater
				than that of oxyhaemoglobin.

				Thus, the exchange is effectively \itl{irreversible}, permanently disabling that particular molecule of haemoglobin.
				If enough molecules are disabled, oxygen fails to reach the cells and death occurs. I've heard it's painless,
				which is a slight consolation.

			% end subsubsection

		% end subsection

	% end section

	\section{Colour of Complexes}

		\subsection{Overview}

			As one might be aware, many complex ions are brightly coloured. Examples include \ch{[Cu(H2O)6]^2+}, which is pale blue,
			\ch{[Cu(NH2)4(H2O)2]^2+}, which is dark blue, and \ch{[Fe(SCN)(H2O)5]^2+}, which is blood red.

			There are a number of reasons for this, which will be explored below.

		% end subsection


		\subsection{Quantum Theory of Light}

			What must first be explained is the quantum theory of light. It states that light consists of discrete particles --- \itl{photons}.
			When a photon interacts with an electron, it must be \itl{completely} absorbed, in a one-to-one interaction. All of the energy of
			the photon must be absorbed at once, or not at all.

			Next, electrons in atomic orbitals can only have certain, \itl{fixed} energy levels. Thus, for an electron to absorb the energy
			of an oncoming photon, the energy of the photon must correspond \itl{exactly} to a valid \itl{energy gap} in the orbitals of
			the atom.

			The energy of a photon is given as such:

			\mathdiagram{
				\[ E = \frac{hc}{\lambda} \]
			}

			$E$ is the energy of the photon, $h$ is the Planck Constant (\SI{6.63e-34}{\square\metre\kilo\gram\per\second}), $c$ is
			the speed of light, and $\lambda$ is the wavelength of the photon.

			As one might know, the wavelength of light corresponds to the \itl{colour} of that light. Hence, the energy of a photon depends
			on its colour.

			The wavelengths of visible light lie between \SI{400}{\nano\metre} and \SI{700}{\nano\metre}. Red has the longest wavelength,
			followed by orange, yellow, green, blue, then violet, at \SI{400}{\nano\metre}.

		% subsection


		\pagebreak
		\subsection{Crystal Field Theory}

			Crystal field theory, or CFT, is a theory that attempts to explain the reason for coloured complex ions. The primary idea
			underlying CFT is that complexes are formed because of an electrostatic attraction between the metal cation, and the lone
			pairs of electrons.

			The 5 d-orbitals have the following shapes in 3d space:

			\imgdiagram{100mm}{../figures/inorganic/ch04/d_orbitals.png}{The 5 orbitals in the d subshell.}

			For an octahedral complex, the ligands will approach along the \itl{x}, \itl{y}, and \itl{z} axes. Thus, they directly approach
			the lobes of the \itl{d\sbs{z\sps{2}}} and \itl{d\sbs{x\sps{2-}}\sbs{y\sps{2}}} orbitals, leaving the lobes of the \itl{d\sbs{yz}},
			\itl{d\sbs{xz}}, and \itl{d\sbs{xy}} orbitals between the ligands.

			A theoretically free transition metal ion has 5 \itl{degenerate} d-orbitals, that are all at the same energy level. However, due
			to their arrangement in space, the electrostatic repulsion between the ligands' lone pairs, and \itl{some} of the d-orbitals,
			leads to a \itl{splitting} of the orbitals, and a loss of degeneracy.

			\imgdiagram{100mm}{../figures/inorganic/ch04/electron_splitting.png}{The 5 degenerate orbitals are split, into 2 and 3.}

			Note that the 2 higher energy orbitals correspond to those where there is electrostatic repulsion with the ligand, and the lower
			3 are the orbitals between the axes.

			\pagebreak
			\subsubsection{d-d Electronic Transition}

				Going back to the quantum theory of light, a d-d electronic transition can occur when an electron in the lower-energy orbitals
				\itl{absorbs} the energy of a photon corresponding to $∆E$, and is \itl{promoted} to the higher-energy orbital.

				Hence, on average, the photons that pass through the complex molecule will be \itl{missing} that particular wavelength of
				light, so the colour of the complex will appear to be the \itl{complementary colour} of the one absorbed.

				The complements of colours are below.

				\begin{bulletlist}
					& Yellow / Violet
					& Red / Green
					& Orange / Blue
				\end{bulletlist}

				Conveniently, the value of $∆E$ for many complexes lie within the wavelength of visible light, so many of them appear
				coloured.

				If $∆E$ lies beyond that, then d-d electronic transitions will still occur (absorption of UV light for example),
				but human eyes cannot detect them.

			% end subsubsection

		% end subsection


		\pagebreak
		\subsection{Charge-Transfer}

			Other than d-d electronic transfers, charge transfers --- LMCT, or ligand-to-metal charge transfer, and MLCT, metal-to-ligand
			charge transfers --- can also be the cause of coloured ions.

			Notice that in \ch{MnO4-}, the central \ch{Mn} atom is at a +7 oxidation state, and has \itl{no} d-electrons. Hence, d-d electronic
			transitions cannot be the cause of its \itl{intense} purple colour. Instead, LMCT is the cause.

			Similar concepts apply, including the quantum theory of light. Instead of $∆E$ being the difference in energy of the split
			d-orbitals, it is now the energy difference between the highest occupied molecular orbital of the donor, and the lowest unoccupied
			molecular orbital of the acceptor.

			In other words, it is the first electron affinity of the acceptor less the first ionisation energy of the donor, plus corrections
			for electrostatic interactions.

			\mathdiagram{
				\[ ∆E = E_{A} - E_{I} + J \]
			}

			Due to certain reasons which limit the kinds of electrons that can be promoted in a d-d electronic transition, the intensity of
			absorption for complexes whose colour depends on d-d transitions is usually small --- hence most of the colours are somewhat
			less intense.

			However, these restrictions (involving the quantum state of the electron) do not apply for charge transfers, so the intensity
			of light absorbed is much greater, leading to much more intense colours in these complexes.

		% end subsection


		\subsection{Factors Affecting Complex Colours}

			\subsubsection{Oxidation State of Cation}

				The oxidation state of the central metal atom will affect the number of electrons present in the d-subshell. This will thus
				affect the energy level of the orbitals in question, thus changing the value of $∆E$.

				Note that \ch{Sc^3+}, \ch{Ti^4+}, \ch{Cu+} and \ch{Zn^2+} are not coloured, even though titanium and copper are transition metals,
				because they either have fully-filled d-subshells, or empty d-subshells. The subshells are, in the case of titanium and copper,
				partially filled in other oxidation states.

			% end subsubsection


			\subsubsection{Nature of Ligand}

				The nature of the ligand will also affect the value of $∆E$. The bonds within the ligands themselves can affect the energy
				level of the lone pair of electrons, and thus change the degree of electrostatic repulsion faced by the orbitals.

			% end subsubsection

		% end subsection

	% end section

% end part














