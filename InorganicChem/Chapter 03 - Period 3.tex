% Chapter 03 - Period 3.tex
% Copyright (c) 2014 - 2016, zhiayang@gmail.com
% Licensed under the Apache License Version 2.0.


\pagebreak
\part{The Third Period}

	\section{Overview}

		This chapter will examine the 7 elements in the third period; sodium (\ch{Na}), magnesium (\ch{Mg}), aluminium (\ch{\Al}), silicon (\ch{Si}),
		phosphorus (\ch{P}), sulphur (\ch{S}), and chlorine (\ch{\Cl}). Argon is... inert. Mostly.

	% end section

	\section{Elemental Reactions}

		Generally speaking, the reactivity of the elements with both oxygen and chlorine \itl{decrease} across the period. Since reducing power
		decreases across the period, elements are less likely to be oxidised by oxygen and chlorine.

		\subsection{Reaction with Oxygen}

			\subsubsection{Sodium}

				Sodium metal tarnishes rapidly in air, and burns readily with a \itl{bright yellow} flame.

				\txtreactioneqn{
					\ch{2 Na \stS} + \ch{\fracHalf O2 \stG} \arrow \ch{Na2O \stS}
				}

			% end subsubsection

			\subsubsection{Magnesium}

				Magnesium metal tarnishes in air, albeit less readily than sodium. It burns very strongly with a \itl{bright white} flame, and
				its combustion is highly exothermic.

				\txtreactioneqn{
					\ch{Mg \stS} + \ch{\fracHalf O2 \stG} \arrow \ch{MgO \stS}
				}

			% end subsubsection

			\pagebreak
			\subsubsection{Aluminium}

				Aluminium metal is readily oxidised by air, and forms a protective layer of aluminium oxide (\ch{\Al2O3}) that shields the
				metal inside from further oxidation.

				High temperatures (above \SI{800}{\celsius}) are required to fully oxidise aluminium.

				\txtreactioneqn{
					\ch{2 \Al \stS} + \ch{\fracThreeTwo O2 \stG} \arrow \ch{\Al2O3 \stS}
				}

			% end subsubsection

			\subsubsection{Silicon}

				Silicon \itl{does not} react with oxygen at room temperature. When strongly heated, some silicon dioxide (\ch{SiO2}) can be
				formed. Silicon is a giant covalent molecule, so large amounts of energy need to be supplied to allow any reaction to
				take place.

				\txtreactioneqn{
					\ch{Si \stS} + \ch{O2 \stG} \arrow \ch{SiO2 \stS}
				}

			% end subsubsection

			\subsubsection{Phosphorus}

				There are two predominant forms of elemental phosphorus; white phosphorus, which are discrete molecules of \ch{P4}, and red
				phosphorus, which takes a polymeric form.

				The former \itl{spontaneously combusts} in air with a bright white flame, due to the strained tetrahedral bonds in the molecule.
				On the other hand, red phosphorus is more stable, and will burn on heating with a soft orange flame; the products for both
				forms of phosphorus are identical:

				\txtreactioneqn{
					\ch{P4 \stS} + \ch{3 O2 \stG} \arrow \ch{P4O6 \stS}
					\arrow(@c1.east--.east){0}[-90,.4]
					\ch{P4 \stS} + \ch{5 O2 \stG} \arrow \ch{P4O10 \stS}
				}

				With limited oxygen, the predominant product will be \ch{P4O6}; in excess oxygen, the major product will be \ch{P4O10}.

			% end subsubsection

			\pagebreak
			\subsubsection{Sulphur}

				Sulphur is stable at room temperature, and requires some heating to burn --- with a pale blue flame:

				\txtreactioneqn{
					\ch{S \stS} + \ch{O2 \stG} \arrow \ch{SO2 \stG}
				}

				With a catalyst (eg. vanadium oxide \ch{V2O5}), \ch{SO2} can be further oxidised:

				\txtreactioneqn{
					\ch{SO2 \stG} + \ch{\fracHalf O2 \stG} \arrow \ch{SO3 \stG}
				}

			% end subsubsection

			\subsubsection{Chlorine}

				Chlorine rarely reacts directly with oxygen, because it is a stronger oxidising agent than oxygen and hence unlikely to be
				oxidised.

			% end subsubsection
		% end subsection


		\subsection{Reaction with Chlorine}

			\subsubsection{Sodium}

				Sodium metal is readily oxidised by chlorine and burns in it, with highly exothermic reaction:

				\txtreactioneqn{
					\ch{2 Na \stS} + \ch{\Cl2 \stG} \arrow \ch{Na\Cl \stS}
				}

			% end subsubsection

			\subsubsection{Magnesium}

				Magnesium's reaction with chlorine is similar to that of sodium:

				\txtreactioneqn{
					\ch{Mg \stS} + \ch{\Cl2 \stG} \arrow \ch{Na\Cl \stS}
				}

			% end subsubsection

			\subsubsection{Aluminium}

				Aluminium does not \itl{quite} react readily with chlorine. The reaction can be sped up by passing \itl{dry} chlorine gas
				over heated aluminium foil (larger surface area):

				\txtreactioneqn{
					\ch{\Al \stS} + \ch{\fracThreeTwo \Cl2 \stG} \arrow \ch{\Al\Cl3 \stS}
				}

				\ch{\Al\Cl3} typically sublimes into the gaseous state.

			% end subsubsection


			\pagebreak
			\subsubsection{Silicon}

				The reaction of silicon with chlorine is even slower than with aluminium, but if heated silicon powder is used, the
				reaction can be sped up slightly.

				\txtreactioneqn{
					\ch{Si \stS} + \ch{2 \Cl2 \stG} \arrow \ch{Si\Cl4 \stL}
				}

				Silicon tetrachloride is a liquid at room temperature.

			% end subsubsection

			\subsubsection{Phosphorus}

				White phosphorus (\ch{P4}) burns vigorously in chlorine gas to form \ch{P\Cl3} and \ch{P\Cl5}:

				\txtreactioneqn{
					\ch{P4 \stS} + \ch{6 \Cl2 \stG} \arrow \ch{4 P\Cl3 \stL}
					\arrow(@c1.east--.east){0}[-90,.4]
					\ch{P4 \stS} + \ch{10 \Cl2 \stG} \arrow \ch{4 P\Cl5 \stS}
				}

				The formation of \ch{P\Cl5} occurs in an equilibrium, so excess chlorine gas should be used to favour the production of the latter,
				or limited chlorine to prefer the former:

				\txtreactioneqn{
					\ch{P\Cl3 \stL} + \ch{\Cl2 \stG} \arrow{<=>} \ch{P\Cl5 \stS}
				}

			% end subsubsection

			\subsubsection{Sulphur}

				Heated sulphur can react with chlorine gas:

				\txtreactioneqn{
					\ch{2 S \stS} + \ch{\Cl2 \stG} \arrow \ch{S2\Cl2 \stL}
				}

			% end subsubsection

		% end subsection


		\pagebreak
		\subsection{Reaction with Water}

			\subsubsection{Sodium}

				Sodium, characteristic of group 1 metals, reacts vigorously even with cold water.

				\txtreactioneqn{
					\ch{2 Na \stS} + \ch{2 H2O \stL} \arrow \ch{2 NaOH \stAq} + \ch{H2 \stG}
				}

			% end subsubsection

			\subsubsection{Magnesium}

				Magnesium reacts slowly with cold water, due to the formation of a mostly-insoluble hydroxide layer around the metal.

				\txtreactioneqn{
					\ch{Mg \stS} + \ch{2 H2O \stL} \arrow \ch{Mg(OH)2 \stS} + \ch{H2 \stG}
				}

				With steam however, magnesium reacts directly to form magnesium oxide:

				\txtreactioneqn{
					\ch{Mg \stS} + \ch{H2O \stG} \arrow \ch{MgO \stS} + \ch{H2 \stG}
				}

			% end subsubsection

			\subsubsection{Aluminium}

				Aluminium does not appear to react with with water, but this is due to the strong and unreactive layer of
				\ch{\Al2O3} on the surface, which is formed by the fast oxidation of aluminium by oxygen in the air.

				If the coating is not present, aluminium will form \ch{\Al(OH)3} and \ch{H2} as usual.



			% end subsubsection


			\subsubsection{Silicon}

				Silicon is fairly unreactive towards water, due to the strong bonds in its giant covalent structure.
				However, when heated strongly with steam, silicon dioxide can be formed:

				\txtreactioneqn{
					\ch{Si \stS} + \ch{2 H2O \stG} \arrow \ch{SiO2 \stS} + \ch{H2 \stG}
				}

			% end subsubsection

			\subsubsection{Phosphorus and Sulphur}

				Neither phosphorus nor sulphur react with water.

			% end subsubsection

			\subsubsection{Chlorine}

				Chlorine disproportionates reversibly in water, as covered in the previous chapter on halogens:

				\txtreactioneqn{
					\ch{\Cl2 \stG} + \ch{H2O \stL} \arrow{<=>} \ch{H\Cl \stAq} + \ch{HO\Cl \stAq}
				}

			% end subsubsection
		% end subsection

	% end section



	\pagebreak
	\section{Properties of the Oxides}

		In general, the acidities of the oxides increase across the period. Sodium and magnesium oxide are basic, aluminium oxide is
		amphoteric, and the other oxides are acidic. This is due to the decreasing ionic character, or conversely, the increasing
		covalent character in the bonding of the oxides.

		The sections below will also cover the reactions of the hydroxides when appropriate.

		\subsection{Bonding, Structure, and Melting Points}

			Across the period, the bonding in the oxides moves from ionic to covalent. This is due to the decreasing difference
			in electronegativity between the element and oxygen; thus it is less likely to \itl{completely lose} electrons, and hence
			covalent bonds become prevalent.

			The change in bonding naturally creates a change in structure, from ionic lattices to discrete covalent molecules. Hence, this
			also causes a general decrease in melting points across the period.

			\begin{table}[htb]\renewcommand{\arraystretch}{1.5}\begin{center}
			\begin{tabu} to 0.7\textwidth { X[c,m] | X[c,m] | X[c,m] }

				% headings
				Thing       &   Structure       &   Melting Point / \si{\celsius}   \\ \hline
				\ch{Na2O}   &   ionic           &   \num{1130}                      \\
				\ch{MgO}    &   ionic           &   \num{2850}                      \\
				\ch{\Al2O3} &   ionic           &   \num{2070}                      \\
				\ch{SiO2}   &   giant covalent  &   \num{1700}                      \\
				\ch{P4O6}   &   covalent        &   \num{24}                        \\
				\ch{P4O10}  &   covalent        &   \num{360} (sublimes)            \\
				\ch{SO2}    &   covalent        &   \num{-75}                       \\
				\ch{SO3}    &   covalent        &   \num{17}                        \\

			\end{tabu}\end{center}
			\end{table}\vspace{-1em}

			The melting point of \ch{MgO} is much larger than that of \ch{Na2O} due to the much higher charge density of \ch{Mg^2+}, which
			greatly increases the lattice energy of \ch{MgO}.

			However, the \itl{even higher} charge density of \ch{\Al^3+} instead serves to \itl{polarise} the \ch{O^2-} ion, resulting
			in some \itl{covalent character}, hence \itl{reducing} the lattice energy of \ch{\Al2O3}.

			The reason for the high melting point of silicon dioxide is simply due to its giant covalent structure.

		% end subsection


		\pagebreak
		\subsection{Sodium Oxide}

			Sodium oxide dissolves completely in water to give \ch{OH-} ions, and as such a it will give a \pH{} of around \num{13}. This
			reaction is vigorous and exothermic, occurring even with cold water.

			Sodium oxide also reacts directly with acids, to give an aqueous salt solution.

			\txtreactioneqn{
				\ch{Na2O \stS} + \ch{H2O \stL} \arrow \ch{2 Na+ \stAq} + \ch{2 OH- \stAq}
				\arrow(@c1.east--.east){0}[-90,.4]
				\ch{Na2O \stS} + \ch{2 H+ \stAq} \arrow \ch{2 Na+ \stAq} + \ch{H2O \stL}
			}

		% end subsection


		\subsection{Magnesium Oxide}

			Magnesium oxide dissolves much less readily in water, and reacts \itl{very slowly}. This is due to the high lattice energy of
			\ch{MgO}, which naturally needs a lot of energy to break to create ions.

			Furthermore, whatever pithy amounts of \ch{Mg(OH)2} formed \itl{barely} dissociates to give \ch{OH-} ions, and so the reaction
			mixture is a mostly solid lump with a slightly-higher-than-neutral \pH{} of around \num{9}. This is due to the similarly high lattice
			energy of \ch{Mg(OH)2}.

			\txtreactioneqn{
				\ch{MgO \stS} + \ch{H2O \stL} \arrow \ch{Mg(OH)2 \stS}
				\arrow(@c1.east--.east){0}[-90,.4]
				\ch{Mg(OH)2 \stS} \arrow{<=>} \ch{Mg^2+ \stAq} + \ch{2 OH- \stAq}
			}

			On the other hand, both \ch{MgO} and \ch{Mg(OH)2} can react directly with acids to form aqueous salt solutions:

			\txtreactioneqn{
				\ch{MgO \stS} + \ch{2 H+ \stAq} \arrow \ch{Mg^2+ \stAq} + \ch{H2O \stL}
				\arrow(@c1.east--.east){0}[-90,.4]
				\ch{Mg(OH)2 \stS} + \ch{2 H+ \stAq} \arrow \ch{Mg^2+ \stAq} + \ch{2 H2O \stL}
			}

		% end subsection


		\pagebreak
		\subsection{Aluminium Oxide}

			Aluminium oxide, \ch{\Al2O3}, is partially covalent, due to the polarising power of the \ch{\Al^3+} ion. Thus, the combination of
			ionic and covalent characters allows \ch{\Al2O3}, as well as \ch{\Al(OH)3}, to be amphoteric; ie. they react with both acids and
			bases.

			Due to the high lattice energy of both \ch{\Al2O3} and \ch{\Al(OH)3}, they are insoluble in water. However, they both react
			with acids to form an aqueous salt solution containing \ch{\Al^3+} ions:

			\txtreactioneqn{
				\ch{\Al2O3 \stS} + \ch{6 H+ \stAq} \arrow \ch{\Al^3+ \stAq} + \ch{3 H2O \stL}
				\arrow(@c1.east--.east){0}[-90,.4]
				\ch{\Al(OH)3 \stS} + \ch{3 H+ \stAq} \arrow \ch{\Al^3+ \stAq} + \ch{3 H2O \stL}
			}

			They can react with strong bases as well, to form \itl{aluminate complex ions}, \ch{[\Al(OH)4]-}:

			\txtreactioneqn{
				\ch{\Al2O3 \stS} + \ch{2 OH- \stAq} \arrow \ch{2 [\Al(OH)4]- \stAq}
				\arrow(@c1.east--.east){0}[-90,.4]
				\ch{\Al(OH)3 \stS} + \ch{OH- \stAq} + \ch{3 H2O \stL} \arrow \ch{[\Al(OH)4]- \stAq}
			}

		% end subsection



		\subsection{Silicon Dioxide}

			Silicon dioxide neither reacts with nor dissolves in water, due to the strong covalent bonds between the \ch{Si} and \ch{O} atoms.
			In fact, it does not succumb even when heated with aqueous bases. For any reaction to take place, there must be strong heating
			with concentrated bases, eg. \ch{NaOH}.

			\txtreactioneqn{
				\ch{SiO2 \stS} + \ch{2 OH- \stL} \arrow \ch{SiO3^2- \stAq} + \ch{H2O \stL}
			}

			When it does react, a silicate ion is formed.

		% end subsection



		\pagebreak
		\subsection{Phosphorus Oxides}

			Both forms of phosphorus oxide, \ch{P4O6} and \ch{P4O10}, react with water to give acidic solutions. Both react with cold water,
			although the reaction of \ch{P4O10} is a lot more violent than that for \ch{P4O6}.

			\diagram[1.0]{
				\schemestart[0,1.5,thick]
					\chemname[1em]{\chemfig{P(=[:90]!\molO)(-[:210]H)(-[:330]!\molO-[:30]H)(-[:280]!\molO-[:330]H)}}{\ch{H3PO3}}
					\arrow{0}
					\chemname[1em]{\chemfig{P(=[:90]!\molO)(-[:210]!\molO-[:150]H)(-[:330]!\molO-[:30]H)(-[:280]!\molO-[:330]H)}}{\ch{H3PO4}}
				\schemestop
			}

			\ch{H3PO3} and \ch{H3PO4} are both weak acids. \ch{H3PO3} is a diprotic acid, and its \pKa{} values are \num{1.1} and \num{6.7};
			\ch{H3PO4} is a triprotic acid, with \pKa{} values of \num{2.15}, \num{7.20}, and \num{12.32}. Note that the third hydrogen
			in \ch{H3PO3} isn't attached to anything electronegative, so it does not dissociate.

			\txtreactioneqn{
				\ch{P4O6 \stS} + \ch{6 H2O \stL} \arrow \ch{4 H3PO3 \stAq}
				\arrow(@c1.east--.east){0}[-90,.4]
				\ch{H3PO3 \stAq} \arrow{<=>} \ch{H2PO3- \stAq} + \ch{H+ \stAq}
				\arrow(@c3.east--.east){0}[-90,.4]
				\ch{H2PO3-} \arrow{<=>} \ch{HPO3^2- \stAq} + \ch{H+ \stAq}
				\arrow(@c5.east--.east){0}[-90,.8]
				\ch{P4O10 \stS} + \ch{6 H2O \stL} \arrow \ch{4 H3PO4 \stAq}
				\arrow(@c7.east--.east){0}[-90,.4]
				\ch{H3PO4 \stAq} \arrow{<=>} \ch{H2PO4- \stAq} + \ch{H+ \stAq}
				\arrow(@c9.east--.east){0}[-90,.4]
				\ch{H2PO4- \stAq} \arrow{<=>} \ch{HPO4^2-} + \ch{H+ \stAq}
				\arrow(@c11.east--.east){0}[-90,.4]
				\ch{HPO4^2- \stAq} \arrow{<=>} \ch{PO4^3- \stAq} + \ch{H+ \stAq}
			}

			Both \ch{P4O6} and \ch{P4O10} can react directly with bases to give a salt solution, containing anions of the fully deprotonated
			acids (\ch{HPO3^2-} and \ch{PO4^3-} respectively):

			\txtreactioneqn{
				\ch{P4O6 \stS} + \ch{8 OH- \stAq} \arrow \ch{4 HPO4^2- \stAq} + \ch{2 H2O \stL}
				\arrow(@c1.east--.east){0}[-90,.4]
				\ch{P4O10 \stS} + \ch{12 OH- \stAq} \arrow \ch{4 PO4^3- \stAq} + \ch{6 H2O \stL}
			}


		% end subsection


		\pagebreak
		\subsection{Sulphur Oxides}

			\subsubsection{Sulphur Dioxide (\texorpdfstring{\ch{SO2}}{SO₂})}

				Sulphur dioxide is mostly soluble in water, giving a solution of the weak diprotic sulphurous acid, \ch{H2SO3}. It has a
				\pKa{} value of around \num{1.8}, so it is slightly stronger than \ch{H3PO3} and \ch{H3PO4}.

				\diagram[1.0]{
					\schemestart[0,1.5,thick]
						\chemfig{S(=[:90]!\molO)(-[:330]!\molO-[:0]H)(-[:210]!\molO-[:180]H)}
					\schemestop
				}[Sulphurous acid, \ch{H2SO3}]

				\txtreactioneqn{
					\ch{SO2 \stG} + \ch{H2O \stL} \arrow \ch{H2SO3 \stAq}
					\arrow(@c1.east--.east){0}[-90,.4]
					\ch{H2SO3 \stAq} \arrow{<=>} \ch{HSO3- \stAq} + \ch{H+ \stAq}
					\arrow(@c3.east--.east){0}[-90,.4]
					\ch{HSO3-} \arrow{<=>} \ch{SO3^2- \stAq} + \ch{H+ \stAq}
				}

				\ch{SO2 \stG} can react directly with bases to form a salt solution containing the sulphite ion, \ch{SO3^2-}, which
				is the fully deprotonated version of the acid.

				\txtreactioneqn{
					\ch{SO2 \stG} + \ch{2 OH- \stL} \arrow \ch{SO3^2- \stAq} + \ch{H2O \stL}
				}


			% end subsubsection


			\subsubsection{Sulphur Trioxide (\texorpdfstring{\ch{SO3}}{SO₃})}

				Sulphur trioxide, on the other hand, reacts violently with water, producing a mist of \ch{H2SO4} droplets.

				\diagram[1.0]{
					\schemestart[0,1.5,thick]
						\chemfig{S(=[:280]!\molO)(=[:90]!\molO)(-[:330]!\molO-[:0]H)(-[:210]!\molO-[:180]H)}
					\schemestop
				}[Sulphuric acid, \ch{H2SO4}]

				Sulphuric acid is the strong acid that we know and love; \ch{SO3 \stG} can also react directly with bases to
				form an aqueous salt solution with sulphate ions.

				\txtreactioneqn{
					\ch{SO3 \stG} + \ch{2 OH- \stL} \arrow \ch{SO4^2- \stAq} + \ch{H2O \stL}
				}

				It should be noted that while the first proton of \ch{H2SO4} is a undoubtedly a strong acid with a \pKa{} of \num{-3}, the second
				proton (from \ch{HSO4-}) has a \pKa{} of only \num{1.99}, which is technically a weak acid. In most cases however, \ch{H2SO4} will
				still fully deprotonate to give the sulphate ion.

			% end subsubsection

		% end subsection

	% ends section


	\pagebreak
	\section{Properties of the Chlorides}

		\subsection{Bonding, Structure, and Melting Points}

			The trend for the chlorides is mostly identical to that for the oxides; bonding moves from ionic to covalent due to decreasing
			electronegativity difference between the element and chlorine, causing a shift from ionic solids to discrete covalent molecules.

			\begin{table}[htb]\renewcommand{\arraystretch}{1.5}\begin{center}
			\begin{tabu} to 0.9\textwidth { X[c,m] | X[c,m] | X[c,m] | X[c,m] }

				% headings
				Thing       &   Structure           &   Melting Point / \si{\celsius}   &   Boiling Point / \si{\celsius}   \\ \hline
				\ch{Na\Cl}  &   ionic               &   \num{801}                       &   \num{1413}                      \\
				\ch{Mg\Cl2} &   ionic               &   \num{714}                       &   \num{1412}                      \\
				\ch{\Al\Cl3}&   discrete covalent   &   ---                             &   \num{180} (sublimes)            \\
				\ch{Si\Cl4} &   discrete covalent   &   \num{-70}                       &   \num{58}                        \\
				\ch{P\Cl3}  &   discrete covalent   &   \num{-112}                      &   \num{76}                        \\
				\ch{P\Cl5}  &   discrete covalent   &   ---                             &   \num{160} (sublimes)            \\
				\ch{SO2}    &   discrete covalent   &   \num{-72}                       &   \num{-10}                       \\
				\ch{SO3}    &   discrete covalent   &   \num{16.9}                      &   \num{45}                        \\

			\end{tabu}\end{center}
			\end{table}\vspace{-1em}



			\subsubsection{Aluminium Chloride}

				In the solid state (at room temperature), it is a very pale solid, and exists as an ionic lattice with a \itl{very high}
				degree of covalent character.

				However, at around \SI{180}{\celsius}, the heat causes the ionic lattice to expand, dissolving the ionic character of the
				compound. Thus, \ch{\Al\Cl3} sublimes immediately into a gas, since the electrostatic forces have now disappeared,
				leaving only id-id interactions for intermolecular bonding.

				A caveat, however, is that the \ch{\Al} atom in \ch{\Al\Cl3} is \itl{electron deficient}, having only 3 electron pairs. Thus,
				it tends to form dative bonds with another molecule of \ch{\Al\Cl3}, creating a dimer of \ch{\Al2\Cl6}.

				% note: we're drawing this unconventionally because apparently we cannot use the -Stealth arrow in the ?[] hook syntax
				% so we draw a bond from Al to Cl, but with Stealth-, which is in fact a reverse arrow.

				\diagram[1.0]{
					\schemestart[0,1.5,thick]
						\chemfig{\Al?(-[:135]\Cl)(-[:225]\Al)(-[:315]\Cl-[:45]\Al(-[:315]\Cl)(-[:45]\Cl)(-[:135]\Cl?))}
					\schemestop
				}[The dimer \ch{\Al2\Cl6}.]

				\pagebreak
				\ch{\Al\Cl3} and \ch{\Al2\Cl6} actually exist in an equilibrium:

				\txtreactioneqn{
					\ch{2 \Al\Cl3 \stG} \arrow{<=>} \ch{\Al2\Cl6 \stG}
				}

				At higher temperatures, the equilibrium shifts to the left, and \ch{\Al\Cl3} exists predominantly as discrete molecules, with
				no dimerisation.

			% end subsubsection
		% end subsection


		\subsection{Sodium Chloride}

			Sodium chloride, \ch{Na\Cl}, should be familiar; it unremarkably dissolves in water to give \ch{Na+ \stAq} and
			\ch{\Cl- \stAq} ions:

			\txtreactioneqn[1.5]{
				\ch{Na\Cl \stS} \arrow{->[\tinytext{water}]} \ch{Na+ \stAq} + \ch{\Cl- \stAq}
			}

			The \pH{} of the solution is \num{7} --- neutral.

		% end subsection



		\subsection{Magnesium Chloride}

			Magnesium chloride, \ch{Mg\Cl2}, dissolves in water as well to give \ch{Mg^2+ \stAq} and \ch{\Cl- \stAq} ions. Note that because
			of magnesium's somewhat high charge density, it forms a hydrated aqua complex in water:

			\txtreactioneqn{
				\ch{Mg\Cl2 \stS} + \ch{6 H2O \stL} \arrow \ch{[Mg(H2O)6]^2+ \stAq} + \ch{2 \Cl- \stAq}
			}

			In a mechanism similar to that of the hydrolysis of aqua complex ions of metals with high charge density, magnesium is strong
			enough to polarise the \ch{O-H} bond slightly, forming a slightly acidic solution:

			\txtreactioneqn{
				\ch{[Mg(H2O)6]^2+ \stAq} + \ch{H2O} \arrow{<=>} \ch{[Mg(H2O)5OH]+ \stAq} + \ch{H3O+ \stAq}
			}

			Note that the polarising power is much weaker than, for instance, \ch{\Al^3+} or \ch{Cr^3+}, so the \pH{} of such a
			solution is only around \num{6.5}.

		% end subsection



		\subsection{Aluminium Chloride}

			The reactions of \ch{\Al\Cl3} with water differ mainly based on the amount of water or base used.

			\subsubsection{Limited Water}

				When \itl{a few drops} of water are added to \ch{\Al\Cl3}, white fumes of \ch{H\Cl} are formed, leaving behind a white
				precipitate of \ch{\Al(OH)3}. The aluminium hydroxide formed is insoluble in water.

				\txtreactioneqn{
					\ch{\Al\Cl3 \stS} + \ch{3 H2O \stL} \arrow \ch{\Al(OH)3 \stS} + \ch{3 H\Cl \stG}
				}

			% end subsubsection

			\pagebreak
			\subsubsection{Excess Water}

				In excess water, \ch{\Al\Cl3} instead forms the hydrated aluminium complex ion, and chloride ions:

				\txtreactioneqn{
					\ch{\Al\Cl3 \stS} + \ch{6 H2O \stL} \arrow \ch{[\Al(H2O)6]^3+ \stAq} + \ch{3 \Cl- \stAq}
				}

				As covered previously, this hydrated aluminium ion can undergo hydrolysis to give an acidic solution:

				\txtreactioneqn{
					\ch{[\Al(H2O)6]^3+ \stAq}\hspace{2mm} + \hspace{2mm}\ch{H2O \stL}
					\arrow{<=>}
					\ch{[\Al(H2O)5OH]^2+ \stAq}\hspace{2mm} + \hspace{2mm}\ch{H3O+ \stAq}
				}

				The extent of hydrolysis is much greater than for \ch{[Mg(H2O)6]^2+}, so the \pH{} is typically around \num{3}.

			% end subsubsection



			\subsubsection{Limited \texorpdfstring{\ch{OH-}}{OH⁻}}

				When a limited amount of base (\ch{OH-}) is added, \ch{[\Al(H2O)6]^3+ \stAq} behaves as a triprotic acid, being
				able to release 3 \ch{H+ \stAq} ions before being depleted and forming a precipitate of \ch{[\Al(H2O)3(OH)3] \stS}.

				\txtreactioneqn{
					\ch{[\Al(H2O)6]^3+ \stAq}\hspace{2mm} + \hspace{2mm}\ch{H2O \stL}
					\arrow{<=>}
					\ch{[\Al(H2O)5OH]^2+ \stAq}\hspace{2mm} + \hspace{2mm}\ch{H3O+ \stAq}
					\arrow(@c1.east--.east){0}[-90,.4]
					\ch{[\Al(H2O)5OH]^2+ \stAq}\hspace{2mm} + \hspace{2mm}\ch{H2O \stL}
					\arrow{<=>}
					\ch{[\Al(H2O)4(OH)2]^+ \stAq}\hspace{2mm} + \hspace{2mm}\ch{H3O+ \stAq}
					\arrow(@c3.east--.east){0}[-90,.4]
					\ch{[\Al(H2O)4(OH)2]^+ \stAq}\hspace{2mm} + \hspace{2mm}\ch{H2O \stL}
					\arrow{<=>}
					\ch{[\Al(H2O)3(OH)3] \stS}\hspace{2mm} + \hspace{2mm}\ch{H3O+ \stAq}
				}

				A solution of \ch{[\Al(H2O)6]^3+ \stAq} is usually acidic enough to react with carbonate ions to liberate carbon dioxide
				gas (which, remember, neither alcohols nor phenols can do), forming the insoluble precipitate.

			% end subsubsection


			\subsubsection{Excess \texorpdfstring{\ch{OH-}}{OH⁻}}

				When faced with excess amounts of base, the white precipitate of \ch{[\Al(H2O)3(OH)3] \stS} will appear to dissolve, instead
				forming the aluminate complex ion \ch{[\Al(OH)4]- \stAq}.

				\txtreactioneqn{
					\ch{[\Al(H2O)3(OH)3] \stS} + \ch{OH- \stAq} \arrow{<=>} \ch{[\Al(OH)4]- \stAq} + \ch{3 H2O \stL}
				}

				Note that, since this is an equilibrium reaction, the concentration of \ch{OH-} must be high enough to move the position of
				equilibrium to the right, to favour the formation of the soluble aluminate ion.

			% end subsubsection

		% end subsection



		\pagebreak
		\subsection{Silicon Chloride}

			Silicon chloride \itl{readily} and \itl{violently} hydrolyses in water completely, forming \ch{H\Cl}, which can either effervesce
			away as gaseous hydrogen chloride, or dissolve in the water to form a strongly acidic solution of \pH{} \num{1}.

			\txtreactioneqn{
				\ch{Si\Cl4 \stL} + \ch{2 H2O \stL} \arrow \ch{SiO2 * 2 H2O \stS} + \ch{4 H\Cl \stAq}
			}

			\ch{SiO2 * 2 H2O \stS} is a hydrated form of silicon dioxide, but it is still in the solid state.

		% end subsection



		\subsection{Phosphorus Chlorides}

			Both phosphorus chlorides, \ch{P\Cl3} and \ch{P\Cl5}, completely hydrolyse in water in a similar fashion to silicon chloride,
			forming \ch{H\Cl} and hence a strongly acidic solution.

			\txtreactioneqn{
				\ch{P\Cl3 \stL} + \ch{3 H2O \stL} \arrow \ch{H3PO3 \stAq} + \ch{3 H\Cl \stAq}
				\arrow(@c1.east--.east){0}[-90,.4]
				\ch{P\Cl5 \stS} + \ch{4 H2O \stL} \arrow \ch{H3PO4 \stAq} + \ch{4 H\Cl \stAq}
			}

			Note that, of course, the contributions of \ch{H3PO3} and \ch{H3PO4} to the \pH{} of the solution will not be
			forgotten, but they are quite insignificant in the face of \ch{H\Cl}.

			Furthermore, on dropwise addition of water with \ch{P\Cl5}, it tends to form \ch{PO\Cl3}, phosphorus oxychloride, instead.

			\txtreactioneqn{
				\ch{P\Cl5 \stS} + \ch{H2O \stL} \arrow \ch{PO\Cl3 \stL} + \ch{2 H\Cl \stG}
			}

		% end subsection

	% end section

% end part




















