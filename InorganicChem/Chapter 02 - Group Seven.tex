% Chapter 02 - Group Seven.tex
% Copyright (c) 2014 - 2016, zhiayang@gmail.com
% Licensed under the Apache License Version 2.0.


\newcommand {\rsv}{\bld{\romannum{VII}}}

\pagebreak
\part{Group \romannum{VII}}

	\section{Overview}

		As should be familiar by now, group \rsv{} elements are the \itl{halogens}, and are the most reactive
		group of non-metals. They are reactive to the point of not occurring as free elements (\ch{\Cl2}, \ch{F2})
		in nature, almost always being found as part of other compounds in an ionic state.

	% end section

	\section{Physical Properties}

		\subsection{Electronic Configuration}

			Halogens have a \itl{n}p\sps{5} valence shell, and so tend to form \ch{X-} ions.


			\tabto{0mm}\ch{F}:\tabto{10mm}1s\sps{2}2s\sps{2}2p\sps{5}
			\tabto{0mm}\ch{\Cl}:\tabto{10mm}1s\sps{2}2s\sps{2}2p\sps{6}3s\sps{2}3p\sps{5}
			\tabto{0mm}\ch{Br}:\tabto{10mm}1s\sps{2}2s\sps{2}2p\sps{6}3s\sps{2}3p\sps{6}4s\sps{2}4p\sps{5}
			\tabto{0mm}\ch{I}:\tabto{10mm}1s\sps{2}2s\sps{2}2p\sps{6}3s\sps{2}3p\sps{6}3d\sps{10}4s\sps{2}4p\sps{6}5s\sps{2}5p\sps{5}

		% end subsection


		\subsection{Various Radii}

			All the halogens exist as diatomic molecules with a covalent bond between the two atoms. Both the ionic radii, and the covalent
			radii, which is half the distance between the two atomic centres in \ch{X2}, increase down the group as expected.

			The ionic radius for any halogen is always greater than its corresponding covalent radius, since the anion has one more electron,
			leading to greater interelectronic repulsion and hence a larger radius.

		% end subsection



		\pagebreak
		\subsection{Bond Energy}

			Generally speaking the bond energies of the halogens decrease down the group, as the atomic radius increases and hence the
			orbital overlap decreases; this leads to a lengthening of the covalent bond, which makes it weaker.

			\begin{center}\begin{table}[htb]\renewcommand{\arraystretch}{1.5}
			\begin{tabu} to \textwidth {X[c,m] | X[c,m]}

				Element		&	\ch{X-X} bond energy/\si{\kilo\joule\per\mole}\\	\hline
				\ch{F}		&	\num{159}										\\	\hline
				\ch{\Cl}	&	\num{242}										\\	\hline
				\ch{Br}		&	\num{193}										\\	\hline
				\ch{I}		&	\num{151}										\\	\hline

			\end{tabu}
			\end{table}\end{center}\vspace{-10mm}

			The discrepancy in the \ch{F-F} bond energy is due to electrostatic repulsion between the lone pairs on each \ch{F} atom, which
			naturally would weaken the bond. This repulsion is due to the length of the bond and the size of the atoms.

		% end subsection


		\subsection{Electron Things}

			\subsubsection{First Ionisation Energy}

				Although the ionisation energy for halogens is typically not of great interest, it is still useful to know their trend. The trend
				is completely regular, with the ionisation energy decreasing down the group.

				This is due to the increasing distance between the valence shell and the nucleus, increasing the shielding effect and thus
				reducing the energy needed to remove one valence electron.

			% end subsubsection


			\subsubsection{First Electron Affinity}

				Of more interest is the first electron affinity, which measures the enthalpy change when adding one mole of electrons to one
				mole of gaseous atoms, forming one mole of singly-charged anions (if you do not recall).

				In short, it is the enthalpy change to in creating \ch{X-} from \ch{X} by adding \ch{e-}.

				\begin{center}\begin{table}[htb]\renewcommand{\arraystretch}{1.5}
				\begin{tabu} to \textwidth {X[c,m] | X[c,m]}

					Element		&	First Electron Affinity/\si{\kilo\joule\per\mole}	\\	\hline
					\ch{F}		&	\num{-328}											\\	\hline
					\ch{\Cl}	&	\num{-349}											\\	\hline
					\ch{Br}		&	\num{-325}											\\	\hline
					\ch{I}		&	\num{-295}											\\	\hline

				\end{tabu}
				\end{table}\end{center}\vspace{-10mm}


				The general trend is that the electron affinity decreases down the group (becomes less exothermic) since the electrostatic
				attraction between the new electron to be added and the nucleus decreases with increasing atomic radius.

				Again, the anomaly for fluorine is due to its small size, and electron lone pairs --- it takes slightly more energy to fit
				a new electron into the already crowded area around the fluorine atom.

			% end subsubsection


			\subsubsection{Electronegativity}

				The trend for electronegativity is fairly regular in decreasing down the group; fluorine is the most electronegative element
				(ever), and iodine is the least electronegative in the group.

				The decrease in electronegativity is explained by the increasing distance between the bond pair of electrons and the nucleus,
				which reduces the electrostatic force attracting the electrons to the nucleus.

			% end subsubsection

		% end subsection


		\subsection{Melting and Boiling Points}

			The melting and boiling points of the halogens are largely determined by their atomic size, and hence number of electrons. Since
			they exist as covalent diatomic molecules, the only intermolecular attractions are instantaneous dipole-induced dipole interactions.

			Naturally, as non-polar molecules the polarisability depends only on the number of electrons. As the number of electrons increase,
			then so does the intermolecular force, and hence so does the melting and boiling point.

			TL;DR: melting and boiling points increase down the group. \ch{F2} and \ch{\Cl2} are gasses at room temperature, \ch{Br2} is a
			liquid, and \ch{I2} is a solid.

		% end subsection


		\subsection{Solubility}

			\subsubsection{In Polar Solvents (Water)}

				Generally, all the halogens except fluorine dissolve in water to \itl{some} extent. Fluorine reacts explosively due to its high
				reduction potential, and oxidises water.

				Chlorine and bromine react to form a number of ionic products in water in a disproportionation equilibrium, which will be discussed
				later. Iodine neither dissolves in water nor disproportionates (due to its low reduction potential), but can form a soluble complex
				\ch{I3-} in the presence of \ch{I-} ions:

				\txtdiagram{
					\schemestart[0,1.5,thick]
						\ch{I2 \stS} + \ch{I- \stAq}\arrow{<=>}\ch{I3- \stAq}
					\schemestop
				}{\vspace{-1.5em}}

			% end subsubsection


			\pagebreak
			\subsubsection{In Non-polar Solvents}

				It should be no surprise that the solubility of the halogens in non-polar solvents is simply a function of their polarisability,
				and as with their melting and boiling points, said solubility increases down the group.

			% end subsubsection

		% end subsection

		\subsection{Colours}

			Generally speaking the colour of the halogens increase in intensity down the group.

			\begin{center}\begin{table}[htb]\renewcommand{\arraystretch}{1.5}
			\begin{tabu} to \textwidth {X[c,m] | X[c,m] | X[c,m] | X[c,m] | X[c,m]}

				Element		&	Gaseous			&	Liquid/Solid	&	Aqueous					&	Non-polar			\\	\hline
				\ch{F}		&	pale yellow		&		-			&		- goes boom -		&	very pale yellow	\\	\hline
				\ch{\Cl}	&	yellow-green	&		-			&	very pale yellow		&	very pale green		\\	\hline
				\ch{Br}		&	reddish-brown	&	reddish-brown	&	yellow/orange			&	orange/reddish-brown\\	\hline
				\ch{I}		&	dark purple		&	shiny black		&	pale yellow				&	violet				\\	\hline

			\end{tabu}
			\end{table}\end{center}\vspace{-10mm}

			Note that when concentrated in aqueous solution, the colour intensity naturally increases. Also, the \ch{I3-} complex is brown
			in colour when aqueous.

		% end subsection

	% end section



	\pagebreak
	\section{Chemical Properties of the Elements}

		\subsection{Oxidation State}

			The most stable oxidation state for the halogens is of course -1, due to having a filled valence shell. However, chlorine, bromine
			and iodine have \itl{low-lying} d-orbitals (in the number 3, 4, and 5 quantum shells respectively) that can accommodate more
			electrons and form more bonds. Thus, they can exist in the +1, +3, +5 and +7 oxidation states.

			Examples of compounds with the halogen in positive oxidation states include \ch{H\Cl{}O}, \ch{\Cl{}F}, \ch{H\Cl{}O2}, \ch{H\Cl{}O3},
			and \ch{H\Cl{}O4}.

		% end subsection


		\subsection{Oxidising Power}

			As explained above, the electronegativity of the elements decrease down the group, and since they describe basically the same
			fundamental electron behaviour, it is no surprise that the reduction potential of the halogens decrease down the group as well.

			\begin{center}\begin{table}[htb]\renewcommand{\arraystretch}{1.5}
			\begin{tabu} to \textwidth {X[c,m] | X[c,m]}

				% headings
				Reaction					&	$E^{\stdst} / V$	\\	\hline
				\ch{F2 + 2 e- -> 2 F-}		&	\num[retain-explicit-plus]{+2.80}	\\	\hline
				\ch{\Cl2 + 2 e- -> 2 \Cl-}	&	\num[retain-explicit-plus]{+1.36}	\\	\hline
				\ch{Br2 + 2 e- -> 2 Br-}	&	\num[retain-explicit-plus]{+1.07}	\\	\hline
				\ch{I2 + 2 e- -> 2 I-}		&	\num[retain-explicit-plus]{+0.54}	\\	\hline

			\end{tabu}
			\end{table}\end{center}\vspace{-10mm}

			This means that oxidising power decreases down the group as well.

		% end subsection


		\subsection{Halogen Displacement}

			A halogen that is more reactive (ie. has a higher reduction potential) can oxidise a less reactive halogen and itself get reduced,
			thereby \itl{displacing} the less reactive halogen.

			For example, with \ch{\Cl2} displacing \ch{Br-}:

			\txtdiagram{
				\schemestart[0,1.0,thick]
					\ch{\Cl2 \stG} + \ch{2 e-}\arrow\ch{2 \Cl-}			\hspace{25mm}$E^{\stdst} = +1.36V$
					\arrow(@c1.south east--.north east){0}[-90,.20]
					\ch{2 Br- \stAq}\arrow\ch{Br2 \stL} + \ch{2 e-}		\hspace{10mm}$E^{\stdst}_{ox} = -1.07V$
				\schemestop
			}{\vspace{-1em}}

			Since the \Ecell{} value ($1.36 + (-1.07) = +0.29V$) is positive, the reaction is feasible. Naturally were the conditions reversed,
			nothing would happen; bromine would not displace chloride.


		% end subsection



		\subsection{Reaction with Water}

			Fluorine reacts with water to form hydrogen fluoride and a mixture of oxygen and ozone, explosively --- even when cold.

			\txtdiagram{
				\schemestart[0,1.0,thick]
					\ch{2 F2 \stG} + \ch{2 H2O \stL}\arrow\ch{4 HF \stAq} + \ch{O2 \stG}
					\arrow(@c1.south east--.north east){0}[-90,.20]
					\ch{3 F2 \stG} + \ch{3 H2O \stL}\arrow\ch{6 HF \stAq} + \ch{O3 \stG}
				\schemestop
			}{\vspace{-1em}}


			Chlorine and bromine, on the other hand, undergo a disproportionation reaction in water, in an equilibrium reaction:

			\txtdiagram{
				\schemestart[0,1.0,thick]
					\ch{\Cl2 \stG} + \ch{H2O \stL}\arrow{<=>}\ch{H\Cl \stAq} + \ch{HO\Cl \stAq}
					\arrow(@c1.south east--.north east){0}[-90,.20]
					\ch{Br2 \stG} + \ch{H2O \stL}\arrow{<=>}\ch{HBr \stAq} + \ch{HOBr \stAq}
				\schemestop
			}{\vspace{-1em}}

			Of note is that \ch{HO\Cl} is the primary agent responsible for the bleaching mechanism in chlorine water. Also, the position
			of equilibrium for bromine's disproportionation is further to the left than chlorine, aka. disproportionation occurs to
			a \itl{smaller} extent.


			Iodine does not react with pure water.

		% end subsection


		\subsection{Reaction with Aqueous \ch{Fe^2+} Ions}

			A simple calculation of the \Ecell{} values for the reaction would show that only chlorine and bromine are able to oxidise
			\ch{Fe^2+ \stAq} to \ch{Fe^3+ \stAq} (producing the relevant colour change). Iodine does not oxidise \ch{Fe^2+}.

			Fluorine would first explode, so it cannot react with the \itl{aqueous} solution.

		% end subsection




		\pagebreak
		\subsection{Reaction with Thiosulphate (\ch{S2O3^2-}) Ions}

			Thiosulphate ions can be oxidised by chlorine and bromine to give sulphate ions:

			\txtdiagram{
				\schemestart[0,1.0,thick]
					\ch{4 \Cl2} + \ch{S2O3^2-} + \ch{5 H2O}\arrow\ch{8 \Cl-} + \ch{2 SO4^2-} + \ch{10 OH-}
					\arrow(@c1.south east--.north east){0}[-90,.20]
					\ch{4 Br2} + \ch{S2O3^2-} + \ch{5 H2O}\arrow\ch{8 Br-} + \ch{2 SO4^2-} + \ch{10 OH-}
				\schemestop
			}{\vspace{-1em}}

			Above, sulphur is oxidised from a +2 oxidation state in thiosulphate to a +6 state in sulphate. However, iodine, given its weaker
			oxidation strength, can only oxidise thiosulphate to a +2.5 (yes, that's right) oxidation state, forming \ch{S4O6^2-}.

			\txtdiagram{
				\schemestart[0,1.0,thick]
					\c{I2 \stAq} + \ch{2 S2O3^2- \stAq}\arrow\ch{2 I- \stAq} + \ch{S4O6^2- \stAq}
				\schemestop
			}{\vspace{-1em}}

		% end subsection



		\subsection{Disproportionation in Alkaline Solutions}

			Chlorine, bromine and iodine disproportionate readily and non-reversibly in alkaline solutions to give various ions.
			Under cold conditions, both \ch{\Cl2} and \ch{Br2} disproportionate to form the halogen in the +1 oxidation state.

			\txtdiagram{
				\schemestart[0,1.0,thick]
					\ch{\Cl2 \stAq} + \ch{2 OH- \stAq}\arrow\ch{\Cl- \stAq} + \ch{\Cl{}O- \stAq} + \ch{H2O \stL}
					\arrow(@c1.south east--.north east){0}[-90,.20]
					\ch{Br2 \stAq} + \ch{2 OH- \stAq}\arrow\ch{Br -\stAq} + \ch{BrO- \stAq} + \ch{H2O \stL}
				\schemestop
			}{\vspace{-1em}}

			The maximum temperature for the formation of \ch{\Cl{}O-} is around room temperature, while that for \ch{BrO-} is around
			\SI{0}{\celsius}. When heated, the halogen enters the +5 oxidation state instead of the +1 oxidation state.

			\txtdiagram{
				\schemestart[0,1.0,thick]
					\ch{3 \Cl2 \stAq} + \ch{6 OH- \stAq}\arrow\ch{5 \Cl- \stAq} + \ch{\Cl{}O3- \stAq} + \ch{3 H2O \stL}
					\arrow(@c1.south east--.north east){0}[-90,.20]
					\ch{3 Br2 \stAq} + \ch{6 OH- \stAq}\arrow\ch{5 Br -\stAq} + \ch{BrO3- \stAq} + \ch{3 H2O \stL}
				\schemestop
			}{\vspace{-1em}}


			Iodine being iodine, it enters the +5 oxidation state to form \ch{IO3-} regardless of temperature.

			\txtdiagram{
				\schemestart[0,1.0,thick]
					\ch{3 I2 \stAq} + \ch{6 OH- \stAq}\arrow\ch{5 I -\stAq} + \ch{IO3- \stAq} + \ch{3 H2O \stL}
				\schemestop
			}{\vspace{-1em}}



		% end subsection



		\subsection{Reaction with Metals}

			Given that the halogens are oxidising agents, they can easily oxidise most reactive metals to give ionic compounds. Fluorine
			exceptionally can react directly with all metals, including traditionally unreactive ones such as gold and silver.

			Unsurprisingly the vigour of the reaction reduces down the group.

			\subsubsection{Reaction with Sodium}

				All 4 halogens can react with sodium, which results in the metal burning with a \itl{bright orange} flame, forming a solid
				ionic product \ch{NaX}.

				\txtdiagram{
					\schemestart[0,1.0,thick]
						\ch{X2 \stG} + \ch{2 Na \stS}\arrow\ch{2 NaX \stS}
					\schemestop
				}{\vspace{-1em}}

			% end subsubsection


			\subsubsection{Reaction with Iron}

				The first 3 halogens, fluorine, chlorine, and bromine, can oxidise iron metal from the 0 oxidation state directly to the +3
				oxidation state.

				\txtdiagram{
					\schemestart[0,1.0,thick]
						\ch{3 X2 \stG} + \ch{2 Fe \stS}\arrow\ch{2 FeX3 \stS}
					\schemestop
				}{\vspace{-1em}}

				Naturally, iodine being what it is, can only oxidise iron to the +2 oxidation state.

				\txtdiagram{
					\schemestart[0,1.0,thick]
						\ch{I2 \stG} + \ch{Fe \stS}\arrow\ch{FeI2 \stS}
					\schemestop
				}{\vspace{-1em}}

			% end subsubsection

		% end subsection


		\pagebreak
		\subsection{Reaction with Phosphorus}

			All of the halogens can react vigorously and irreversibly with solid phosphorus at room temperature to give phosphorus trihalides.

			\txtdiagram{
				\schemestart[0,1.0,thick]
					\ch{3 X2 \stG} + \ch{2 P \stS}\arrow\ch{2 PX3 \stL}
				\schemestop
			}{\vspace{-1em}}


			In the case of chlorine and bromine, there exists an equilibrium in excess halogen that results in the formation of \ch{PX5}.

			\txtdiagram{
				\schemestart[0,1.0,thick]
					\ch{PX3 \stL} + \ch{X2 \stG}\arrow{<=>}\ch{PX5 \stS}
				\schemestop
			}{\vspace{-1em}}

			These compound should be familiar as chlorinating and brominating agents in the nucleophilic substitution of \ch{OH} groups in
			organic chemistry.

		% end subsection



		\subsection{Reaction with Hydrogen}

			The halogens react with hydrogen gas with \itl{decreasing vigour} down the group to form the respective hydrogen halide.

			\txtdiagram{
				\schemestart[0,1.0,thick]
					\ch{X2 \stG} + \ch{H2 \stG}\arrow\ch{2 HX \stS}
				\schemestop
			}{\vspace{-1em}}

			As usual, the reaction with fluorine is explosive even at \SI{-200}{\celsius} and in the dark, while chlorine reacts explosively
			in sunlight and rapidly otherwise. Bromine reacts slowly, and typically requires heating at \SI{300}{\celsius} with a \ch{Pt}
			catalyst.

			As usual, iodine's reaction is reversible and occurs slowly; the reaction is typically incomplete and a mixture of \ch{HX} and \ch{I2}
			is obtained; this is due to the positive enthalpy change of formation.

		% end subsection

	% end section




	\pagebreak
	\section{Chemical Properties of Hydrogen Halides}

		\subsection{Physical Properties}

			The hydrogen halides are polar, diatomic molecules. The melting and boiling points generally increase down the group, due to
			the larger electron cloud, which at larger sizes is the main determinant of the intermolecular attraction force.

			\begin{center}\begin{table}[htb]\renewcommand{\arraystretch}{1.5}
			\begin{tabu} to \textwidth {X[c,m] | X[c,m] | X[c,m]}

				% headings
				Thing		&	Melting Point / \si{\celsius}	&	Boiling Point / \si{\celsius}	\\	\hline
				\ch{HF}		&	\num{-80}						&	\num{20}						\\	\hline
				\ch{H\Cl}	&	\num{-115}						&	\num{-85}						\\	\hline
				\ch{HBr}	&	\num{-89}						&	\num{-67}						\\	\hline
				\ch{HI}		&	\num{-51}						&	\num{-35}						\\	\hline

			\end{tabu}
			\end{table}\end{center}\vspace{-10mm}

			The large discrepancy with \ch{HF} is due to its ability to form \itl{hydrogen bonds}, which are stronger than the pd-pd interactions
			and of course id-id interactions of the other halogens.

			Note that other than electronegativity, the size of the negative atom also contributes to the ability to form hydrogen bonds, which
			is why \ch{H\Cl} is not considered to be able to form them even though it has similar electronegativity to nitrogen and oxygen.


		% end subsection



		\subsection{Thermal Stability}

			The weaker hydrogen halides can be decomposed back into \ch{X2} and \ch{H2} when heated. Naturally the temperature that is required
			for this decomposition decreases down the group, due to the decreasing strength of the \ch{H-X} bond.
			\ch{HF} and \ch{H\Cl} cannot even be decomposed when heated strongly.

			\begin{bulletlist}
				& \ch{HF} and \ch{H\Cl} do not decompose
				& \ch{HBr} decomposes on string heating, yielding reddish-brown fumes of \ch{Br2 \stG}
				& \ch{HI} decomposes simply by contact with a hot metal rod, yielding violet fumes of \ch{I2 \stG}.
			\end{bulletlist}

		% end subsection


		\subsection{Acid Strength}

			\ch{H\Cl}, \ch{HBr} and \ch{HI} are strong acids that dissociate fully in water to give \ch{H+ \stAq} ions.

			The acid strength \itl{increases} down the group, since the \ch{H-X} bond becomes weaker, which allows for easier
			dissociation. However, \ch{HF} notably behaves as a weak acid due to incomplete dissociation which can be attributed to the
			high strength of the \ch{H-F} bond.

		% end subsection

	% end section

	\section{Chemical Properties of Halide Ions}

		\subsection{Precipitation Reactions}

			Silver ions react with halide ions in an equilibrium reaction to form insoluble \ch{AgX}.

			\txtdiagram{
				\schemestart[0,1.0,thick]
					\ch{Ag+ \stAq} + \ch{X- \stAq}\arrow\ch{AgX \stS}
				\schemestop
			}{\vspace{-1em}}

			As discussed in previous chapters, the solubility of the \ch{AgX} salt depends on the concentration of \ch{Ag+} and \ch{X-} ions
			in the solution; recall that ionic product = $[Ag^{+}][I^{-}]$. If IP is less than \Ksp{}, then the salt will dissolve. Otherwise,
			the salt will precipitate out of the solution.

			Upon adding \ch{NH3 \stAq}, a complex ion is formed with the silver ions.

			\txtdiagram{
				\schemestart[0,1.0,thick]
					\ch{Ag+ \stAq} + \ch{NH3 \stAq}\arrow\ch{[Ag(NH3)2]+ \stAq}
				\schemestop
			}{\vspace{-1em}}

			The formation of this complex reduces the concentration of \ch{Ag+} in the solution, thus decreasing the value of IP in the original
			reaction. When the ionic product becomes smaller than the \Ksp{}, the precipitate will dissolve.

			For \ch{Ag\Cl}, dilute \ch{NH3 \stAq} is enough to lower the ionic product below \Ksp{}, and allow the salt to dissolve. For
			\ch{AgBr}, concentrated \ch{NH3 \stAq} must be used to lower the concentration of \ch{Ag+} enough such that the ionic product
			becomes smaller than the \Ksp{} of \ch{AgBr}.

			For \ch{AgI}, the \Ksp{} value is so small that no amount of \ch{NH3 \stAq} will lower the concentration of \ch{Ag+} enough for
			the salt to dissolve.


			\begin{center}\begin{table}[htb]\renewcommand{\arraystretch}{1.5}
			\begin{tabu} to \textwidth {X[c,m] | X[c,m] | X[c,m] | X[c,m]}

				% headings
				Ion		&	Colour of \ch{AgX}	&	Required \ch{NH3 \stAq}	&	\Ksp{}/\si{\mole\squared\per\cubic\deci\metre\tothe{6}}	\\	\hline
				\ch{\Cl-}	&	white			&	dilute					&	\num{1.6e-10}			\\	\hline
				\ch{Br-}	&	cream			&	concentrated			&	\num{7.7e-13}			\\	\hline
				\ch{I-}		&	yellow			&	nothing works			&	\num{8.3e-17}			\\	\hline

			\end{tabu}
			\end{table}\end{center}\vspace{-10mm}

		% end subsection


		\pagebreak
		\subsection{Reaction with Lead (II) Ions}

			As you might know, lead halides are insoluble. Only lead iodide (\ch{PbI2}) is yellow, while the precipitates of the other halides
			are white.

			\txtdiagram{
				\schemestart[0,1.0,thick]
					\ch{Pb^2+ \stAq} + \ch{2 X- \stAq}\arrow\ch{PbX2 \stS}
				\schemestop
			}{\vspace{-1em}}

			Addition of excess \ch{X-} will cause the precipitate to re-dissolve, since a soluble complex ion, \ch{[PbX4]^2-} is formed.

			\txtdiagram{
				\schemestart[0,1.0,thick]
					\ch{Pb^2+ \stAq} + \ch{4 X- \stAq}\arrow\ch{[PbX4]^2- \stAq}
				\schemestop
			}{\vspace{-1em}}



		% end subsection


		\subsection{Reaction with Concentrated \ch{H2SO4}}

			Generally speaking, \ch{X-} ions can act as a Brønsted base, accepting \ch{H+} ions from acids. When concentrated sulphuric acid
			is added to solid halide salts, this indeed occurs.

			\txtdiagram{
				\schemestart[0,1.0,thick]
					\ch{NaX \stS} + \ch{H2SO4 \stL}\arrow\ch{HX \stG} + \ch{NaHSO4 \stS}
				\schemestop
			}{\vspace{-1em}}

			However, \ch{H2SO4} is also a strong oxidising agent; thus, it is able to oxidise the \ch{HBr} and \ch{HI} formed in the acid-base
			reaction above. Note that it is insufficiently strong to oxidise \ch{HF} and \ch{H\Cl}.


			\txtdiagram{
				\schemestart[0,1.0,thick]
					\ch{NaBr \stS} + \ch{H2SO4 \stL}\arrow\ch{HBr \stG} + \ch{NaHSO4 \stS}
					\arrow(@c1.south east--.north east){0}[-90,.20]
					\ch{2 HBr \stG} + \ch{H2SO4 \stL}\arrow\ch{Br2 \stG} + \ch{SO2 \stG} + \ch{2 H2O \stL}
				\schemestop
			}{\vspace{-1em}}

			In this reaction, most of the \ch{HBr} remains untouched, and some \ch{Br2 \stG} is produced, leading to the observation of
			slight reddish-brown fumes.


			\pagebreak
			Below, \ch{H2SO4} continues to oxidise \ch{HI} in a complex and frankly ridiculous manner.


			\txtdiagram{
				\schemestart[0,1.0,thick]
					\ch{NaI \stS} + \ch{H2SO4 \stL}\arrow\ch{HI \stG} + \ch{NaHSO4 \stS}
					\arrow(@c1.south east--.north east){0}[-90,.20]
					\ch{2 HI \stG} + \ch{H2SO4 \stL}\arrow\ch{I2 \stG} + \ch{SO2 \stG} + \ch{2 H2O \stL}
					\arrow(@c3.south east--.north east){0}[-90,.20]
					\ch{6 HI \stG} + \ch{H2SO4 \stL}\arrow\ch{3 I2 \stG} + \ch{S \stS} + \ch{4 H2O \stL}
					\arrow(@c5.south east--.north east){0}[-90,.20]
					\ch{8 HI \stG} + \ch{H2SO4 \stL}\arrow\ch{4 I2 \stG} + \ch{H2S \stG} + \ch{4 H2O \stL}
				\schemestop
			}{\vspace{-1em}}

			Due to the ease of oxidising \ch{HI}, \ch{H2SO4} is itself reduced, with sulphur going from a +6 oxidation state to a -2 state
			in \ch{H2S}. As a sidenote, it is \ch{H2S} that smells like rotten eggs. Furthermore, \ch{SO2} is also a pungent gas. This is
			a smelly reaction.

			Most of the \ch{HI} is oxidised, thus there will be a large output of violet fumes, with some white fumes of \ch{HI}.



			\subsubsection{Reaction in the Presence of \ch{MnO2}}

				Manganese dioxide is a stronger oxidising agent than \ch{H2SO4}, and thus it can serve to oxidise the \ch{H\Cl} that sulphuric
				acid cannot.

				\txtdiagram{
					\schemestart[0,1.0,thick]
						\ch{Na\Cl \stS} + \ch{H2SO4 \stL}\arrow\ch{H\Cl \stG} + \ch{NaHSO4 \stS}
						\arrow(@c1.south east--.north east){0}[-90,.20]
						\ch{4 H\Cl \stG} + \ch{MnO2 \stL}\arrow\ch{\Cl2 \stG} + \ch{Mn\Cl2 \stG} + \ch{2 H2O \stL}
					\schemestop
				}{\vspace{-1em}}

			% end subsubsection


			\subsubsection{\ch{H3PO4} as a Weaker Oxidising Agent}

				Given that sulphuric acid will obliterate \ch{HBr} and \ch{HI}, a less powerful oxidising agent can be used to obtain these
				instead of getting \ch{Br2} and \ch{I2} --- concentrated phosphoric acid is such an acid.

				\txtdiagram{
					\schemestart[0,1.0,thick]
						\ch{NaX \stS} + \ch{H3PO4 \stL}\arrow\ch{HX \stG} + \ch{NaH2PO4 \stS}
					\schemestop
				}{\vspace{-1em}}

			% end subsubsection

		% end subsection

	% end section

% end part
















