% Chapter 02 - Group Seven.tex
% Copyright (c) 2014 - 2016, zhiayang@gmail.com
% Licensed under the Apache License Version 2.0.


\pagebreak
\part{Group 7}

	\section{Overview}

		Group 7 elements are also known as \itl{halogens}, and are the most reactive group of non-metals.
		They are reactive to the point of rarely occurring as free elements (\ch{\Cl2}, \ch{F2}) in nature.

	% end section

	\section{Physical Properties}

		\subsection{Electronic Configuration}

			Halogens have a \itl{n}p\sps{5} valence shell, and so tend to form \ch{X-} ions.


			\tabto{0mm}\ch{F}:\tabto{10mm}1s\sps{2}2s\sps{2}2p\sps{5}
			\tabto{0mm}\ch{\Cl}:\tabto{10mm}1s\sps{2}2s\sps{2}2p\sps{6}3s\sps{2}3p\sps{5}
			\tabto{0mm}\ch{Br}:\tabto{10mm}1s\sps{2}2s\sps{2}2p\sps{6}3s\sps{2}3p\sps{6}4s\sps{2}4p\sps{5}
			\tabto{0mm}\ch{I}:\tabto{10mm}1s\sps{2}2s\sps{2}2p\sps{6}3s\sps{2}3p\sps{6}3d\sps{10}4s\sps{2}4p\sps{6}5s\sps{2}5p\sps{5}

		% end subsection


		\subsection{Various Radii}

			All the halogens exist as diatomic molecules with a covalent bond between the two atoms. Both the ionic radii, and the covalent
			radii (defined as half the distance between two atomic centres), increase down the group as expected.

			The ionic radius for any halogen is always greater than its corresponding covalent radius, since the anion has one more electron,
			leading to greater interelectronic repulsion and hence a larger radius.

		% end subsection



		\subsection{First Ionisation Energy}

			As expected, the ionisation energies decrease down the group, due to the increased shielding effect caused by the addition
			of electron shells.

		% end subsection

		\pagebreak

		\subsection{First Electron Affinity}

			Recall that the first electron affinity is the enthalpy change created by adding one mole of electrons to one mole of gaseous atoms of a
			given element. This is of particular interest for halogens since they tend to ionise to \ch{X-} forms.

			\begin{table}[htb]\renewcommand{\arraystretch}{1.5}\begin{center}
			\begin{tabu} to 0.6\textwidth {X[c,m] | X[c,m]}

				Element     &   First EA / \si{\kilo\joule\per\mole}    \\  \hline
				\ch{F}      &   \num{-328}                              \\
				\ch{\Cl}    &   \num{-349}                              \\
				\ch{Br}     &   \num{-325}                              \\
				\ch{I}      &   \num{-295}                              \\

			\end{tabu}\end{center}
			\end{table}\vspace{-1em}

			The trend is that electron affinity decreases down the group (becomes less exothermic), since the electrostatic
			attraction between the joining electron and the nucleus decreases with increasing atomic radius.

			The anomaly for fluorine is due to its small size and electron lone pairs --- it takes slightly more energy to fit
			a new electron into the already crowded area around the fluorine atom.

		% end subsection


		\subsection{Bond Energy}

			Generally speaking, the bond energies of the halogens decrease down the group as the atomic radii increase. The orbital overlap
			thus decreases, which lengthens the covalent bond and hence weakens it.

			\begin{table}[htb]\renewcommand{\arraystretch}{1.5}\begin{center}
			\begin{tabu} to 0.6\textwidth {X[c,m] | X[c,m]}

				Element     &   bond energy / \si{\kilo\joule\per\mole} \\  \hline
				\ch{F}      &   \num{159}                               \\
				\ch{\Cl}    &   \num{242}                               \\
				\ch{Br}     &   \num{193}                               \\
				\ch{I}      &   \num{151}                               \\

			\end{tabu}\end{center}
			\end{table}\vspace{-1em}

			The discrepancy in the \ch{F-F} bond energy is due to the small atomic radius (and hence bond length) of the \ch{F} atom,
			which leads to abnormally-high levels of electrostatic repulsion between its lone pairs.

		% end subsection


		\subsection{Electronegativity}

			Electronegativity decreases down the group; fluorine is the most electronegative element, and iodine is the least
			electronegative in the group.

			The decrease in electronegativity is explained by the increasing atomic distnace which reduces the electrostatic force
			attracting valence electrons to the nucleus.

		% end subsection



		\pagebreak
		\subsection{Melting and Boiling Points}

			The melting and boiling points of the halogens are largely determined by their atomic size, and hence number of electrons. Since
			they exist as covalent diatomic molecules, the only intermolecular attractions are \idid{} interactions.; as the number of electrons
			increase, so do the strength of the id-id interactions, leading to higher melting and boiling points.

			All that is to say that melting and boiling points increase down the group; \ch{F2} and \ch{\Cl2} are gases at room temperature,
			\ch{Br2} is a liquid, and \ch{I2} is a solid.

		% end subsection


		\subsection{Colours}

			Generally speaking the colours of the halogens increase in intensity down the group:

			\begin{table}[htb]\renewcommand{\arraystretch}{1.5}\begin{center}
			\begin{tabu} to 0.9\textwidth {X[c,m] | X[c,m] | X[c,m] | X[c,m] | X[c,m]}

				Element  &  Gaseous         &   Liquid/Solid    &   Aqueous     &   Non-polar           \\  \hline
				\ch{F}   &  pale yellow     &       -           &   - boom -    &   very pale yellow    \\
				\ch{\Cl} &  yellow-green    &       -           &   pale yellow &   very pale green     \\
				\ch{Br}  &  reddish-brown   &   reddish-brown   &   yellow      &   orange              \\
				\ch{I}   &  dark purple     &   shiny black     &   brown       &   violet              \\

			\end{tabu}\end{center}
			\end{table}\vspace{-1em}

			Note that colour intensity increases with concentration as well.

		% end subsection


		\subsection{Solubility}

			\subsubsection{Polar Solvents (Water)}

				Generally, all the halogens except fluorine dissolve in water to \itl{some} extent. Fluorine reacts explosively due to its high
				reduction potential, and oxidises water.

				Chlorine and bromine react to form a number of ionic products in water in a disproportionation equilibrium, which will be discussed
				later. Iodine neither dissolves in water nor disproportionates (due to its low reduction potential), but can form a \itl{brown}
				soluble complex \ch{I3-} in the presence of \ch{I-} ions:

				\txtreactioneqn{
					\ch{I2 \stS} + \ch{I- \stAq}\arrow{<=>}\ch{I3- \stAq}
				}

			% end subsubsection


			\subsubsection{Non-polar Solvents}

				The solubility of the halogens in non-polar solvents is simply a function of their polarisability, and as with their melting and boiling
				points, increases down the group.

			% end subsubsection
		% end subsection
	% end section



	\pagebreak
	\section{Chemical Properties of the Elements}

		\subsection{Oxidation State}

			The most stable oxidation state for the halogens is of course -1, due to having a filled valence shell. However, chlorine, bromine
			and iodine can often be found in positive oxidation states, particularly when bonded to oxygen or other halogens.

			The primary reason for this behaviour is the fact that halogens other than fluorine are no longer the most electronegative, so more
			electronegative elements like oxygen can attract electrons away from the halogen atom. Examples include \ch{\Cl{}F}, \ch{\Cl{}O-}, and
			\ch{\Cl{}O2-}.

		% end subsection


		\subsection{Oxidising Power}

			As explained above, the electronegativities of the elements decrease down the group, and since they describe basically the same
			fundamental electron behaviour, it is no surprise that the reduction potentials decrease down the group as well.

			\begin{table}[htb]\renewcommand{\arraystretch}{1.5}\begin{center}
			\begin{tabu} to 0.6\textwidth {X[c,m] | X[c,m]}

				% headings
				Reaction                                        &   $E^{\stdst} / V$                    \\  \hline
				$\ch{F2 + 2 e-} \longrightarrow \ch{2 F-}$      &   \num[retain-explicit-plus]{+2.80}   \\
				$\ch{\Cl2 + 2 e-} \longrightarrow \ch{2 \Cl-}$  &   \num[retain-explicit-plus]{+1.36}   \\
				$\ch{Br2 + 2 e-} \longrightarrow \ch{2 Br-}$    &   \num[retain-explicit-plus]{+1.07}   \\
				$\ch{I2 + 2 e-} \longrightarrow \ch{2 I-}$      &   \num[retain-explicit-plus]{+0.54}   \\

			\end{tabu}\end{center}
			\end{table}\vspace{-1em}

			This means that oxidising power decreases down the group as well.

		% end subsection


		\subsection{Halogen Displacement}

			A halogen that is more reactive (ie. has a higher reduction potential) can oxidise a less reactive halogen and itself get reduced,
			thereby \itl{displacing} the less reactive halogen.

			For example, with \ch{\Cl2} displacing \ch{Br-}:

			\txtreactioneqn{
				\ch{\Cl2 \stG} + \ch{2 e-} \arrow \ch{2 \Cl-}
				\arrow(.east--eq1.west){0}[,1.5] \large$\;= \SI{+1.36}{\volt}$ \arrow(@eq1.west--.east){0}[-180,0] \large$\MEo$
				\arrow(@c1.east--.east){0}[-90,.4]
				\ch{2 Br- \stAq} \arrow \ch{Br2 \stL} + \ch{2 e-}
				\arrow(@eq1.east--eq2.east){0}[-90,.4] \large$\;= \SI{-1.07}{\volt}$ \arrow(@eq2.west--.east){0}[-180,0] \large$\MEox$
			}

			Since the \Ecell{} value ($\SI{1.36}{\volt} + (\SI{-1.07}{\volt}) = \SI{+0.29}{\volt}$) is positive, the reaction is feasible.
			Naturally were the conditions reversed, nothing would happen; bromine would not displace chloride.


		% end subsection



		\subsection{Reaction with Water}

			Fluorine reacts with water to form hydrogen fluoride and a mixture of oxygen and ozone, explosively --- even when cold.

			\txtreactioneqn{
				\ch{2 F2 \stG} + \ch{2 H2O \stL} \arrow \ch{4 HF \stAq} + \ch{O2 \stG}
				\arrow(@c1.east--.east){0}[-90,.4]
				\ch{3 F2 \stG} + \ch{3 H2O \stL} \arrow \ch{6 HF \stAq} + \ch{O3 \stG}
			}


			Chlorine and bromine, on the other hand, undergo a disproportionation reaction in water, in an equilibrium reaction:

			\txtreactioneqn{
				\ch{\Cl2 \stG} + \ch{H2O \stL} \arrow{<=>} \ch{H\Cl \stAq} + \ch{HO\Cl \stAq}
				\arrow(@c1.east--.east){0}[-90,.4]
				\ch{Br2 \stG} + \ch{H2O \stL} \arrow{<=>} \ch{HBr \stAq} + \ch{HOBr \stAq}
			}

			Of note is that \ch{HO\Cl} is the primary agent responsible for the bleaching mechanism in chlorine water. Also, the position
			of equilibrium for bromine's disproportionation is further to the left than chlorine, aka. disproportionation occurs to
			a \itl{smaller} extent.


			Iodine does not react with pure water.

		% end subsection


		\subsection{Reaction with Aqueous \texorpdfstring{\ch{Fe^2+}}{Fe²⁺} Ions}

			A simple calculation of the \Ecell{} values for the reaction would show that only chlorine and bromine are able to oxidise
			\ch{Fe^2+ \stAq} to \ch{Fe^3+ \stAq} (producing the relevant colour change). Iodine does not oxidise \ch{Fe^2+}.

			Fluorine would first explode, so it cannot react with the \itl{aqueous} solution.

		% end subsection




		\subsection{Reaction with Thiosulphate (\texorpdfstring{\ch{S2O3^2-}}{S₂O₃²⁻}) Ions}

			Thiosulphate ions can be oxidised by chlorine and bromine to give sulphate ions:

			\txtreactioneqn{
				\ch{4 \Cl2 \stAq} + \ch{S2O3^2- \stAq} + \ch{5 H2O \stAq} \arrow \ch{8 \Cl- \stAq} + \ch{2 SO4^2- \stAq} + \ch{10 OH- \stAq}
				\arrow(@c1.east--.east){0}[-90,.4]
				\ch{4 Br2 \stAq} + \ch{S2O3^2- \stAq} + \ch{5 H2O \stAq} \arrow \ch{8 Br- \stAq} + \ch{2 SO4^2- \stAq} + \ch{10 OH- \stAq}
			}

			Above, sulphur is oxidised from a +2 oxidation state in thiosulphate to a +6 state in sulphate. However, iodine, given its weaker
			oxidation strength, can only oxidise thiosulphate to a +2.5 oxidation state, forming \ch{S4O6^2-}.

			\txtreactioneqn{
				\ch{I2 \stAq} + \ch{2 S2O3^2- \stAq} \arrow \ch{2 I- \stAq} + \ch{S4O6^2- \stAq}
			}

		% end subsection



		\subsection{Disproportionation in Alkaline Solutions}

			Chlorine and bromine disproportionate readily and non-reversibly in alkaline solutions, to form ions in the +1 and
			-1 oxidation states:

			\txtreactioneqn{
				\ch{\Cl2 \stAq} + \ch{2 OH- \stAq} \arrow \ch{\Cl- \stAq} + \ch{\Cl{}O- \stAq} + \ch{H2O \stL}
				\arrow(@c1.east--.east){0}[-90,.4]
				\ch{Br2 \stAq} + \ch{2 OH- \stAq} \arrow \ch{Br -\stAq} + \ch{BrO- \stAq} + \ch{H2O \stL}
			}

			Simutaneously, the hypochlorite and hypobromite ions (\ch{\Cl{}O-} and \ch{BrO-} respectively) themselves undergo a
			disproportionation reaction to form the halogens in -1 and +5 oxidation states:

			\txtreactioneqn{
				\ch{3 BrO-} \arrow \ch{2 Br-} + \ch{BrO3-}
			}

			The hypohalite ion is thermally unstable, and decomposes through the secondary disproportionation reaction. Its stability
			decreases down the group, due to the increased bond length which decreases bond strength. Since the reactions are simultaneous,
			the proportion each product is dependent on the rate of each reaciton.

			For chlorine, the primary reaction is fast below \SI{20}{\celsius}, while the secondary reaction dominates above \SI{70}{\celsius}.
			For bromine, the first reaction is fast fast around \SI{20}{\celsius} as well, but the hypobromite ion is already unstable and
			the secondary reaction dominates; at \SI{0}{\celsius}, it is slow enough for \ch{BrO-} to be the primary product.

			\txtreactioneqn{
				\ch{3 \Cl2 \stAq} + \ch{6 OH- \stAq} \arrow \ch{5 \Cl- \stAq} + \ch{\Cl{}O3- \stAq} + \ch{3 H2O \stL}
				\arrow(@c1.east--.east){0}[-90,.4]
				\ch{3 Br2 \stAq} + \ch{6 OH- \stAq} \arrow \ch{5 Br -\stAq} + \ch{BrO3- \stAq} + \ch{3 H2O \stL}
			}[The overall reaction at high temperatures (respectively for each species)]

			Note that the hypoiodite ion is thermally unstable enough to spontaneously decompose to directly form the iodate ion at any temperature.

		% end subsection



		\subsection{Reaction with Metals}

			Given that the halogens are oxidising agents, they can easily oxidise most reactive metals to give ionic compounds. Fluorine
			exceptionally can react directly with all metals, including traditionally unreactive ones such as gold and silver (with heating);
			the vigour of the reaction reduces down the group.

			\subsubsection{Reaction with Sodium}

				All 4 halogens can react with sodium, which results in the metal burning with a \itl{bright orange} flame, forming a solid
				ionic salt \ch{NaX}.

				\txtreactioneqn{
					\ch{X2 \stG} + \ch{2 Na \stS} \arrow \ch{2 NaX \stS}
				}

			% end subsubsection


			\subsubsection{Reaction with Iron}

				Fluorine, chlorine, and bromine can oxidise iron metal from oxidation state 0 directly to
				oxidation state +3:

				\txtreactioneqn{
					\ch{3 X2 \stG} + \ch{2 Fe \stS} \arrow \ch{2 FeX3 \stS}
				}

				Iodine can only oxidise iron to oxidation state +2:

				\txtreactioneqn{
					\ch{I2 \stG} + \ch{Fe \stS} \arrow \ch{FeI2 \stS}
				}

			% end subsubsection
		% end subsection


		\subsection{Reaction with Phosphorus}

			All of the halogens can react vigorously and irreversibly with solid phosphorus at room temperature to give phosphorus trihalide:

			\txtreactioneqn{
				\ch{3 X2 \stG} + \ch{2 P \stS} \arrow \ch{2 PX3 \stL}
			}


			In the case of chlorine and bromine, there exists an equilibrium in excess halogen that results in the formation of \ch{PX5}.

			\txtreactioneqn{
				\ch{PX3 \stL} + \ch{X2 \stG} \arrow{<=>} \ch{PX5 \stS}
			}

			These compounds should be familiar as chlorinating and brominating agents in the nucleophilic substitution of \ch{OH} groups in
			organic chemistry.

		% end subsection



		\subsection{Reaction with Hydrogen}

			The halogens react with hydrogen gas with \itl{decreasing vigour} down the group to form the respective hydrogen halide.

			\txtreactioneqn{
				\ch{X2} + \ch{H2} \arrow \ch{2 HX}
			}

			As usual, the reaction with fluorine is explosive even at \SI{-200}{\celsius} and in the dark; chlorine reacts explosively
			in sunlight, and rapidly otherwise. Bromine reacts slowly, and typically requires heating at \SI{300}{\celsius} with a \ch{Pt}
			catalyst.

			Iodine's reaction is reversible and occurs slowly. The reaction is typically incomplete and a mixture of \ch{HI} and \ch{I2}
			is obtained; this is due to the positive enthalpy change of formation of \ch{HI}.

		% end subsection
	% end section




	\pagebreak
	\section{Chemical Properties of Hydrogen Halides}

		\subsection{Physical Properties}

			The hydrogen halides are polar, diatomic molecules. The melting and boiling points generally increase down the group, due to
			the larger electron cloud, which at larger sizes is the main determinant of the intermolecular attraction force (as opposed to
			permanent dipole interactions)

			\begin{table}[htb]\renewcommand{\arraystretch}{1.5}\begin{center}
			\begin{tabu} to 0.7\textwidth {X[c,m] | X[c,m] | X[c,m]}

				% headings
				Thing       &   Melting Point / \si{\celsius}   &   Boiling Point / \si{\celsius}   \\  \hline
				\ch{HF}     &   \num{-80}                       &   \num{20}                        \\
				\ch{H\Cl}   &   \num{-115}                      &   \num{-85}                       \\
				\ch{HBr}    &   \num{-89}                       &   \num{-67}                       \\
				\ch{HI}     &   \num{-51}                       &   \num{-35}                       \\

			\end{tabu}\end{center}
			\end{table}\vspace{-1em}

			The large discrepancy with \ch{HF} is due to its ability to form \itl{hydrogen bonds}, which are stronger than the pd-pd interactions
			(and id-id interactions) of the other halogens. Note that \ch{H\Cl} is generally unable to form hydrogen bonds (even though it has
			similar electronegativity to \ch{O} and \ch{N}) due to the atomic size of \ch{\Cl}.

		% end subsection



		\subsection{Thermal Stability}

			The weaker hydrogen halides can be decomposed back into \ch{X2} and \ch{H2} when heated. Naturally the temperature that is required
			for this decomposition decreases down the group, due to the decreasing strength of the \ch{H-X} bond.
			\ch{HF} and \ch{H\Cl} cannot even be decomposed when heated strongly.

			\begin{bulletlist}
				& \ch{HF} and \ch{H\Cl} do not decompose
				& \ch{HBr} decomposes on strong heating, yielding reddish-brown fumes of \ch{Br2 \stG}
				& \ch{HI} decomposes simply by contact with a hot metal rod, yielding violet fumes of \ch{I2 \stG}.
			\end{bulletlist}

		% end subsection


		\subsection{Acid Strength}

			\ch{H\Cl}, \ch{HBr} and \ch{HI} are strong acids that dissociate fully in water to give \ch{H+ \stAq} ions.

			The acid strength \itl{increases} down the group, since the \ch{H-X} bond becomes weaker, which allows for easier
			dissociation. However, \ch{HF} notably behaves as a weak acid due to incomplete dissociation which can be attributed to the
			high strength of the \ch{H-F} bond.

			At high (approaching 100\%) concentrations however, \ch{HF} undergoes autoionisation and homoassociation to form the strong acid
			\ch{HF2-}:

			\txtreactioneqn{
				\ch{3 HF} \arrow{<=>} \ch{H2F+} + \ch{HF2-}
			}[\ch{HF2-} (or \ch{FHF-}) is stablised by the strong hydrogen bond between \ch{HF} and \ch{H}.]

		% end subsection

	% end section

	\section{Chemical Properties of Halide Ions}

		\subsection{Precipitation Reactions}

			Silver ions react with halide ions in an equilibrium reaction to form insoluble \ch{AgX}.

			\txtreactioneqn{
				\ch{Ag+ \stAq} + \ch{X- \stAq} \arrow \ch{AgX \stS}
			}

			As discussed in previous chapters, the solubility of the \ch{AgX} salt depends on the concentration of \ch{Ag+} and \ch{X-} ions
			in the solution; recall that ionic product = $[\ch{Ag+}][\ch{I-}]$. If IP is less than \Ksp{}, then the salt will dissolve; otherwise,
			the salt will precipitate out of the solution.

			Upon adding \ch{NH3 \stAq}, a complex ion is formed with the silver ions.

			\txtreactioneqn{
				\ch{Ag+ \stAq} + \ch{NH3 \stAq} \arrow \ch{[Ag(NH3)2]+ \stAq}
			}

			The formation of this complex reduces the concentration of \ch{Ag+} in the solution, thus decreasing the value of IP in the original
			reaction. When the ionic product becomes smaller than \Ksp{}, the precipitate will dissolve.

			For \ch{Ag\Cl}, dilute \ch{NH3 \stAq} is enough to lower the ionic product below \Ksp{}, and allow the salt to dissolve. For
			\ch{AgBr}, concentrated \ch{NH3 \stAq} must be used to lower the ionic product below the \Ksp{} of \ch{AgBr}.

			For \ch{AgI}, the \Ksp{} value is so small that no amount of \ch{NH3 \stAq} will lower the concentration of \ch{Ag+} enough for
			the salt to dissolve.


			\begin{table}[htb]\renewcommand{\arraystretch}{1.5}\begin{center}
			\begin{tabu} to 0.8\textwidth {X[c,m] | X[c,m] | X[c,m] | X[c,m]}

				% headings
				Ion         &   Colour of \ch{AgX}  &   Required \ch{NH3 \stAq} &   \Ksp{} / \si{\mole\squared\per\cubic\deci\metre\tothe{6}} \\  \hline
				\ch{\Cl-}   &   white               &   dilute                  &   \num{1.6e-10}                                           \\
				\ch{Br-}    &   cream               &   concentrated            &   \num{7.7e-13}                                           \\
				\ch{I-}     &   yellow              &   nothing works           &   \num{8.3e-17}                                           \\

			\end{tabu}\end{center}
			\end{table}\vspace{-1em}

		% end subsection


		\pagebreak
		\subsection{Reaction with Lead (II) Ions}

			As you might know, lead halides are insoluble. Only lead iodide (\ch{PbI2}) is yellow, while the precipitates of the other halides
			are white.

			\txtreactioneqn{
				\ch{Pb^2+ \stAq} + \ch{2 X- \stAq} \arrow \ch{PbX2 \stS}
			}

			Addition of excess \ch{X-} will cause the precipitate to re-dissolve, since the soluble complex ion \ch{[PbX4]^2-} is formed.

			\txtreactioneqn{
				\ch{Pb^2+ \stAq} + \ch{4 X- \stAq} \arrow \ch{[PbX4]^2- \stAq}
			}



		% end subsection


		\subsection{Reaction with Concentrated \texorpdfstring{\ch{H2SO4}}{H₂SO₄}}

			Generally speaking, \ch{X-} ions can act as a Brønsted base, accepting \ch{H+} ions from acids. When concentrated sulphuric acid
			is added to solid halide salts, this indeed occurs:

			\txtreactioneqn{
				\ch{NaX \stS} + \ch{H2SO4 \stL} \arrow \ch{HX \stG} + \ch{NaHSO4 \stS}
			}

			However, \ch{H2SO4} is also a strong oxidising agent; thus, it is able to oxidise the \ch{HBr} and \ch{HI} formed in the acid-base
			reaction above. Note that it is insufficiently strong to oxidise \ch{HF} and \ch{H\Cl}.


			\txtreactioneqn{
				\ch{NaBr \stS} + \ch{H2SO4 \stL} \arrow \ch{HBr \stG} + \ch{NaHSO4 \stS}
				\arrow(@c1.east--.east){0}[-90,.4]
				\ch{2 HBr \stG} + \ch{H2SO4 \stL} \arrow \ch{Br2 \stG} + \ch{SO2 \stG} + \ch{2 H2O \stL}
			}

			In this reaction, most of the \ch{HBr} remains unreacted, and some \ch{Br2 \stG} is produced, leading to the observation of
			slight reddish-brown fumes.


			Below, \ch{H2SO4} continues to oxidise \ch{HI} in a complex and frankly ridiculous manner:


			\txtreactioneqn{
				\ch{NaI \stS} + \ch{H2SO4 \stL} \arrow \ch{HI \stG} + \ch{NaHSO4 \stS}
				\arrow(@c1.east--.east){0}[-90,.4]
				\ch{2 HI \stG} + \ch{H2SO4 \stL} \arrow \ch{I2 \stG} + \ch{SO2 \stG} + \ch{2 H2O \stL}
				\arrow(@c3.east--.east){0}[-90,.4]
				\ch{6 HI \stG} + \ch{H2SO4 \stL} \arrow \ch{3 I2 \stG} + \ch{S \stS} + \ch{4 H2O \stL}
				\arrow(@c5.east--.east){0}[-90,.4]
				\ch{8 HI \stG} + \ch{H2SO4 \stL} \arrow \ch{4 I2 \stG} + \ch{H2S \stG} + \ch{4 H2O \stL}
			}

			Due to the ease of oxidising \ch{HI}, \ch{H2SO4} is itself reduced, with sulphur going from a +6 oxidation state to a -2 state
			in \ch{H2S} (which smells like rotten eggs). Furthermore, \ch{SO2} is also a pungent gas --- this is
			a smelly reaction.

			Most of the \ch{HI} is oxidised, thus there will be a large output of violet fumes, with some white fumes of \ch{HI}.


			\pagebreak
			\subsubsection{Reaction in the Presence of \texorpdfstring{\ch{MnO2}}{MnO₂}}

				Manganese dioxide is a stronger oxidising agent than \ch{H2SO4}, and thus it can serve to oxidise the \ch{H\Cl} that sulphuric
				acid cannot:

				\txtreactioneqn{
					\ch{Na\Cl \stS} + \ch{H2SO4 \stL} \arrow \ch{H\Cl \stG} + \ch{NaHSO4 \stS}
					\arrow(@c1.east--.east){0}[-90,.4]
					\ch{4 H\Cl \stG} + \ch{MnO2 \stL} \arrow \ch{\Cl2 \stG} + \ch{Mn\Cl2 \stG} + \ch{2 H2O \stL}
				}

			% end subsubsection


			\subsubsection{\texorpdfstring{\ch{H3PO4}}{H₃PO₄} as a Weaker Oxidising Agent}

				Given that sulphuric acid will obliterate \ch{HBr} and \ch{HI}, a less powerful oxidising agent can be used to obtain these
				instead of getting \ch{Br2} and \ch{I2} --- concentrated phosphoric acid is such an acid:

				\txtreactioneqn{
					\ch{NaX \stS} + \ch{H3PO4 \stL} \arrow \ch{HX \stG} + \ch{NaH2PO4 \stS}
				}

			% end subsubsection

		% end subsection

	% end section

% end part
















