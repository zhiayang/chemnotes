% Chapter 01 - Group Two.tex
% Copyright (c) 2014 - 2016, zhiayang@gmail.com
% Licensed under the Apache License Version 2.0.


\newcommand{\rtwo}{\bld{\romannum{II}}}

\pagebreak
\part{Group \romannum{II}}

	\section{Overview}

		Group \rtwo{} metals are the \itl{alkaline earth metals}, and have this name for two reasons; their oxides form alkaline solutions
		when dissolved in water, and some another mostly irrelevant and historical reason. Mainly the alkaline solution thing.

	% end section

	\section{Physical Properties}

		\subsection{Electronic Configuration}

			All compounds containing group \rtwo{} metals have them at a fixed oxidation state of +2, and all group \rtwo{} metals have a
			fully filled outermost \itl{s} subshell:

			\tabto{0mm}\ch{Be}:\tabto{10mm}1s\sps{2}2s\sps{2}
			\tabto{0mm}\ch{Mg}:\tabto{10mm}1s\sps{2}2s\sps{2}2p\sps{6}3s\sps{2}
			\tabto{0mm}\ch{Ca}:\tabto{10mm}1s\sps{2}2s\sps{2}2p\sps{6}3s\sps{2}3p\sps{6}4s\sps{2}
			\tabto{0mm}\ch{Sr}:\tabto{10mm}1s\sps{2}2s\sps{2}2p\sps{6}3s\sps{2}3p\sps{6}3d\sps{10}4s\sps{2}4p\sps{6}5s\sps{2}
			\tabto{0mm}\ch{Ba}:\tabto{10mm}1s\sps{2}2s\sps{2}2p\sps{6}3s\sps{2}3p\sps{6}3d\sps{10}4s\sps{2}4p\sps{6}4d\sps{10}5s\sps{2}5p\sps{6}6s\sps{2}


			Group \rtwo{} metals form +2 compounds two reasons; first, the 3rd ionisation energy (to get to a +3 state) is significantly greater
			than either the 1st or 2nd --- this required energy cannot be sufficiently compensated for by an exothermic lattice energy.

			Secondly, they do not exist in the +1 state because, in many cases, the lattice energy of compounds containing the metal in the +2
			state is more exothermic, thus favouring the more stable \ch{MX2} compound as opposed to the \ch{MX} compound.

		% end subsection





		\subsection{Ionic and Atomic Radii}

			Naturally, both the ionic and atomic radii of the group \rtwo{} metals \itl{increase} down the group, due simply to the increasing
			number of quantum shells.

		% end subsection



		\pagebreak
		\subsection{Ionisation Energy}

			As should be expected, the first and second ionisation energies of the elements in the group \itl{decrease} progressively. This is
			due to the increasing number of electron shells between the nucleus and the valence electron shell. Thus, the shielding effect
			increases, the electrostatic force of attraction decreases, and less energy is required to remove the two valence electrons.

		% end subsection

		\subsection{Melting and Boiling Points}

			\subsubsection{General Trend}

				Generally speaking, the melting and boiling points decrease down the group, although there are rather large irregularities in
				the trend. The broad reason for decreasing melting and boiling points is the increasing cationic size down the group, leading
				to weaker electrostatic forces between the delocalised electron cloud and the ions --- hence weaker metallic bonding.

				There is unfortunately no simple explanation for the irregularity in the trend.

			% end subsubsection


			\subsubsection{Comparison with Group \bld{\romannum{I}} Metals}

				In the same period, group \rtwo{} metals have higher melting and boiling points than group \bld{\romannum{I}} metals, primarily
				due to the increase in nuclear charge (+1 to +2), and more electrons in the delocalised cloud, leading to stronger metallic
				bonding.

			% end subsubsection

		% end subsection



		\subsection{Solubility of Hydroxides and Sulphates}

			The solubility of group \rtwo{} hydroxides \itl{increases} down the group, while the solubility of the sulphates \itl{decrease}
			down the group; \ch{Mg(OH)2} and \ch{BaSO4} are insoluble, while \ch{MgSO4} and \ch{Ba(OH)2} are soluble.

			\txtdiagram{
				\schemestart[0,1.0,thick]
					\enth{sol} $= -LE + ∆H_{hyd}$
				\schemestop
			}{\vspace{-2em}}

			\txtdiagram{
				\schemestart[0,1.0,thick]
					$LE \propto \frac{q_{+}\times q_{-}}{r_{+} + r_{-}}$\hspace{20mm}\enth{hyd} $\propto \frac{q_{+}}{r_{+}} + \frac{q_{-}}{r_{-}}$
				\schemestop
			}{\vspace{-2em}}


			For the sulphates, the anionic radius is far larger than the cationic radius, hence the lattice energy remains relatively constant
			down the group. However, looking at the enthalpy change of hydration, the increasing cationic radius decreases the \enth{hyd}.
			Hence, the overall effect is that \enth{sol} increases down the group, becoming \itl{less exothermic}, and hence solubility decreases.

			For the hydroxides, the reverse is true; the anionic radius is small compared to the cation, so the lattice energy decreases
			down the group as the cationic radius has a greater effect. Thus, combined with the decreasing \enth{hyd}, solubility for
			the hydroxides increases down the group.

		% end subsection
	% end section


	\pagebreak
	\section{Chemical Properties of the Elements}

		\subsection{Redox Reactivity}

			Due to the decrease in first two ionisation energies down the group, the group \rtwo{} metals are more easily oxidised down the
			group. Therefore, their strength as reducing agents also increases down the group.

			Behold a table of \Eox{} values (lol you thought you had escaped them, didn't you):

			\begin{center}\begin{table}[htb]\renewcommand{\arraystretch}{1.5}
			\begin{tabu} to \textwidth {X[c,m] | X[c,m]}

				% headings
				Reaction							&	\Eox{}$ / V$	\\	\hline
				\ch{Be \stS -> Be^2+ + 2 e-}		&	\num[retain-explicit-plus]{+1.99}		\\	\hline
				\ch{Mg \stS -> Mg^2+ + 2 e-}		&	\num[retain-explicit-plus]{+2.37}		\\	\hline
				\ch{Ca \stS -> Ca^2+ + 2 e-}		&	\num[retain-explicit-plus]{+2.87}		\\	\hline
				\ch{Sr \stS -> Sr^2+ + 2 e-}		&	\num[retain-explicit-plus]{+2.89}		\\	\hline
				\ch{Ba \stS -> Ba^2+ + 2 e-}		&	\num[retain-explicit-plus]{+2.91}		\\	\hline


			\end{tabu}
			\end{table}\end{center}\vspace{-10mm}

		% end subsection


		\subsection{Reaction with Oxygen}

			The trend of the metals' reactions with oxygen is simply a product of their increasing reducing power down the group. They all
			tarnish in air (before exploding) without external agents, and the reactivity increases down the group.

			\txtdiagram{
				\schemestart[0,1.0,thick]
					\ch{2 Mg \stS} + \ch{O2 \stG}\arrow\ch{2 MgO \stS}
				\schemestop
			}{\vspace{-1.5em}}

			Barium reacts at room temperature quickly, so it is stored under oil to prevent rapid tarnishing. Beryllium on the other hand needs
			to be burning before \itl{any} reaction takes place; its curiosities will be explored later.

		% end subsection


		\pagebreak
		\subsection{Reaction with Water}

			Again, the reactivity of the metals with water increase down the group; they reduce water to
			hydrogen gas. Note that beryllium does not react with water (or steam), even when heated.

			All the metals form hydroxides with the exception of magnesium --- when reacting with steam,
			it forms \ch{MgO} directly, and reacts slowly with cold water otherwise to form \ch{Mg(OH)2}.

			\txtdiagram{
				\schemestart[0,1.0,thick]
					\ch{2 Mg \stS} + \ch{H2O \stG}\arrow\ch{MgO \stS} + \ch{H2 \stG}
				\schemestop
			}{\vspace{-1.5em}}

			For the rest of the metals, hydroxides are formed directly, with increasing vigour down the group.

			\txtdiagram{
				\schemestart[0,1.0,thick]
					\ch{Ba \stS} + \ch{2 H2O \stL}\arrow\ch{Ba(OH)2 \stAq} + \ch{H2 \stG}
				\schemestop
			}{\vspace{-1.5em}}

			Note that the solubility of \ch{Ca(OH)2} is an equilibrium, and exists as either slaked lime in
			the solid form, or limewater in the aqueous form. The latter is produced when reacted with excess
			water.

			Finally, all the group \rtwo{} hydroxides are basic (duh), except for beryllium hydroxide
			(\ch{Be(OH)2}) which is amphoteric.

		% end subsection

	% end section


	\section{Chemical Properties of the Oxides}

		\subsection{Physical Properties}

			Strictly does not belong in this section, but whatever. All group \rtwo{} oxides are solid at room
			temperature, are white, are basic, and have very high melting and boiling points. Melting and
			boiling points for the oxides decrease down the group, due to a decreasing lattice energy.

		% end subsection


		\subsection{Reaction with Water}

			All group \rtwo{} oxides react with water to form their respective hydroxides (hence they are basic
			oxides), except \ch{BeO} which is amphoteric, but does not react with water.

			The vigour of the reaction increases down the group, since the magnitude of the lattice energy
			of the ionic oxide decreases.

			\txtdiagram{
				\schemestart[0,1.0,thick]
					\ch{BaO \stS} + \ch{H2O \stL}\arrow\ch{Ba(OH)2 \stAq}
				\schemestop
			}{\vspace{-1.5em}}

		% end subsection

	% end section


	\section{Thermal Stability of Group \romannum{II} Compounds}

		\subsection{Overview}

			All group \rtwo{} hydroxides, carbonates and nitrates thermally decompose at appropriate
			temperatures. The trend for all 3 is that the temperature required \itl{increases} down
			the group, aka. the stability of these ionic compounds increases down the group.

			The data for beryllium is generally sparse, so... it's not there. Oops.

			\begin{center}\begin{table}[htb]\renewcommand{\arraystretch}{1.5}
			\begin{tabu} to \textwidth {X[c,m] | X[c,m] | X[c,m] | X[c,m]}

				% headings
				Element		&	\ch{MCO3}	&	\ch{M(NO3)2}	&	\ch{M(OH)2}	\\	\hline
				\ch{Mg}		&	\SI{400}{\celsius}	&\SI{450}{\celsius}	&\SI{300}{\celsius}		\\	\hline
				\ch{Ca}		&	\SI{900}{\celsius}	&\SI{575}{\celsius}	&\SI{390}{\celsius}		\\	\hline
				\ch{Sr}		&	\SI{1280}{\celsius}	&\SI{635}{\celsius}	&\SI{466}{\celsius}		\\	\hline
				\ch{Ba}		&	\SI{1360}{\celsius}	&\SI{675}{\celsius}	&\SI{700}{\celsius}		\\	\hline

			\end{tabu}
			\end{table}\end{center}\vspace{-10mm}

			The main factor affecting the thermal stability is the polarising power of the ion, which depends
			on charge density --- this decreases down the group, leading to more stable ionic compounds.

		% end subsection


		\subsection{Mechanism of Action}

			The decomposition of all 3 categories of ionic compounds follow the same principle. Given the
			relatively high charge density of the group \rtwo{} metals, they are able to \itl{polarise} the
			covalent bonds \itl{within the anion}, weakening them.

			This weakening of the covalent bonds in the anion allows said bond to break when heat is applied,
			typically resulting in a metal oxide as the product.

			\diagram[1.0]{
				\schemestart[0, 1.5, thick]
					\chemfig{Mg\sps{2+}-[:0,1.2,,,dashed]!\molO(-[@{bond}:0,1.3,,,lddbond]C(
					-[:60,,,,lddbond]!\molO)(-[:300,,,,lddbond]!\molO)(-[:120,1.5,,,draw=none]@{txt}weak))}
				\schemestop

				\chemmove{\draw[-Stealth,line width=0.4mm,shorten <=2mm,shorten >=2mm](txt) -- (bond);}
			}{\vspace{0em}}


			In this case, when heat is applied, the weak bond will be broken, leading to the formation of
			\ch{MO} and \ch{CO2}. Generally speaking this is why metal oxides are usually formed.

			As the polarising power of the metal ion decreases down the group (due to increasing radius but
			constant charge), the bond is weakened to a lesser degree, leading to a higher stability.

			Clearly, only polyatomic anions can be decomposed; monatomic anions cannot be broken down further
			by heat.


			\subsubsection{Carbonates}

				The thermal decomposition of group \rtwo{} carbonates generally produces the metal oxide,
				and carbon dioxide gas.

				\txtdiagram{
					\schemestart[0,1.0,thick]
						\ch{BaCO3 \stS}\arrow{->[\tinytext{heat}]}\ch{BaO \stS} + \ch{CO2 \stG}
					\schemestop
				}{\vspace{-1.5em}}

			% end subsubsection


			% small note: moved up here to make the arrow misalignment less jarring
			\subsubsection{Hydroxides}

				Decomposing hydroxides produces the metal oxide and water.

				\txtdiagram{
					\schemestart[0,1.0,thick]
						\ch{Ba(OH)2 \stS}\arrow{->[\tinytext{heat}]}\ch{BaO \stS} + \ch{H2O \stL}
					\schemestop
				}{\vspace{-1.5em}}


			% end subsubsection


			\subsubsection{Nitrates}

				The thermal decomposition of nitrates generally yields the metal oxide, as well as nitrogen
				and oxygen gas.

				\txtdiagram{
					\schemestart[0,1.0,thick]
						\ch{2 Ba(NO3)2 \stS}\arrow{->[\tinytext{heat}]}\ch{2 BaO \stS} + \ch{4 NO2 \stG} + \ch{O2 \stG}
					\schemestop
				}{\vspace{-1.5em}}

			% end subsubsection

		% end subsection


		\subsection{Comparison with Group \romannum{I} Metals}

			Most group \bld{\romannum{I}} metals do not have sufficient charge density to polarise the
			carbonate anion, however they do polarise the nitrate ion enough for some decomposition:

			\txtdiagram{
				\schemestart[0,1.0,thick]
					\ch{2 NaNO3 \stS}\arrow{->[\tinytext{heat}]}\ch{2 NaNO2 \stS} + \ch{O2 \stG}
				\schemestop
			}{\vspace{-1.5em}}

			Lithium is the exception to this rule --- it has a sufficiently high charge density (due to its
			tiny ionic radius) to polarise nitrates fully to decompose in the same way as group 2 ions, and
			can also polarise carbonates.

			\txtdiagram{
				\schemestart[0,1.0,thick]
					\ch{Li2CO3 \stS}\arrow{->[\tinytext{heat}]}\ch{Li2O \stS} + \ch{CO2 \stG}
				\schemestop
			}{\vspace{-1.5em}}

		% end subsection

	% end section


	\section{Beryllium Being Boneheaded}

		Nice alliteration, eh?

		The majority of the discrepancies with beryllium can be attributed to the highly covalent character
		of bonds formed with \ch{Be}, partially due to its high charge density which can polarise the electron
		cloud of anions.

		Compounds with large anions, such as \ch{Be\Cl2}, are completely covalent.



		\subsection{Non-reaction with Water and Oxygen}

			Similar to magnesium but more extreme, beryllium forms an \itl{impervious} oxide layer when
			exposed to oxygen --- this reacts with neither water nor oxygen, leading to its unreactivity.

		% end subsection


		\subsection{Amphoteric Nature}

			Beryllium oxides and hydroxides are amphoteric unlike its other group \rtwo{} compatriots, and
			this is due to the partially covalent nature of the bonds with \ch{Be}.

			\begin{bulletlist}
				& Acid:\tabto{15mm} \ch{BeO + 2 NaOH + H2O -> Na2[Be(OH)4]}
				& Base:\tabto{15mm} \ch{BeO + H2SO4 -> BeSO4 + H2O}
			\end{bulletlist}

			Note the formation of the \ch{[Be(OH)4]^2-} complex, which will be discussed below.

		% end subsection


		\subsection{Formation of \ch{Be} complexes}

			\subsubsection{Hydrated Ions}

				As will be discussed later, \ch{Mg^2+} has sufficient charge density to form hydrated complex
				ions in water (\ch{[Mg(H2O)6]^2+}). Thus, it is clear that \ch{Be} should form these complexes
				as well, and it does --- \ch{[Be(H2O)4]^2+}.

				However, \ch{Be^2+} is only 4-coordinate, due to the lack of vacant d-orbitals (there is no 2d
				subshell). Hence, by using the 2s and 2p orbitals, it can form 4 bonds with ligands. By contrast
				\ch{Mg^2+} has a 3d subshell, allowing for more ligands.

			% end subsubsection


			\pagebreak
			\subsubsection{Other Complexes}

				While \ch{Be^2+} can form complexes, \ch{Be} can also form complexes when bonded with other
				atoms, since it still has 2 vacant p-orbitals to fill with ligand bonds.

				An example seen above is the formation of \ch{[Be(OH)4]^2-} when \ch{Be(OH)2} is exposed to
				more \ch{OH-}, and the formation of \ch{[BeF4]^2-}.

				\txtdiagram{
					\schemestart[0,1.0,thick]
						\ch{Be(OH)2 \stAq} + \ch{2 OH- \stAq} \arrow \ch{[Be(OH)4]^2- \stAq}
						\arrow(@c1.south east--.north east){0}[-90,.25]
						\ch{BeF2 \stAq} + \ch{2 F- \stAq} \arrow \ch{[BeF4]^2- \stAq}
					\schemestop
				}{\vspace{-1.5em}}

				A point to note is that \ch{[Be(OH)4]^2-} is created from \ch{Be(OH)2}, which is itself formed
				from the hydrolysis of a beryllium salt in water (since \ch{BeO} does not react with water).

				On adding a base (aka \ch{OH-}), the hydrated \ch{[Be(H2O)4]^2+} complex acts as an acid,
				forming the \ch{[Be(OH)4]^2-} by losing 4 \ch{H+} ions. This behaviour is conveniently
				discussed below.

			% end subsubsection

		% end subsection

		\subsection{Acidity of Beryllium Salts}

			Similar to other cations of high charge density like \ch{\Al^3+}, \ch{Fe^3+} and \ch{Cr^3+} (and to
			a certain extent \ch{Mg^2+}), \ch{Be^2+} forms a hydrated complex in water, \ch{[Be(H2O)4]^2+},
			which was seen above.

			Just like the other complexes, it can hydrolyse to give \ch{H3O}+, making it acidic in water.

			\txtdiagram{
				\schemestart[0,1.0,thick]
					\ch{[Be(H2O)4]^2+} + \ch{H2O}\arrow\ch{[Be(H2O)3(OH)]+} + \ch{H3O+}
				\schemestop
			}{\vspace{-1.5em}}

			Naturally it can undergo further hydrolysis to finally form \ch{[Be(OH)4]^2-}. This behaviour
			is what enables beryllium to react with bases, and be amphoteric.


		% end subsection

	% end section

% end part



























