% Chapter 01 - Group Two.tex
% Copyright (c) 2014 - 2016, zhiayang@gmail.com
% Licensed under the Apache License Version 2.0.


\pagebreak
\part{Group 2}

	\section{Overview}

		Group 2 metals are the \itl{alkaline earth metals}, and have this name for two reasons; their oxides form alkaline solutions
		when dissolved in water, and some another mostly irrelevant and historical reason. Mainly the alkaline solution thing.

	% end section

	\section{Physical Properties}

		\subsection{Electronic Configuration}

			Group 2 metals are present in compounds with a fixed oxidation state of +2, with a fully filled outermost \itl{s} subshell:

			\tabto{0mm}\ch{Be}:\tabto{10mm}1s\sps{2}2s\sps{2}
			\tabto{0mm}\ch{Mg}:\tabto{10mm}1s\sps{2}2s\sps{2}2p\sps{6}3s\sps{2}
			\tabto{0mm}\ch{Ca}:\tabto{10mm}1s\sps{2}2s\sps{2}2p\sps{6}3s\sps{2}3p\sps{6}4s\sps{2}
			\tabto{0mm}\ch{Sr}:\tabto{10mm}1s\sps{2}2s\sps{2}2p\sps{6}3s\sps{2}3p\sps{6}3d\sps{10}4s\sps{2}4p\sps{6}5s\sps{2}
			\tabto{0mm}\ch{Ba}:\tabto{10mm}1s\sps{2}2s\sps{2}2p\sps{6}3s\sps{2}3p\sps{6}3d\sps{10}4s\sps{2}4p\sps{6}4d\sps{10}5s\sps{2}5p\sps{6}6s\sps{2}


			Group 2 metals form +2 compounds for two reasons; first, the 3rd ionisation energy (to get to a +3 state) is significantly greater
			than either the 1st or 2nd, which cannot be sufficiently compensated for by an exothermic lattice energy.

			Secondly, they do not exist in the +1 state because in many cases, the lattice energy of compounds containing the metal in the +2
			state is more exothermic, thus favouring the more stable \ch{MX2} compound as opposed to the \ch{MX} compound.

		% end subsection





		\subsection{Ionic and Atomic Radii}

			Naturally, both the ionic and atomic radii of the group 2 metals \itl{increase} down the group, due to the increasing
			number of quantum shells.

		% end subsection

		\pagebreak
		\subsection{Ionisation Energy}

			The first and second ionisation energies of the elements \itl{decrease} down the group. This is due to the increasing number of
			electron shells between the nucleus and the valence electron shell, increasing the shielding effect.

			Recall that the shielding effect decreases the attractive electrostatic forces between the valence electrons and the nucleus, so less
			energy is required to remove said electron.

		% end subsection

		\subsection{Melting and Boiling Points}

			\subsubsection{General Trend}

				Generally speaking, the melting and boiling points decrease down the group, although there are rather large irregularities in
				the trend. The broad reason for decreasing melting and boiling points is the increasing cationic size down the group, which leads
				to weaker electrostatic forces between the delocalised electron cloud and the ions. Thus, the strength of the metallic bond decreases.

			% end subsubsection


			\subsubsection{Comparison with Group 1 Metals}

				In the same period, group 2 metals have higher melting and boiling points than group 1 metals, primarily
				due to the increase in nuclear charge (+1 to +2) along with more electrons in the delocalised cloud, leading to stronger metallic
				bonding.

			% end subsubsection

		% end subsection



		\subsection{Solubility of Hydroxides and Sulphates}

			The solubilities of group 2 hydroxides \itl{increase} down the group, while the solubilities of the sulphates \itl{decrease};
			\ch{Mg(OH)2} and \ch{BaSO4} are insoluble, while \ch{MgSO4} and \ch{Ba(OH)2} are soluble.

			\mathdiagram{
				\[ \Menth{sol} = -LE + \Menth{hyd} \]
			}

			\mathdiagram{
				\[ LE \propto \frac{q_{+}\times q_{-}}{r_{+} + r_{-}}   \qquad
					\Menth{hyd} \propto \frac{q_{+}}{r_{+}} + \frac{q_{-}}{r_{-}} \]
			}


			For the sulphates, the anionic radius is far larger than the cationic radius, hence the $LE$ remains relatively constant
			down the group. However, looking at the enthalpy change of hydration, the increasing cationic radius decreases the \enth{hyd}.
			Overall, \enth{sol} increases down the group and is \itl{less exothermic}, decreasing solubility.

			For the hydroxides, the reverse is true; the anionic radius is small compared to the cation, so the the cationic radius has a
			greater effect and $LE$ decreases. Thus, combined with the decreasing \enth{hyd}, hydroxides become increasingly soluble down the group.

		% end subsection
	% end section


	\pagebreak
	\section{Chemical Properties of the Elements}

		\subsection{Redox Reactivity}

			Due to the decrease in the first two ionisation energies down the group, the ease of oxidation increases as well --- implying that
			group 2 metals increase in strength as reducing agents as well.

			Behold a table of \Eox{} values:

			\begin{table}[htb]\renewcommand{\arraystretch}{1.5}\begin{center}
			\begin{tabu} to 0.6\textwidth {X[c,m] | X[c,m]}

				% headings
				Reaction                                            &   $\MEox{} / \si{\volt}$              \\  \hline
				$\ch{Be \stS} \longrightarrow \ch{Be^2+ + 2 e-}$    &   \num[retain-explicit-plus]{+1.99}   \\
				$\ch{Mg \stS} \longrightarrow \ch{Mg^2+ + 2 e-}$    &   \num[retain-explicit-plus]{+2.37}   \\
				$\ch{Ca \stS} \longrightarrow \ch{Ca^2+ + 2 e-}$    &   \num[retain-explicit-plus]{+2.87}   \\
				$\ch{Sr \stS} \longrightarrow \ch{Sr^2+ + 2 e-}$    &   \num[retain-explicit-plus]{+2.89}   \\
				$\ch{Ba \stS} \longrightarrow \ch{Ba^2+ + 2 e-}$    &   \num[retain-explicit-plus]{+2.91}   \\


			\end{tabu}\end{center}
			\end{table}\vspace{-1em}

		% end subsection


		\subsection{Reaction with Oxygen}

			The trend of the metals' reactions with oxygen is simply a product of their increasing reducing power down the group. They all
			tarnish in air (before exploding) without external agents, and the reactivity increases down the group.

			\txtreactioneqn{
				\ch{2 Mg \stS} + \ch{O2 \stG} \arrow \ch{2 MgO \stS}
			}

			Barium reacts at room temperature quickly, so it is stored under oil to prevent rapid tarnishing. Beryllium on the other hand needs
			to be burning before \itl{any} reaction takes place; its curiosities will be explored later.


			\begin{table}[htb]\renewcommand{\arraystretch}{1.5}\begin{center}
			\begin{tabu} to 0.85\textwidth {X[2,c,m] | X[1,c,m] | X[1,c,m]}

				% headings
				Reaction                                                        &   Flame Colour        &   Reactivity  \\  \hline
				$\ch{2 Mg \stS} + \ch{O2 \stG} \longrightarrow \ch{2 MgO \stS}$ &   brilliant white     &   very slow   \\
				$\ch{2 Ca \stS} + \ch{O2 \stG} \longrightarrow \ch{2 CaO \stS}$ &   brick red           &   slow        \\
				$\ch{2 Sr \stS} + \ch{O2 \stG} \longrightarrow \ch{2 SrO \stS}$ &   crimson red         &   fast        \\
				$\ch{2 Ba \stS} + \ch{O2 \stG} \longrightarrow \ch{2 BaO \stS}$ &   green               &   very fast   \\

			\end{tabu}\end{center}
			\end{table}\vspace{-1em}

		% end subsection


		\pagebreak
		\subsection{Reaction with Water}

			The reactivities of the metals with water increase down the group, reducing water to hydrogen gas.
			Note that \ch{Be} does not react with water (or steam), even when heated.

			All the metals form hydroxides with the exception of \ch{Mg} --- when reacting with steam,
			it forms \ch{MgO} directly, and ptherwise reacts slowly with cold water to form \ch{Mg(OH)2}.

			\txtreactioneqn{
				\ch{2 Mg \stS} + \ch{H2O \stG} \arrow \ch{MgO \stS} + \ch{H2 \stG}
			}

			The hydroxide formed, since it is only sparingly soluble, acts as a protective layer that slows down the further reaction of
			the underlying metal with water.

			For the rest of the metals, hydroxides are formed directly with increasing vigour.

			\txtreactioneqn{
				\ch{Ba \stS} + \ch{2 H2O \stL} \arrow \ch{Ba(OH)2 \stAq} + \ch{H2 \stG}
			}

			Note that the solubility of \ch{Ca(OH)2} is an equilibrium, and exists as either slaked lime in
			the solid form, or limewater in the aqueous form. The latter is produced when reacted with excess
			water.

			Finally, all the group 2 hydroxides are basic, except for beryllium hydroxide (\ch{Be(OH)2}) which is amphoteric.

		% end subsection
	% end section


	\section{Properties of the Oxides}

		\subsection{Physical Properties}

			All group 2 oxides are solid at room temperature, white, basic, and have very high melting
			and boiling points. Melting and boiling points for the oxides decrease down the group, due to a decreasing lattice energy.

		% end subsection


		\subsection{Reaction with Water}

			All group 2 oxides react with water to form their respective hydroxides (hence they are basic
			oxides). The exeption is \ch{BeO} which is amphoteric but does not react with water.

			The vigour of the reaction increases down the group, since the magnitude of the lattice energy
			of the ionic oxide decreases. In particular, magnesium oxide reacts \itl{very slowly} with cold water.

			\txtreactioneqn{
				\ch{BaO \stS} + \ch{H2O \stL} \arrow \ch{Ba(OH)2 \stAq}
			}

		% end subsection

	% end section


	\pagebreak
	\section{Thermal Stability of Group 2 Compounds}

		\subsection{Overview}

			All group 2 hydroxides, carbonates and nitrates thermally decompose at appropriate
			temperatures. The trend for all 3 is that the temperature required \itl{increases} down
			the group, aka. the stability of these ionic compounds increases down the group.

			The data for beryllium is generally sparse, so... it's not there. Oops.

			\begin{table}[htb]\renewcommand{\arraystretch}{1.5}\begin{center}
			\begin{tabu} to 0.8\textwidth {X[3,c,m] | X[c,m] | X[c,m] | X[c,m]}

				% headings
				Element &   \ch{MCO3}           &   \ch{M(NO3)2}        &   \ch{M(OH)2}         \\ \hline
				\ch{Mg} &   \SI{400}{\celsius}  &   \SI{450}{\celsius}  &   \SI{300}{\celsius}  \\
				\ch{Ca} &   \SI{900}{\celsius}  &   \SI{575}{\celsius}  &   \SI{390}{\celsius}  \\
				\ch{Sr} &   \SI{1280}{\celsius} &   \SI{635}{\celsius}  &   \SI{466}{\celsius}  \\
				\ch{Ba} &   \SI{1360}{\celsius} &   \SI{675}{\celsius}  &   \SI{700}{\celsius}  \\

			\end{tabu}\end{center}
			\end{table}\vspace{-1em}

			The main factor affecting the thermal stability is the polarising power of the ion, which depends
			on charge density --- this decreases down the group, leading to more stable ionic compounds.

		% end subsection

		\subsection{Mechanism of Action}

			The decomposition of all 3 categories of ionic compounds follow the same principle. Given the
			relatively high charge density of the group 2 metals, they are able to \itl{polarise} the
			covalent bonds \itl{within the anion}, weakening them.

			This weakening of the covalent bonds in the anion allows it to break when heat is applied,
			typically resulting in a metal oxide as the product.

			\diagram[1.0]{
				\schemestart[0, 1.5, thick]
					\chemfig{Mg\sps{2+}-[:0,1.3,,,dotted]!\molO(-[@{bond}:0,,,,lddbond]C(
					-[:60,,,,lddbond]!\molO)(-[:300,,,,lddbond]!\molO)(-[:110,,,,draw=none]@{txt}weak))}
				\schemestop
				\chemmove{\draw[-Stealth,line width=0.4mm,shorten <=2mm,shorten >=2mm](txt) -- (bond);}
			}[The concept is similar for nitrates and hydroxides, and shows why the product is usually the metal oxide.]

			When heat is applied, the weak bond will be broken, leading to the formation of
			\ch{MgO} and \ch{CO2}.

			As the polarising power of the metal ion decreases down the group (due to increasing radius but
			constant charge), the bond is weakened to a lesser degree, leading to a higher stability.

			\pagebreak
			\subsubsection{Carbonates}

				The thermal decomposition of group 2 carbonates produces the metal oxide and carbon dioxide gas.

				\txtreactioneqn{
					\ch{BaCO3 \stS} \arrow{->[\tinytext{heat}]} \ch{BaO \stS} + \ch{CO2 \stG}
				}

			% end subsubsection


			% small note: moved up here to make the arrow misalignment less jarring
			\subsubsection{Hydroxides}

				Thermally decomposing hydroxides produces the metal oxide and water.

				\txtreactioneqn{
					\ch{Ba(OH)2 \stS} \arrow{->[\tinytext{heat}]} \ch{BaO \stS} + \ch{H2O \stL}
				}


			% end subsubsection


			\subsubsection{Nitrates}

				The thermal decomposition of nitrates yields the metal oxide, nitrogen, and oxygen.

				\txtreactioneqn{
					\ch{2 Ba(NO3)2 \stS} \arrow{->[\tinytext{heat}]} \ch{2 BaO \stS} + \ch{4 NO2 \stG} + \ch{O2 \stG}
				}

			% end subsubsection

		% end subsection


		\subsection{Comparison with Group 1 Metals}

			Most group 1 metals do not have sufficient charge density to polarise the
			carbonate anion, however they do polarise the nitrate ion enough for some decomposition:

			\txtreactioneqn{
				\ch{2 NaNO3 \stS} \arrow{->[\tinytext{heat}]} \ch{2 NaNO2 \stS} + \ch{O2 \stG}
			}

			Lithium is the exception to this rule --- it has a sufficiently high charge density (due to its
			tiny ionic radius) to polarise nitrates fully, decomposing them in the same way as group 2 ions, and
			can also do the same to carbonates.

			\txtreactioneqn{
				\ch{Li2CO3 \stS} \arrow{->[\tinytext{heat}]} \ch{Li2O \stS} + \ch{CO2 \stG}
			}

		% end subsection

	% end section


	\pagebreak
	\section{Beryllium Being Boneheaded}

		The majority of the discrepancies with beryllium can be attributed to the highly covalent character
		of bonds formed with \ch{Be}, partially due to its high charge density which can polarise the electron
		cloud of anions.

		Compounds with large anions, such as \ch{Be\Cl2}, are completely covalent.



		\subsection{Non-reaction with Water and Oxygen}

			Similar to magnesium but more extreme, beryllium forms an \itl{impervious} oxide layer when
			exposed to oxygen --- this layer reacts with neither water nor oxygen, leading to beryllium's unreactivity.

		% end subsection


		\subsection{Amphoteric Nature}

			Beryllium oxides and hydroxides are amphoteric unlike its other group 2 compatriots, due to the partially
			covalent nature of the bonds with \ch{Be}.

			\begin{bulletlist}
				& Acid:\tabto{15mm} $\ch{BeO + 2 NaOH + H2O} \longrightarrow \ch{Na2[Be(OH)4]}$
				& Base:\tabto{15mm} $\ch{BeO + H2SO4} \longrightarrow \ch{BeSO4 + H2O}$
			\end{bulletlist}

			Note the formation of the \ch{[Be(OH)4]^2-} complex, which will be discussed below.

		% end subsection


		\subsection{Formation of \ch{Be} complexes}

			\subsubsection{Hydrated Ions}

				As will be discussed later, \ch{Mg^2+} has sufficient charge density to form hydrated complex
				ions in water (\ch{[Mg(H2O)6]^2+}). Thus, it is clear that \ch{Be} should form these complexes
				as well, and it does --- \ch{[Be(H2O)4]^2+}.

				However, \ch{Be^2+} is only 4-coordinate, due to the lack of vacant d-orbitals (there is no 2d
				subshell). By making do with only the 2s and 2p orbitals, it bonds with 4 ligands. By contrast
				\ch{Mg^2+} has a 3d subshell, allowing for up to 6 ligands.

			% end subsubsection


			\pagebreak
			\subsubsection{Other Complexes}

				While \ch{Be^2+} can form complexes, \ch{Be} can also form complexes when bonded with other
				atoms, since it still has 2 vacant p-orbitals that can be used to bond with ligands.

				Examples include the formation of \ch{[Be(OH)4]^2-} when \ch{Be(OH)2} is exposed to
				more \ch{OH-}, and the formation of \ch{[BeF4]^2-}:

				\txtreactioneqn{
					\ch{Be(OH)2 \stAq} + \ch{2 OH- \stAq} \arrow \ch{[Be(OH)4]^2- \stAq}
					\arrow(@c1.east--.east){0}[-90,.4]
					\ch{BeF2 \stAq} + \ch{2 F- \stAq} \arrow \ch{[BeF4]^2- \stAq}
				}

				A point to note is that \ch{[Be(OH)4]^2-} is created from \ch{Be(OH)2}, which is itself formed
				from the hydrolysis of a beryllium salt in water (since \ch{BeO} does not react with water).

				On adding a base (aka \ch{OH-}), the hydrated \ch{[Be(H2O)4]^2+} complex acts as an acid,
				forming \ch{[Be(OH)4]^2-} by losing 4 \ch{H+} ions.

			% end subsubsection

		% end subsection

		\subsection{Acidity of Beryllium Salts}

			Similar to other cations of high charge density like \ch{\Al^3+}, \ch{Fe^3+} and \ch{Cr^3+} (and to
			a certain extent \ch{Mg^2+}), \ch{Be^2+} forms a hydrated complex in water, \ch{[Be(H2O)4]^2+},
			which was seen above.

			Just like the other complexes, it can hydrolyse to give \ch{H3O}+, making it acidic in water.

			\txtreactioneqn{
				\ch{[Be(H2O)4]^2+} + \ch{H2O} \arrow \ch{[Be(H2O)3(OH)]+} + \ch{H3O+}
			}

			\ch{[Be(H2O)3(OH)]+} can undergo further hydrolysis three more times, to finally form \ch{[Be(OH)4]^2-}.
			This enables beryllium to react with bases, and hence be amphoteric.


		% end subsection

	% end section

% end part



























